\chapter{La diffraction}
\begin{wrapfigure}[8]{r}{2cm}
\vspace{-6mm}
\includegraphics[scale=0.30]{ch2/image4.png}
\captionof{figure}{ }
\end{wrapfigure}
Un faisceau laser se propageant à tendance à s'élargir et se déformer : il s'agit du 
phénomène de diffraction. Ce phénomène est bien précis et peut être décrit mathématiquement. 
Il dépend notamment de la forme du laser. En faisant passer un laser par une section carrée, 
on sera amener à observer des distributions d'intensités particulières.

\section{Équations de Maxwell et transformée de Fourier}
La linéarité des équations de Maxwell permet un traitement efficace par les transformées 
de Fourier. Voici les fameuses équations dans le vide
\begin{equation}
\left\{\begin{array}{llll}
\rot \vec{E} &= -\dfrac{\partial \vec{B}}{\partial t},\qquad &\div \vec{E} &= 0\\
\rot \vec{B} &= \epsilon_0\mu_0\dfrac{\partial \vec{E}}{\partial t},\qquad &\div \vec{B} &= 0
\end{array}\right.
\end{equation}
Sachant que $\rot [\rot \vec{E}] = \grad[\div \vec{E}] - \Delta\vec{E}$ où $\div \vec{E}=0$ dans 
le vide on retrouve l'équation d'onde électromagnétique
\begin{equation}
\Delta \vec{E} = \mu_0\epsilon_0\dfrac{\partial^2\vec{E}}{\partial t^2}\quad \text{où } \Delta 
\vec{E} = \Delta E_x\vec{1_x}+\Delta E_y\vec{1_y}+\Delta E_z\vec{1_z}
\end{equation}
On ne considérera ici que la composante en $x$ du champ
\begin{equation}
\Delta E_x = \mu_0\epsilon_0\dfrac{\partial^2E_x}{\partial t^2}
\end{equation}
On remplacera souvent $E_x,E_y$ ou $E_z$ par $E$ mais il faut garder à l'idée qu'il ne s'agit 
qu'une seule des composantes du champ. L'équation d'onde scalaire s'écrit
\begin{equation}
\dfrac{\partial E}{\partial x^2}+\dfrac{\partial E}{\partial y^2}+\dfrac{\partial E}{\partial z^2} 
= \mu_0\epsilon_0\dfrac{\partial^2E}{\partial t^2}
\end{equation}
Cette équation est \textit{linéaire}, $E$ n'apparaissant qu'à la première puissance. Si $E_1$ et 
$E_2$ sont solution, une combili de ces deux solutions est également solution. On va profiter 
de cette linéarité pour exprimer la solution de cette équation comme une somme d'onde harmonique, 
d'où l'utilité des transformée de Fourier. Le champ électrique se verra décomposé en une somme 
de \textit{fonctions harmonique}\footnote{Écrit ci-dessous dans le domaine des phaseurs.} :
\begin{equation}
E(x,y,z,t) = \iiiint_{-\infty}^\infty \tilde{E}(k_x,k_y,k_z,\omega)e^{ik_xx}e^{ik_yy}e^{ik_zz}
e^{-i\omega t}\ dk_xdk_ydk_zd\omega
\end{equation}
Si le facteur harmonique $e^{ik_xx}e^{ik_yy}e^{ik_zz}e^{-i\omega t}$  est solution des équations 
de Maxwell $\forall k_i,\omega$, alors $E$ sera solution. La résolution de ces équations ne sont 
pas aisées. Pour le faire de façon analytique, on travaillera avec la décomposition. Vérifions 
que ces fonctions harmoniques sont bien solution
\begin{equation}
\dfrac{\partial^2 e^{ik_xx}e^{ik_yy}e^{ik_zz}e^{-i\omega t}}{\partial x^2} = -k_x^2 e^{ik_xx}e^{ik_yy}
e^{ik_zz}e^{-i\omega t}
\end{equation}
En faisant de même pour les deux autres dérivées partielles pour obtenir
\begin{equation}
(-k_x^2-k_y^2-k_z^2)e^{ik_xx}e^{ik_yy}e^{ik_zz}e^{-i\omega t} = -\mu_0\epsilon_0\omega^2 
e^{ik_xx}e^{ik_yy}e^{ik_zz}e^{-i\omega t}
\end{equation}
C'est-à-dire
\begin{equation}
k_x^2+k_y^2+k_z^2 = \mu_0\epsilon_0\omega^2
\end{equation}
Il s'agit d'une \textbf{contrainte} sur les modes de Fourier. On peut voir l'expression de $E$ 
comme une transformée de Fourier où $\tilde{E}$ est le spectre de Fourier généralisé à 4 
dimensions. En terme de transformée de Fourier, on peut dire qu'il s'agit d'une contrainte 
sur les \textit{modes de Fourier} (les quatre facteurs exponentiels), soit une \textbf{relation 
de dispersion généralisée}. Si les modes satisfont cette contrainte, $E$ sera solution.\\

\begin{wrapfigure}[9]{r}{3cm}
%\vspace{-6mm}
\includegraphics[scale=0.45]{ch2/image5.png}
\captionof{figure}{ }
\end{wrapfigure}
Intéressons-nous aux aspects physiques sous-jacent à cette décomposition de Fourier du champ 
électrique. Commençons par la ré-écriture suivante en supposant que les $k_i$ sont les 
composantes d'un vecteur
\begin{equation}
\tilde{E}(k_x,k_y,k_z,\omega)e^{ik_xx}e^{ik_yy}e^{ik_zz}e^{-i\omega t} = \tilde{E}(\vec{k},
\omega)e^{i\vec{k}.\vec{r}}e^{-i\omega t}
\end{equation}
\danger $\vec{r}$ est le vecteur position, il désigne le point de l'espace considéré. On peut 
ainsi exprimer la condition pour laquelle le mode de Fourier satisfait les équations de Maxwell  :
\begin{equation}
|k|^2 = k^2 = \mu_0\epsilon_0\omega^2
\end{equation}

\begin{wrapfigure}[9]{l}{3cm}
\vspace{-6mm}
\includegraphics[scale=0.45]{ch2/image6.png}
\captionof{figure}{ }
\end{wrapfigure}
Cela signifie que le vecteur d'onde est tendu d'être sur une sphère de rayon $k=
\sqrt{\mu_0\epsilon_0}\omega$. Figeons le temps et cherchons le lieux des points de phase 
constante. Pour avoir une phase constante, $\vec{k}.\vec{r}$ doit être constant : tous les 
points qui ont la même projection auront la même phase, il s'agit d'une \textit{onde plane}.
Ci-contre, la représentation des fronts d'ondes. Analysons cela de façon analytique
\begin{equation}
\varphi = \vec{k}.\vec{r} = \text{cste}\quad \Rightarrow k_xx+k_yy+k_zz = \text{ cste}
\end{equation}
Il s'agit de l’équation d'un plan perpendiculaire au vecteur d'onde. Les modes de Fourier 
réprésentent bien une onde plane dont la direction $\vec{k}/k$. On peut toujours considérer 
$\vec{k}/k = \vec{1_z} \rightarrow \vec{k}= k\vec{1_z}$ de sorte à écrire
\begin{equation}
E = \tilde{E}e^{ikz}e^{-i\omega t}
\end{equation}
Ceci montre que lorsque $z$ est fixé, le champ ne varie pas en $x$. Si l'on considère la 
partie réelle de ceci, on retrouve $\tilde{E}\cos(kz-\omega t)$ ce qui est bien l'équation 
d'une onde plane.\\

Que se passe-t-il si on libère le temps? Intéressons-nous d'abord à la périodicité en 
$z : \lambda = 2\pi/k$ ce qui correspond à l'espacement des fronts d'ondes. Considérons 
une phase constante
\begin{equation}
\text{Front d'onde : } \varphi = \text{ cste} \Rightarrow  kz-\omega t = \text{ cste} \qquad 
\Longleftrightarrow z = \dfrac{\omega}{k}t+\dfrac{\text{cste}}{k}
\end{equation}
Pour garder la constante, quand le temps augmente $z$ doit également augmenter. On obtient 
ici la \textit{vitesse de phase} $\omega/k = v_\phi$ où $k = \sqrt{\mu_0\epsilon_0}\omega$. 
Cette dernière condition est obligatoire pour avoir quelque chose de physique vérifiant 
les équations de Maxwell. Par substitution
\begin{equation}
v_\varphi = \dfrac{1}{\sqrt{\mu_0\epsilon_0}} \equiv c
\end{equation}
On en tire que $k=\omega/c$.\\

\underline{Remarque}. Considérons un champ scalaire solution des équations d'ondes 
\begin{equation}
\vec{E} = (A_x\vec{1_x}+A_y\vec{1_y}+A_z\vec{1_z})e^{ikz}e^{ikz}e^{-i\omega t}
\end{equation}
où les amplitudes $A_i$ ne sont pas nécessairement connues. Nous allons voir qu'il y a 
une contrainte sur ces amplitudes. Appliquons la divergence nulle du champ électrique 
dans le vide
\begin{equation}
\div \vec{E} = 0 \quad \Longrightarrow\quad \dfrac{\partial E_x}{\partial x}+
\dfrac{\partial E_y}{\partial y} + \dfrac{\partial E_z}{\partial z} = 0
\end{equation}

\begin{wrapfigure}[6]{l}{3cm}
\vspace{-16mm}
\includegraphics[scale=0.45]{ch2/image7.png}
\captionof{figure}{ }
\end{wrapfigure}
Or, il n'y a pas de dépendance en $x$ et $y$ comme le montre le schéma ci-contre. Avoir 
une composante selon $y$, ici $A_y$ n'implique pas une dépendance en $y$! Dès lors, 
seule la dérivée par rapport à $z$ est non-nulle. On en tire
\begin{equation}
iA_zke^{ikz}e^{-i\omega t} = 0\qquad \forall z,t
\end{equation}
Il en résulte que $A_z = 0$. Conclusion : le champ électrique est toujours transverse à 
l'axe $z$ pour une onde se déplaçant sur ce même axe.\\

Revenons à la description mathématique de la SG des EDP de Maxwell. Rappelons la \textbf{contrainte 
sur les modes de Fourier} 
\begin{equation}
k = \dfrac{\omega}{c}
\end{equation}
Le vecteur d'onde doit obligatoirement se balader sur une sphère :
\begin{equation}
k_x^2+k_y^2+k_z^2 = k^2 = \dfrac{\omega^2}{c}
\end{equation}
Rentrons cette contrainte dans l'expression de $E$. Or, nous n'avons que trois dimension : on 
fait le choix d'exprimer $k_z$ en fonction de la contrainte :
\begin{equation}
k_z = \sqrt{k^-k_x^2-k_y^2}
\end{equation}
Si $k_z$ satisfait cette relation, c'est gagné. Pour être solution, il faut que le spectre ai 
la forme particulière définie ci-dessous, qui ne dépend plus de $k_z$ la variable ayant été 
"sacrifiée" :
\begin{equation}
\tilde{E}(k_x,k_y,k_z,\omega) = \tilde{E}(k_x,k_y,\omega)\delta\left(k_z-\sqrt{k^-k_x^2-k_y^2}\right)
\end{equation}
où l'on introduit le delta de Dirac : si le spectre est limité à des spectres de cette forme là, 
d'office ce sera solution des équations de Maxwell\footnote{C'est une façon mathématique d'imposer 
la valeur de $k_z$.}. En substituant dans mon équation $\iiiint$, une des intégrales sera simplifiée 
par le delta de Dirac.
\begin{equation}
E(x,y,z,t) = \iiint_{-\infty}^\infty \tilde{E}(k_x,k_y,\omega) e^{ik_xx}e^{ik_yy}e^{i 
\sqrt{k^-k_x^2-k_y^2} z}e^{-i\omega t}\ dk_xdk_yd\omega
\end{equation}
Ce qui n'est rien d'autre que la solution générale des équations de Maxwell. Nous allons nous limiter 
à des ondes monochromatiques, c'est à dire pour un seul $\omega_0$. On se limite ainsi à un spectre 
spatial pour une fréquence donnée
\begin{equation}
\tilde{E}(k_x,k_y,\omega) = A(k_x,k_y)\delta(\omega-\omega_0)
\end{equation}
Notre intégrale triple devient
\begin{equation}
E(x,y,z,t) = \iint_{-\infty}^\infty A(k_x,k_y) e^{ik_xx}e^{ik_yy}e^{i 
\sqrt{k^-k_x^2-k_y^2} z}\ dk_xdk_ye^{-i\omega_0 t}
\end{equation}
où $\omega$ n'apparaît plus : il est caché dans la définition de $k$. On considère ici des ondes 
monochromatiques, on va dès lors s'affranchir du caractère temporel qui ne nous intéresse pas ici. 
Soit 
\begin{equation}
E(x,y,z,t) = a(x,y,z)e^{-i\omega_0t}
\end{equation}
où $a$ est l'amplitude indépendante du temps. On a donc
\begin{equation}
a(x,y;z) = \iint_{-\infty}^\infty A(k_x,k_y) e^{ik_xx}e^{ik_yy}e^{i 
\sqrt{k^-k_x^2-k_y^2} z}\ dk_xdk_y
\end{equation}
où $k^2 = \omega_0^2/c^2$. Ceci est à coup sur une solution des équations de Maxwell. Allégeons 
les notations : $k_x=\rho, k_y=\sigma, k_z=\beta=\sqrt{k^2-\rho^2-\sigma^2}$ pour avoir
\begin{equation}
a(x,y;z) = \iint_{-\infty}^\infty A(\rho,\sigma)e^{i\rho x}e^{i\sigma y} e^{i\sqrt{k^2-\rho^2-
\sigma^2}z}\ d\rho d\sigma
\end{equation}
Il s'agit d'une intégrale de Fourier, une combinaison linéaire d'onde plane qui ont certaines 
amplitudes qui représentent le spectre du champ. Il s'agit de l'expression d'une \textit{solution 
générale des équations de Maxwell pour une onde monochromatique}, le point de départ de la 
théorie de diffraction.


\newpage
\section{Diffraction et transformée de Fourier}
Histoire de nous habituer aux notations, reprenons la définition de la transformée de Fourier
\begin{equation}
F(\rho) = \int_{-\infty}^\infty f(x)e^{-i\rho x}\ dx
\end{equation}
\danger Il y a un signe négatif. Ce signe ne change physiquement rien, il faudra simplement 
inverser le signe de la transformée inverse. Il faut juste remarquer que le $i$ de l'exponentielle 
n'a pas le même signe dans la partie temporelle et spatiale : pour le temporelle, on utilise un 
moins et pour les variation spatiales un plus\footnote{Confusion, à éclaircir}. La transformée 
inverse vaudra alors
\begin{equation}
f(x) = \dfrac{1}{2\pi}\int_{-\infty}^\infty F(\rho)e^{i\rho x}\ d\rho = TF^{-1}[F(\rho)]
\end{equation}
Il faut remarquer que $a(x,y;z)$ est une transformée inverse. L'idée est que comme nous ne nous 
mesurerons jamais de valeur précise, il n'est pas utile de garder le facteur $1/2\pi$.\\

On remarque que le rôle de $z$ est différente des autres variables, les bornes d'intégrations 
ne portent pas sur lui : il n'apparaît que comme un simple paramètre. Il s'agit donc bien d'une 
fonction de $x$ et $y$. On remarque que l'on calcule la transformée inverse d'une fonction dans 
l'expression de $a(x,y;z)$ :
\begin{equation}
A(\rho,\sigma)e^{i\sqrt{k^2-\rho^2-\sigma^2}z}
\end{equation}
Il s'agit du \textit{spectre de Fourier} de la distribution transverse du champ en une valeur $z$, 
$a(x,y;z)$. Les $\rho,\sigma$ sont les fréquences spatiales. On peut ainsi écrire
\begin{equation}
a(x,y;z) = TF^{-1}\left[ A(\rho,\sigma)\underbrace{e^{i\sqrt{k^2-\rho^2-\sigma^2}z} }_{(*)}\right]
\end{equation}
Commençons l'étude en $z=0$ (C.I., connue)
\begin{equation}
a(x,y;z) = TF^{-1}\left[ A(\rho,\sigma) \right]\quad \Rightarrow \quad A(\rho,\sigma) = TF[
a(x,y;0)]
\end{equation}
On remarque que ceci n'est rien autre que le spectre de la transformée de Fourier de la distribution 
initiale. Le problème de diffraction est que connaissant la distribution initiale, on veut connaître 
la distribution $a(x,y;z)$ pour tout $z$. Si $z\neq0$, un facteur exponentiel supplémentaire apparaît 
: ce n'est rien d'autre que le \textbf{propagateur} $(*)$ de spectre. Le spectre initial est multiplié 
par un simple facteur de phase, une sorte de phaseur. Hélas, s'exprimer dans le spectre ne donne pas 
grand chose, le problème est la conversion spatiale.\\


Intéressons-nous avant tout à l'interprétation physique de la S.G., rappelée ici
\begin{equation}
a(x,y;z) = \iint_{-\infty}^\infty A(\rho,\sigma)e^{i\rho x}e^{i\sigma y} e^{i\sqrt{k^2-\rho^2-
\sigma^2}z}\ d\rho d\sigma
\end{equation}
Posons $\beta = \sqrt{k^2-\rho^2-\sigma^2}$
\begin{equation}
a(x,y;z) = \iint_{-\infty}^\infty A(\rho,\sigma)e^{i(\rho x + \sigma y + \beta z)}\ d\rho d\sigma
\end{equation}
On peut alors dire que le phaseur a une phase $\varphi \rho x + \sigma y + \beta z$. Si on considère 
cette phase constante, on considère le lieu des points vérifiant 
\begin{equation}
\varphi \rho x + \sigma y + \beta z = 2m\pi
\end{equation}
Il s'agit d'une équation de plan dont $\rho,\sigma$ et $\beta$ sont les cosinus directeur, les 
composantes du vecteur $k$. Ce phaseur n'est qu'une onde plane qui se propage avec un certain 
angle par rapport à l'axe $z$. Analysons l'intégrale dans sa globalité en faisant passé le 
faisceau laser par une fonction fenêtre (une fente infinie en $y$) :
\begin{equation}
\left\{\begin{array}{ll}
a(x';0) = a_0 & \text{ si } |x'| < l\\
a(x';0) = 0 & \text{ si } |x'| > l
\end{array}\right.\quad \Rightarrow\quad A(\rho) = 2l\ \text{sinc}(l\rho)
\end{equation}


\begin{wrapfigure}[7]{l}{3cm}
\vspace{-4mm}
\includegraphics[scale=0.45]{ch2/image8.png}
\captionof{figure}{ }
\end{wrapfigure}
Admettons que l'on rentre le spectre dans l'intégrale, on remarque que pour des valeurs de $z>0$, 
le champ peut s'exprimer comme une sorte d'onde plane. Chaque valeur de $\rho$ représente un certain 
angle $\theta$ (projection sur $x$) et à chaque valeur de $\rho$ est associé une certaine amplitude : 
a chaque angle est associé une amplitude. Les équations de Maxwell (ici, sa solution) dit que l'on 
peut représenter par une somme d'onde plane dont l'amplitude est donnée par la transformée de 
Fourier.\\

Pour des $z>0$, on a une ensemble d'onde plane dont l'amplitude est donnée par la transformée de 
Fourier de la condition initiale : $\rho$ représente l'angle qui est $\theta = \arcsin(\rho/k)$. 
On a bien une somme d'onde plane qui donne lieu au phénomène de diffraction. Passons en revue 
quelques exemples triviaux.
\begin{enumerate}
\item Onde plane en $z=0 : a(x,y;0) = a_0$.\\
Le spectre recherché est le spectre de la condition initiale, c'est-à-dire la constante $a_0$\footnote{
$A(\rho) = \int_{-\infty}^\infty a_0e^{-i\rho x}\ dx = 2\pi a_0\delta(\rho)$}
\begin{equation}
A(\rho,\sigma) = TF[a(x,y;0)] = a_0\delta(\rho,\sigma)
\end{equation}
Après substitution
\begin{equation}
a(x,y;z) = a_0\iint_{-\infty}^\infty \delta(\rho,\sigma)e^{i\rho x}e^{i\sigma y} e^{i\sqrt{k^2-\rho^2-
\sigma^2}z}\ d\rho d\sigma
\end{equation}
Dès lors
\begin{equation}
a(x,y;z) = a_0e^{ikz}
\end{equation}
Ce qui est bien la réponse attendue. Il n'y a pas de diffraction, c'est un "mode de propagation" qui 
est invariant.
\item Onde plane inclinée selon $x = a(x,y;0) = a_0e^{i\rho_0x}$.\\
Ceci représente bien une onde avec un certain angle car $\rho_0 = l\sin\theta_0$. Après 
substitution (on choisit $\sigma=0$).
\begin{equation}
A(\rho,\sigma) = a_0\int e^{i\rho_0x}e^{-i\rho x}\ dx \int e^{-i\sigma y}\ dy
\end{equation}
On obtient le produit de deux Dirac, noté 
\begin{equation}
A(\rho,\sigma) = a_0\delta(\rho-\rho_0,\sigma)
\end{equation}
On retrouve une onde plane qui ne sera pas non plus modifiée par sa propagation
\begin{equation}
a(x,y;z) = a_0e^{i\rho_0x}e^{i\sqrt{k^2-\rho_0^2}z}
\end{equation}
Notons que $\beta_0 = k\cos\theta_0$. Les fronts d'onde seront les lieux de phase 
constantes
\begin{equation}
\varphi = k\sin\theta_0x + k\cos\theta_0 z = 2m\pi
\end{equation}
Un mode de Fourier est une onde plane, il s'agit de quelque chose qui ne se déforme pas.

\item Modulation harmonique transverse : $a(x,y;0) = a_0\cos(\rho_0x)$.\\
On retrouvera le même résultat si on se souvient que l'on peut écrire
\begin{equation}
a(x,y;0) = a_0\frac{e^{i\rho_0x}+e^{-i\rho_0x}}{2}
\end{equation}
Le spectre vaut
\begin{equation}
A(\rho,\sigma) = a_0\frac{\delta(\rho-\rho_0,\sigma)+\delta(\rho+\rho_0,\sigma)}{2}
\end{equation}
Le spectre est composé de deux delta de Dirac qui sélectionneront deux $\rho_0$ et 
$\sigma$ sera toujours bien nul:
\begin{equation}
a(x,y;z) = a_0\cos(\rho_0x)e^{i\sqrt{k^2-\rho_0^2}z}
\end{equation}
Il s'agit d'un autre mode. Le cosinus n'est la somme de deux ondes planes : une se 
déplaçant à un angle $\theta_0$ et l'autre avec un angle $-\theta_0$.

\item Diffraction par une fente infinie. En $z=0$ :
\begin{equation}
\left\{\begin{array}{ll}
a(x;0) = a_0 & \text{ si } |x| < l\\
a(x;0) = 0 & \text{ si } |x| > l
\end{array}\right.
\end{equation}
On considère une onde plane sur un écran qui va l'absorber sauf sur une largeur $2l$. 
Réduisons le problème en $2D$ ($x$ et $z$- : rien ne change en y, la fente est infinie.  
La transformée de Fourier est bien connue
\begin{equation}
A(\rho) = 2l\ sinc(\rho l)\qquad a_0=1
\end{equation}
Ci-dessous, la densité spectrale du spectre $|A(\rho)|^2$. 
\begin{center}
\includegraphics[scale=0.45]{ch2/image9.png}
\captionof{figure}{L'angle en abscisse (indirect) est celui de propagation. Pour une certaine 
direction, ceci donne l'amplitude correspondante.}
\end{center}
Cette fois-ci, la fonction est 
moins triviale : on ne calculera pas on se contentera de la physique. Si l'on place un 
détecteur, un certain $\rho$ va sélectionné un certain $\theta$ et cette valeur donne une 
certaine amplitude. On tombe sur une difficulté conceptuelle : on trouve la valeur de $k$ 
puis $\rho$ dépasse $k$ ; on recherche l'arc dont le sinus est supérieur à 1\footnote{$\theta =
\arcsin\frac{\rho}{k}$}.\\

Regardons le problème de la diffraction
\begin{equation}
a(x;z) = \int 2l\ \text{sinc}(l\rho)e^{i\sqrt{k^2-\rho_0^2}z}e^{i\rho x}\ d\rho
\end{equation}
A cause des bornes d'intégration, nous aurons d'office des valeurs de $\rho$ dépassant 
$k$. Passons outre et calculons :
\begin{equation}
\text{Si } \rho > k\quad \text{alors}\quad \beta = \sqrt{k^2-\rho^2} = i\sqrt{\rho^2-k^2} = 
i\alpha
\end{equation}
Ceci implique que l'exponentielle imaginaire présente dans $a(x;z)$ devient réelle :
\begin{equation}
e^{i\beta z} = e^{-\alpha z}
\end{equation}
On a ce que l'on appelle des \textbf{ondes évanescentes} : plutôt que d'avoir un comportement 
périodique à un comportement amorti. Ce que l'on va faire, c'est séparer l’intégrale en deux 
parties :
\begin{equation}
a(x;z) = \int_{-k}^k  2l\ \text{sinc}(l\rho)e^{i\sqrt{k^2-\rho_0^2}z}e^{i\rho x}\ d\rho + 
\int_{|\rho|>k}  2l\ \text{sinc}(l\rho)e^{-\sqrt{k^2-\rho_0^2}z}e^{i\rho x}\ d\rho
\end{equation}
Le deuxième terme somme des ondes évanescentes. En $z$, elles évoluent en exponentielles 
négatives. Quelle est cette absorption ? Il s'agit d'un phénomène de réflexion (il représente 
des photons qui passent l'écran et retournent de la ou ils viennent). 

\begin{center}
\includegraphics[scale=0.25]{ch2/image10.png}
\captionof{figure}{ }
\end{center}
Par simplification, 
on ne va s'arranger pour qu'elles soient négligeables. Pour se faire, analysons le spectre. 
La fonction sinus cardinal a un lobe principal : toute l'énergie est proche de l'origine. Nous 
allons travailler de sorte que les objets diffractant vérifient
\begin{equation}
\dfrac{2\pi}{l} \ll k = \dfrac{2\pi}{k}
\end{equation}
A ce moment la, le spectre est fort ramené à l'origine de sorte à ce que les ondes évanescentes 
soient négligées. Autrement dit
\begin{equation}
\int_{|\rho|>l} |A(\rho)|^2\ d\rho \ll \int_{-k}^{+k} |A(\rho)|^2\ d\rho
\end{equation}
Ce qui exprime que le spectre soit centrée sur l'origine et très étroit. Il faut dès lors que 
les objets diffractant sont significativement plus grand que la longueur d'onde : $\lambda \ll l$.\\

Le calcul de l'intégrale n'est pas évident (le facteur de phase gène). Nous allons utiliser 
l'approximation faite pour les ondes évanescentes :
\begin{equation}
\rho \approx \dfrac{2\pi}{l} \ll k
\end{equation}
On va pouvoir approximer le propagateur permettant de  ce cas, mais ceci est l'objet de la 
suivante section.
\end{enumerate}










































