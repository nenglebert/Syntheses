\chapter{Inductances et transformateurs}
\section{Tensions appliquées et induites}
Si un circuit circulaire fermé est traversé par un flux variant, 
une f.e.m. induite se créera dans le même sens que le courant $i$ 
qu'elle génère : $e = -\dfrac{d\phi}{dt}$. Par la convention 
récepteur, la tension qui équilibre cette force doit avoir un 
sens opposé au courant. On a donc
\begin{equation}
\begin{array}{ll}
v &= Ri - e\\
 &= Ri + \dfrac{d\phi}{dt}
\end{array}
\end{equation}
On considérera que $e$ est définie dans le même sens que $v$ (on 
la voit comme une tension appliquée).

\section{Le transformateur idéal}
Soit l'illustration ci-dessus avec $N_1$ et $N_2$ spires à gauche 
et à droite. Supposons que la résistivité du fer soit nulle : tout 
le flux va passer dans le fer et le flux perçu par les deux bobines 
sera identique. On définit alors le \textbf{flux totalisé} $\Psi$ :
\begin{equation}
\Psi = N\phi
\end{equation}
où $\phi$ est le flux d'une spire. La loi de Maxwell devient $v = 
\frac{d\Psi}{dt} = N\frac{d\phi}{dt}$. Comme le flux est le même 
dans les deux enroulements
\begin{equation}
\frac{\Psi_1}{\Psi_2} = \dfrac{N_1}{N_2},\qquad \frac{v_1}{v_2} = 
\frac{N_1}{N_2} = \mu.
\end{equation}
où $\mu$ est le rapport théorique des tensions. La loi des f.m.m 
donne
\begin{equation}
N_1i_1 + N_2i_2 = \underbrace{\oint_l \vec{H}.\vec{dl}}_{=0
\Leftrightarrow \mu_{Fe}=\infty}
\end{equation}
On sait que $\mu_0 \ll \mu_{Fe}$. Poussons le bouchon un peu plus 
loin : $\mu_{Fe} = \infty$. Imposer $v_1$ au montage donne un champ 
d'induction fini, mais un champ magnétique tendant vers 0 : dans un 
fer parfait, il faut une très petite force magnétomotrice ($\sum i$) 
pour faire circuler un flux. On a alors
\begin{equation}
\frac{i_1}{i_2} = -\frac{N_2}{N_1} = -\frac{1}{\mu}
\end{equation}
Ceci décrit le transformateur idéal.


\section{Inductances}
	\subsection{Inductances monophasées dans l'air}
		\subsubsection{Cas d'une seule spire}
		La flux passant à travers une spire est donné par $\phi = 
		LI$ où 		$L$ est l'inductance du circuit.\\
		\textsc{Exemple : calcul de $L$}. Si la spire est constituée 
		de deux conducteurs infini de rayon $a$, distant de $d$, 
		véhiculant un courant $i$, le flux s'écrit
		\begin{equation}
		\phi = \frac{\mu_0}{2\pi}\int_{a}^{d-a}\left(\frac{1}{x}+
		\frac{1}{d-x}\right) i \text{ dx} = \frac{\mu_0}{\pi}\ln
		\frac{d-a}{a}i
		\end{equation}
		L'inductance par unité de longueur vaut alors
		\begin{equation}
		l = \frac{\mu_0}{\pi}\ln\frac{d-a}{a}\approx\frac{\mu_0}{
		\pi}\ln\frac{d
		}{a}
		\end{equation}
		
		
		\subsubsection{Cas de plusieurs spires - Notions de flux 
		totalisé}
		La généralisation à $N$ spires est immédiate : $\Psi = 
		N\phi$ où $\phi$ est le flux d'une seule spire. Maxwell 
		se généralise de la même façon :
		\begin{equation}
		v = \frac{d\Psi}{dt} = N\frac{d\phi}{dt}
		\end{equation}
		Le milieu restant linéaire $\Psi = Li$. Grâce aux notions 
		du circuit magnétique et à la relation des Ampère-tours, on 
		peut écrire $Ni = \mathfrak{R}\phi$ où $\mathfrak{R}$ est la 
		\textbf{réluctance} du circuit d'induction. La valeur de l'
		inductance se calcule alors
		\begin{equation}
		L = \frac{\Psi}{i} = N\frac{\phi}{u} = \frac{N^2}{\mathfrak{R}} 
		= N^2 \mathcal{P}
		\end{equation}
		où $\mathcal{P}$ est la perméance du circuit. \\
		Attention cependant : si les spires ne sont pas confondues la 
		relation $\Psi = N\phi$ n'est \textbf{plus} valable ! En effet, 
		le flux ne sera pas le même pour chaque spire : on appelle 
		flux de dispersion le flux non-commun. Bonne nouvelle : on est 
		encore dans une phase linéaire : $\Psi = Li$ reste valable.
		
		
		
	\subsection{Inductances monophasées à noyau magnétique}		
		\subsubsection{Flux et inductance}
		Soit circuit magnétique fermé de longueur $l$ et de section 
		constante $S$, constitué de $N$ spires. On suppose que le flux
		reste entièrement canalisé dans le fer\footnote{Très bonne 
		approximation} de sorte que $\Psi = N\phi$ reste valable.\\
		Le souci vient de l'imperfectibilité du fer : le relation entre 
		$\Psi$ et $i$ n'est plus linéaire : $\Psi = N\phi = BNS$ et 
		$i= H l/N$. Cette dernière relation est obtenue par la courbe 
		d'hystérèse magnétique qui n'est ni linéaire, ni univoque.\\
		
		Appliquons une tension sinusoïdale $v = V_M\cos\omega t = V
		\sqrt{2}\cos\omega t$ à l'	enroulement. Le flux résultat sera 
		sinusoïdal car $v = d\psi/dt$.	En première approximation, notre 
		tension vaut (toujours vrai:)
		\begin{equation}
		v = Ri + \frac{d\Psi}{dt}
		\end{equation}
		Si la résistance est non-nulle, il faut résoudre un système à 
		deux inconnues (dont une équation est donnée par le cycle d'
		hystérèse, Oh joie), la présence de $i$ compliquant l'ED. Par 
		contre, si $R=0$ :
		\begin{equation}
		\phi = \frac{\Psi}{N} = \frac{1}{N}\int_0^tv\text{ dt} + \text{ 
		cste}
		\end{equation}
		ce qui vaut\footnote{Q: supposez qu'on ai une bobine comme ça et 
		que l'on met 12V. Si le courant est alternatif c'est un courant 
		alternatif. Représentez le ? Si on met du ctn, on détruit la bobine 
		(ça fume, savoir expliqué).} $\phi = \frac{V_M}{N\omega}\cos\left(
		\omega t - \frac{\pi}{2}\right)$. Le phaseur $\underline{\phi}$ est 
		déphasé de $-\frac{\pi}{2}$ par rapport à $\underline{V}$.\\
		Comme $V_M = N\omega\phi_M$ :\footnote{??}
		\begin{equation}
		\phi_M : \dfrac{\sqrt{2}}{2\pi f N}V
		\end{equation}
		Si la tension appliquée est sinusoïdale, la tension et donc 
		l'induction le sera également, mais le courant absorbé par la 
		bobine ne l'est pas.
			
		\subsubsection{Courant absorbé}
		Sur le schéma ci-contre\footnote{Ajouter des explications}, $i_m$ 
		est le courant obtenu en ne considérant que la courbe d'aimantation 
		moyenne auquel il faut ajouter $i_{pH}$, le courant de pertes 
		hystérétiques pour donner le courant total $i$.
		
		
		\subsubsection{Pertes hystérétiques et par courants de Foucault}
		Si on place $i_{pH}$ sur un phrase, on va remarquer que sa courbe 
		est en phase sur celle de la tension : on va pouvoir le modéliser 
		par une résistance.
		
		\subsubsection{Schéma équivalent}
		Le courant absorbé par la bobine est composé d'un courant 
		magnétisant $I_m$ et un courant de pertes par hystérèse et par 
		Foucault $I_p$. On a vu que les pertes $\propto V^2$ et qu'elles 
		peuvent être représentées par une résistance $R_p$ de sort que 
		\begin{equation}
		\underline{V} = R_p\underline{I_p}
		\end{equation}
		Le courant magnétisant n'est pas sinusoïdal mais on peut définir 
		un \textit{courant magnétisant sinusoïdal équivalent} déphasé de 
		$\pi/2$ sur la tension et dont la valeur efficace vaut 
		\begin{equation}
		I_m= \sqrt{I_v^2-I_p^2}
		\end{equation}
		où $I_V = \sqrt{I_1^2+I_3^2+I_5^2+\dots}$. Dans cette expression, 
		$I_j$ est la valeur efficace de l'harmonique $j$ du courant. Il est 
		en effet habituel de définir un courant sinusoïdal équivalent de  
		même valeur efficace que le courant $i$ : $I_V = I$.\\
		Pour le fun, on peut définir une réactance de magnétisation $X_m : 
		\underline{V} = jX_m\underline{I}_m$, réactance qui dépend de l'état 
		de magnétisation du circuit. Ces constatations nous donnent le 
		schéma équivalent suivant :\\
		\begin{center}
		Photoooo
		\end{center}
		
	\subsection{Inductance à circuit magnétique à entrefer}
	La réluctance d'un tube de flux d'air d'1mm est équivalent à la 
	réluctance d'un tube de flux de 5000mm d'épaisseur dans le fer : l'
	inductance est essentiellement déterminée par l'entrefer : on peut le 
	voir comme un blindage magnétique.
	
	\subsection{Phénomènes transitoire de mise sous tension d'une bobine 
	de fer}
	Si on applique brusquement en $t=0$ la tension $v=V\sqrt{2}\cos(\omega 
	t+\xi_V)$ aux bornes d'une bobine idéale, le flux vaut\footnote{Par 
	intégration de $v = N\frac{d\phi}{dt}$.} (si l'on néglige le rémanent)
	\begin{equation}
	\begin{array}{ll}
	\phi &= \int_0^t \frac{V\sqrt{2}}{N}\cos(\omega t + \xi_V)\text{ dt}\\
	 &= \frac{V\sqrt{2}}{\omega N}\left[\cos(\omega t + \xi_V - \frac{\pi}{2}
	 -\cos(\xi_V-\frac{\pi}{2})\right]
	\end{array}	
	\end{equation}
	En tenant compte des résistances/pertes, le flux rejoint progressivement 
	sa valeur de régime sinusoïdal, de même pour le courant. Si l'enclenchement 
	se fait quand la tension est maximale ($\xi_V=0$-, le flux est directement 
	en régime
	\begin{equation}
	\phi = \frac{V\sqrt{2}}{\omega N}\cos\left(\omega t - \frac{\pi}{2}\right)
	\end{equation}	
	Mais si on enclenche quand la tension est nulle ($xi_v = -\frac{\pi}{2}$) :
	\begin{equation}
	\phi = \frac{V\sqrt{2}}{\omega N}[1-\cos\omega t]
	\end{equation}
	ce qui montre que le flux atteint deux fois la valeur de régime : le fer 
	peut se saturer.
		
\section{Transformateurs monophasés}
	\subsection{Bobines à spires confondues, couplées dans l'air}
		
		
		
		
		
		
		
		
		