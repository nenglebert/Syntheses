\documentclass{book}
\usepackage{preambule}
\usepackage{fullpage}
\usepackage[T1]{fontenc}
\usepackage{mathpazo}
\usepackage[scaled=0.95]{helvet}
\usepackage{courier}
\usepackage{caption}
\usepackage{adjustbox}
\usepackage{listings}
\usepackage{color} %red, green, blue, yellow, cyan, magenta, black, white
\definecolor{mygreen}{RGB}{28,172,0} % color values Red, Green, Blue
\definecolor{mylilas}{RGB}{170,55,241}
 \setlength{\parindent}{0pt}
\newcommand{\HRule}{\rule{\linewidth}{0.5mm}}
\renewcommand*{\figureautorefname}{Fig.}
\usepackage[style=alphabetic,backend=biber,url=true,natbib=true]{biblatex}
\DeclareFieldFormat{url}{\space\url{#1}}
\DeclareNameAlias{default}{last-first}
\renewcommand\nameyeardelim{, }
\addbibresource{a.bib}
\usepackage[autostyle]{csquotes}
\usepackage{multicol}
\newcommand{\prop}[1]{\adjustbox{minipage=\linewidth-2\fboxsep-2\fboxrule,fbox}{
#1}}

\begin{document}
\AddToShipoutPicture*{\BackgroundPic}

\begin{titlepage}
	\begin{center}	
			
		\newcommand{\HRule}{\rule{\linewidth}{0.5mm}}   			            %Titre en gros
		\includegraphics[scale=0.11]{../../Builder/titlepage/logo.jpg}~\\[1cm]				%Logo
			
			\textsc{\LARGE Université Libre de Bruxelles}\\[1.5cm]
			\textsc{\Large Synthèse}\\[0.5cm]
			
			\HRule \\[0.4cm]
			{ \huge \bfseries \cours \ \\\memo \\[0.4cm] }
			
			
			\HRule \\[1.5cm]
			\begin{minipage}{0.6\textwidth}
				\begin{flushleft}%\large
					\emph{Auteur:}\\
					\prenom~\textsc{\nom}\\
				\end{flushleft}
			\end{minipage}
			\begin{minipage}{0.25\textwidth}
				%\begin{flushright}
				%\large
				\emph{Professeur :}\\
				\pprenom~\textsc{\pnom}
				%\end{flushright}
			\end{minipage}
			
			\vfill
			
			% Bottom of the page
			{\large Année \adebut~-~\afin}
			
		\end{center}
	\end{titlepage}
\chapter{ELDA}

\section{Première manipulation}
\subsection{Montage expérimental}


\chapter{TFO}
\section{Couplages}
	\prop{Réaliser un couplage $Yd11$. Que représente cette notation ? Quels problèmes 
	liés à l'observation de l'étoile des tensions va se présenter ? Relier les bobines 
	de manière à réaliser l'indice horaire et la vérifier en pratique.}\ 
	
	Considérons le schéma le plus basique du transformateur. Si le fer est parfait, 
	tout le flux reste confiné à l’intérieur de celui-ci. Les tensions sont dès 
	lors égales, au nombre de spires près 
	\begin{equation}
	V_1  = N_1\dfrac{d\phi}{dt},\qquad\qquad\qquad V_2 = N_2\dfrac{d\phi}{dt}.
	\end{equation}
	On s'intéresse ici à l'indice horaire suivant : $Yd11$. Pour rappel, nous 
	travaillons ici avec un systèmes triphasé : le $Y$ signifie que la disposition 
	du primaire est en étoile alors que le secondaire est disposé en triangle :
	\begin{center}
	Inclure schéma de l'énoncé complété
	\end{center}
	Les deux dispositions, celle du primaire et du secondaire, vont être "assemblées" 
	de sorte que $V_{ab}$ soit en phase avec $V_a$, $V_{bc}$ avec $V_B$ et $V_{ca}$ 
	avec $V_C$. Notons que chaque bobine capte la même tension, car chacune des 
	bobines est connectée à l'autre.\\
	Représentons maintenant le diagramme des phaseurs $V_A,V_B$ et $V_C$. Comme on 
	sait que $V_{ab}$ est en phase avec $V_{A}$, il faut tracer une droite parallèle 
	à $V_A$. En suivant le même raisonnement pour $V_{bc}$ et $V_{ca}$, on obtient 
	un "nouveau" triangle de sommets $a,b$ et $c$. Il faut ensuite relier chacun de 
	ces sommets par une flèche partant de l'origine. Si les échelles sont respectées, 
	l'angle entre $a$ et $A$ est de $-30^\circ$, correspondant bien à 11h.\\
	\begin{center}
	Inclure schéma final
	\end{center}
	En pratique, il est difficile de déterminer l'indice  horaire avant de câbler : 
	on commencera donc par câbler puis en on déduira l'indice horaire.\\
	
	Pour se faciliter la tâche, il serait bénéfique d'obtenir un point neutre pour 
	notre secondaire, comme c'est le cas pour le primaire. Il est possible d'obtenir 
	un point neutre \textit{virtuel} (100\% mathématique) en disposant des résistances, 
	identiques de la façon suivante, pour le peu que celles-ci ne déforment pas tout :
	\begin{center}
	Inclure schéma $R$
	\end{center}
	On aura ainsi créer un point neutre virtuel. Le câblage étant réalisé\footnote{
	Diagramme ?}, le PC nous montre bien le $Yd11$.
	
\section{Paramètres de transformateur}
	\prop{Décrire, grâce à deux essais, comment calculer les paramètres d'un 
	transformateur triphasé afin de pouvoir représenter son équivalent monophasé dont 
	le secondaire est ramené au primaire.}\ 
	
	On va ici tenter une modélisation des imperfections du transformateur. La première 
	chose à faire est de se débarrasser du couple de magnétisation en ramenant le 
	secondaire au primaire, en multipliant la tension du secondaire par le rapport de 
	transformation
	\begin{center}
	Image ou deux bobines => ramené
	\end{center}
	Il faut maintenant modéliser les différentes pertes possibles :
	\begin{itemize}
	\item[$\bullet$] Pertes d'hystérèses et courant de Foucault : $R_p$.
	\item[$\bullet$] Imperfection du fer ; relation $I-\phi$ non linéaire : $X_m$ car 
	courant magnétisant
	\item[$\bullet$] Pertes par effet Joule : $R_1$ et $\mu^2R_2$
	\item[$\bullet$] Flux de dispersion $X_1$ et $\mu^2X_2$
	\end{itemize}
	Si l'on représente le flux magnétisant et le courant magnétisant sur le même 
	graphique en fonction du temps, on se rend compte que le flux est en avance 
	sur le courant de $\pi/2$ : $X_m$ sera modélisé par une bobine.
	\begin{center}
	Inclure schéma avec ordre de grandeur
	\end{center}
	En pratique, on utilise les \textit{per unit}. Il s'agit d'une normalisation de 
	la grandeur afin de pouvoir les comparer (indispensable pour comparer des puissances). 
	Par exemple, si l'on possède un $I_{nominal} = 5A$, on divisera chacune mesure par 
	$5A$ pour l'exprimer en $pu$. On peut également calculer $Z_b = V_b/I_b$.\\
	
	Le premier test à réaliser est de faire un \textit{court-circuit à droite}. En faisant 
	ça, on remarque que l'impédance des pertes est beaucoup plus importante que celles 
	dans l'entrefer : on peut négliger le courant magnétisant et prendre la mesure de 
	$|Z|$. Si en plus on mesure le déphasage $\phi$ entre la tension et le courant, on 
	peut tout retrouver grace aux relations suivantes :
	\begin{equation}
	\Re(Z) = R_1,\qquad\qquad\qquad \Im(Z) = X_1
	\end{equation}

		
	Le second test à réaliser est cette fois en \textit{circuit ouvert à droite}. Cette 
	fois-ci, c'est l'impédance de l'entrefer qui est beaucoup plus élevée que celles des 
	pertes et je peux mesurer celle-ci. Comme vu au cours, on fait l'hypothèse que 
	$R_1\approx \mu^2R_2$ et $X_1\approx \mu^2X_2$.\\
		
	
	\prop{Donner le schéma de câblage et décrire les manœuvres permettant de réaliser 
	l'essai.}\ 
	
	Schéma de câblage ?\\
	
	
	\prop{Quels sont les grandeurs à surveiller durant cet essai.}\ 
	
	Pour le court-circuit, il est important de monitorer le courant, afin de ne pas 
	griller les circuits! Ceci n'est pas important dans le second cas\footnote{Justif?} 
	(circuit ouvert).\\
		
	\prop{Indiquez les résultats et calculs de cet essai.}\ 
	
	Il faut considérer les courants, tensions et puissances moyennes (on est en circuit 
	triphasé, ne l'oublions pas!) On obtient :
	\begin{equation}
	\overline{I} := 5.30\ A\,\qquad\qquad \overline{V} = 6.21\ V,\qquad\qquad \overline{W} 
	= 27.09\ W.
	\end{equation}	
	On peut dès lors calculer le $\cos\phi$ :
	\begin{equation}
	\overline{P} = \overline{V}\overline{I}\cos\phi \Rightarrow \cos\phi = 0.82
	\end{equation}
	
	
\section{Saturation}
	\prop{Qu'est ce que le phénomène de saturation d'un cœur ferromagnétique ? Comment se 
	manifeste-t-il ? Comment l'observer ?}\ 
	
	Le matériau magnétique d'un transformateur est constitué de fer (en tôles) ou de ferrite. 
	Ce sont des aimants microscopiques qui sont, à priori, orientés dans n'importe quelle 
	direction (aucune direction privilégiée). Les aimantations de chaque aimant microscopique 
	se compensent donc globalement. Lorsque ce matériau est soumis à un champ magnétique 
	croissant, les aimants vont s'orienter progressivement selon ce champ. L'aimantation 
	globale tendra à suivre le champ appliqué. Si on continue d'augmenter le champ magnétique 
	extérieur, il arrive un moment où tous les aimants sont orientés selon le champ et ne
	peuvent donc plus faire augmenter l'aimantation globale du matériau. L'aimantation globale 
	n'est plus proportionnelle au champ magnétique extérieur : c'est la saturation magnétique.\\
	
	On peut observer un tel effet en imposant une tension importante au primaire puis en 
	continuant d'augmenter celle-ci.\\
		
	\prop{Donner le schéma de câblage et décrire les manœuvres permettant de réaliser 
	l'essai.}\ 
	
	\begin{center}
	Schéma de câblage
	\end{center}

	Le transformateur du laboratoire est dimensionné pour que les bobines puissent recevoir 
	400V. Pour rendre la saturation pleinement visible, on peut utiliser la seconde encoche 
	permettant de n'utiliser que 90\% des bobines, et non l'entièreté. De cette façon, si 
	l'on applique à nouveau 400V, on sera beaucoup plus loin dans la courbe de saturation. \\
	
	On cherche à obtenir une courbe de $\phi \backsim v$ en fonction du courant nominal. Pour 
	obtenir une telle courbe, on augmente petit à petit la tension et on mesure le courant 
	de magnétisation. Pour mesurer le courant de magnétisation, il faut mettre le secondaire 
	à vide.\\
		
	\prop{Quelles sont les grandeurs à surveiller durant cet essai ?}\ 
		
	Comme précédemment, nous somme dans le cas d'un circuit ouvert, à vide. Il ne faut donc 
	pas surveiller de grandeurs ici.\\
		
	\prop{Donner la courbe de saturation du transformateur ainsi que l'allure temporelle 
	de $i$.}\ 
	
	\begin{center}
	Inclure graphe Badr-Ali
	\end{center}
	
	Notons que la pente de cette courbe vaudra toujours, au minimum, $\mu_0$.
	
	
	
	
	
	
\section{Enclenchement}
	\prop{Quel phénomène apparaît lors d'un enclenchement de la tension sur un transformateur ?}\ 
	
	On sait que la tension peut être donnée par $v = \dfrac{d\phi}{dt}$. Dès lors, on a 
	\begin{equation}
	\phi = \int v\text{d}t
	\end{equation}
	Deux situations peuvent se produire :
	\begin{itemize}
	\item[$\bullet$] Si on enclenche au moment ou la tension est à son maximum, le flux est 
	directement en régime.
	\item[$\bullet$] Si on enclenche au moment ou la tension est à son minimum, il y a un 
	risque de surintensité.
	\end{itemize}
	On peut justifier ceci mathématiquement (cf. cours). Une autre façon de voir est de 
	regarder l'aire sous la courbe de $v$, représentant le flux. On remarque que l'aire 
	sous la courbe de $v$ si on enclenche à un minimum est deux fois plus grande que 
	dans l'autre cas : le flux est dès lors deux fois plus grand et le risque de saturation 
	est bien présent.\\
		
	
	\prop{Donner le schéma de câblage et décrire les manœuvres permettant de réaliser 
	l'essai.}\ 
	
	\begin{center}
	Schéma de câblage
	\end{center}
	
	Le souci se situe au niveau de l'enclenchement et donc au niveau du transitoire. Pour 
	réaliser cet essai, il faudra connecter directement le transformateur au secteur et 
	l'allumer rapidement. Cette fois ci encore, on cherche à mesurer le courant magnétisant :
	il faut placer le secondaire en court-circuit.\\
		
		
	\prop{Donner une courbe représentative d'un enclenchement}\ 
	
	\begin{center}
	Inclure graphe Badr-Ali
	\end{center}
	
	Si l'on enclenche au mauvais moment, on peut entendre un bruit : le flux trop 
	important a déplacé les bobines. Heureusement c'est sans danger, ce n'est qu'un 
	effet transitoire.
	
\end{document}