\chapter{Représentations position - impulsion}
Commençons par quelques rappels
\begin{equation}
\ket{\psi} = \int d\vec r\ \psi(\vec{r})\ket{\vec{r}}
\end{equation}
où $\psi(\vec{r}) = \bra{\vec{r}}\ket{\psi}$.
\begin{equation}
\bra{\psi}\ket{\psi} = \iint d\vec r\ d\vec r'\psi^*(\vec{r'})\psi(\vec{r})\underbrace{
\bra{\vec{r'}}\ket{\vec{r}}}_{\delta(\vec{r},\vec{r'})}\qquad\qquad \int d\vec r |\psi(\vec{r}|^2=1
\end{equation}
Avec la décomposition spectrale $\hat{\vec{r}} = \int d\vec{r}\ \vec{r}\ket{\vec{r}}\bra{\vec{r}}$, 
on peut écrire
\begin{equation}
\mathbb{P}(\vec{r}) = \underbrace{\bra{\psi}\ket{\vec{r}}}_{\psi^*(\vec{r})}\underbrace{
\bra{\vec{r}}\ket{\psi}}_{\psi(\vec{r})} = |\psi(\vec{r})|^2,\qquad\qquad\langle\hat{\vec{r}}\rangle = 
\bra{\psi}\hat{\vec{r}}\ket{\psi} = \int d\vec{r}\ \vec{r}\underbrace{\bra{\psi}\ket{\vec{r}}
\bra{\vec{r}}\ket{\psi}}_{|\psi(\vec{r})|^2}
\end{equation}

\section{Opérateur impulsion}
Pour motiver physiquement l'opérateur impulsion, repartons de la notion d'onde de de Broglie que 
l'on peut associer à une particule libre ayant une certaine vitesse. 
\subsection{Onde de De Broglie, paquets d'onde}
		\subsubsection{Onde de De Broglie}
		On peut voir une particule comme une onde, de Broglie a réussi à donner les "paramètres" de 
		cette onde. Voyons ça comme une onde, par analogie à l'électromagnétisme
		\begin{equation}
		\psi(\vec{r},t) = \psi_0e^{i(\vec{k}\vec{r}-\omega t)}
		\end{equation}
		De Broglie pose les valeurs qu'il faut associer à $\vec k$ et $\omega$ pour une particule ayant 
		une certaine vitesse et énergie. Le point de départ est
		\begin{equation}
		\lambda = \dfrac{h}{p},\qquad k=\dfrac{2\pi}{\lambda},\qquad \vec k=\dfrac{\vec{p}}{\hbar}
		\end{equation}
		Pour la fréquence, par analogie à l'optique
		\begin{equation}
		\nu = \dfrac{E}{h},\qquad \omega = 2\pi\nu = \dfrac{E}{\hbar}
		\end{equation}
		En substituant ces expressions dans l'expression de l'onde plane
		\begin{equation}
		\psi(\vec{r},t) = \psi_0\ e^{\frac{i}{\hbar}\left(\vec{p}\vec{r}-Et\right)}\qquad \text{ où }\ \psi_0 =
		\dfrac{1}{(2\pi\hbar)^{3/2}}
		\end{equation}
		Cette solution étant écrite par analogie avec l'EM, il est possible d'écrire une équation 
		d'onde que cette onde va vérifier. Regardons par exemple si l'on effectue
		\begin{equation}
		\begin{array}{lll}
		\bullet i\hbar\frac{\partial}{\partial t}\psi &= i\hbar \psi_0\frac{E}{i\hbar}e^{\dots} &=
		E\psi(\vec{r},t)\\
		\bullet \Delta \psi(\vec{r},t) &= \psi_0\dfrac{p_x^2+p_y^2+p_z^2}{(-i\hbar)^2}e^{\dots} &= -
		\dfrac{p^2}{\hbar^2}\psi(\vec{r},t)
		\end{array}
		\end{equation}
		En en tire
		\begin{equation}
		-\dfrac{\hbar^2}{2m}\Delta \psi(\vec{r},t) = \dfrac{p^2}{2m}\psi(\vec{r},t) = E\psi(\vec{r},t)
		\end{equation}
		On retrouve le classique
		\begin{equation}
		\underline{i\hbar\dfrac{\partial}{\partial t}\psi(\vec{r},t)= -\dfrac{\hbar}{2m}\Delta\psi(\vec{r},t)}
		\end{equation}
		Il s'agit d'une ED de type d'onde qui est bien vérifiée par l'onde plane construite ci-dessus. 
		Cette ED correspond à une particule libre. Elle est linéaire : toute combili de solution est 
		solution. Au lieu de prendre une onde de de Broglie monocinétique correspondant à une onde 
		monochromatique, on peut considérer un paquet d'onde. Par exemple
		\begin{equation}
		\left.\begin{array}{ll}
		p_1 \rightarrow \psi_1\\
		p_2 \rightarrow \psi_2
		\end{array}\right\}\longrightarrow \alpha\psi_1+\beta\psi_2
		\end{equation}
		Un paquet d'onde est une superposition continue de particules avec toute une gamme de vitesse 
		possible. Il y correspondra une onde de de Broglie nommée \textit{paquet d'onde}.

		\subsubsection{Paquet d'onde}
		En substituant le préfacteur $\psi_0$, on obtient
		\begin{equation}
		\psi(\vec{r},t) = \dfrac{1}{(2\pi\hbar)^{3/2}}\int d\vec{p}\ \phi(\vec{p})e^{\frac{i}{\hbar}(\vec{p}
		\vec{r}-Et)}
		\end{equation}
		La même expression, en notation de Dirac :
		\begin{equation}
		\ket{\psi} = \int d\vec{p}\ \phi(\vec{p})\ket{\vec{p}}
		\end{equation}
		Il s'agit d'un mélange (avec un certain poids) de kets $\vec{p}$. En refermant
		\begin{equation}
		\psi(\vec{r},t) = \bra{\vec{r}}\ket{\psi} = \int d\vec{p}\ \phi(\vec{p})\underbrace{
		\bra{\vec{r}}\ket{\vec{p}}}_{(*)}
		\end{equation}
		où $\DS (*) = \frac{1}{(2\pi\hbar)^{3/2}}e^{\frac{i}{\hbar}(\vec{p}\vec{r}-Et)}$. On voudrait 
		faire la même chose mais en intégrant la phase dans le ket
		\begin{equation}
		\psi(\vec{r},t) = \frac{1}{(2\pi\hbar)^{3/2}}\int d\vec{p} \underbrace{\phi(\vec{p})e^{-\frac{i}
		{\hbar}Et}}_{\phi(\vec{p},t)}e^{\frac{i}{\hbar}\vec{p}.\vec{r}}
		\end{equation}
		\textbf{Attention} : ce n'est plus le même $\ket{\vec{p}}$ qu'à la précédente expression, la 
		définition à ici changée afin de pouvoir écrire
		\begin{equation}
		\ket{\psi} = \int d\vec{p}\ \phi(\vec{p},t)\ket{p}
		\end{equation}
		En refermant
		\begin{equation}
		\psi(\vec{r},t) = \bra{\vec{r}}\ket{\psi} = \int d\vec{p}\ \phi(\vec{p},t)
		\underbrace{\bra{\vec{r}}\ket{\vec{p}}}_{(**)}
		\end{equation}
		où $\DS (**)= \frac{1}{(2\pi\hbar)^{3/2}}e^{\frac{i}{\hbar}\vec{p}\vec{r}}$ est le 
		\textbf{noyau de la transformée de Fourier}.\\
		
		Résumons
		\begin{equation}
		\ket{\psi}\left\{\begin{array}{ll}
		= \int d\vec{r}\ \psi(\vec{r},t)\ket{\vec{r}} & \rightarrow\text{Base pos.}\\
		= \int d\vec{p}\ \phi(\vec{p},t)\ket{\vec{p}} & \rightarrow\text{Base imp.}		
		\end{array}\right.
		\end{equation}
		avec
		\begin{equation}
		\underline{\bra{\vec{r}}\ket{\vec{p}} = \frac{1}{(2\pi\hbar)^{3/2}}e^{\frac{i}{\hbar}\vec{p}\vec{r}}}
		\end{equation}
		Il s'agit de la \textit{formule de changement de base}. Elle est particulièrement intéressante :
		\begin{equation}
		\begin{array}{ll}
		\psi(\vec{r},t) = \bra{\vec{r}}\ket{\psi} &= \int d\vec{p}\ \phi(\vec{p},t)\ \bra{\vec{r}}\ket{\vec{p}}\\
		&= \frac{1}{(2\pi\hbar)^{3/2}}\int d\vec{p}\ \phi(\vec{p},t)e^{\frac{i}{\hbar}\vec{p}\vec{r}}\\
		\phi(\vec{p},t) = \bra{\vec{p}}\ket{\psi} &= \int d\vec{r}\ \psi(\vec{r},t)\ \bra{\vec{p}}\ket{\vec{r}}\\
		&= \frac{1}{(2\pi\hbar)^{3/2}}\int d\vec{r}\ \psi(\vec{r},t)e^{-\frac{i}{\hbar}\vec{p}\vec{r}}		
		\end{array}
		\end{equation}
		On peut écrire
		\begin{equation}
		\underline{\phi(\vec p) = TF[\psi(\vec{r})]},\qquad\qquad \underline{\psi(\vec r) = TF^{-1}
		[\phi(\vec{p})]}
		\end{equation}
		
		
		Il en découle une série de propriétés
		\begin{itemize}
		\item[i.] Théorème de Parseval : le produit scalaire de deux fonctions vaut le produit scalaire 
		des TF de ces deux fonctions. 
		\begin{equation}
		\int f_1(\vec{r})f_2^*(\vec{r}) d\vec{r} = \int F_1(\vec{p})F_2(\vec{p})d\vec{p}
		\end{equation}
		Dans notre cas :
		\begin{equation}
		\bra{\psi_1}\ket{\psi_2} \left\{\begin{array}{ll}
		= \int d\vec{r}\ \underbrace{\bra{\psi_1}\ket{\vec{r}}}_{\psi_1^*(\vec{r})}
		\underbrace{\bra{\vec{r}}\ket{\psi_2}}_{\psi_2(\vec{r})}\\
		= \int d\vec{p}\ \underbrace{\bra{\psi_1}\ket{\vec{p}}}_{\phi_1^*(\vec{p})}
		\underbrace{\bra{\vec{p}}\ket{\psi_2}}_{\phi_2(\vec{p})}		
		\end{array}\right.
		\end{equation}
		La normalisation est toujours valable
		\begin{equation}
		1 = \bra{\psi}\ket{\psi} = \int d\vec{r}\ |\psi(\vec{r})|^2 = \int d\vec{p}\ |\phi(\vec{p})|^2
		\end{equation}
		
		\item[ii.] On peut voir Robertson comme une conséquence de ces TF : la TF d'une fonction étroite 
		sera large, il y a quelque chose qui peut s'apparenter à une relation d'incertitude $x-p$
		\begin{equation}
		11 (prop ii) )
		\end{equation}
		Si on a une paire de fct (ici psi et phi) qui sont connectée par une TF, dans la th. des TF le 
		produit des cste peut etre majoré par une cste, ici on démontre rien mais c'est fort similaire.\\

		\end{itemize}
		
		
		
		



La dérivée d'une fct, au niveau de sa tf, c'est juste multiplié par i* variable conj. Ceci va 
nous permettrede définir proporement l'opérateur impulsion en base position.
\begin{equation}
12 (prop iii) )
\end{equation}
SI on dérivé dans le domaine position dans le domaine impulsion ça revient  mult par i/hbar pj. Une 
façon simple est d'écrire la valeur moyennede l'impulsion
\begin{equation}
13
\end{equation}