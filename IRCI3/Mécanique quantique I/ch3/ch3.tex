\chapter{z}
qques rappels
\begin{equation}
1
\end{equation}
Pour motiver physiquement l'opérateur impulsion, repartons de la notion d'onde de de broglie que 
'lon peut associer à une particule libre ayant une certaine vitesse. 
\subsection{Onde de De Broglie}
On peut voir une particule comme une onde, De broglie a réussi à donner les "paramètres" de cette 
onde. Voyons ça comme une onde, par analogie à l'électromagnétisme
\begin{equation}
2
\end{equation}
De Broglie posse les valeurs qu'il faut associer à k et omega pour une particule ayant une certaine 
vitesse et energie. Le point de départ c'est
\begin{equation}
3
\end{equation}
Pour la fréquence, analogie à l'optique
\begin{equation}
4
\end{equation}
Commeécrit par analogie avec l'EM, je peux écrire une équation d'onde que cette onde va vérifier. 
Regardons par exemple si l'on effectue
\begin{equation}
5
\end{equation}
ED de type eq d'onde qui est vérifiée par l'onde plane construite ci-dessous. Cette ED correspond
à une particule libre. Elle estlinéaire : toute combili de solution est solution. Au lieu de prendre
une onde de De Broglie monocinétique correspondant à une onde monocrhomatique on peut considérer
un paquet d'onde. Par ex
\begin{equation}
6
\end{equation}
Un paquet d'onde est une superposition continue de particules avec toute une gamme de vitesse 
possible. Il y correspondra une onde de de broglie nommée \textit{paquet d'onde}.

\subsection{Paquet d'onde}
Le psi 0 étant le préfacteur, on obtient
\begin{equation}
7
\end{equation}
On voudrait faire la meme chose mas en intégrant la phase dans le ket
\begin{equation}
8
\end{equation}
Prenons par exemple le théorème de Parseval : le produit scalaire de deux fonction vaut le 
produit scalaire des TF des deux fonctions, onva voir ce que ça va donner ici
\begin{equation}
9
\end{equation}
La normalisation est tjs valable
\begin{equation}
10 (prop i) )
\end{equation}
On peut voir Robertson comme une conséquence de ces TF : la TF d'une fct étroite sera large, il y 
a quelque chose qui peut s'apparenter à une relation d'incertitude
\begin{equation}
11 (prop ii) )
\end{equation}
Si on a une paire de fct (ici psi et phi) qui sont connectée par une TF, dans la th. des TF le 
produit des cste peut etre majoré par une cste, ici on démontre rien mais c'est fort similaire.\\

La dérivée d'une fct, au niveau de sa tf, c'est juste multiplié par i* variable conj. Ceci va 
nous permettrede définir proporement l'opérateur impulsion en base position.
\begin{equation}
12 (prop iii) )
\end{equation}
SI on dérivé dans le domaine position dans le domaine impulsion ça revient  mult par i/hbar pj. Une 
façon simple est d'écrire la valeur moyennede l'impulsion
\begin{equation}
13
\end{equation}