
	
	
	
CHAPITRE SIX : 
\section{Méthodes des perturbations dépendantes du temps}
\subsection{Principe de base}
On veut résoudre
\begin{equation}
18
\end{equation}
Petite parenthèse dans le cas ou H ne dépend as du temps. 
On avait alors utilisé les opérateurs dévolutions
\begin{equation}
19
\end{equation}
On retrouve cette phase tournante ou on avait réussi à une 
fois pourtout fixer l'évolution. Considérons maintenant 
$H(t) = H_0 + W(t)$. La différence est que la perturbation 
dépend du temps alors que l'hamiltonien non perturbé est 
\textbf{indépendant} du temps. Le terme perturbationest 
"petit" et lui dépend du temps. On suppose que l'on connait 
la solution du problème non pertrubé. On peut donc écrire
\begin{equation}
20
\end{equation}
On suppose que le système est initialement (jusqu'à l'instant
t=0) dans un des états propre ket(psi i) initial. Nous 
avons un état stationnaire jusqu'à l'instant 0 ou on allume 
une perturbation petite par rapport à H( par exemple un champ
magnétique faible). On va supposer dans la suite que
\begin{equation}
21
\end{equation}

\subsection{Système d'équation diff}
Le point de départ c'est d'essayer de résoudre l'équation 
de schrod qui dépend du temps. On va écrire un SD à partir
d'une ED
\begin{equation}
22
\end{equation}
Si on projette par psi n seule le terme n va rester dans la 
somme (états orthonormé, base). On va avoir
\begin{equation}
23
\end{equation}
A cause du dernier terme, on aura des termes couplés (éléménts 
hors diag). On a W nk(t) la matrice des éléménts de matrice. On 
pose $C_n(t) = b_n(t).e^{-i/\hbar.E_nt}= b_n(t).e^{1/i\hbar.E_nt}$. Le fait qu'il y ai une 
perturbation fait que l'on va s'écarter de la solution analytique 
toute simple connue. En dérivant par rapport au temps
\begin{equation}
24
\end{equation}