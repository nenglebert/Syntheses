\chapter{Méthodes d'approximations (Schrödinger indépendant du temps)}
Deux méthodes seront vue dans ce chapitre : la méthode des perturbations 
et la méthode des variations. \\

Pour les perturbations, on considère un hamiltonien proche du problème non perturbé. 
On part d'un problème proche ou l'on a un hamiltonien diagonalisable et, à partir de 
ce système, on va "allumer" la perturbation et se rapprocher de la situation réelle. 
Ceci sera possible en utilisant divers développement en série afin de se rapprocher 
du cas réel.\\

Pour la méthode des variations, on va définir une fonctionnelle énergie sur laquelle 
o va appliquer la méthode des variations de sortte à minimiser l'énergie par rapport 
à une famille de "fonction d'essai". On va ainsi définir une classe de fonction d'onde, 
les fonctions d'essai, et on va essayer de trouver quelle est celle qui s'approche le 
plus de la vrai fonction d'onde exacte. \\

On peut s'intéresser soit au problème stationnaire ou son évolution dans le temps. Pour 
la méthode des perturbations on verra comment l'utiliser pour l'équation stationnaire 
(ou indépendante du temps) =:soit comment utiliser cette méthode pour approximer les états 
propres) ou alors la dépendante du temps (comment peut-on approximer la dynamique d'un 
système toujours avec la méthode des perturbations). On en déduira la règle d'or de Fermi. 

	\subsection{Méthodes des perturbations - Eq stationnaire}
		\subsubsection{Principe base - notation}
		On veut résoudre le probleme aux valeurs propre 
		\begin{equation}
		1
		\end{equation}
		Il existe un autre hamiltonien qui est le \textit{non-perturbé}
		\begin{equation}
		2
		\end{equation}
		On peut réécrire l'équation qu'on veut résoudre en fonction de 
		l'hamiltionien non perturbé
		\begin{equation}
		3
		\end{equation}
		Toute l'idée est de dire que la perturbation est petite : $\lambda\ll 
		1$.  Toute l'idée va être dé développer en série de puissance 
		$\lambda$. Sans trop s'écarter de H0 on va pouvoir s'intéresser à 
		ce qui nous intéresse dans le voisinage H0. L'équation que l'on veut 
		résoudre est donc
		\begin{equation}
		4
		\end{equation}
		On va maintenant écrire ces fonction comme un dev en série de lambda
		\begin{equation}
		5
		\end{equation}
		Le terme d'ordre 0 est connu mais les différentes corrections
		(soulignés sont inconnues). 
		
		Comme ces deux expressions doivent être valable pour tout lambda, on 
		va pouvoir identifier terme à terme
		\begin{equation}
		6
		\end{equation}
		Plus on descend en ordre plus on aura une précision dans les séries
		de puissances. Il ne faut pas oublier la condition de normalisation 
		\begin{equation}
		7
		\end{equation}
		On peut exprimer cette condtion de normalisation en terme de 
		puissances de lambda.
		\begin{equation}
		8
		\end{equation}
		On peut à nouveau faire les identifications
		\begin{equation}
		9
		\end{equation}
		On va résoudre ce système de proche en proche amenant au final une 
		meilleure précision. On peut voir psi n comme un dev de série en 
		lambda, mais chaque terme peut etre vu comme un certain ket. On peut 
		les réexprimer dans une base : une base naturelle est celle du 
		problème non-perturbé. 
		\begin{equation}
		10
		\end{equation}
		
	\subsection{b. Perturbation d'un niveau non-dégénéré}
	On verra dans le développement que la dégénéresence change quelque 
	chose
	
		\subsubsection{1. Énergie au premier ordre}
		On a notre problème non pertrubé
		\begin{equation}
		11
		\end{equation}
		Nous allons partir de la premiere équation non triviale de (6) (soit 
		elle de lambda1). On va multiplier à gauche par le bra de psi n 0.
		\begin{equation}
		12
		\end{equation}
		C'est lélément de matrice diagonal (valeur moyenne) de la perturbation
		W dans l'état propre non perturbé.  Il s'agit de la \textit{formule 
		des perturbations}.\\
		
		On va maintenant regarder la perturbation au premier ordre de l'état 
		propres. Nous avons fait pour l'énergie, il faut aussi faire pour 
		l'état propre correspondant
		
		\subsubsection{2. État propre au premier ordre}
		On va cette fois refermer avec un bra psi k (0) ou k neq n. On refait 
		exactement la même chose
		\begin{equation}
		13
		\end{equation}
		On peut développer le terme suivant dans la base des états propre 
		de l'état non perturbé
		\begin{equation}
		14
		\end{equation}
		On a une relation "semblable" à l'énergie avec l'élément de matrice 
		diagonal de la perturbation (soit la valeur moyenne). Ici ce qui 
		change c'est qu'on regarde tous les autres états propres non 
		perturbés.  On peut remarquer que pour calculer la perturbation du 
		n eme état on somme sur tous les auters états. C'est bien dégénéré 
		on aurait un zéro au dénominateur. Si pas dégénéré l'expression est 
		bien correcte.
		
		
		Avant d'aller plus loin, un exemple. Prenons un oscillateur 
		harmonique d'une autre raideur.
		\begin{equation}
		15
		\end{equation}
		Voir feuiille
		
		
		Fin exemple. Nous avions
		\begin{equation}
		15
		\end{equation}
		Il faut que ça soit "petit" (lambda). On a un coeff 1 devant 
		l'état non perturbé. On part d'un vect col (1 0 0 ..) puis quand 
		on allume la perturbation on a (1 ... ). La condition de validité 
		de la méthode dit que tous ces "nouveaux" coeff doivent etre petit 
		en norme. SI on regarde un élément de matrice non diagonale
		\begin{equation}
		16
		\end{equation}
		
		On va maintenant voir l'ordre 2. On avait un système d'équation 
		ou il faut identifier : le terme suivant sera le terme quadratique.
		
		\subsubsection{3. État propre au second ordre}
		On referme à gauche par psi n 0
		\begin{equation}
		17
		\end{equation}
		Pour détermine rl'énergie à l'ordre 2 il faut l'énergie à l'ordre 0 
		et à l'ordre 1. Il faut donc travailler de proche en proche. En 
		substituant l'ordre 1
		\begin{equation}
		18
		\end{equation}
		Ceci ne fait plus qu'apparaître les énergie et états propres de 
		l'état non perturbé. On peut voir que si l'état qui nous intéresse 
		est l'état fondamental et que l'on s'intérsse à la correction du à 
		l'allumage de la perturbation. Le numérateur sera toujours positif. 
		Comme En0 est fond, les Ek0 seront plus grand : la correction de 
		l'énergie au second ordre est leq 0. 
		\begin{equation}
		19
		\end{equation}
		
	\subsection{c. Perturbation état dégénéré}
	i est la degen
	\begin{equation}
	20
	\end{equation}
	On pourrait avoir une levée de la dégénérescence en allumant la 
	perturbation. De même, si on a par exemple 4 niveau en faisant tendre
	lambda vers 0 (on éteind progressivement la perturbation) on pourrait
	 obtenir une quadruple dégénérescence. 
	\begin{equation}
	21
	\end{equation}
	Il est faut de croire que psi n,i va tendre vers psi n i (0) étant 
	donné qu'il y a une dégénérescence. Il n'y a pas une correspondance mais 
	onsait qu'ls vont tous tendre vers des état qui appartiennent à un 
	certain espace propre. 
	
	\begin{equation}
	22
	\end{equation}
	Ce qui change c'est qu'ici à cause de la degen, il faut calculer l'ordre
	zéro de la correction (ce qui était avant trivial). 
	\begin{equation}
	23
	\end{equation}		
	Il faut maintenant exprimer les états phi dans la base. tout ce qu'on 
	fera par rapport au cas précédent c'est qu'il faura diagonaliser une 
	matrice. La solution finale serait
	\begin{equation}
	24
	\end{equation}
	On va devoir diagonalise : les vecteurs propres diront comment 
	diagonaliser et les énergies. On obtiendra gn réponse, le déplacement en 
	énergie des sous niveau qui correspond aux valeurs propres de cette 
	matrice. Avant on avait un unique état et on regardait l'unique état. Ici 
	on a un espace propre : matrice à diagonaliser. Le cas non dégénéré n'est 
	que le cas particulier de la matrice 1x1. On aura dans le cas d'une 
	degen un matr gn x gn. Les caleurs propres de cette matrice donneront le 
	déplacement de l'énergie....