\chapter{Introduction générale}
\section{L'électronique, c'est quoi ?}
Les slides sont assez informatifs, l'essentiel est repris ci-dessous :\\

\prop{\begin{itemize}
	\item L'électronique est la principale technologie pour traiter et
	transmettre l'information.
	\item Cette information est codée dans un signal électrique
	\item Faire de l'électronique, c'est notamment choisir et assembler
	des composants
	\end{itemize}}


\section{Une vision plus large}
\subsection{Comme ingénieur, votre job est de "contrôler" les choses}
Le \textit{contrôle de processus} est \textbf{la} situation de base
de l'ingénieur. L'électronique est ainsi au cœur des boucles de 
contrôles.

\subsection{Système embarqué}
On appelle \textit{système embarqué} tout objet utilisant de l'électronique, que ce soit pour une fonction précise, pour de l'électronique "enfouie" pour doper une fonction de base ou encore
pour satisfaire des contraintes spécifiques.\\

Ajoutons un petit mot sur l'\textit{intelligence}. Celle-ci désigne
une capacité de calcul, c'est à dire l'électronique et donc comme
nous venons de le voir les systèmes embarqués.


\section{Un peu plus d'électronique}
Les disciplines de l'électronique se divisent en deux grandes parties:
\begin{enumerate}
	\item Signal ou puissance
	\item Analogique ou numérique
\end{enumerate}

Ainsi, suivant le \textit{niveau de puissance utilisé} on distinguera
l'électronique de \textbf{signal} dont le but est de traiter l'information et
ce avec une faible puissance ($mW$ à $W$). C'est l'objet de ce cours. L'électronique de \textbf{puissance} dont le but est de mettre en forme
une puissance électrique ne sera - hélas - pas vu ici. Notons que l'\textit{électronique analogique} travaille par correspondance directe alors que l'\textit{électronique numérique} passe par un codage intermédiaire sous forme binaire.