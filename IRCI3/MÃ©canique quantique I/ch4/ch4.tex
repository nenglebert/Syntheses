\chapter{Algèbre des moments cinétiques}
\section{Moment cinétique orbital}
	\subsection{Règle de correspondance, relation de commutation}
	Classiquement, un moment cinétique est défini par la relation
	\begin{equation}
	\vec{L_{cl}} = \vec{r_{cl}}\times\vec{p_{cl}}
	\end{equation}
	On pourrait être tenté d'appliquer le principe de correspondance mais c'est faux car 
	$\hat{x}\hat{p}$ n'est pas hermitien
	\begin{equation}
	(\hat{x}\hat{p})^\dagger = \hat{p}^\dagger\hat{x}^\dagger = \hat{p}\hat{x}\neq\hat{x}\hat{p}
	\end{equation}		
	Il est cependant possible de le rendre hermitien	en le symétrisant correctement : $\frac{1}{2}(\hat{x}\hat{p}
	+\hat{p}\hat{x})$. L'opérateur "quantique" est donné par la forme symétrisée 
	\begin{equation}
	\vec{L} = \frac{1}{2}\left(\hat{\vec{r}}\times\hat{\vec{p}}-\underbrace{\hat{\vec{p}}\times
	\hat{\vec{r}}}_{-\hat{\vec{r}}\times\hat{\vec{p}}}\right) = \hat{\vec{r}}\times\hat{\vec{p}}
	\end{equation}
	où le signe négatif compense le changement de signe du produit vectoriel. Calculons ce moment 
	cinétique\footnote{\danger\ Les opérateurs ne commutent pas forcément, ne pas être "trop rapide" !}
	\begin{equation}
	\vec{L}=\vec{r}\times\vec{p} = \left|\begin{array}{ccc}
	\vec{1_x} & \vec{1_y} & \vec{1_z}\\
	\hat{x} & \hat{y} & \hat{z}\\
	\hat{p_x} & \hat{p_y} & \hat{p_z}
	\end{array}\right|\qquad\Longrightarrow\qquad\left\{\begin{array}{ll}
	L_x &= \hat{y}\hat{p_z} - \hat{z}\hat{p_y}\\
	L_y &= \hat{z}\hat{p_x}-\hat{x}\hat{p_z}\\
	L_z &= \hat{x}\hat{p_y}-\hat{y}\hat{p_x}
	\end{array}\right.
	\end{equation}
	Imaginons que l'on ai défini l'opérateur $\hat{L'}=\hat{p}\times\hat{r}$. Calculons une de 
	ses composante :
	\begin{equation}
	\hat{L_z'} = \hat{p_x}\hat{y}-\hat{p_y}\hat{x} = \hat{y}\hat{p_x}-\hat{x}\hat{p_y}=-\hat{L_z}
	\end{equation}
	L'inversion de $\hat{r}$ et $\hat{p}$ donne un signe négatif comme différence (pfpfp). On a 
	défini un opérateur antisymétrique, vérifions qu'il s'agit d'une \textbf{observable}% $\hat{\vec{r}}\times\hat{\vec{p}}$
	. Vérifions que celui-ci est bien hermitien
	\begin{equation}
	\hat{L_z}^\dagger = (\hat{x}\hat{p_y}-\hat{y}\hat{p_x})^\dagger = \hat{p_y}^\dagger\hat{x}^\dagger
	-\hat{p_x}^\dagger\hat{y}^\dagger=\hat{p_y}\hat{x}-\hat{p_x}\hat{y} = \hat{x}\hat{p_y}-\hat{y}
	\hat{p_x} = L_z
	\end{equation}
	Il est intéressant de réaliser au moins une fois le commutateur entre deux composantes
	\begin{equation}
	\begin{array}{ll}
	[L_x,L_y] &= [yp_z-zp_y, zp_x-xp_z]\\
	&= [yp_z,zp_x]-[yp_z,xp_z]-[zp_y-zp_x]+[zp_y,xp_z]\\
	&=y[p_z,zp_x]+[y,zp_x]p_z+z[p_y,xp_z]+[z,xp_z]p_y\\
	&= yz[p_z,p_x]+y[p_z,z]p_x+x[z,p_z]p_y+[z,x]p_zp_y\\
	&= i\hbar(xp_y-yp_x) = i\hbar L_z
	\end{array}
	\end{equation}
	Les commutateurs 2 et 3 de la seconde ligne sont nuls ($p_z$ commute avec lui même et $y$ 
	commute avec $x$). De même pour la troisième ligne. Quatrième ligne, le premier et le dernier 
	commutateur sont nuls (deux éléments de deux espaces distincts commutent). Pour la dernière 
	ligne, on a utilisé $[p_z,z] = -i\hbar$ et $[z,p_z]=i\hbar$.\\
	
	En résumé
	\begin{equation}
	\left\{\begin{array}{ll}
	\left[L_x,L_y\right] &= i\hbar L_z\\
	\left[L_y,L_z\right] &= i\hbar L_x\\
	\left[L_z,L_x\right] &= i\hbar L_y		
	\end{array}\right.\qquad\Longrightarrow\qquad \hat{\vec{L}}\times\hat{\vec{L}} = i\hbar
	\hat{\vec{L}}
	\end{equation}
	A droite, une notation condensée qui donnerait zéro dans un cas classique (mais nous sommes 
	en quantique héhé). On peut vérifier que cela donne bien le résultat attendu
	\begin{equation}
	\left|\begin{array}{ccc}
	\vec{1_x} & \vec{1_y} & \vec{1_z}\\
	L_x & L_y & L_z\\
	L_x & L_y & L_z	
	\end{array}\right|\qquad \longrightarrow \qquad(\hat{\vec{L}}\times\hat{\vec{L}})_z = L_xL_y-L_yL_x
	 = i\hbar L_z
	\end{equation}
	Pour la composition, on s'intéresse à l'opérateur suivant
	\begin{equation}
	\hat{L}^2 \equiv \hat{L_x^2}+\hat{L_y^2}+\hat{L_z^2}
	\end{equation}
	Calculons son commutateur avec ses différentes composantes
	\begin{equation}
	\left\{\begin{array}{ll}
	\begin{array}{ll}
	\left[L_x,L_x^2+L_y^2+L_z^2\right] &= \left[L_xL_y^2, L_xL_z^2\right]\\
	&= L_y\overbrace{\left[L_x,L_y\right]}^{i\hbar L_z}+\left[L_x,L_y\right]L_y + L_z\overbrace{\left[L_x,L_z\right]}^{-i\hbar L_y}
	+\left[L_x,L_z\right]L_z\\
	&= 0
	\end{array}\\
	\left[L_y,L^2\right] = 0\\
	\left[L_z,L^2\right] = 0
	\end{array}\right.
	\end{equation}
	Ceci implique que
	\begin{equation}
	[\hat{\vec{L}},L^2] = 0
	\end{equation}
	On peut voir que $\left\{\hat{L_z},\hat{L^2}\right\}$ va commuter et former un \textit{ECOC} 
	: il existe une \textbf{base propre commune} formée de l'ensemble des états
	\begin{equation}
	\left\{\ket{l,m}\right\}
	\end{equation}
	où $l$ est le nombre quantique orbital, associé à la distribution des valeurs propres
	 de $L^2$ et $m$ le nombre quantique magnétique associé au spectre de $L_z$. 


\section{Moment cinétique total}
L'idée est de regarder plus loin que l'orbital. Imaginons que l'on ait $N$ particules : 
chacune a une position $\hat{r_i}$ et impulsion $\hat{p_i}$. On peut créer un moment total
\begin{equation}
\vec{L}^{(tot)} = \sum_{i=1}^N \vec{L_i} = \sum_{i=1}^N r_i\times p_i
\end{equation}
Dans ce cas la déjà, cet opérateur va toujours vérifier les mêmes relations de 
commutation
\begin{equation}
[L^{(tot)}_x,L^{(tot)}_y] = \left[ \sum_{i=1}^N L_x^{(i)}, \sum_{j=1}^N L_y^{(j)}\right] 
= \sum_{i=1}^N[L_x^{(i)},L_y^{(i)}] = i\hbar L_z^{(tot)}
\end{equation}
En effet, pour donner un commutateur non-nul, il faut nécessairement que $i=j$ : les seules 
composantes restantes sont alors celles désignant la même particule.\\


On s’intéresse à tout triplet de trois opérateurs qui satisfont ces relation 
de commutation : c'est ce qu'on appellera moment cinétique. On va pour ça 
s'intéresser aux valeurs propres et tout ce qu'on peut dire. Pour les distinguer, 
on va les appeler $\vec J$, le \textit{moment cinétique} : ça pourrait être un orbital, une 
combili d'orbital, un spin, \dots
\begin{equation}
\text{Moments cinétiques : }\ \hat{\vec{J}} \equiv (\hat{J_x},\hat{J_y},\hat{J_z})\ \text{ satisfont }\ 
\left\{\begin{array}{ll}
\vec{J}\times\vec{J} &= i\hbar\vec{J}\\
\left[\vec{J},\vec{J^2}\right] &= 0\quad \text{ avec } \vec{J^2}=J_x^2+J_y^2+J_z^2
\end{array}\right.
\end{equation}
De 	façon similaire
\begin{equation}
\left\{\hat{J_z}, J^2\right\}\quad\rightarrow\quad \text{ECOC}
\end{equation}
Il existe donc une \textbf{base propre commune}
\begin{equation}
\left\{\ket{j,m}\right\}	
\end{equation}
où $j$ est associé à la quantification des valeurs propres de $J^2$. Le but de la sous-section 
suivante sera de montrer que $j$ est discret et en nombre fini. Pour se faire, on utilisera les 
opérateurs élévateurs et abaisseurs. Notons la relation d'orthogonalité
suivante
\begin{equation}
\bra{j',m'}\ket{j,m} = \delta_{jj'}\delta_{mm'}
\end{equation}

	\subsection{Quantification, opérateurs élévateurs $J_+$ et abaisseurs $J_-$}
	Par définition
	\begin{equation}
	\hat{J}_+ = J_x + iJ_y,\qquad\qquad
	\hat{J}_- = J_x - iJ_y	
	\end{equation}
	Il ne s'agit pas d'observables, mais ils forment une paire d'opérateurs adjoints : $(\hat{J_+})^\dagger = \hat{J_-}$
	et inversément. Par additions et différences
	\begin{equation}
	\left\{\begin{array}{ll}
	\hat{J_x} &= \frac{1}{2}\left(J_++J_-\right)\\
	\hat{J_y} &= \frac{1}{2i}\left(J_+-J_-\right)	
	\end{array}\right.
	\end{equation}
	Deux relations de commutations sont directement visibles
	\begin{enumerate}
	\item $[\hat{J^2},J_\pm] = 0$
	\item $[\hat{J_z},\hat{J_\pm}] = \underbrace{[J_z,J_x]}_{i\hbar J_y}\pm i\underbrace{[J_z,J_y]}_{
	-i\hbar J_x} = i\hbar J_y\pm\hbar J_x = \pm \hbar(J_x\pm iJ_y) = \pm \hbar  \hat{J}_\pm$
	\end{enumerate}

Nous pouvons appliquer les éléments de notre ECOC sur leurs états propres communs, $\ket{jm}$. On 
ne sait rien de ces deux nombres complexes, on sait juste qu'ils sont reliés aux valeurs propres 
des éléments de notre ECOC :
\begin{equation}
\left\{\begin{array}{llll}
\hat{J^2} &\ket{j,m}&=j(j+1)\hbar^2\ket{j,m}&\qquad j\in\mathbb{R}\\
\hat{J_z} & \ket{j,m}&=m\hbar\ket{j,m}&\qquad m \in \mathbb{R}
\end{array}\right.
\end{equation}
On peut montrer que $j\geq 0$, sachant qu'une norme est forcément positive
\footnote{Il manque une partie de la démo ici :(}
\begin{equation}
\begin{array}{ll}
\bra{\psi}J^2\ket{\psi} &\geq 0\qquad\forall \psi\\
\bra{\psi}J^\dagger J\ket{\psi} &\geq 0\\
\bra{\phi}\ \ \ket{\phi} &\geq 0\qquad \longrightarrow j\geq 0
\end{array}
\end{equation}
Il faut maintenant montrer que ce nombre des discret. Sachant que $J^2$ et $J_\pm$ 
commute, nous pouvons écrire la première ligne ci-dessous. Cependant, pour la seconde 
ligne, le commutateur est non nul :
\begin{equation}
\begin{array}{ll}
\hat{J^2}\underline{\hat{J_\pm}\ket{jm}}&\DS = \hat{J_\pm}\hat{J^2}\ket{jm} =
 j(j+1)\hbar^2\underline{\hat{J_\pm}\ket{jm}}\\
\hat{J_z}\hat{J_\pm}\ket{jm}&\DS= (\hat{J_\pm}\hat{J_z}\pm\hbar\hat{J_\pm})\ket{jm}\\
&\DS= m\hbar\hat{J_\pm}\ket{jm}\pm \hbar\hat{J_\pm}\ket{jm}\\
&\DS= (m\pm 1)\hbar\underline{\hat{J_\pm}\ket{jm}}
\end{array}
\end{equation}
Les relations ci-dessous nous montre deux liens de proportionnalité
\begin{equation}
\begin{array}{ll}
\hat{J_+}\ket{jm} &\propto \ket{j,m+1}\quad \text{ ou } 0\\
\hat{J_-}\ket{jm} &\propto \ket{j,m-1}\quad \text{ ou } 0
\end{array}
\end{equation}
On va maintenant montrer qu'on peut jamais monter de plus que un. D'un point de vue 
classique (et ici peu rigoureux), on peut montrer que $m$ ne peut pas descendre 
trop bas. Sachant qu'une composante est toujours inférieure ou égale à la norme du 
même vecteur
\begin{equation}
\sqrt{J^2} \geq |J_z|\quad\Leftrightarrow\quad \sqrt{j(j+1)}\hbar \geq |m|\hbar
\end{equation}
Voyons ce que vaut la norme du ket
\begin{equation}
\|J_\pm\ket{jm}\|^2 = \bra{jm}J_\mp J_\pm\ket{jm}
\label{eq:7.6}
\end{equation}
où nous avons utilisé le fait que $J_\pm$ est l'adjoint l'un de l'autre. Faisons 
une petite parenthèse pour calculer cet élément de matrice
\begin{equation}
\begin{array}{ll}
J_\mp J_\pm &= (J_x\mp iJ_y)(J_x\pm iJ_y) = J_x^2\pm iJ_xJ_y+iJ_yJ_x+J_y^2\\
&=J^2-J_z^2 \pm i\underbrace{[J_x,J_y]}_{i\hbar J_z} = J^2-J_z^2 \mp \hbar J_z
\end{array}
\end{equation}
Nous pouvons maintenant calculer \eqref{eq:7.6}
\begin{equation}
\begin{array}{ll}
\|J_\pm\ket{jm}\|^2 &=  \overbrace{\bra{jm}J^2\ket{jm}}^{j(j+1)\hbar^2} - \overbrace{\bra{jm}J_z^2\ket{jm}}^{
m^2\hbar^2}\mp\overbrace{\hbar \bra{jm}J_z\ket{jm}}^{m\hbar}\\
&= \hbar^2\{j(j+1)-m(m\pm 1)\} \geq 0
\end{array}
\end{equation}
Il s'agit de l'expression du module carré (d'où le $\geq 0$) ou l'on a appliqué un opérateur élévateur 
ou abaisseur. C'est cette relation qui va empêcher $m$ de monter trop haut ou descendre trop bas.\\

Regardons successivement ce qui se passe pour un élévateur et abaisseur.\\
	\textsc{Opérateur élévateur}
	\begin{equation}
	\|J_+\ket{jm}\| = \hbar^2(j^2+j-m^2-m) = \hbar^2(j-m)(j+m+1)\geq 0
	\end{equation}
	Pour satisfaire cette relation deux cas sont possibles : les deux parenthèses positives, ou négatives.
	\begin{equation}
	\left\{\begin{array}{ll}
	m &\leq j\\
	m &\geq -1-m
	\end{array}\right.\qquad\qquad\qquad	\left\{\begin{array}{ll}
	m &\geq j\\
	m &\leq -1-j
	\end{array}\right.\quad\rightarrow \text{Impossible}
	\end{equation}
	
	
	\textsc{Opérateur abaisseur}
	\begin{equation}
	\|J_+\ket{jm}\| = \hbar^2(j^2+j-m^2+m) = \hbar^2(j+m)(j-m+1)\geq 0
	\end{equation}
	Pour satisfaire cette relation deux cas sont possibles : les deux parenthèses positives, ou négatives.
	\begin{equation}
	\left\{\begin{array}{ll}
	m &\geq -j\\
	m &\leq j+1
	\end{array}\right.\qquad\qquad\qquad		\left\{\begin{array}{ll}
	m &\leq -j\\
	m &\geq j+1
	\end{array}\right.\quad\rightarrow \text{Impossible}
	\end{equation}\ \\
	
	Nous avons donc quatre inégalités, mais certaines sont plus fortes que d'autres. Il reste
	\begin{equation}
	\left\{\begin{array}{ll}
	m &\leq j\\
	m &\geq j
	\end{array}\right.\qquad\Longrightarrow\qquad \underline{-j\leq m\leq j}
	\end{equation}
	Les valeurs de $m$ sont délimitées les droites $m=\pm j$ formant un cône de valeurs possibles. 
	La valeur maximale se situe forcément sur une de ces deux droites après avoir ajouté $p$ à $m$ 
	ou soustrait $q$ à $m$):
	\begin{equation}
	\left\{\begin{array}{llll}
	J_+\ket{j,j} = 0 & \quad \exists p\in\mathbb{N} : m+p=j &\quad \rightarrow j-m=p\in\mathbb{N}\\
	J_-\ket{j,-j} = 0 & \quad \exists q\in\mathbb{N} : m-p=-j &\quad \rightarrow j+m=q\in\mathbb{N}	
	\end{array}\right.
	\end{equation}
	En sommant ces deux relations
	\begin{equation}
	j = \frac{p+q}{2} = \left\{0,\frac{1}{2},1,\frac{3}{2},2\dots\right\},\qquad
	m = \frac{q-p}{2} = \{j,-j+1,\dots, j-1,j\}
	\end{equation}
	Nous avons bien deux nombres quantiques : ils ne peuvent prendre que des valeurs discrètes. Pour
	$j$ fixé, nous avons $2j+1$ valeurs de $m$ possibles. \\
	
	En réalité, on ne peut jamais être totalement aligné sur un axe $J_x$, $J_y$ ou $J_z$ car cela 
	voudrait dire que les deux autres composantes sont totalement connues : impossible en vertu du 
	principe d'incertitude.



	\subsection{Mesure de $J_x$ et $J_y$ dans l'état $\ket{j,m}$}
	Intéressons-nous aux mesures des différentes projection. Les valeurs propres associées sont 
	les suivantes : $\hat{J_x}\rightarrow m'\hbar, \hat{J_y}\rightarrow m''\hbar$ et 
	$\ket{j,m}\rightarrow m\hbar$. Regardons les valeurs moyennes et les variances
	\begin{enumerate}
	\item \textit{Valeurs moyennes}.
	\begin{equation}
	\bra{jm}J_x\ket{jm} = \bra{jm}\frac{1}{2}(J_++J_-)\ket{jm} = \frac{1}{2}\bra{jm}J_+\ket{jm}+
	\frac{1}{2}\bra{jm}J_-\ket{jm} =0
	\end{equation}
	Or $J_\pm\ket{jm}\propto \ket{j,m\pm 1}$. Comme $\ket{j,m}\perp\ket{j,m\pm1}$, les valeurs 
	moyennes de $J_x$ et $J_y$ sont nulles.
	\item \textit{Variances}.\\
	Il est plus simple de faire apparaître $J^2$ pour faire apparaître les états propres communs
	\begin{equation}
	\bra{jm} J_x^2+J_y^2 \ket{jm}= \bra{jm}J^2-J_z^2 \ket{jm} = j(j+1)\hbar^2-m^2\hbar^2 = \hbar^2(j(j+1)
	-m^2)
	\end{equation}
	On a alors
	\begin{equation}
	\overbrace{\bra{jm}J_x^2\ket{jm}}^{\Delta J_x^2} = \overbrace{\bra{jm}J_y^2\ket{jm}}^{\Delta J_x^2} 
	= \frac{\hbar^2}{2}\{j(j+1)-m^2\}\quad \longrightarrow \Delta J^2_{\min} = \frac{\hbar^2m}{2}
	\end{equation}	
	\end{enumerate}
	En effectuant le produit des variances, on obtient %\footnote{TT : peu de notes ici, à compléter plz.}
	\begin{equation}
	\begin{array}{ll}
	\Delta J_x\Delta J_y &= \frac{\hbar^2}{2}(j(j+1)-m^2)\\
	&\geq \frac{\hbar^2}{2}(|m|(|m|+1)-m^2) = \frac{\hbar^2}{2}|m|
	\end{array}
	\end{equation}
	
	qui est exactement ce que l'on trouve en appliquant le théorème de Robertson	
	\begin{equation}
	\Delta J_x\Delta J_y \geq \frac{1}{2}|[J_x,J_y]| \geq \frac{\hbar}{2}\underbrace{|\langle J_z\rangle|}_{m
	\hbar} \geq \frac{\hbar^2}{2}|m|
	\end{equation}
	Les bornes obtenues sont en fait les états qui saturent le principe d'incertitude et ils
	correspondent aux états permis les plus proches des pôles nord et sud sur la sphère des
	moments cinétiques\dots
	
	\subsection{Convention de Codon Sortley}
	Rappelons ce qui a été précédemment vu. 
	\begin{equation}
	\left.\begin{array}{l}
	J^2\\
	J_z
	\end{array}\right\}\ket{j,m},\qquad J_\pm\ket{j,m} \propto \ket{j,m+1}
	\end{equation}
	Nous avions également
	\begin{equation}
	\|J_\pm\ket{j,m}\|^2 = [j(j+1)-m(+1)]\hbar^2 >0,\qquad j\in\mathbb{Z}\ \text{ou }\ \mathbb{Z}/2, 
	\quad -j\leq m \leq j
	\end{equation}
	La relation de proportionnalité ci-dessus est vrai à une constante près. Par convention, 
	on suppose que la phase de cette proportionnalité est nulle
	\begin{equation}
	J_\pm \ket{j,m} = \sqrt{j(j+1)-m(m\pm 1)}\ket{j,m+1}
	\end{equation}
	où cette constante de proportionnalité est la constante de Codon Sortley et la phase vaut bien 
	1.
	


\section{Quantification du moment cinétique orbital en base position ($l$ entier)}
Les harmoniques sphériques peuvent s'écrire
\begin{equation}
\bra{\Omega}\ket{l,m} = Y_l^m
\end{equation}
Intéressons-nous maintenant au cas particulier du moment cinétique lorsque l'origine est orbitale. 
On notera ce moment cinétique $\vec{L}$.\\

	Nous pouvons décrire $\vec{L}$ en terme d'opérateur
	\begin{equation}
	\begin{array}{ll}
	\hat{L_z} &= \hat{x}\hat{p_y}-\hat{y}\hat{p_x}\qquad (\vec{p} = -i\hbar\vec{\nabla})\\
	&= -i\hbar(x\partial_y - y\partial_x)
	\end{array}
	\end{equation}
	Pour démontrer que $j$ (ici particularisé à $j=l$) doit être entier, on gagne à passer aux 
	coordonnées sphériques $r,\theta,\phi$. Par changement de variables
	\begin{equation}
	\left\{\begin{array}{ll}
	x &= r\sin\theta\cos\phi\\
	y &= r\sin\theta\sin\phi\\
	z &= r\cos\theta
	\end{array}\right. 
	\end{equation}
	Réécrit dans ces coordonnées, nous obtenons
	\begin{equation}
	\hat{L_z} = -i\hbar\dfrac{\partial}{\partial\phi}
	\end{equation}
	Que devient alors un état propre de $\hat{L_z}$? Par définition 
	\begin{equation}
	\begin{array}{lll}
	& \hat{L_z}\psi_m(\vec{r}) &= m\hbar\psi_m(\vec{r})\\
	\Leftrightarrow& -i\hbar \partial_\phi \psi_m(\vec{r}) &= m\hbar\psi_m(\vec{r})
	\end{array}
	\end{equation}
	On va utiliser le méthode de Fourier en supposant une séparation des 
	variables
	\begin{equation}
	\psi_m(\vec{r}) = \Psi_m(r,\theta)\Phi_m(\phi)
	\end{equation}
	En substituant cette solution
	\begin{equation}
	\begin{array}{lll}
	& -i\hbar\partial_\phi \Psi_m(\phi) &= m\hbar\Phi_m(\phi)\\
	\Leftrightarrow& \partial_\phi \Phi_m(\phi) &= im\Phi_m(\phi)\\
	\Leftrightarrow& \Phi_m(\phi) &= e^{im\phi}
	\end{array}
	\end{equation}
	La solution devient alors
	\begin{equation}
	\psi_m(r,\theta,\phi) = \Psi_m(r,\theta)e^{im\phi}
	\end{equation}
	Par périodicité, $\phi\rightarrow \phi+2\pi$. Dès lors
	\begin{equation}
	\begin{array}{ll}
	e^{im(\phi+2\pi)} &= e^{im\phi}\\
	e^{im2\pi} &= 1
	\end{array}\qquad \Longrightarrow m \in\mathbb{Z}
	\end{equation}
	Or, nous avons vu précédemment que
	\begin{equation}
	\left\{\begin{array}{ll}
	l-m \in \mathbb{Z}\\
	l+m \in \mathbb{Z}	
	\end{array}\right.
	\end{equation}
	Ceci démontre que 
	\begin{equation}
	l \in\mathbb{Z}
	\end{equation}
	
	Si l'on considère une valeur demi-entière de $l$, en tournant de $2\pi$, la 
	fonction d'onde se voit multiplier d'un facteur -1 : le module n'est donc pas 
	modifier. Dès lors, un spin $1/2$ est quelque chose de bizarre car il faudrait 
	"tourner deux fois" pour retrouver ce que nous avions initialement. Ceci montre 
	que le spin n'a pas d'équivalent classique. La vision de la "rotation sur soi-même" 
	n'est donc pas correcte (à cause du nombre demi-entier, $s=1/2$). Néanmoins, cette 
	vision "intuitive" reste utilisée.
	
	\section{Moment magnétique orbital (spin), rapport gyromagnétique}
	Il s'agit de la mise en évidence de l'existence du spin. Lorsqu'on a un 
	moment cinétique d'origine orbitale on peut y associer à un moment dipolaire. 
	L'orientation du champ magnétique va jouer avec le moment magnétique : cela 
	déplace les niveaux d'énergie, permettant une mise en évidence du spin. Trois 
	grande expériences ont montré l'existence du spin
	\begin{enumerate}
	\item Expérience de Sten-Gerlach : celle-ci met en évidence le caractère discret 
	du moment cinétique, son caractère quantique. Elle met également en évidence l'effet 
	de spin $1/2$.
	\item Effet Zeeman 
	\item Précession de Larmor : un dipôle va tourner lorsqu'il est placé dans un 
	champ magnétique, causant des effets quantiques.
	\end{enumerate}
	
	Introduisons l'observable moment cinétique
	\begin{equation}
	\hat{\vec{\mu}} = \gamma_0\hat{\vec{L}}
	\end{equation}
	où $\gamma_0$ est le rapport gyromagnétique : il s'agit du rapport entre 
	l'aspect magnétique $\mu$ et l'aspect cinétique $\vec{L}$. L'hamiltonien décrivant 
	cette situation est donné par 
	\begin{equation}
	\hat{H}_M = -\hat{\vec{\mu}}.\vec{B}
	\end{equation}
	Si on prend le cas particulier d'un atome, la particule tourne dans un potentiel central. 
	
		\subsubsection{Pas de champ magnétique}	
		Imaginons qu'il n'y ai 	pas de champ magnétique. Nous avons comme ECOC
		\begin{equation}
		\{H, L^2, L_z\}\quad \rightsquigarrow \ket{n_r,l,m} \rightarrow \left\{\begin{array}{ll}
		E &= E(n_r,l)\\
		L^2 &= l(l+1)\hbar^2\\
		L_z &= \hbar m
		\end{array}\right.
		\end{equation}
		Ces observables ont donc un état propre commun renseigné ci-dessus. Il est important de remarquer 
		que $E$ n'est \textbf{pas} fonction de $m$ (car $B=0$). Comme $-l\leq m\leq l$ et que 
		l'énergie n'est pas fonction de $m$, cette énergie est $2l+1$ fois dégénérée.\\
		
		Dans ce cas sans champ magnétique, nous avons
		\begin{equation}
		\gamma_0 = -\frac{e}{2m_e},\qquad \mu_z = -\frac{e}{2m_e}L_z = -m\frac{e\hbar}{2m_e} = 
		-m\mu_B
		\end{equation}
		où $\mu_z$ est la projection su moment magnétique suivante l'axe $z$, $m$ est entier 
		et $\mu_B \approx 9*10^{-24} J.T^{-1}$ est le \textit{magnéton de Bohr}. Sa valeur est 
		toute petite car il est associé à l'interaction d'un unique électron dans un champ
		magnétique. Nous avons ici montré que la projection du moment magnétique ne peut pas 
		prendre n'importe quelle valeur, mais seulement une nombre entier multiplié par une 
		certaine constante.
		
		\subsubsection{Avec un champ magnétique}
		Allumons maintenant un champ magnétique suivant l'axe $z$ 
		\begin{equation}
		\vec{B}=B\vec{1_z}
		\end{equation}
		Dans notre hamiltonien $\hat H$ se rajoute un terme lié à l'interaction magnétique
		\begin{equation}
		\hat{H}_M = -\hat{\mu_z}B = m\mu_BB
		\end{equation}
		Il s'agit donc d'un nombre entier multiplier par le magnéton de Bohr que multiplie 
		encore le champ magnétique. Ici, l'énergie sera fonction de $m$, il y aura donc une 
		levée de la dégénérescence avec l'apparition de $(2l+1)$ sous-niveaux. L'écart entre 
		deux sous-niveau est donné par $\mu_BB$.
		
	
\section{Moment magnétique intrinsèque, notion de spin}
	\subsection{Généralisation : moment cinétique $\vec{J}$}
	Nous allons faire l'hypothèse que les formules précédemment vue vont s'étendre au cas 
	étudié à l'instant et généraliser. \\
	On va toujours pouvoir avoir un observateur moment cinétique
	\begin{equation}
	\hat{\vec{\mu}} = \gamma\hat{\vec{J}},\qquad \hat{H}_M = -\hat{\vec{\mu}}\vec{B}
	\end{equation}
	On peut étendre ces formules à n'importe quel observable moment cinétique. On 
	retrouvera la même levée de dégénérescence lorsqu'un champ est allumé. En effet,
	la différence est que ici, contrairement à $l$, $j$ peut être demi-entier. Si 
	$j$ est entier, nous observerons un nombre impair de raies. Inversement, si $j$ 
	est demi-entier, le nombre de raies observées serait paires.\\
	
	Si le spin vaut $1/2\rightarrow j = \pm 1/2$ et on observe deux raies. C'est cette 
	observation qui a mis en évidence l'existence du spin $1/2$.	
	C'est ce qui a été découvert en regardant les atomes alcalins.  Si on regarde 
	les raies on voit apparaître un nombre pair de raie (exp. de spectroscopie). 
	On a donc une double division du moment cinétique : un orbital $\vec{L}$ et un 
	intrinsèque $\vec{S}$. On trouve les relation semblables
	\begin{equation}
	\hat{\vec{\mu_S}} = \gamma\hat{\vec{S}},\qquad \hat{H}_M = -\hat{\vec{\mu_S}}.\vec{B}
	\end{equation}
	Considérons le cas particulier d'un atome. Le facteur gyromagnétique va être 
	traditionnellement 	écrit comme
	\begin{equation}
	\gamma = g\gamma_0\qquad \text{où }\ \gamma_0 = -\frac{e}{2m_e}
	\end{equation}
	où $g\approx 2$, le \textit{facteur gyromagnétique}.
	
	Le rapport donne le lien entre moment cinétique et magnétique, dans le 
	cas d'un spin il faut tenir compte de ce facteur. Ce qui et intéressant (sans 
	démonstration) c'est que ce facteur gyromagnétique peut être calculé. Si on 
	prend la MQ relativiste  (généralisation de l'équation de Schrödinger dans un 
	cadre relativiste) on peut prouver que ce facteur\footnote{Il n'y a en effet 
	que la relativité qui peut rigoureusement justifier la nécessite d'utiliser le spin.}, 
	après de rigoureux calculs, vaut exactement 2. Si on utilise une version encore 
	plus complète (la théorie quantique des champs) où les champs sont également quantifiés, 
	il est possible de recalculer encore ce facteur. Ce qui est intéressant comme 
	anecdote est que le calcul de $g$ a été fait à plus de dix décimales, toutes vérifiées 
	expérimentalement.
	\begin{center}
	$g = 2.00231930438$
	\end{center}
	Un traitement 100\% quantique est vérifié expérimentalement, via les spectres. Il 
	s'agit d'une des prédictions de la MQ les plus spectaculaires.
	

	\subsection{Spin 1/2}
	Comme nous l'avons fait pour tout moment cinétique, nous avons toujours
	\begin{equation}	
	\vec{S},\qquad \vec{S}\times\vec{S} = i\hbar\vec{S},\qquad [\vec{S},S^2]=0
	\end{equation}
	Nous avons également un ECOC : il existe alors une base de fonction propres communes 
	aux observables de cet ECOC.
	\begin{equation}
	\{S^2,S_z\}\qquad \rightsquigarrow\qquad \left\{\begin{array}{ll}
	S^2\ket{s,m_s} &= s(s+1)\hbar^2 \ket{s,m_s}\\
	S_z\ket{s,m_s} &= m_s\hbar \ket{s,m_s}	
	\end{array}\right.\quad -s\leq m_s\leq s
	\end{equation}
	Pour $s=1/2, m_s = \pm 1/2 \rightarrow \dim\mathcal{E}_H = 2$. Il est possible de construire 
	une base des états propre
	\begin{equation}
	\text{\textsc{Base} : }\ \left\{\begin{array}{ll}
	\ket{+} &=\DS \ket{\frac{1}{2},\frac{1}{2}}\\
	\ket{-} &=\DS \ket{\frac{1}{2},-\frac{1}{2}}	
	\end{array}\right.
	\end{equation}
	Appliquons les observables de notre ECOC à ces vecteurs de base
	\begin{equation}
	\begin{array}{ll}
	S^2\ket{\pm} &= \frac{1}{2}\left(\frac{1}{2}+1\right)\hbar^2\ket{\pm} = \frac{3\hbar^2}{4}\ket{\pm}\\
	S_z\ket{\pm} &= \pm\frac{\hbar}{2}\ket{\pm}
	\end{array} 
	\end{equation}
	Les états propres commun on donc une projection selon $z$ valant $\pm \hbar/2$.
	On peut aussi définir des opérateurs de montée et descente
	\begin{equation}
	\left\{\begin{array}{ll}
	S_+\ket{+} &= 0\\
	S_-\ket{+} &= \hbar\ket{-}
	\end{array}\right.\qquad\qquad\left\{\begin{array}{ll}
	S_+\ket{-} &= \hbar\ket{+}\\
	S_-\ket{+} &= 0
	\end{array}\right.
	\end{equation}
	Nous avions obtenu les relations suivantes	
	\begin{equation}
	\left\{\begin{array}{ll}
	S_x &= \frac{1}{2}(S_++S_-)\\
	S_y &= \frac{1}{2i}(S_+-S_-)	
	\end{array}\right.
	\end{equation}
	Regardons ce que donne l'application de $S_{x,y}$ sur les vecteurs de base
	\begin{equation}
	\left\{\begin{array}{ll}
	S_x\ket{+} &= \frac{\hbar}{2}\ket{-}\\
	S_x\ket{-} &= \frac{\hbar}{2}\ket{+}	
	\end{array}\right.\qquad\qquad\left\{\begin{array}{ll}
	S_y\ket{+} &= -\frac{i\hbar}{2}\ket{-}\\
	S_y\ket{-} &= -\frac{i\hbar}{2}\ket{+}	
	\end{array}\right.
	\end{equation}
	On nomme parfois les vecteurs de la base \textit{up} et \textit{down} et on leur associe 
	une représentation matricielle
	\begin{equation}
	\ket{+} \equiv \ket{\uparrow} = \left(\begin{array}{c}
	1\\
	0
	\end{array}\right)\qquad\qquad\ket{-} \equiv \ket{\downarrow} = \left(\begin{array}{c}
	0\\
	1
	\end{array}\right)
	\end{equation}
	On peut exprimer les trois opérateurs dans cette base. 
	\begin{equation}
	S_z = \dfrac{\hbar}{2}\underbrace{\left(\begin{array}{cc}
	1 & 0\\
	0 & -1
	\end{array}\right)}_{\sigma_z},\qquad S_x = \dfrac{\hbar}{2}\underbrace{\left(\begin{array}{cc}
	0 & 1\\
	-1 & 0
	\end{array}\right)}_{\sigma_x},\qquad S_y = \dfrac{\hbar}{2}\underbrace{\left(\begin{array}{cc}
	0 & -i\\
	i & 0
	\end{array}\right)}_{\sigma_z},\qquad S^2  = \dfrac{3\hbar^2}{4}\underbrace{\left(\begin{array}{cc}
	1 & 0\\
	0 & 1
	\end{array}\right)}_{\mathcal{I}}
	\end{equation}
	On retrouve les matrices de Pauli. On peut facilement remarquer que
	\begin{equation}
	\hat{S}_{x,y,z} = \frac{\hbar}{2}\hat{\sigma}_{x,y,z},\qquad\qquad \sigma_x^2 = \sigma_y^2 
	=\sigma_z^2 = \mathbb{1}
	\end{equation}
	Et donc
	\begin{equation}
	S^2 = \frac{\hbar^2}{4}(\mathbb{1}+\mathbb{1}+\mathbb{1}) = \frac{3\hbar^2}{4}\mathbb{1}
	\end{equation}

	Nous pouvons maintenant passer au dernier aspect important : addition ou composition des 
	moments cinétiques. La subtilité se cache dans les nombres quantiques.
	
\section{Règles de couplage (addition) de moments cinétiques}
Ces règles ont une importance capitale lorsque l'on analyse finement les spectres 
(un niveau est en fait constitué de très fins sous-niveau) afin d'arriver à la 
structure fine et hyperfine. Ceci résulte du couplage spin orbite $\vec L\vec S$. 
Ces deux moments vont interagir et lever une dégénérescence très légère. Hyperfine, 
c'est encore plus précis : il s'agit du couplage entre plusieurs spins (le spin de 
l'électron et du proton pour un atome d'hydrogène, on va pouvoir associer un moment 
magnétique à ces deux spin ce qui va donner lieu à une levée de la dégénérescence). 
Ceci mène à la raie à 21cm de l'hydrogène neutre : le fondamental est constitué de 
deux énergie très peu différente correspondant à une raie de 21 cm. C'est utilisé pour 
détecter la présence d’hydrogène atomique dans les nuages interstellaires. \\

Trêve de bavardage, additionnons deux moments cinétiques
\begin{equation}
\left.\begin{array}{ll}
\vec{J_1}\in\mathcal{E}_1\\
\vec{J_2}\in\mathcal{E}_2
\end{array}\right\}\qquad \begin{array}{ll}
\hat{\vec{J}} &= \hat{\vec{J_1}}\times\hat{\vec{J_2}}\\
&= \hat{\vec{J_1}}\times\mathbb{1}_2+\mathbb{1}_1\hat{\vec{J_2}}
\end{array}\qquad\quad \hat{\vec{J}} \in \mathcal{E}_1\times\mathcal{E}_2
\end{equation}
Les deux relations suivantes sont également vérifiées
\begin{equation}
\left\{\begin{array}{ll}
\vec{J}\times\vec{J} &= i\hbar\vec{J}\\
\left[J^2,\vec{J}\right] &=0
\end{array} \right.
\end{equation}
Montrons le 
\begin{enumerate}
\item Sachant que les commutateurs entre deux opérateurs appartenant à deux espaces distincts 
sont nuls, nous pouvons directement écrire 
 \begin{equation}
\begin{array}{ll}
[J_x,J_y] &= [J_{1x}+J_{2x}, J_{1y}+J_{2y}]\\
&= \underbrace{[J_{1x},J_{1y}]}_{i\hbar J_{1z}} +\underbrace{[J_{2x}J_{2y}]}_{i\hbar J_{1z}} 
= i\hbar J_z
\end{array}
\end{equation}
\item Commençons par développer le carré
\begin{equation}
[J_1^2 + 2J_1J_2 + J_2^2, J_{1z}+J_{2z}]
\end{equation}
$J_1^2$ et $J_2^2$ commutent respectivement avec $J_1$ et $J_2$ et leur commutent entre eux 
(deux opérateurs appartenant à deux espaces distincts\dots). Nous avons alors
\begin{equation}
[2J_1J_2] = 2[J_{1x}J_{2x}+J_{1y}J_{2y}+J_{1z}J_{2z}, J_{1z}+J_{2z}]
\end{equation}
où $J_{1z}J_{2z}$ ne joue pas de rôle, pour les même raison que précédemment. Par linéarité
\begin{equation}
2[J_{1x}J_{2x},J_{1z}J_{2z}] + 2[J_{1y}J_{2y}, J_{1z}J_{1z}]
\end{equation}
Par les règles sur le commutateur, nous avons 
\begin{equation}
2J_{1x}(-i\hbar)J_{2y} + 2(-i\hbar)J_{2y}J_{2x} + 2J_{1y}(i\hbar)J_{2x}+2(i\hbar)J_{1x}J_{2y} = 0
\end{equation}
\end{enumerate}

On peut montrer (exercice, en vérifiant la commutation) que nous avons bien l'ECOC suivant
\begin{equation}
\{J^2, J_z, J_1^2, J_2^2\}\qquad\rightsquigarrow\qquad \ket{j,m,j_1,j_2}
\end{equation}
Il s'agit de la \textit{base couplée}.\\
Nous aurions également pu former une autre base, en partant de l'ECOC trivial suivant
\begin{equation}
\{J_1^2, J_{1z}, J_2^2, J_{2z}\qquad\rightsquigarrow\qquad \ket{j_1,m_1;j_2,m_2}
\end{equation}
Il s'agit de la \textit{base découplée}.

\subsection{Coefficients de Clebsch-Gordan}
Ce sont les coefficients qui interviennent quand on écrit la base couplée en fonction 
de la base découplée
\begin{equation}
\ket{j,m,j_1,j_2} = \sum_{m_1, m_2} C^{j,m}_{j_1,m_1,j_2,m_2} \ket{j_1,m_1;j_2,m_2}
\end{equation}
Parfois, ce coefficient est nul. Il nous reste à voir comment traiter les différents 
nombres quantiques. Commençons par voir que
\begin{equation}
\vec{J} = \vec{J_1}+\vec{J_2},\qquad J_z = J_{1z}+J_{2z}\qquad \rightarrow m=m_1+m_2
\end{equation}

Une explication "schématique" a été vue en cours (cours 8) mais il n'est malheureusement 
pas possible de l'inclure ici. On retiendra qu'il "suffit" d'appliquer les relations 
triangulaires
\begin{equation}
|j_1-j_2| \leq j \leq j_1+j_2
\end{equation}
































