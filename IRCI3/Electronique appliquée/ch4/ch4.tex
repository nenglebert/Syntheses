\chapter{L'amplificateur opérationnel}
\setcounter{section}{-1}
\section{Pourquoi amplifier}
L'amplification est omniprésente en électronique analogique. On la réalise avec l'\textit{amplificateur-opérationnel} (composant) bien que le \textit{transistor} soit également fondamentalement un amplificateur. Trois des $6*10^{34}$ raisons d'amplifier sont par exemple : 
\begin{enumerate}
\item Signaux très faibles
\item Réglage de niveau
\item Amplification en amont d'un convertisseur analogique/numérique (CAN)
\end{enumerate}

Dans une  table de mixage analogique signal initial étant relativement faible (venant par exemple d'un micro), les \textit{pré-amplis} servent à augmenter son niveau vis-à-vis des parasites. Ensuite, dans chaque voie, des \textit{amplis} servent à régler le niveau des signaux (pour le mixage).



\section{Ampli-op : propriétés de base}
\subsection{Qu'est ce qu'un ampli-op ?}
\begin{wrapfigure}[5]{r}{2cm}
\includegraphics[scale=0.25]{img/image37}
\captionof{figure}{AOP}
\end{wrapfigure}

Un ampli-op se présente le plus classiquement sous forme d'un circuit intégré à huit "pattes". Il s'agit d'un composant à \textbf{deux} bornes d'entrée et \textbf{une} borne de sortie. Conventionnellement, en électronique analogique, le triangle est son symbole.
\begin{center}
\includegraphics[scale=0.4]{img/image38}
\captionof{figure}{Représentation schématique d'un AOP}
\end{center}
Les deux bornes sont respectivement notée "+" et "-" et correspondent respectivement à l'\textit{entrée inverseuse} et à l'\textit{entrée non-inverseuse}.

\subsubsection{Tension d'entrée et de sortie}
Chacune des \textit{entrées} reçoit un signal de tension mesurée par rapport à la masse : $V^+$ et $V^-$. Même topo pour la tension de sortie $V_{out}$.\\
La fonction d'un AOP est d'amplifier la \textbf{différence} $V_d$ des tensions présentes à ses bornes. Il s'agit de la \textit{tension d'entrée différentielle\footnote{On retrouve la notion de tension différentielle vu au cours \textit{Électricité : ELEC-H-200} signifiant ici qu'aucune des deux bornes n'est la masse.}}

\subsubsection{Loi fondamentale et gain}
Cette fonction d'amplification se traduit mathématiquement par la loi : 
\begin{equation}
V_{out} = A.V_d
\end{equation}
où $A$ est le gain de l'AOP, c'est à dire le facteur\footnote{Sans dimension.} d'amplification de son entrée différentielle et sa sortie.\\
Quelques remarque importantes sont à considérer :
\begin{itemize}
\item Le gain $A$ est toujours très élevé : 30,000 à 100,000 $\rightarrow$ il sera considéré infini.
\item Un tel gain n'a de sens que si le signal d'entrée est très faible : c'est le cas pour $V_d$ ($\ll 1\ mV$).
\item $V_d$ et $V_{out}$ sont de signes quelconques.
\end{itemize}
Le schéma ci-dessous récapitule les concepts précédents et les deux relations à retenir :
\begin{center}
\includegraphics[scale=0.5]{img/image39}
\captionof{figure}{Schéma de synthèse}
\end{center}


\subsubsection{Tensions et bornes d'alimentation}
\begin{wrapfigure}[8]{l}{3cm}
\includegraphics[scale=0.3]{img/image40}
\captionof{figure}{Alimentation}
\end{wrapfigure}
Afin d'amplifier le signal, l'AOP doit être lui même alimenté en énergie : il possède en plus de ses bornes d'entrée et de sortie des bornes d'alimentations $+V_{al}$ et $-V_{al}$. \\

Ces tensions à fournir sont le plus souvent symétriques et valent typiquement +12V/-12V ou +15V/-15V. L'AOP est donc un composé \textit{actif}.\\

Un \textit{quadripôle actif} est défini par le fait qu'un signal de puissance élevée (sortie) est contrôlé par un signal de puissance plus faible (entrée), ce qui est typiquement le cas de notre AOP.\\

\prop{La tension de sortie ne peut pas sortir de la gamme fixée par ces tensions d'alimentations.
\begin{equation}
-V_{al} \leq V_{out} \leq V_{al}
\end{equation}}



\newpage

\subsection{Impédances d'entrée et de sortie}\begin{wrapfigure}[1]{r}{3cm}
\includegraphics[scale=0.23]{img/image41}
\captionof{figure}{E. Thévenin}
\end{wrapfigure}
\subsubsection{Équivalent de Thévenin}

Pour modéliser l'équivalent de Thévenin, il faut tenir compte de deux particularités de l'AOP :
\begin{enumerate}
\item L'impédance d'entrée $Z_{in}$ est différentielle
\item Au contraire, la source $E_{out}$ est référencée à la masse.
\end{enumerate}


\subsubsection{Valeur des impédances}
L'impédance d'entrée (différentielle) d'un AOP est \textbf{très élevée} ($>\ 10\ M\Omega$). Cela à pour conséquence que le courant signalant entre la borne "+" et "-" est très faible. On considèrera cette impédance comme infinie.\\
L'impédance de sortie est \textbf{très faible} ($\approx 1\ \Omega$). On fera souvent l'approximation que celle-ci est nulle. En résumé : 1. Impédance d'entrée (très) élevée et 2. Impédance de sortie (très) faible.\\
Un AOP convient donc, à priori, pour une \textit{adaptation d'impédance en tension}. Pour rappel :\\

\prop{Lorsqu'on désire transmettre un signal de tension, l'impédance de sortie doit être faible devant l'impédance d'entrée.}

\subsubsection{Source commandée}
Il reste à définir la source commandée $E_{out}$ de l'équivalent de Thévenin. Pour rappel :\\

\prop{La source commandée fait le "lien" entre l'entrée et la sortie de l'équivalent de Thévenin et modélise ainsi la "fonction" de l'AOP.}\ \\

Soit la loi fondamentale $V_{out} = A.V_d$. En négligeant $Z_{out}$ (supposée nulle), on peut assimiler $V_{out}$ à $E_{out}$. On a donc : 
\begin{equation}
E_{out} = A.V_{in} = A.V_d
\end{equation}

\subsection{Caractéristique de transfert et principe du zéro virtuel}\begin{wrapfigure}[6]{r}{3cm}
\includegraphics[scale=0.25]{img/image42}
\captionof{figure}{Caractéristique AOP}
\end{wrapfigure}
\subsubsection{Caractéristiques de transfert}

Un AOP possède plusieurs caractéristiques, mais nous nous intéressons ici à la \textit{caractéristique de transfert} c'est à dire le rapport entre la tension d'entrée $V_d$ et la tension de sortir $V_{out}$. Cette relation est déjà connue comme la loi fondamentale de l'AOP, loi qui se traduit par une droite de pente $A$.

\subsubsection{Caractéristique : zone linéaire}
\begin{wrapfigure}[8]{l}{3cm}
\includegraphics[scale=0.25]{img/image43}
\captionof{figure}{Zone linéaire}
\end{wrapfigure}
Il ne faut pas oublier de tenir compte d'un phénomène supplémentaire, à savoir :
\begin{equation}
-V_{al} \leq V_{out} \leq +V_{al}
\end{equation}
Ces limites se traduisent pas des horizontales dans le plan de la caractéristique. Seule la zone entre ces horizontales est accessible : il s'agit de la \textbf{zone linéaire}.\\
Les limites de cette zone sont simple à calculer, il s'agit des points de sortie pour lesquels $V_{d} = +V_{al}/A$ et $-V_{al}/A$. La grande valeur de $A$ a pour conséquence que la zone linéaire est très étroite sur l'axe horizontal.

\subsubsection{Caractéristique : saturation}
\begin{wrapfigure}[8]{r}{3cm}
\includegraphics[scale=0.25]{img/image44}
\captionof{figure}{Saturation}
\end{wrapfigure}
Si la tension d'entrée est au-dessus de $+V_{al}/A$, la tension de sortie est \textit{limitée} à la valeur $+V_{al}$ : l'AOP ne peut aller plus haut que sa propre tension d'alimentation positive (et pareil pour la tension négative). Dans ces deux cas, lorsque l'on sort de la zone linéaire on dit que l'AOP \textbf{sature}.\\
\textbf{Attention !} Dans les zones de saturations, la loi fondamentale n'est plus vérifiée : il n'y a plus d'amplification.




\subsubsection{Principe du zéro virtuel}
Il s'agit d'une propriété générale de tous les AOP, propriété directement liée à la caractéristique de ce dernier. Cette propriété fondamentale permet de faire gagner beaucoup de temps et s'énonce :\\

\prop{\textsc{Principe du zéro virtuel :} Tant qu'un amplificateur opérationnel \textbf{ne sature pas}, sa tension différentielle d'entrée est virtuellement nulle.}\ \\

Il est important de ne considérer cette théorique seulement si l'AOP est en zone linéaire (\textbf{erreur classique!}). Une tension différentielle virtuellement nulle signifie que $V_d$ est tellement faible que l'on peut la négliger devant les autres tensions (souvent, $V_d \ll 1\ mV$). Négliger $V_d$ signifie l'approximation :
\begin{equation}
V_d = 0\ V
\end{equation}
C'est cette approximation que l'on appelle \textit{zéro virtuel}.\footnote{Zéro car $V_d$ est nulle, virtuel pour rappeler l'approximation.}


\subsubsection{Principe du zéro virtuel : remarques}
Quelques remarques sont à prendre en compte :
\begin{enumerate}
\item Compte tenu de la définition de $V_d$, le zéro virtuel revient encore à faire l'approximation :
\begin{equation}
V^+ = V^-
\end{equation}
\item La grande valeur de $A$ justifie ce principe. En effet, dans la zone linéaire $V_d$ est négligeable ce qui est évident au moment de tracer la caractéristique de transfert.
\item Le principe ne garantit \textbf{pas} que l'AOP est dans la zone linéaire.
\item Si l'on suit le principe, la tension d'entrée est nulle, or c'est elle que l'on veut amplifier... Problem ? No ! Le zéro virtuel fait \textbf{simultanément} deux approximations :
\begin{enumerate}
\item $V_d = 0\ V$
\item $A = \infty$
\end{enumerate}
La loi fondamentale devient une "forme indéterminée" et la valeur n'est pas forcément nulle; pas de contradictions.
\item Si l'on fait une des deux approximation du point précédent sans faire l'autre, on est foutu.
\end{enumerate}

\begin{wrapfigure}[9]{r}{3cm}
\includegraphics[scale=0.45]{img/image45}
\captionof{figure}{Schéma non-inverseur}
\end{wrapfigure}
\section{Deux montages amplificateurs}
\subsection{Ampli non-inverseur}
Un montage amplificateur inverseur est un ampli-op auquel on a ajouté deux résistances $R_1$ et $R_2$ dans le but de "contrôler" le gain. 

\subsubsection{Montage >< composant}
Il est fondamental de faire la différence entre "deux" niveau du schéma : l'AOP qui est le composant et le montage construit sur base de cet AOP. C'est l'ensemble complet que l'on appelle \textit{amplificateur non-inverseur}. Son signal d'entrée est $V_{in}$, son signal de sortie $V_{out}$. Ces deux tensions sont référencées à la masse, elle sont donc \textbf{non-}différentielles.\\

Sur la figure du schéma non-inverseur, la résistance $R_2$ placée entre la sortie de l'AOP et son entrée inverseuse : elle est en \textbf{rétroaction}. 


\subsubsection{Calcul du gain $A_{\text{non\_inv}}$}
Nous allons chercher à calculer le \textit{gain du montage} défini\footnote{$A_{\text{non\_inv}}$ est le gain "à vide", on suppose qu'aucune charge n'est connectée à la sortie du montage.} :
\begin{equation}
A_{\text{non\_inv}} = \frac{V_{out}}{V_{in}}
\end{equation}
Pour calculer ce gain, il faut choisir le modèle d'AOP que l'on utilise dans les calculs. Faisons les hypothèses suivantes et démontrons une expression du gain :
\begin{itemize}
\item Le gain $A$ est fini (élevé, mais pas infini).
\item Son impédance d'entrée $Z_d$ est infinie.
\item Son impédance de sortie est nulle.
\end{itemize}
\begin{proof}
\ \\
Considérons les deux relations fondamentales de départ, ré-écrite sous une seule identité :
\begin{equation}
\left\{\begin{array}{ll}
V_{out} & = A.V_d\\
V_d & = V^+ - V^-
\end{array}\right.\ \ \ \ \ \ \ \Rightarrow\ \ \ \ V_{out} = A.(V^+ - V^-)
\end{equation}
Essayons d'éliminer $V^+$ et $V^-$ au profit de $V_{in}$ et $V_{out}$ qui sont les variables à garder en finale. Comme $V_{in}$ est directement appliquée à l'entrée "+", on peut écrire :
\begin{equation}
V^+ = V_{in}
\end{equation}
On peut calculer $V^-$ par la méthodes des courants de branche, mais les propriétés de l'AOP permettent d'aller un peu plus vite. Posons qu'il existe un courant $I$ traversant $R_2$. Lorsque ce courant quitte $R_2$, il ne peut rentrer dans l'AOP car l'impédance de celui-ci est infinie : $I$ doit donc passer en totalité dans la résistance $R_1$.

\begin{center}
\includegraphics[scale=0.45]{img/image46}
\captionof{figure}{Schéma pour le calcul de $V^-$}
\end{center}

 En appliquant la loi d'Ohm sur chacune des résistances, on peut éliminer $I$ pour obtenir notre troisième équation : 
\begin{equation}
\left\{\begin{array}{ll}
V^- & = R_1I\\
V_{out} - V^- & = R_2I\\
\rightarrow V_{out} & = (R_1+R_2)I
\end{array}\right.\ \ \ \ \ \ \ \Rightarrow\ \ \ \ V^- = \dfrac{R_1}{R_1+R_2}V_{out}
\end{equation}
Notons que ce résultat aurait pu être immédiat par application du diviseur résistif, sous l'hypothèse que $Z_d = \infty$.\\ En combinant nos trois équations obtenues, on trouve l'équation
\begin{equation}
V_{out} = A\left(V_{in} - \dfrac{R_1}{R_1 + R_2}V_{out}\right)
\end{equation}
qu'il suffit de réarranger pour obtenir le gain recherché :
\begin{equation}
A_{\text{non\_inv}} = \frac{V_{out}}{V_{in}} = \frac{A(R_1+R_2)}{AR_1+R_1+R_2}
\end{equation}
La valeur élevée de $A$ permet de négliger $R_1$ et $R_2$ par rapport à $AR_1$. On trouve finalement le gain de notre AOP :\\

\prop{$A_{\text{non\_inv}} \approx 1 + \frac{R_2}{R_1}$}
\end{proof}

\subsubsection{Amplificateur non-inverseur : remarques}
Si ce montage est dit "non-inverseur", c'est parce que son gain est positif. On remarque également que le gain de l'AOP ($A$) n'est pas dans l'expression de $A_{\text{non\_inv}} \rightarrow$ le gain du montage ne dépend pas du gain de l'AOP mais uniquement des valeurs des deux résistances. Les deux sous-sections précédentes sont consacrées à deux de ses propriétés.

\subsubsection{Montage non-inverseur à gain variable}
Régler le gain peut être fondamental, comme par exemple pour régler le volume de votre amplificateur audio. Pour se faire, il suffit de remplacer une des deux résistances (le plus souvent $R_2$, car étant au numérateur elle donne une variation linéaire du gain) par un \textit{potentiomètre}. Pour rappel :\\

\prop{Un potentiomètre est une résistance variable dont l'utilisateur règle la valeur via un curseur mécanique.}\ \\



%\subsubsection{Impédance d'entrée}

\subsection{Ampli inverseur}
\begin{wrapfigure}[7]{r}{3cm}
\includegraphics[scale=0.35]{img/image47}
\captionof{figure}{Schéma inverseur}
\end{wrapfigure}
Ce montage à base d'un AOP est quasi-identique au non-inverseur à une exception près : le rôle de la masse et de l'entrée $V_{in}$ ont été permuté :
\begin{itemize}
\item $V_{in}$ est maintenant connectée à $R_1$
\item L'entrée "+" est maintenant mise à la masse
\end{itemize}
Notons cependant que $R_2$ est toujours montée en rétroaction (entre l'entrée et la sortie) et que $R_1$ et $R_2$ forment toujours un diviseur résistif (un peu particulier, car aucune des extrémités n'est à la masse).

\subsubsection{Calcul du gain $A_{inv}$}
En utilisant la méthode de calcul classique du point précédent (même hypothèses et démarches de calcul), on trouve que le gain à vide vaut :

\prop{\begin{equation}
A_{inv} \approx -\frac{R_2}{R_1}
\end{equation}}

\begin{proof}\ \\
Afin de calculer la réponse en fréquence de ce circuit, il faut tout d'abord modéliser
    l'amplificateur opérationnel par un élément d'impédance d'entrée infinie, d'impédance
    de sortie nulle et de tension de sortie :
    
        \begin{equation}
        V_{\text{out}}=A(V^+ - V^-)
        \end{equation}
    où A est le gain propre à l'amplificateur, très élevé, mais fini. Le gain à vide du montage
    étant:
        \begin{equation}
        G = \frac{V_{\text{out}}}{V_{\text{in}}}
        \end{equation}
    
    Deux autres équations sont encore nécessaires pour arriver à  calculer le gain de notre
    amplificateur inverseur. Premièrement, nous savons que la borne positive est directement
    connectée à la masse. Deuxièmement, en utilisant la loi des mailles et en supposant qu'un
    même courant $I$ circule dans les résistances $R_1$ et $R_2$ (l'amplificateur possédant
    une impédance d'entrée infinie). Nous obtenons dès lors :

        \begin{equation}
            \left\{\begin{array}{l}
             V^+=0  \\
             V_{\text{in}}+R_1 I + R_2 I = V_{\text{out}}
            \end{array}\right.
        \end{equation}
    Sachant que $V^-=V_{\text{in}}+R_1 I$, nous obtenons l'équation :
        \begin{equation}
        V^-=V_{\text{in}}+R_1\frac{V_{\text{out}}-V_{\text{in}}}{R_1+R_2}
        \end{equation}

    Nous pouvons désormais exprimer $V_{\text{out}}$ en fonction de $V_{\text{in}}$ :
        \begin{equation}
        V_{\text{out}}=-A\left( \dfrac{V_{\text{in}}(R_1+R_2)+R_1 V_{\text{out}}-R_1 V_{\text{in}}}{R_1+R_2}\right) = -\dfrac{R_2 V_{\text{in}}+R_1 V_{\text{out}}}{R_1+R_2}
        \end{equation}
    Ou encore, après ré-arrangement :
        \begin{equation}
        (A R_1 + R_1 + R_2) V_{\text{out}} = -A R_2 V_{\text{in}}
        \end{equation}
    En supposant que $A R_1$ est beaucoup plus grand que $R_1 + R_2$, nous pouvons exprimer le 
    gain de notre amplificateur inverseur :
     \begin{equation}
     G =-\frac{A R_2}{A R_1 + R_1 + R_2} \approx -\frac{R_2}{R_1}
     \label{eq:gain}
     \end{equation}
\end{proof}

\subsubsection{Amplificateur inverseur : remarques}
Si ce montage est dit "inverseur" c'est parce que son gain est négatif. Comme pour le non-inverseur, le gain $A$ à également disparu de l'expression de $A_{inv}$. Comme avant, le gain peut être rendu variable à l'aide d'un potentiomètre si ce n'est qu'ici le gain minimal est nul !

\newpage
\subsubsection{Impédance d'entrée : calcul}
\begin{wrapfigure}[8]{l}{3.5cm}
\includegraphics[scale=0.35]{img/image48}
\captionof{figure}{Calcul de $Z$}
\end{wrapfigure}
Calculons l'impédance d'entrée du montage inverseur. 
\begin{equation}
Z_{in} = \frac{V_{in}}{Z_{in}}\left|_{I_{out} = 0}\right.
\end{equation}
Le courant $I_{in}$ circulant n'est autre que celui circulant dans $R_1$, mais également celui circulant dans $R_2$ (l'impédance de l'AOP étant infinie). On peut donc écrire que ce courant vaut (loi d'Ohm sur l'ensemble des deux résistances) :
\begin{equation}
I_{in} = \frac{V_{in} - V_{out}}{R_1+R_2}
\end{equation}
Par définition de l'AOP inverseur, $V_{out}$ vaut :
\begin{equation}
V_{out} = A_{inv}.V_{in} = -\frac{R_2}{R_1}V_{in}
\end{equation}
En combinant les deux équations fraichement trouvées, on trouvé la valeur de $I_{in}$
\begin{equation}
I_{in} = \frac{V_{in} + \frac{R_2}{R_1}V_{in}}{R_1+R_2} = \frac{V_{in}}{R_1}
\end{equation}
L'impédance d'entrée vaut dès lors :
\begin{equation}
Z_{in} = R_1
\end{equation}
Il ne faut surtout pas conclure que l'impédance d'entrée du \textbf{montage} c'est "l'impédance qui se trouve à l'entrée du montage", résultat totalement faux dans le cas général ! Il faut dire que $Z_{in}$ à la même valeur que $R_1$ (sans l'être).\\

Cette impédance d'entrée est donc relativement faible (on pouvait s'attendre à ce qu'elle soit très élevée au vu de l'analyse du non-inverseur) doit être vue comme un inconvénient du montage inverseur.

\subsubsection{Méthode de calcul classique}
La méthode de résolution que nous avons utilisé tout au long du chapitre permet de résoudre tout circuit AOP avec rétroaction. Elle permet de calculer le gain à vide mais aussi de calculer l'impédance d'entrée sous hypothèses que l'impédance d'entrée d'un AOP soit aussi hight que Buzz.

\subsection{Calcul rapide par le principe du zéro virtuel}
Le principe du zéro virtuel énoncé précédemment permet d'effectuer un calcul beaucoup plus rapide qu'avec la méthode classique\footnote{Cela implique qu'on suppose $A = \infty$ dès le départ.}.

\newpage
\subsubsection{Calcul du gain de l'AOP non-inverseur}
\begin{wrapfigure}[8]{r}{3.5cm}
\includegraphics[scale=0.35]{img/image49}
\captionof{figure}{AOP N-I}
\end{wrapfigure}
Commençons par poser nos différentes hypothèses sur l'AOP : impédance différentielle $Z_d$ infinie, impédance de sortie null et l'on a un zéro virtuel (gain de l'AOP infini et $V_d = 0$).\\
La "loi" n'est \textbf{plus} à utiliser ici ! De plus utiliser $A$ peut poser problème (comme supposé infini). A la place, ayant un zéro virtuel, on peut écrire $V^+ = V^- \rightarrow$ cette équation nous servira de "première équation". \\
Notre système d'équation est ainsi :
\begin{equation}
\left\{\begin{array}{ll}
V^+ &= V^-\\
V^+ &= V_{in}\\
V^- &= \frac{R_1}{R_1+R_2}V_{out}
\end{array}\right.
\end{equation}
En vertu de la première équation, on peut immédiatement égaler les deux équations de droite et trouver directement le gain du montage
\begin{equation}
A_{non_inv} = \frac{V_{out}}{V_{in}} = 1+\frac{R_2}{R_1}
\end{equation}
Ce qui est bien le même que celui obtenu avec la méthode classique.\\

C'est bien beau, mais on peut encore faire mieux en raisonnant "directement sur le schéma" tout en limitant les développement mathématiques :
\begin{enumerate}
\item Dans l'hypothèse du zéro virtuel, $V^- = V^+$
\item Or $V^+$ vaut $V_{in}$, ceux-ci étant en connexion directe
\item Donc, $V^-$ vaut $V_{in}$
\item Par ailleurs, $R_1$ et $R_2$ forment un diviseur résistif (car aucun courant ne passe dans l'AOP ($Z_d = \infty$)
\item Pour ce diviseur résistif, on peut donc écrire
\begin{equation}
V_{in} = \frac{R_1}{R_1+R_2}V_{out}
\end{equation}
Dont on déduit directement le gain
\begin{equation}
A_{non_inv} = \frac{V_{out}}{V_{in}} = 1+\frac{R_2}{R_1}
\end{equation}
\end{enumerate}

En terme de rapidité, on ne peut faire mieux ! Mais : il faut bien maîtriser les concepts de la théorie et les appliquer au bon moment! 

\subsubsection{Calcul du gain de l'AOP inverseur}
\begin{wrapfigure}[8]{r}{3.5cm}
\includegraphics[scale=0.35]{img/image50}
\captionof{figure}{AOP I.}
\end{wrapfigure}
Travaillons directement sur le schéma :
\begin{enumerate}
\item Dans l'hypothèse du zéro virtuel, $V^- = V^+$
\item Or $V^+$ vaut $0V$ : connexion directe à la masse
\item Donc $V^-$ vaut 0 $V$ (on parle de "masse virtuelle")
\item Par ailleurs, $R_1$ et $R_2$ forment un diviseur résistif (car aucun courant ne passe dans l'AOP ($Z_d = \infty$) (non "classique" car on a 0V au milieu des deux résistances)(j'aime les parenthèses)
\item Le courant dans $R_1$ est facile à trouver :
\begin{equation}
I = V_{in}/R_1
\end{equation}
En effet, comme l'extrémité droite de $R_1$ est (virtuellement à à la masse, la ddp sur $R_1$ vaut bien $V_{in}-0V = V_{in}$.
\item Comme ce même courant passe dans $R_2$, on peut écrire : $0V-V_{out} = R_2I$. On trouve finalement
\begin{equation}
V_{out} = -R_2I = -R_2\left(\frac{V_{in}}{R_1}\right)
\end{equation}
ou encore
\begin{equation}
A_{inv} = \frac{V_{out}}{V_{in}} = - \frac{R_2}{R_1}
\end{equation}
\end{enumerate}


\subsection{Analyse de la rétroaction}
Les deux montages vu précédemment comportent tous deux une résistance $R_2$ en rétroaction. Clarifier le fonctionnement de cette rétroaction peut répondre à un certains nombres de questions : pourquoi amplifier $V_d$ et pas $V_{in}$, que justifie l'hypothèse du zéro virtuel ?

\subsubsection{Analyse de la rétroaction : principe}
Au cours de notre raisonnement, nous adopterons deux règles :
\begin{enumerate}
\item Raisonner strictement en termes de "cause" et d'"effet"
\item L'AOP régit avec un certain délai : raisonnons plus vite que lui ; supposons que la tension de sortie de l'AOP ne peut subir de discontinuité (échelon)
\end{enumerate}


\subsection{Compléments sur la rétroaction}




\section{Montages à ampli-op}
\section{Ampli-op : propriétés avancées}