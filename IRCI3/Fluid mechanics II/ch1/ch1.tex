%%%%%%%%%%%%%
%  Ch1 : Generalities  %
%%%%%%%%%%%%%

\chapter{Generalities}
\section{Fundamental laws}
\subsubsection{Reminder}
	Let's first remind the 3 basic principles of \textit{Fluid mechanics I} : \\
	
	\begin{itemize}
		\item[•] \textbf{Mass conservation :} \textit{The mass of a closed system remains constant in time.}\\
		This is much a definition of a closed system than a principle. We have to notice that related to Einstein law of relativity, $E = mc^2$, mass must vary with energy. But if we exclude nuclear reactions, our approximation is valid. Indeed, the square of light velocity has a greater impact on energy than the mass term. If the energy exchange is huge like in nuclear reaction, mass vary, but in smaller energies domain (combustion for example), the mass can be considered as constant. \\
		
		\item[•] \textbf{Newton's law :} \textit{the time rate of change of momentum of a closed system is equal to the sum of the forces applied on the system.} \\
		
		\item[•] \textbf{First principle of thermodynamics :} \textit{the time rate of change of the total energy of a closed system is equal to the sum of the power of the forces applied on the system and the thermal power provided to the system.}
	\end{itemize}		
	
	\subsubsection{Useful equations}
	
	\begin{wrapfigure}[9]{l}{4cm}
	\vspace{-5mm}
	\includegraphics[scale=0.3]{ch1/1}
	\captionof{figure}{}
	\label{fig:1.1}
	\end{wrapfigure}
	Let's consider the integral on a moving volume of a function depending on time and position $f(\vec{x},t)$. Imagine that \autoref{fig:1.1} represents the moving volume at initial time containing mass $m$. An infinitesimal part of that volume contains an infinitesimal mass $dm = \rho dV$, where $\rho$ is mass density. We deduce the expression of the total mass at any time by that of the initial time  
	\begin{equation}
		M(t_0) = \int _{V(t_0)}\rho (\vec{x},t_0)\, dV \qquad \Rightarrow M(t) = \int _{V(t)}\rho (\vec{x},t)\, dV
	\end{equation}
	By considering $\rho (\vec{x},t)$ as $f(\vec{x},t)$, the derivative of the integral is given by
	
	\begin{center}
	\theor{\textbf{Reynolds transport theorem}
	\begin{equation}
		\frac{d}{dt}\int _{V(t)}f (\vec{x},t)\, dV = \int _{V(t)} \frac{\D f}{\D t} (\vec{x},t)\, dV + \oint _{S(t) = \D V(t)} f(\vec{x},t) \vec{b}\, \vec{n}\, dS
	\end{equation}
	where $\vec{b}$ is the surface displacement velocity. }
	\end{center}
	
	The second equation that will be used in the developement is given by
	\begin{center}
	\theor{\textbf{Gauss theorem}
	\begin{equation}
		\oint _{S = \D V} \vec{a} \, \vec{n} \, dS = \int _V \nabla \vec{a}\, dV
	\end{equation}}
	\end{center}
	
	\subsection{Mass conservation equation}
	If $V(t)$ is the moving volume occupied by the closed system as time varies, then by definition of a closed system $\frac{dM(t)}{dt} = 0$. The corresponding equation using Reynolds transport theorem is 
	\begin{equation}
		M(t) = \int _{V(t)} \rho\, dV \qquad \Rightarrow \frac{dM(t)}{dt} = \int _{V(t)} \frac{\D \rho}{\D t}\, dV + \oint _{S(t) = \D V(t)} \rho\, \vec{b}\, \vec{n}\, dS = 0
	\end{equation}
	\begin{wrapfigure}[9]{l}{3cm}
	\vspace{-5mm}
	\includegraphics[scale=0.3]{ch1/2}
	\captionof{figure}{}
	\label{fig:1.2}
	\end{wrapfigure}
	We have to express that this volume is not traversed by material. There is no flux of fluid and the particles in the volume are always the same. By definition, the infinitesimal distance traveled by the surface and the fluid are
	\begin{equation}
		d\vec{x} = \vec{b}\, dt \qquad and \qquad d\vec{x}' = \vec{u}\, dt
	\end{equation}	 
	where $\vec{u}$ is the fluid velocity. Under wich condition do we know that the fluid has not traversed the boundary? We have to define the relative dispolacement $d\vec{x}'-d\vec{x}$ of the fluid in regard to the fluid. For a closed system 
	\begin{equation}
	\begin{aligned}
		(d\vec{x}'-d\vec{x})\cdot \vec{n} = 0 \quad &\Leftrightarrow \quad dt (\vec{u}-\vec{b})\cdot \vec{n} = 0 \quad \Leftrightarrow \quad (\vec{u}-\vec{b})\cdot \vec{n} = 0 \\
		&\Rightarrow \quad \vec{b} \vec{n} = \vec{u} \vec{n}
		\end{aligned}
	\end{equation}
	\begin{center}
	\theor{\textbf{Mass conservation equation for closed systems (integral form)}
	\begin{equation}
		\int _{V(t)} \frac{\D \rho}{\D t}\, dV + \oint _{S(t) = \D V(t)} \rho\, \underbrace{\vec{b}\, \vec{n}}_{=\vec{u}\, \vec{n}}\, dS =0
		\label{eq:1.7}
	\end{equation}}
	\end{center}
	
	How to write this equation in a different way? Let's consider now a fixed open system composed of fluid particles in the fixed volume $V_0(t) = V(t_0)$. Similarly to the previous point, the mass variation in this fixed volume is expressed like 
	\begin{equation}
		M_0(t) = \int _{V_0(t)} \rho\, dV \qquad 
		\Rightarrow \int _{V_0(t)} \frac{\D \rho}{\D t}\, dV + \oint _{S_0(t) = \D V_0(t)} \rho\, \vec{b}\, \vec{n}\, dS.
	\end{equation}
	The volume integral expresses the variable mass in the fixed volume and the surface integral is nul due to the nul surface velocity (since the volume is fixed). This relation enables us to write the 
	\begin{center}
	\theor{\textbf{Mass conservation equation for fixed open systems (integral form)}
	\begin{equation}
		\frac{dM_0}{dt} + \underbrace{\oint _{S_0(t) = \D V_0(t)} \rho\, \vec{u}\, \vec{n}\, dS = 0}_{\mbox{mass flow out of the system}}
	\end{equation}}
	\end{center}	
	Let's finally consider an arbitrary open system containing fluid particles in a moving volume $V_*(t)$ such that $V_*(t_0) = V(t_0) = V_0$. Similarly we have using the Reynolds transport theorem
	\begin{equation}
		M_*(t) = \int _{V_*(t)} \rho\, dV \qquad \Rightarrow \frac{dM_*(t)}{dt} = \int _{V_*(t)} \frac{\D \rho}{\D t}\, dV + \oint _{S_*(t) = \D V_*(t)} \rho\, \vec{b}\, \vec{n}\, dS
	\end{equation}
	Using the definition of the volume at $t=t_0$, we can equalize the volume integral with that of \eqref{eq:1.7} to find 
	\begin{center}
	\theor{\textbf{Mass conservation equation for arbitrary open systems (integral form)}
	\begin{equation}
		\frac{dM_*(t_0)}{dt} + \oint _{S(t_0) = \D V(t_0)} \rho (\vec{u}-\vec{b})\, \vec{n}\, dS = 0
	\end{equation}}
	\end{center}	
	
	Let's now take \eqref{eq:1.7} again and apply Gauss theorem
	\begin{equation}
	\begin{aligned}
		\int _{V(t)} \frac{\D \rho}{\D t}\, dV + \oint _{S(t) = \D V(t)} \rho \underbrace{\vec{u}\, \vec{n}}_{\vec{a}}\, &dS =0 \qquad with \qquad 
		\oint _{S(t)} \rho\, \underbrace{\vec{u}\, \vec{n}}_{\vec{a}}\, dS = \int _{V(t)} \nabla \rho \vec{u} \, dV \\
		&\Leftrightarrow \int _{V(t)} \left[ \frac{\D \rho}{\D t} + \nabla \rho \vec{u} \right] \,dV = 0
		\end{aligned}
	\end{equation}
	For this last equation to be true for all systems, the integrated term must be equal to zero 
	\begin{center}
	\theor{\textbf{Mass conservation equation (differential form (1) - divergent form)}
	\begin{equation}
		\frac{\D \rho}{\D t} + \nabla \rho \vec{u} = 0
		\label{eq:1.13}
	\end{equation}}
	\end{center}
	
	\begin{wrapfigure}[4]{l}{4cm}
	\vspace{-5mm}
	\includegraphics[scale=0.35]{ch1/3}
	\captionof{figure}{}
	\label{fig:1.3}
	\end{wrapfigure}
	In order to find the second differential form, let's consider 2 points Q and P as described in \autoref{fig:1.3}. The difference of density between the 2 points is \\\\\\
	\begin{equation}
	\begin{aligned}
		\rho _Q(t+dt) - \rho _P(t) &= \rho (x_1+dx_1,x_2+dx_2,x_3+dx_3,t+dt) - \rho (x_1,x_2,x_3)  \\
		&= d\rho = \frac{\D \rho}{\D x_1}dx_1 + \frac{\D \rho}{\D x_2}dx_2 + \frac{\D \rho}{\D x_3}dx_3 + \frac{\D \rho}{\D t}dt
		\end{aligned}
		\label{eq:1.14}
	\end{equation}
	In general, the fluid particles at $P(t)$ and $Q(t+dt)$ are different. However, if $dx_1 = u_1 dt, dx_2 = u_2 dt, dx_3 = u_3 dt$, then the fluid particles at the 2 points are the same. By making appear these velocities in \eqref{eq:1.14}, 
	\begin{equation}
		d\rho = \left( \frac{\D \rho}{\D x_1}u_1 + \frac{\D \rho}{\D x_2}u_2 + \frac{\D \rho}{\D x_3}u_3 + \frac{\D \rho}{\D t}\right) dt
	\end{equation}
	Finally, after dividing by $dt$ the 2 members of the equation, we obtain the definition of the time rate of change of density when I follow the fluid $\dot{\rho}$. As \eqref{eq:1.13} can be expressed in term of indicial notation like 
	\begin{equation}
		\frac{\D \rho}{\D t} + u_i \frac{\D\rho}{\D x_i} + \rho \frac{\D u_i}{\D x_i} = 0 
	\end{equation}
	Replacing the sum of first and second term by $\dot{\rho}$ gives the last form
	\begin{center}
	\theor{
	\textbf{Mass conservation equation (differential form (2) - substancial form)}
	\begin{equation}
	\dot{\rho} + \rho \nabla \vec{u} = 0
	\label{eq:1.17}
	\end{equation}
	}
	\end{center}
	
	\subsection{Newton's second law : Momentum equation}
	\begin{wrapfigure}[8]{l}{4cm}
	\vspace{-5mm}
	\includegraphics[scale=0.3]{ch1/4}
	\captionof{figure}{}
	\label{fig:1.4}
	\end{wrapfigure}
	Momentum in this course is noted $\vec{P}(t)$. For closed systems,
	\begin{equation}
		\frac{d\vec{P}(t)}{dt} = \sum \vec{F} = \frac{d}{dt}\int _{V(t)} \rho \vec{u}\, dV
		\label{eq:1.18}
	\end{equation}
	where $\rho \vec{u}$ is the momentum density. We will spell out the expression of the 2 members. First, the derivative, using the Reynolds transport theorem gives 
	\begin{equation}
		\frac{d\vec{P}}{dt} = \int _{V(t)} \frac{\D \rho \vec{u}}{\D t}\, dV + \oint _{S(t) = \D V(t)} \rho \vec{u} (\vec{u} \vec{n}) \, dS
	\end{equation}
	This written in indicial notation
	\begin{equation}
	\begin{aligned}
		\frac{dP_i}{dt} &= \int _{V(t)} \frac{\D \rho u_i}{\D t}\, dV + \oint _{S(t) = \D V(t)} \underbrace{\rho u_i u_j}_{tensor :\, \vec{u} \otimes \vec{u}} n_j \, dS\\
		&= \int _{V(t)} \frac{\D \rho u_i}{\D t}\, dV + \oint _{S(t) = \D V(t)} \rho (\vec{u} \otimes \vec{u}) \vec{n}\, dS 
		\end{aligned}
	\end{equation}
	and by applying Gauss theorem to the surface integral

	\begin{center}
	\begin{equation}
		\frac{d\vec{P}}{dt} = \int _{V(t)} \left[\frac{\D \rho \vec{u}}{\D t} + \nabla \rho \vec{u} \otimes \vec{u} \right] dV \qquad and \qquad \frac{dP_i}{dt} = \int _{V(t)} \left[ \frac{\D \rho u_i}{dt} + \frac{\D}{\D x_j} (\rho u_i u_j) \right] dV
	\end{equation}
	\end{center}
	
	Based on the previous forms, we can generalize this for any arbitrary function $\phi$
	\begin{equation}
	\begin{aligned}
		\frac{d}{dt} \int _{V(t)} \rho \phi\, dV  &= \int _{V(t)} \left[\frac{\D \rho \phi}{\D t} + \frac{\D}{\D x_j} \rho \phi u_j \right] dt \\
		&= \int _{V(t)} \left[ \rho \frac{\D \phi}{dt} + \phi \right. \underbrace{\left( \frac{\D \rho}{\D t} + \frac{\D \rho u_j}{\D x_j} \right)}_{= 0\ \eqref{eq:1.13}}\left. + \rho u_j\frac{\D \phi}{\D x_j} \right] dV \\
		&= \int _{V(t)} \rho \left[ \frac{\D \phi}{\D t} + u_j \frac{\D \phi}{\D x_j} \right] dV
	\end{aligned}
	\end{equation}
	Similarly to thermodynamics courses, we can introduce an extensive variable $\Phi$ and an intensive $\phi$ to have 
	\begin{center}
	\theor{
	\textbf{General relation for any arbitrary function in closed systems}
	\begin{equation}
		\frac{d\Phi }{dt} = \int _{V(t)} \left[ \frac{\D \rho \phi}{\D t} + \nabla \rho \phi \vec{u} \right] dV = \int _{V(t)} \rho \dot{\phi} \, dV
	\end{equation}
	}
	\end{center}
	For the specific case where $\Phi = \vec{P}$ and $\phi = \vec{u}$, we obtain
	\begin{equation}
		\frac{d\vec{P}}{dt} = \int _{V(t)} \rho \underbrace{\left[ \frac{\D \vec{u}}{\D t} + \vec{u} \nabla \vec{u} \right]}_{\dot{u}} dV 
		\label{eq:1.24}
	\end{equation}
	We can now express the forces applied on the system. There are 2 main classes : \\
	\begin{itemize}
		\item[•] \textbf{Distance forces (volume)} $\vec{F}_V$ : \\
		This type of force allows a body to influence another without being in contact with. 
		\begin{itemize}
			\item The most present one is gravity which is applied on each fluid particles ($d\vec{F} = dm \vec{g}$). We can imagine that there exists a force density $\vec{f}$ such that 
		\begin{equation}
			\vec{F}_V = \int _{V(t)} \vec{f} \, dV = \int _{V(t)} \rho \vec{a} \, dV
			\label{eq:1.25}
		\end{equation}
		where $\vec{a}$ is a force per unit mass, so an acceleration (gravity : $\vec{f} = \rho \vec{g}$). 
		
			\item If we have an electric material, we can talk about electromagnetic forces, which can be modelled as 
		\begin{equation}
			\vec{f} = \rho _c (\vec{E}+\vec{u}\times \vec{B}) + \vec{J}\times \vec{ B}
		\end{equation}
		where $\rho _c$ is the charge density $[C/m^3]$ and the second term is the Lorentz force. Indeed, if we have a lot of particles, we can talk of an average velocity $\vec{v}_k = \vec{u}+\vec{C}_k$, where $C_k$ is a particular velocity due to molecular agitation.  The force applied on the system is 
		\begin{equation}
			\vec{F}_k = q_k [\vec{E} + \vec{v}_k \times \vec{B}] \qquad \Leftrightarrow \qquad \underbrace{\frac{\sum \vec{F}_k}{V}}_{\rho _c} = \frac{\sum q_k}{V} (\vec{E}+\vec{u}\times \vec{B}) + \underbrace{\frac{\sum q_k \vec{C}_k}{V}}_{\vec{J}} \times \vec{B}
		\end{equation}
		Mollecules are in general neutral, but containing non-neutral regions. Fluids are essentially neutral, $\vec{F}_V = 0$ in most cases. They are called quasi-neutral fluids. Electric influenced fluids will not be considered in that course but they existence has to be known. 
		
			\item They are also entrainment and Coriolis forces in rotating frame of references. These forces due to the rotation of Earth are not considered due to the small rotative velocity, unlike pomps and turbines. \\
		\end{itemize}
		\item[•] \textbf{Contact forces (surface)} $\vec{F}_S$ : \\
		
		\begin{minipage}{0.23\textwidth}
		\includegraphics[scale=0.3]{ch1/5}
		\captionof{figure}{ }
		\label{fig:1.5}
		\end{minipage}
		\begin{minipage}{0.7\textwidth}
		These forces results from the contact of an internal and external fluid in regard of a region of surface $dS(t)$. We have 
		\begin{equation}
			d\vec{F}_S = \vec{T}\, dS \qquad \Rightarrow \vec{F}_S = \oint _S \vec{T}\, dS
			\label{eq:1.28}
		\end{equation}
		$\vec{T}$ is a force per unit area, a continuous function of space depending on location $\vec{x}$ and also linearly on the infinitesimal surface orientation $\vec{n}$. If $\vec{T}$ is the force per unit area for a surface element normal
		\end{minipage}
		 to the unit vector in the $j$ direction $e_j$, $\vec{T}(\vec{x}) = \vec{T}_jn_j$. \eqref{eq:1.28} becomes
		 \begin{equation}
		 \vec{F}_S = \oint _S \vec{T}_j n_j \, dS \qquad and \qquad F_{S_i} = \oint _S \underbrace{\tau _{j,i}}_{\sigma _{ji}} n_j\, dS 
		 \label{eq:1.29}
		 \end{equation}
		 where $\sigma _{ji}$ is the stress tensor. 
		\end{itemize}
		We can now take \eqref{eq:1.18} and replace the terms using \eqref{eq:1.24}, \eqref{eq:1.25} and \eqref{eq:1.29} to obtain the 
		
		\begin{center}
		\theor{\textbf{Momentum equation (integral form)}
		\begin{equation}
			\frac{dP_i}{dt} = \int _{V(t)} \left[\frac{\D \rho u_i}{\D t} + \nabla \rho u_i \vec{u} \right] dV = \int _{V(t)} \rho \dot{u}\, dV = \int _{V(t)} \rho a_i\, dV + \oint _{S(t)} \sigma _{ji} n_j \, dS
			\label{eq:1.30}
		\end{equation}
		}
		\end{center}
		We can see that $\sigma _{ji}$ and $\rho u_iu_j$ have the same mathematical nature. This is not surprising because in fact these forces result from molleculare agitation in fluids. Let's discuss it. We said that $\vec{v}_k = \vec{u} + \vec{C}_k$. Let's consider a surface element and make the hypothesis that the fluid is in rest, so the average velocity $\vec{u} = 0$. It doesn't mean that the particles are immobile, but that if all particles have the same mass (pure fluid) and if a certain number of particles are going from right to left with velocity $\vec{c}$, there are the same number of particles going from left to right with velocity $-\vec{c}$. There is no global mass flux. So for n particles going in one direction, the mass flux
		\begin{equation}
			2nm\vec{u} = nm\vec{c} + 2nm(-\vec{c}) = 0 
		\end{equation}
		To obtain the momentum in direction $x_1$, we have to multiply the mass flow in this direction by the velocity in this direction 
		\begin{equation}
			nm (\vec{c}\cdot \vec{e_1})c_1 + nm (-\vec{c_1}\cdot \vec{e_1})(-c_1) = 2nmc_1^2
		\end{equation}
		The global momentum flux traversing the unit surface is so positive going out of the volume. We need so a balance force in the opposite direction to keep the mass in. This explains the presence and nature of $\sigma _{ji}$ which is a momentum flux. \\
	Let's finally establish the differential form of the momentum equation, applying Gauss theorem to the second right side of \eqref{eq:1.30}
	\begin{equation}
		\int _{V(t)} \rho a_i\, dV + \oint _{S(t)} \sigma _{ji} n_j \, dS = \int _{V(t)} \left[ \rho a_i + \frac{\D \sigma _{ji}}{\D x_j} \right] dV
	\end{equation}
	and by considering the whole equation
	\begin{equation}
		\int _{V(t)} \left[ \frac{\D \rho u_i}{\D t} + \frac{\D \rho u_i u_j}{\D x_j} - \rho a_i - \frac{\D \sigma _{ji}}{\D x_j} \right] dV = 0 = \int _{V(t)} \left[ \rho \dot{u}_i - \rho a_i - \frac{\D \sigma _{ji}}{\D x_j} \right] dV
	\end{equation}
	and for this to be true for all systems we consider, we obtain
	\begin{center}
		\theor{\textbf{Momentum equation (differential form (1) - divergent form)}
		\begin{equation}
			\frac{\D \rho u_i}{\D t} + \frac{\D \rho u_i u_j}{\D x_j} = \rho a_i + \frac{\D \sigma _{ji}}{\D x_j} \qquad and \qquad \frac{\D \rho \vec{u}}{\D t} + \nabla \rho \vec{u}\otimes \vec{u} = \rho \vec{a} + \nabla \bar{\bar{\sigma}}
			\label{eq:1.35}
		\end{equation}
		}
		\end{center}
		
	\begin{center}
		\theor{\textbf{Momentum equation (differential form (2) - substancial form)}
		\begin{equation}
			\rho a_i + \frac{\D \sigma _{ji}}{\D x_j} = \rho \dot{u}_i \qquad and \qquad \rho \vec{a} + \nabla \bar{\bar{\sigma}} = \rho \dot{\vec{u}}
			\label{eq:1.36}
		\end{equation}
		}
		\end{center}
		
	\subsection{Angular momentum equation}
		This is a corollary of the momentum equation that states that \textit{the time rate of change of the angular momentum of a closed system is equal to the sum of the torques applied to the system.} There is no additional information except that the stress tensor should be symetric 
		\begin{equation}
			\sigma _{ji} = \sigma _{ij}
		\end{equation}
		
	\subsection{Energy equation - First principle of thermodynamics}
		If we note $\mathcal{E}$ the total energy of the system, the first principle tells that
		\begin{equation}
			\frac{d\mathcal{E}}{dt} = \dot{W} + \dot{Q}
			\label{eq:1.38}
		\end{equation}
		where $\dot{W}$ is the mechanical power provided by the forces applied on the system and $\dot{Q}$ the thermal power provided to the system. We will proceed like the previous equation expressing first the left side then the right side. If we note $E$ the total energy per unit mass, $e$ the internal energy per unit mass and $k$ the kinetic energy per unit mass (potential energy is not considered in order not to take into account power coming from potential forces)
		\begin{equation}
			\mathcal{E} = \int _{V(t)} E\, dm = \int _{V(t)} \rho E \, dV = \int _{V(t)} \rho (e+k) \, dV \qquad with \quad k = \frac{\vec{u} \vec{u}}{2}
		\end{equation}
		The time derivative of the energy using the Reynolds transport theorem, then the Gauss theorem is 
		\begin{equation}
		\begin{aligned}
			\frac{d\mathcal{E}}{dt} &= \int _{V(t)} \frac{\rho (e+k)}{dt} \, dV + \oint _{S(t)} \rho (e+k)\vec{u} \vec{n} \, dS \\
			&= \int _{V(t)} \left[\frac{\rho (e+k)}{dt} + \nabla \rho (e+k)\vec{u}  \right] dV = \int _{V(t)} \rho \dot{(e+k)}\, dV 
		\end{aligned}
		\label{eq:1.40}
		\end{equation}
		Let's now go on with with the mechanical power expression. We expressed in \eqref{eq:1.30} that there are volume and surface forces. These multiplied by the velocity and using Gauss gives 
		\begin{equation}
			\dot{W} = \int _{V(t)} \rho a_i u_i \, dV + \oint _{S(t)} \sigma _{ji} u_i n_j \, dS = \int _{V(t)} \left[ \rho a_i u_i + \frac{\D}{\D x_j}\sigma _{ji} u_i\right] dV
			\label{eq:1.41}
		\end{equation}
		\begin{wrapfigure}[6]{l}{3cm}
		\vspace{-5mm}
		\includegraphics[scale=0.3]{ch1/6}
		\captionof{figure}{}
		\end{wrapfigure}
		For the thermal power expression, we need to introduce a new concept that is the heat flux vector $\vec{q}$ which qualifies a thermal power per unit area leaving the surface. Physically, there is only two heat transport mecanism which are radiation and conduction. Indeed, convection is a specific conduction case where the temperature gradient region becomes thinner and favorises the exchange. The thermal power is 
		\begin{equation}
			\dot{Q} =  - \oint _{S(t)} \vec{q} \vec{n} \, dS = - \int _{V(t)} \nabla \vec{q} \, dV
			\label{eq:1.42}
		\end{equation}
		Replacing the terms of \eqref{eq:1.38} by \eqref{eq:1.40}, \eqref{eq:1.41} and \eqref{eq:1.42} gives 
		\begin{center}
		\theor{
		\textbf{Total energy equation (integral form)}
		\begin{equation}
			\int _{V(t)} \frac{\rho (e+k)}{dt} \, dV + \oint _{S(t)} \rho (e+k)\vec{u} \vec{n} \, dS = \int _{V(t)} \rho \vec{a}\vec{u} \, dV + \oint _{S(t)} (\bar{\bar{\sigma}} \vec{n}) \vec{u}\, dS - \oint _{S(t)} \vec{q} \vec{n} \, dS
		\end{equation}
		}
		\end{center}
		
		The differantial form is obtained using Gauss theorem for the two sides and regrouping all the terms into one integral
		\begin{equation}
		\begin{aligned}
			&\int _{V(t)} \left[\frac{\rho (e+k)}{dt} + \nabla \rho (e+k)\vec{u} - \rho \vec{a} \vec{u} - \nabla \bar{\bar{\sigma}}\vec{u} + \nabla \vec{q} \right] dV = 0 \\
			\Leftrightarrow \qquad &\int _{V(t)} \left[\rho\dot{(e+k)} - \rho \vec{a} \vec{u} - \nabla \bar{\bar{\sigma}}\vec{u} + \nabla \vec{q} \right] dV = 0
		\end{aligned}
		\end{equation}
		
		And considering the fact that this has to be true for all systems, we obtain the two last forms
		\begin{center}
		\theor{
		\textbf{Total energy equation (differential form (1) - divergent form)}
		\begin{equation}
			\frac{\rho (e+k)}{dt} + \nabla \rho (e+k)\vec{u} = \rho \vec{a} \vec{u} + \nabla \bar{\bar{\sigma}}\vec{u} - \nabla \vec{q}
		\end{equation}
		}
		\end{center}
		\begin{center}
		\theor{
		\textbf{Total energy equation (differential form (2) - substantial form)}
		\begin{equation}
			\rho \dot{(e+k)} = \rho (\dot{e} + \dot{k}) = \rho \vec{a} \vec{u} + \nabla \bar{\bar{\sigma}}\vec{u} - \nabla \vec{q}
			\label{eq:1.46}
		\end{equation}
		}
		\end{center}
		Let's finally establish the distribution of the forces in the different energies. If we multiply \eqref{eq:1.36} by velocity $\vec{u}$ and if we observe that $\dot{k} = \frac{\dot{u}_i u_i+u_i\dot{u}_i}{2} = u_i \dot{u}_i$, we obtain 
		
		\begin{center}
		\theor{
		\textbf{Kinetic - Mechanical energy equation}
		\begin{equation}
			\vec{u}\left(\rho a_i + \frac{\D \sigma _{ji}}{\D x_j} = \rho \dot{u}_i\right) \qquad \Leftrightarrow \qquad \rho \underbrace{u_i \dot{u}_i}_{\dot{k}} = \rho u_i a_i + \frac{u_i \D\sigma _{ji}}{\D x_j}
			\label{eq:1.47}
		\end{equation}
		}
		\end{center}
		The difference between total energy \eqref{eq:1.46} and kinetic energy \eqref{eq:1.47} gives the internal energy
		
		\begin{center}
		\theor{
		\textbf{Internal energy equation}
		\begin{equation}
			\rho \dot{e} = 0 + \sigma _{ji} \frac{\D u_i}{\D x_j} - \nabla \vec{q} 
			\label{eq:1.48}
		\end{equation}
		}
		\end{center}
		We see that volume forces only contributes to the kinetic energy, heat flux only to the internal energy and the suface forces to both. 
		
	\subsection{Summary - Complementary equation}
		Let's make the inventory of the 3 substancial equations that we found. How many equations and unknowns do we have? 
	\begin{itemize}
		\item[•] In continuity equation \eqref{eq:1.17}, $\rho$ and $u_i$ are 4 unknowns in 3D. 
		\item[•] In momentum equation \eqref{eq:1.36}, $a_i$ is an external applied force so is known, $\sigma _{ji}$ consists in 6 unknowns (symetric matrix).
		\item[•] In internal energy equation \eqref{eq:1.48}, $e$ and $\vec{q}$ are 4 most unknowns. 
	\end{itemize}			
	\ \\
	The total unknowns number is 14.  The number of disponible equations is 5, 1 thanks to the energy, 1 thanks to the continuity and 3 thanks to the vectorial momentum equation. In this stage, we haven't made any assumption on the nature of the material we're considering. These equations are valid for an elastic solid as a fluid. The main difference is that solids resist to a deformation whereas fluid doesn't. But fluid resists to a rate of deformation. The way that stress tensor $\sigma _{ji}$ is related to the displacement field is called the constitutuve equations. 
	
	\subsubsection{Constitutive relations} 
		For a fluid, the stress tensor depends on the fluid rate of deformation (rate of strain). To express $\sigma _{ji}$, we have to find a quantity in the field of motion of the fluid that represents the rate of strain. If the velocity field $\vec{u}(\vec{x},t)$ was uniform, not depending on $\vec{x}$, the fluid will be moving as a bulk and there is no rate of deformation. The rate of strain must be somehow related to the velocity gradient tensor $\nabla \otimes \vec{u}$. We know that all tensors can be decomposed in an antisymetric and symetric part like 
		\begin{equation}
			\nabla \otimes \vec{u} = \frac{\D u_j}{\D x_i} = \Omega _{ji} + S _{ij} = \frac{1}{2} \left( \frac{\D u_j}{\D x_i} - \frac{\D u_i}{\D x_j} \right) + \frac{1}{2} \left( \frac{\D u_j}{\D x_i} + \frac{\D u_i}{\D x_j} \right).
		\end{equation}
		
		For a constant gradient velocity field, the velocity field is linear in the coordinates
		\begin{equation}
			u_j = \frac{\D u_j}{\D x_i}x_i = \Omega _{ji} x_i + S _{ij} x_i
			\label{eq:1.51}
		\end{equation}
		
		Let's look to the mathematical nature of the antisymetric part.  If we express using Kronecker $\delta$, we have
		\begin{equation}
			\Omega _{ji} =\frac{1}{2} \left( \frac{\D u_j}{\D x_i} - \frac{\D u_i}{\D x_j} \right) = \frac{1}{2} \delta _{kij}\delta _{kqp} \frac{\D u_p}{\D x_q} \qquad 
			with \quad \delta _{kij}\delta _{kqp} = \delta _{iq}\delta _{jp} - \delta _{ip}\delta _{jq} 
		\end{equation}
		Knowing that $(a \times b)_k = \delta _{kpq} a_p b_q$, we can introduce the \textbf{curle} (rotationnel) of velocity called vorticity $\vec{\omega}$
		\begin{equation}
			\delta _{kqp} \frac{\D u_p}{\D x_q} = (\nabla \times \vec{u})_k \qquad \Rightarrow \nabla \times \vec{u} = \vec{\omega}
		\end{equation}		 
		Let's look to the way this is linked to \eqref{eq:1.51}
		\begin{equation}
			 u_j^{AS} = \Omega _{ji}x_i = \frac{1}{2} \delta _{jki} \omega _k x_i = \frac{1}{2} (\vec{\omega} \times \vec{x})_j
			\qquad \Leftrightarrow \qquad 
			\vec{u}^{AS} = \frac{1}{2} \vec{\omega}\times \vec{x}
		\end{equation}
	 	In conclusion, we see that the antisymetric part consists in a pure rotation velocity field, a rigid body motion of angular velocity $\frac{1}{2}\vec{\omega}$ without strain. $\vec{\omega}$ is twice the angular velocity of fluid particles around themselves. The quantity representative of the fluid rate of strain can only be the symetric part of the velocity gradient tensor called the rate of strain tensor. For a fluid, $\sigma _{ij} = f(S_{ij})$. \\
	 	
	  To determine the nature of this relationship, we will assume that $\sigma _{ij} $ is a linear function of $S_{pq}$. This is called 
	\begin{center}	  
	\theor{
	  \textbf{Newton's assumption for stresses}
	  \begin{equation}
	  	\sigma _{ij} = a_{ij} + b_{ijpq} S_{pq}.
	  \end{equation}
	  }
	  \end{center}
	  In this equation, $b_{ijpq}$ is a tensor with four indices, but we know that it's symetric with respect to $pq$ and $ij$ because $S$ is symetric with respect to $pq$ and $ij$, leading to $6 \times 6 = 36$ coefficients. Symetric tensor $a_{ij}$ counts 6 coefficients, for a total of 42 coefficients. \\
	  If we assume that the fluid is \textbf{isotropic}, meaning that the fluid react in the same way whatever the sollicitation direction. For example, let's take a case of $S_{ij}$ where all coefficients are null except the $S_{11}$ term. Diagonal terms reprensent a rate of elongation/stretch while the off-diagonal terms represent an angular deformation between two perpendicular direction. The assumption means that if the rate of stress is not in 1 direction but 2, the fluid reaction will be the same. In other words, if we make a rotation of coordinates, the relation in the rotated frame of reference must be the same. In that case, the relation reduces to
	  \begin{equation}
	  	\sigma _{ij} = a \delta _{ij} + bS _{ij} + c \delta _{ij}S_{kk}
	  	\label{eq:1.55}
	  \end{equation}
	  where only 3 coefficient must be found. It is natural to think that air and water have no preferential direction unlike certain solid as wood that has a preferential direction related to the orientation of fibers. Blood or dissolved polymer chains are examples of non isotropic fluids. We will from now consider the fluid to be isotropic. \\
In \eqref{eq:1.55} $a$ is a constant that represents the stress present when the fluid is at rest. 	The surface force associated to that component is purely normal 
		\begin{equation}
			\sigma _{ij} n_j = a \delta _{ij} n_j = an_i
		\end{equation}
		This constant corresponds to the pressure exerted by the fluid at rest. Because of its application in the opposite direction to the normal, it's negative. The two other coefficients represents the 2 coefficients of viscosity
		\begin{equation}
			a = -p \qquad b = 2\mu \qquad c = \lambda
		\end{equation}
		The stress tensor equation can so be written with a pressure stress and a viscous stress part like 
		\begin{equation}
			\sigma _{ij} = -p\delta _{ij} + \tau _{ij} \qquad with \qquad \tau _{ij} = 2\mu S_{ij} + \lambda \delta _{ij} S_{kk}
			\label{eq:1.58}
		\end{equation}
	  	An alternative form to that is the following 
	  	\begin{equation}
	  		\tau _{ij} = 2 \mu \underbrace{\left( S_{ij} + \frac{1}{3} \delta _{ij} S_{kk} \right)}_{\equiv S_{ij}^S} +\underbrace{\left( \lambda + \frac{2\mu}{3} \right)}_{\equiv \mu _V} \delta _{ij} S_{kk}
	  	\end{equation}
	  	This notation is necessary to make appear the part of the strain tensor which has no trace $S_{ij}^S$, called the rate of shear. Indeed 
	  	\begin{equation}
	  		S_{ii}^S = S_{ii} - \frac{1}{3}\delta _{ii}S_{kk} = 0
	  	\end{equation}
	  	This means that $S_{ij}^S$ represents the trace less part of the rate of \textbf{strain} tensor called the sheer rate tensor. What is now $S_{kk}$ ? 
	  	\begin{equation}
	  		S_{kk} = \frac{1}{2} \left( \frac{\D u_k}{\D x_k} + \frac{\D u_k}{\D x_k} \right) = \frac{\D u_k}{\D x_k} = \nabla \vec{u}
	  	\end{equation}
	  	The divergence of the velocity is related to the rate of dilatation of the fluid, the change of volume. We decomposed the rate of strain in a part representing the deformation without change of volume (pure deformation) and another with change of volume ($\mu _V$ is the bulk viscosity). Another expression for $\tau _{ij}$ with a final net gain of 3 unknowns is 
	  	\begin{equation}
	  		\tau _{ij} = 2 \mu S_{ij}^S + \mu _V \delta _{ij} \nabla \vec{u}
	  		\label{eq:1.62}
		\end{equation}	  	 
		At this stage, we have to determine still 6 unknowns from the 9 at the beginning. 
		
	\subsubsection{Heat flux}
		We discussed about the fact that heat flux propagates using 2 physical mechanism : conduction and radiation. In the energy equation it's the divergence of the heat fluc that appears. In most application, the radiative effect does not imply heat accumulation or loss. The fluids are so transparent to radiative heat flux $\nabla \vec{q}^{rad} =0$. We only have conduction and the Fourier law says that 
		\begin{equation}
			\vec{q} \propto \nabla T = d\nabla T = -\kappa \nabla T \qquad \Leftrightarrow \qquad q_i = -\kappa \frac{\D T}{\D x_i}
		\end{equation}
	The negative sign comes from the fact that heat goes from hot to cold (decrease of T). We have a net gain of 1 unknown with this equation. 
	
	\subsubsection{Thermodynamics}
		At this stage, we are using 4 thermodynamics intensive variables which are $\rho, e, p, T$. We know that for a single phase fluid, the variance is 2, meaning that we can use 2 thermodynamics equations of state (EoS) relating them. For example, for a alorically and thermally perfect gas, we will have 
		\begin{equation}
			p = \rho R T \qquad and \qquad e = c_v T
		\end{equation}		 
		We have a net gain of 2 unknows, so there remains 2 unknowns.  
		
	\subsubsection{Transport coefficients}
		The remaining variables are the shear viscosity $\mu$, the bulk viscosity $\mu _V$ and thermal conductivity $\kappa$. These are functions of the thermodynamic state. For example for gases we have the relations 
		\begin{equation}
			\mu = f(T) \qquad and \qquad Pr = \frac{\mu c_p}{\kappa} = cst \Leftrightarrow \kappa = \frac{\mu (T) c_p (T)}{Pr}
		\end{equation}
		The bulk viscosity is more difficult to determine, but it can be shown that for monoatomic gases (no internal degres of freedom) $\mu _V=0$. For diatomic gases it's much more delcate to measure, but it has been shown that for many flows, the flows is insensitive to the variation of value of bulk viscosity. In fact, for fluids without divergence of velocity, we don't care about $\mu _V$ because there is no variation of volume  \eqref{eq:1.62}. We will so make the following assumption
		\begin{center}
		\theor{\textbf{Stokes assumption}
		\begin{equation}
			\mu _V = 0 \quad \mbox{even for other gases}.
		\end{equation}
		}
		\end{center}
		 We're done, we have as many equations as variables. We mentioned the first principle of thermodynamics but not the second. Let's analyse that. 
		 
		\subsubsection{Second principle of thermodynamics}
			We will reuse the internal energy equation \eqref{eq:1.48}, replace $\sigma _{ij}$ by it's expression in \eqref{eq:1.58} and use the fact that $\frac{\D u_i}{\D x_j} = \Omega _{ij} + S_{ij}$
			\begin{equation}
				\rho \dot{e} = \sigma _{ij} \frac{\D u_i}{\D x_j} - \frac{\D q_i}{\D x_i} = -p \delta _{ij} \frac{\D u_i}{\D x_j} + (2\mu S_{ij}^S + \mu _V \delta _{ij} \nabla \vec{u})(S_{ij}+\cancel{\Omega _{ij}}) - \frac{\D q_i}{\D x_i}
			\end{equation}
			where $\Omega _{ij}$ doesn't contribute because the contraction of the symetric tensor by the antisymetric tensor is equal to 0.  We have the relation 
			\begin{equation}
			S_{ij}^S = S_{ij}-\frac{1}{3}\delta _{ij}\nabla \vec{u} \qquad \Leftrightarrow \qquad S_{ij} = S_{ij}^S + \frac{1}{3}\delta _{ij}\nabla \vec{u}
	  		\end{equation}
	  		Combined to the fact that $\delta _{ij} \frac{\D u_i}{\D x_j} = \frac{\D u_i}{\D x_i} = \nabla \vec{u}$, we obtain
	  		\begin{equation}
	  			\rho \dot{e} = -p \nabla \vec{u} + (2\mu S_{ij}^S + \mu _V \delta _{ij} \nabla \vec{u})\left( S_{ij}^S + \frac{1}{3}\delta _{ij}\nabla \vec{u}\right) - \nabla \vec{q}
	  		\end{equation}
	  		The mass conservation equation tells us that we can write the divergence as 
	  		\begin{equation}
	  			\dot{\rho} + \rho \nabla \vec{u} = 0 \qquad \Leftrightarrow \qquad \nabla \vec{u} = -\frac{\dot{\rho}}{\rho} = -\rho \left( \frac{\dot{\rho}}{\rho ^2}\right) = \rho \dot{\left( \frac{1}{\rho}\right)} = \rho \dot{v}
	  			\label{eq:1.70}
	  		\end{equation}
	  		If we replace the divergence in the previous equation, we have
	  		\begin{equation}
	  			\rho \dot{e} = -\rho p \dot{v} + (2\mu S_{ij}^S + \mu _V \delta _{ij} \nabla \vec{u})\left(S_{ij}^S + \frac{1}{3}\delta _{ij}\nabla \vec{u}\right) - \nabla \vec{q}
	  		\end{equation}
	  		The first term here looks like the reversible work $-pdv$ in thermodynamics and is so the reversible contribution to the internal energy. Let's make this appear by bringing this to the left side. We make appear $\rho [\dot{e} + p\dot{v}]$, but we have the famous Gibbs relation $de = Tds - pdv \Leftrightarrow \dot{e} = T\dot{s}-p\dot{v}$. We have now 
	  		\begin{equation}
	  			\rho \dot{s} = \frac{(2\mu S_{ij}^S + \mu _V \delta _{ij} \nabla \vec{u})\left(S_{ij}^S + \frac{1}{3}\delta _{ij}\nabla \vec{u}\right)}{T} - \frac{\nabla \vec{q}}{T}
	  			\label{eq:1.72}
	  		\end{equation}
	  		If we remind the relation we demonstrated before for any variable $\dot{\Phi} = \int _V \rho \dot{\phi} dV$, we can make the analogy here to say that when this is integrated over a volume, it gives the time rate of change of the entropy of the closed system that's initially inside this volume. We have to identify the reversible part in this equation. We know that the reversible entropy rate of exchange for a uniform system and its integral over a closed surface is given by
	  		\begin{equation}
	  			\frac{\vec{q}dS}{T} \qquad \Rightarrow \qquad \oint _S \frac{\vec{q}}{T}(-\vec{n})\, dS = -\int _V \nabla \frac{\vec{q}}{T} \, dV.
			\end{equation}	  		 
			We see that we have to make appear a this in the last equation. But we know that
			\begin{equation}
				\frac{\nabla \vec{q}}{T} = \nabla \frac{\vec{q}}{T} - \vec{q} \nabla \left(\frac{1}{T} \right) = \nabla \frac{\vec{q}}{T} + \vec{q} \frac{\nabla T}{T^2}
			\end{equation}
			And by introducing this into the relation \eqref{eq:1.72}, we make appear the reversible entropy rate of exchange
			\begin{equation}
				\rho \dot{s} = -\nabla \frac{\vec{q}}{T} - \frac{\vec{q}\nabla T}{T^2} + \frac{(2\mu S_{ij}^S + \mu _V \delta _{ij} \nabla \vec{u})\left(S_{ij}^S + \frac{1}{3}\delta _{ij}\nabla \vec{u}\right)}{T}
			\end{equation}
			We also know that $\vec{q} = -\kappa \nabla T$, making appear $(\nabla T)^2$
			\begin{equation}
				\rho \dot{s} = -\nabla \frac{\vec{q}}{T} + \frac{\nabla T\nabla T}{T^2} + \frac{(2\mu S_{ij}^S + \mu _V \delta _{ij} \nabla \vec{u})\left(S_{ij}^S + \frac{1}{3}\delta _{ij}\nabla \vec{u}\right)}{T}
			\end{equation}
			If we imagine a fluid at rest with only a heat exchange operating on it, the third term $=0$, the first term is reversible so anyway the sign and the second term must be positive. This implies $\kappa \geq 0$ due to the square of the other varaibles (the heat has to go from hot to cold). Let's expand the third term
			\begin{equation}
				\rho \dot{s} = -\nabla \frac{\vec{q}}{T} + \frac{\nabla T\nabla T}{T^2} + \frac{1}{T}\left[ 2\mu S_{ij}^S S_{ij}^S + \cancel{\mu _V \nabla \vec{u} \delta _{ij} S_{ij}^S} + \cancel{2\mu S_{ij}^S \frac{\delta _{ij}}{3} \nabla \vec{u}} + \mu _V \frac{\delta _{ij}\delta _{ij}}{3}(\nabla \vec{u})^2  \right]			
			\end{equation}			 
			In this last equation, the second and third terms are nul because $S_{ii}^S=0$. Let's imagine that we have a fluid with only dilation and no shear $S_{ij}^S$, the last term must be positive and so $\mu _V$ has to be positive ($\geq 0$). In the other hand, for the first term, we have a quadratic form (sum of squares $\geq 0$), so $\mu$ has to be positive. To verify the second principle, we have to verify these 3 inequalities. In fluid mechanics, we don't have to worry about the second principle, it's built in the equations as long as the transport coefficient are positive.  
			
	\subsection{Boundary conditions}
		We have now to establish the boundary conditions which makes the difference between the flow cases. First of all, we have two main categories of flows : 
		\begin{itemize}
			\item[•] \textbf{External flows} (unbounded domain)\\
				For example, a flow over a wing, assuming that atmosphere extends to infinity. In that case we have far field boundary conditions, what happens far from the body ($u\rightarrow u_\infty, p\rightarrow p_\infty, T\rightarrow T_\infty)$. 
			
			\item[•] \textbf{Internal flows} (bounded domain)\\
				For example, a flow in a pipe or a fluid in a rotating machine like a pump. In that case we don't have the far field conditions but the inlet and outlet boundary conditions but this problem is not discussed here. 
		\end{itemize}
		
		\subsubsection{Solid surfaces}
			In both case we have solid surfaces, we have to make a distinction. We wrote the equation for the general case of a viscous flow, but there is flows where the viscous stresses can be neglected (not $=0$ !) leading to what we call the \textbf{inviscid flows}. Let's analyse the two cases. 
			
		\subsubsection{Viscous flows}
			Viiscosity is associated to the exchange of momentum between neighboring fluid layers due to molecular agitation. If we have a molecule coming from a low velocity region to a high velocity region, it slows down the molecule there and inversely. The same occurs when a fluid particle enter in contact with solid surfaces, it exchange momentum. The result is that velocity and temperature fields must be continuous 
			\begin{equation}
				\vec{u}_{fluid} = \vec{u}_{wall} \qquad and \qquad T_{fluid} = T_{wall}
			\end{equation}
			In particular, for a surface at rest, the fluid must be at rest on the solid surface as well. This is called the \textbf{no-slip condition}. 
		
		\subsubsection{Inviscid flows} 
			For inviscid flows, this mecanism doesn't exist, the fluid may slip. The boundary condition is that the fluid can't go throw the solid
			\begin{equation}
				\vec{u}_{fluid}\vec{n} = \vec{u}_{wall}\vec{n}
			\end{equation}
			This is called the \textbf{slip/no penetration condition}. The previous condition is stronger because in fact $\vec{u} = \vec{u}_n\vec{n}+\vec{u}_t$ includes the tangential condition too. 
			
\section{Special cases}
	\subsection{General case}
		The generale equations are the following : 
		\begin{itemize}
			\item[•] Mass conservation equation 
			\begin{equation}
				\dot{\rho} + \rho \nabla \vec{u} = 0 
			\end{equation}
			
			\item[•] Momentum equation
			\begin{equation}
				\rho \dot{\vec{u}} = - \nabla p + \nabla \bar{\bar{\tau}} + \rho \vec{F}
			\end{equation}
			
			\item[•] Energy equation
			\begin{equation}
				\rho \dot{e} = - p \nabla \vec{u} + \underbrace{\bar{\bar{u}} .. \nabla \otimes \vec{u}}_{\epsilon _V} - \nabla \vec{q}
			\end{equation}
			where $\nabla \otimes \vec{u}$ can be replace by the symetric part, rate of strain tensor $\bar{\bar{S}}$ and $\epsilon _V$ is the viscous dissipation. \\
			
			\item[•] Constituve relation
			\begin{equation}
			\begin{aligned}
				\tau _{ij} &= 2\mu \left( S _{ij} - \frac{1}{3}\delta_{ij}\nabla \vec{u} \right) + \mu _V S_{ij}\nabla \vec{u}\\
				&= \mu \left( \frac{\D u_i}{\D x_j} + \frac{\D u_j}{\D x_i} - \frac{2}{3}\delta _{ij}\nabla \vec{u} \right) + \mu _V S_{ij}\nabla \vec{u}
			\end{aligned}
			\label{eq:1.83}
			\end{equation}
			
			\item[•] Conductive heat flux 
			\begin{equation}
				\vec{q} = -\kappa \nabla T
			\end{equation}
		\end{itemize}
		
	\subsection{Steady flow} 
		This is caracterized by the fact that $\frac{\D }{\D t} = 0$ and implies for example that 
		\begin{equation}
			\dot{\rho} = \frac{\D \rho}{\D t} + \vec{u} \nabla \rho = \vec{u} \nabla \rho
		\end{equation}
		and for the others. 
		
	\subsection{Inviscid flows}
		They are defined as flows in which vicous stresses and conduction heat flux can be neglected.  We are talking about flows and not fluids because there is no fluid with $\mu = 0$ or $\kappa = 0$. This happens for superfluids but we don't care. When we look at the fluid properties tables, in SI units, water and air have very small $\mu$ but we can't say that there are negligible because it depends on the system of refference used. If something is negligible  it is with respect to something else. Let's start with the viscous stresses. Momentum equation can be written as 
		\begin{equation}
			\rho \dot{\vec{u}} = \frac{\D \rho \vec{u}}{\D t} + \nabla \rho \vec{u} \otimes \vec{u} = -\nabla p + \nabla \bar{\bar{\tau}} + \rho \vec{F}
		\end{equation}
		where the viscous stress tensor is a tensor as the momentum flux tensor. They correspond to the same physical phenomenon but at different scales, the viscous stress tensor is due to the molecular agitation whereas the momentum flux tensor is for the macroscopic scale, the average scale. So it makes sense to compare the order of magnitude of the two ones. \\
		
		\begin{wrapfigure}[6]{l}{4cm}
		\vspace{-8mm}
		\includegraphics[scale=0.38]{ch1/7}
		\captionof{figure}{}
		\end{wrapfigure}
		Let's consider a fluid flows of far field velocity $\vec{U}$ around a solid body of characteristic lenght $L$, if we consider the momentum flux tensor, we know that the velocity around the body will vary between 0 and $U$ so the order of magnitude will be $\theta (\rho U^2)$. What about $\tau$? We see that in \eqref{eq:1.83} appears the velocity gradient, derivative. What is the order of magnitude of the derivative of a function?  
		
		\begin{wrapfigure}[8]{l}{3.5cm}
		\vspace{-5mm}
		\includegraphics[scale=0.5]{ch1/8}
		\captionof{figure}{}
		\label{fig:1.8}
		\end{wrapfigure}
		Let's consider a function $f(x)$ represented on \autoref{fig:1.8}. If the function is smooth, so if the function doesn't vary much in the interval, its derivative keeps a constant order of magnitude in the integral. We see it in the figure, the slope varies between $a$ and $b$, can be twice the slope at the center but keeps the same order of magnitude. So for a smooth function, the order of magnitude of $f'$ remains the same over the interval $f' = \theta \left(\frac{\Delta f}{\Delta x}\right)$. Let's use this to have an approximation for the velocity gradient tensor 
		\begin{equation}
			\nabla \otimes \vec{u} = \theta \left( \frac{U}{L} \right) \qquad \Rightarrow \bar{\bar{\tau}} = \theta \left( \mu \frac{U}{L} \right)
		\end{equation}
		The relative order of magnitude of viscous stresses with respect to momentum flow is 
		\begin{equation}
			\frac{\mu \frac{\cancel{U}}{L}}{\rho U^{\cancel{2}}} = \frac{\mu}{\rho UL} = \frac{1}{Re_L}
		\end{equation}
		We conclude that viscous stresses can be neglected in the case of high Reynolds number. \\
		
		\begin{wrapfigure}[7]{r}{3.5cm}
		\vspace{-8mm}
		\includegraphics[scale=0.53]{ch1/9}
		\captionof{figure}{}
		\label{fig:1.9}
		\end{wrapfigure}
		Now we have to verify the assumption that velocity is a smooth function of the coordinates. Examples of not smooth functions are represented on \autoref{fig:1.9} where the green curve is smooth close to the limits but not smooth in a small interval and the yellow one is a periodic function with the characteristic wave length. In these cases, $\Delta x$ is not the appropriate length scale to determine the order of magnitude of the derivative. \\\\\\

		\begin{wrapfigure}[8]{l}{4cm}
		\vspace{-5mm}
		\includegraphics[scale=0.35]{ch1/10}
		\captionof{figure}{}
		\label{fig:1.10}
		\end{wrapfigure}
		Now what about the velocity field? Let's assume that velocity is smooth and the fluid inviscid, the boundary conditions to respect are the slip conditions. But we know that $\mu$ is not strictly 0, so the velocity must be 0 at the wall and therefore, there must exist a region of another caracteristic scale $\delta$ of rapidly changing velocity close to the wall. In this case the study of order of magnitude made is incorrect, $U/L$ must be replaced by $U/\delta$. Due to the smaller scale of $\delta$ compared to L, the viscous stress is more important than outside this region and can be of comparable size to the momentum flux tensor.
		 We conclude that the flow can be decomposed into two regions : a region outside of this viscous layer, a distal region where the viscous stresses and heat conduction term can be neglected and a proximal or inner region where viscous stresses may not be neglected. We complete the definition with high Reynolds number by adding "\textbf{except close to solid bodies and in their wake}\footnote{Sillage : refermement de la couche visqueuse à droite de la \autoref{fig:1.10}.}".
		 
	\subsection{Inviscid flows equations}
		They are the same as the general case, except that the viscous and heat flux terms are neglected. 
		\begin{itemize}
			\item[•] Mass conservation equation 
			\begin{equation}
				\dot{\rho} + \rho \nabla \vec{u} = 0 
			\end{equation}
			
			\item[•] Momentum equation
			\begin{equation}
				\rho \dot{\vec{u}} = - \nabla p + \cancel{\nabla \bar{\bar{\tau}}} + \rho \vec{F}
			\end{equation}
			
			\item[•] Energy equation
			\begin{equation}
				\rho \dot{e} = - p \nabla \vec{u} + \cancel{\bar{\bar{u}} .. \nabla \otimes \vec{u}} - \cancel{\nabla \vec{q}} \qquad \Leftrightarrow \qquad \rho \dot{e} + p\nabla \vec{u} = 0
			\end{equation}
			We already analyzed this expression before, with \eqref{eq:1.70} we can conclude that 
			\begin{equation}
				\rho (\dot{e}+ p\dot{v}) = 0 = \rho T\dot{s} \qquad \Rightarrow \dot{s} = 0
			\end{equation}
			Entropy per unit mass is constant along trajectories. Only viscous term and heat flux are responsible of irreversible entropy variations. So all the particles keeps constant entropy and if the incoming fluid particles are uniform it means that the entropy will be constant across the whole flow.  \\
			\end{itemize}
			
			Let's specify the terminology, when we speek about uniform quantity it means that $\nabla q = 0$ and steady q means that it doesn't vary with time $\frac{\D q}{\D t}= 0$. Let's now see what happens with the momentum equation
			\begin{equation}
			\begin{aligned}
				\rho \dot{\vec{u}} &= - \nabla p + \rho \vec{F} = \rho \left[\frac{\D u}{\D t} + (\vec{u} \nabla)\vec{u} \right]\\
				\rho \dot{u}_i &= - \frac{\D p}{\D x_i} + \rho F_i = \rho \left[ \frac{\D u_i}{\D t} + u_j \frac{\D u_i}{\D u_j} \right]
			\end{aligned}
			\label{eq:1.93}
			\end{equation}
			But if we say that, by adding and removing the needed term
			\begin{equation}
				u_j \frac{\D u_i}{\D x_j} = u_j \left( \frac{\D u_i}{\D x_j} - \frac{\D u_j}{\D x_i}  \right) + u_j \frac{\D u_j}{\D x_i}
			\end{equation}
			But what's the last term? We know that the gradient of kinetic energy corresponds to that because
			\begin{equation}
				 u_j \frac{\D u_j}{\D x_i} = \frac{\D \frac{u_j u_j}{2}}{\D x_i} = \frac{\D k}{\D x_i} \qquad \mbox{additionally} \qquad \frac{\D u_i}{\D x_j} - \frac{\D u_j}{\D x_i} = \delta _{kji} \omega _k
			\end{equation}
			Now if we replace this in \eqref{eq:1.93}, we find the
			
			\begin{center}
			\theor{\textbf{Lamb's form of the momentum equation}
			\begin{equation}
			\begin{aligned}
				&- \frac{\D p}{\D x_i} + \rho F_i = \rho \left[ \frac{\D u_i}{\D t} + \delta _{kji}\omega _k u_j +\frac{\D k}{\D x_i} \right]\\
				\Leftrightarrow & - \frac{\D p}{\D x_i} + \rho F_i = \rho \left[ \frac{\D u_i}{\D t} + (\vec{\omega}\times \vec{u})_i +\frac{\D k}{\D x_i} \right]\\
			\Leftrightarrow	&- \nabla p + \rho \vec{F} = \rho \left[ \frac{\D \vec{u}}{\D t} + \vec{\omega} \times \vec{u} +\nabla k \right]
			\end{aligned}
			\end{equation}}
			\end{center}
			
	\subsection{Barotropic flows - Force deriving from a potential}
		Like the previous one, there are barotropic flows but no barotropic fluids. Let's rewrite the Lamb's equation 
		\begin{equation}
			\frac{\D \vec{u}}{\D t} + \vec{\omega} \times \vec{u} +\nabla k = - \frac{\nabla p}{\rho} + \vec{F} \qquad with \vec{F} = \vec{a}
			\label{eq:1.97}
		\end{equation}
		In general we know that the thermodynamic state of a pure fluid in single phase is determined by 2 thermodynamic variables. This means that in general, $p$ and $\rho$ are independant variables. But when another thermodynamic variable is constant, uniform, then it exists a relation between $p$ and $\rho$ and hence it exists a certain function $P(p)$ such that $\frac{dP}{dp} = \frac{1}{\rho (p)}$. This implies that the gradient
		\begin{equation}
			\nabla P = \frac{dP}{dp} \nabla p = \frac{\nabla p}{\rho}
		\end{equation}
		allowing us to replace this quantity in \eqref{eq:1.97}. Two examples of constant variables : \\
		\begin{itemize}
			\item[•] \textbf{Constant density flows} : $\rho(p) = \rho = cst$ and so $\frac{dP}{dp}=\frac{1}{\rho}\Rightarrow P = \frac{p}{\rho}$.
			\item[•] \textbf{Isentropic flows} : we have the Gibbs relation $dh = Tds + vdp$ simplifying in $dh = \frac{dp}{\rho}$ and so $P = h$. \\
		\end{itemize}
		
		If now in addition to the barotropic flow assumption we assume that $\vec{F}$ derives from a potential, we will make  other assumptions. This means that the curl $\nabla \times \vec{F} =0$, allowing to write $\vec{F}$ as the gradient of a certain potential energy per unit mass $\vec{F} = - \nabla \Phi$. Then we can write \eqref{eq:1.97} as
		\begin{equation}
			\frac{\D \vec{u}}{\D t} + \vec{\omega} \times \vec{u} = -\nabla (P+k +\Phi)
		\end{equation}
		Making the additional assumption that we have a steady flow and multiplying by $\vec{u}$ the two members leads to 
		\begin{center}
		\theor{\textbf{Bernouilli's equation (1)}
		\begin{equation}
		\vec{u} (\vec{\omega \times \vec{u}}) = -\vec{u} \nabla (P+k+\Phi) \qquad \Leftrightarrow \qquad P+k+\Phi = e_m
		\end{equation}
		telling that mecanical energy is constant along streamlines (but can vary between streamlines).}
		\end{center}
		
	\subsection{Irrotational flows}
		This points to flows such that $\vec{\omega} = 0$. In that case we see that 
		\begin{equation}
			-\nabla (P+k+\Phi) = 0 \qquad \Rightarrow P+k+\Phi = cst 
		\end{equation}
		Everywhere in the domain! When the flow is irrotational $\vec{\omega} = \nabla \times \vec{u} = 0$, $\vec{u}$ can be experssed as a velocity potential $\nabla \phi$. Now if we consider an unsteady, barotropic, irrotational inviscid flows with irrotational body forces 
		\begin{equation}
			\frac{\D \vec{u}}{\D t} = \frac{\D \nabla \phi}{\D t} = \nabla \frac{\D \phi }{\D t} = -\nabla (P+k+\Phi) \qquad \Leftrightarrow \qquad \frac{\D \phi}{\D t} + P + k +\phi = cst 
		\end{equation}
		Everywhere in the domain. 
		
	\subsection{Incompressible/quasi incompressible flows}
		Let's consider an inviscid steady flow in absence of body forces (and uniform $S$ due to inlet conditions). Then the momentum equation \eqref{eq:1.93} tells that
		\begin{equation}
			\rho (\vec{u}\nabla )\vec{u} = -\nabla p  = - \left.\frac{dp}{d\rho}\right|_{S} \nabla \rho 
		\end{equation}
		where $\frac{dp}{d\rho}|_S = a^2$, with $a$ the speed of sound. 
		
		\begin{wrapfigure}[4]{l}{2cm}
		\vspace{-5mm}
		\includegraphics[scale=0.4]{ch1/11}
		\captionof{figure}{}
		\end{wrapfigure}
		If $\vec{e}_s$ is the unit vector along the following streamline. We have that 
		\begin{equation}
		\begin{aligned}
			\vec{u}= u \vec{e}_s 
			\qquad &\Rightarrow (\vec{u}\nabla)\vec{u} = u \frac{d}{ds}(u\vec{e}_s) = -\frac{a^2}{\rho} \nabla \rho = -\frac{a^2}{\rho}\left[ \frac{d\rho}{ds}\vec{e}_s + \frac{d\rho}{dn} \vec{e}_n \right]\\
			&\Leftrightarrow u\frac{du}{ds}\vec{e}_s + u^2\frac{d\vec{e}_s}{ds} =  -\frac{a^2}{\rho}\left[ \frac{d\rho}{ds}\vec{e}_s + \frac{d\rho}{dn} \vec{e}_n \right]
			\end{aligned}
		\end{equation}
		We will not discuss the derivative of the unit vector, so if we look to the streamline's direction component, we have
		\begin{equation}
			u\frac{du}{ds} = -\frac{a^2}{\rho} \frac{d\rho}{ds} \quad \Leftrightarrow \quad \frac{u^2}{a^2}\frac{du}{u} = - \frac{d\rho}{\rho} \qquad \Rightarrow \frac{d\rho}{\rho} = -M^2 \frac{du}{u}
		\end{equation}
		where $M = \frac{u}{a}$ is the \textbf{Mach number} and compares the local velocity to the speed of sound. The conclusion is that when M is small, density variations are small. We can so make the approximation of constant density. Even for compressible fluids (like gases), in the conditions provided by the assumptions, density almost does not vary as long as the Mach number is much smaller than 1 (smaller than 0.3 in practice). \\
		A first counter example is the sound waves because they create unsteady flows and natural convection where the density variation is due to temperature variation (we only considered pressure variation here). In these cases density variation is not negligible even for small Mach number. 
		
	\subsection{Two-dimensional (planar) flows}
	
		\begin{wrapfigure}[7]{r}{2.8cm}
		\vspace{-5mm}
		\includegraphics[scale=0.4]{ch1/12}
		\captionof{figure}{}
		\label{fig:1.12}
		\end{wrapfigure}
		They are essentially flows over cylindrical geometries. A general cylilnder is a body made of straightlines parallel to each other and wich lie upon a two dimensional curl. When we take a surface and draw infinite straightlines, there is no reason for the solution to vary in the infinite direction, all the derivative are 0. This means, according to \autoref{fig:1.12}, that $\frac{\D}{\D x_3}=0$. Moreover, in general there is no velocity component in this direction but it isn't necessary $u_3 = 0$.\\
		
		In that case, for \textbf{steady} flows, the continuity equation becomes 
		\begin{equation}
			\dot{\rho} + \rho \nabla \vec{u} = \cancel{\frac{\D \rho}{\D t}} + \nabla \rho \vec{u} = 0 \qquad \Leftrightarrow \frac{\D (\rho u_1)}{\D x_1} + \frac{\D (\rho u_2)}{\D x_2} = 0
		\end{equation}
		This equation can be made satisfied by introducing an auxiliary funciton $\psi$ (called streamfunction), such that  
		\begin{equation}
		\left\{
		\begin{aligned}
			\rho u_1 &= &\rho _0 \frac{\D\psi}{\D x_2}\\
			\rho u_2 &= &-\rho _0 \frac{\D\psi}{\D x_1}
		\end{aligned}
		\right.
		\qquad \Rightarrow \qquad \underbrace{\frac{\D \rho u_1}{\D x_1}}_{\rho _0 \frac{\D ^2 \psi}{\D x_1 \D x_2}} + \underbrace{\frac{\partial ^2 \rho u_2}{\D x_2}}_{-\rho _0 \frac{\D ^2 \psi}{\D x_2 \D x_1}} = 0
		\label{eq:1.107}
		\end{equation}
		We see that continuity equation is satisfied for this function.  We can replace the two velocity variables by this function and reduce the unknowns from 2 to 1. 
		
		\newpage
		\subsubsection{Physical meaning of the streamfunction}
			\begin{wrapfigure}[8]{l}{2.8cm}
			\vspace{-5mm}
			\includegraphics[scale=0.4]{ch1/13}
			\captionof{figure}{}
			\label{fig:1.13}
			\end{wrapfigure}
			If we are in 3D with $x_3$ the direction of homogeneity and a surface composed of straightlines (height h=1) lying upon a curl $C$ in $x_1 x_2$ plan. The mass flow over the surface is ($\vec{b} = 0$)
			\begin{equation}
				\dot{m} = \int _S \rho \vec{u}\vec{n}\, dS = \int _0 ^1 dx_3 \int _C \dot{m}\vec{u}\vec{n}\, ds
				\label{eq:108}
			\end{equation}
		Let's look to the curl from top. We will make a little bit geometry on the magnifier of the curl C. \\
		
			\begin{wrapfigure}[7]{r}{4.5cm}
			\vspace{-5mm}
			\includegraphics[scale=0.4]{ch1/14}
			\captionof{figure}{}
			\label{fig:1.14}
			\end{wrapfigure}
			In order to determine the normal to the circuit, we have to close it following an anticlockwise fashion (A to B). If we write the normal following $x_1$ and $x_2$ direction, according to \autoref{fig:1.14} we have ($dx_1<0$)
			\begin{equation}
			\left.
			\begin{aligned}
				n_1 &= \cos \theta = \frac{dx_2}{ds} \\
				n_2 &= \sin \theta = -\frac{dx_1}{ds}
			\end{aligned}
			\right\}
			 \Rightarrow 
			\begin{aligned}
				&\rho \vec{u}\vec{n}\,  ds = \rho (u_1 n_1 + u_2 n_2)\, ds\\
				&= \rho \left[ u_1 \frac{dx_2}{ds} - u_2 \frac{dx_1}{ds} \right]\, ds\\
				&= \rho (u_1 dx_2 - u_2 dx_1)
			\end{aligned}
			\label{eq:1.109}
			\end{equation}
			And now if we remplace in \eqref{eq:108} and use the definition of the streamline function, we have 
			\begin{equation}
				\dot{m} = \int _C \rho (u_1 dx_2 - u_2 dx_1) = \int _C \rho _0 \underbrace{\left[\frac{\D \psi }{\D x_2}dx_2 - \left( -\frac{\D \psi}{\D x_1} \right) dx_1 \right]}_{d\psi} = \rho _0 (\psi _B - \psi _A)
			\end{equation}
			So, the physical meaning is that $\psi$ on a certain point is the mass flow between this point and a reference point where $\psi = 0$. And because of a steady flow, it's the same mass flow over the surface from A to B whatever the curl used for. If 2 points A and B are on the same streamline then $\psi_A = \psi _B$, so lines with $\psi = cst$ are streamlines. 
			
		\subsubsection{Streamfunction equation (constant-density flow)} 
			We are interested in finding an equation the streamfunction has to satisfied. The assumption of constant density allow us to consider $\rho = \rho _0$ as it's not a function of space in \eqref{eq:1.107} 
			\begin{equation}
				\left\{
		\begin{aligned}
			u_1 &= &\frac{\D\psi}{\D x_2}\\
			u_2 &= &- \frac{\D\psi}{\D x_1}
		\end{aligned}
		\right.
		\qquad \Rightarrow \nabla \vec{u} = 0 \mbox{ (continuity equation) }
		\label{eq:1.111}
			\end{equation}
			If we compute the vorticity vector 
			\begin{equation}
				\vec{\omega} = \nabla \times \vec{u} = 
				\left| 
				\begin{array}{ccc}
				\vec{e}_1 & \vec{e}_2 & \vec{e}_3 \\ 
				\frac{\D}{\D x_1} & \frac{\D}{\D x_2} & \cancel{\frac{\D}{\D x_3}} \\ 
				u_1 & u_2 & \cancel{u_3}
				\end{array} 
				\right|
				= \vec{e}_3 \underbrace{\left(\frac{\D u_2}{\D x_1} - \frac{\D u_1}{\D x_2}\right)}_{\omega _3}
			\end{equation}
			Now if we replace the velocity components by the streamfunction equivalence, we found the
			
			\begin{center}
			\theor{
			\textbf{Streamfunction equation}
			\begin{equation}
				\omega _3 = -\frac{\D ^2 \psi}{\D x_1^2} - \frac{\D ^2 \psi}{\D x_2^2} \qquad \Leftrightarrow \qquad -\omega _3 (x_1,x_2) = \nabla ^2 \psi
				\label{eq:1.113}
			\end{equation}
			}
			\end{center}
			
		 	An equation for vorticity can be obtained by taking the curl of the momentum equation. This last equation compatible with our assumptions is 
		 	\begin{equation}
		 		\rho\dot{\vec{u}} = \rho \left[\frac{\D \vec{u}}{\D t} + (\vec{u}\nabla)\vec{u}\right] = -\nabla p + p\vec{F} + \nabla \bar{\bar{\tau}}
		 	\end{equation}
		 	Combined with \eqref{eq:1.111}, we have a set of 4 equations and 4 unknowns in 3D ($\vec{u}$ and $p$). So the continuity and momentum equations can be solved independently of the energy equation, the thermal problem is dissociated from the hydrodynamic problem. Let's finally point that for irrotational flows we have 
		 	\begin{equation}
		 		\nabla ^2 \psi = 0 \qquad \mbox{(Laplace's equation)} 
		 	\end{equation}