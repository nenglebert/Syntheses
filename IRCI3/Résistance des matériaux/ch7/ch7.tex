\chapter{Flexion composée}
\section{Théorie}
	\subsection{Méthode cinématique}
	\begin{wrapfigure}[7]{r}{6.2cm}
	\vspace{-10mm}
	\includegraphics[scale=0.4]{ch7/image1.png}
	\captionof{figure}{ }
	\end{wrapfigure}
	On commence à connaître la chanson. Considérons une composition de ce qui a 
	été vu précédemment  un déplacement axial $u$ constant \textit{et} une 
	variation linéaire en fonction de la coordonnée Z.
	\begin{equation}
	u=u_0(x) + z\beta_x(x),\qquad v=0,\qquad w=w_0(x).
	\end{equation}
	Notons $O$, le centre géométrique.
	
	\subsection{Déplacements – Déformations – Contraintes}
		\subsubsection{Déformations}
		On combine : $\epsilon_x$ a une répartition constante \textit{et} est 
		linéaire en $z$ dans la section transversale 
		\begin{equation}
		\epsilon_x = \dfrac{\partial u_0}{\partial x}+z\dfrac{\partial\beta_x}{
		\partial x}
		\end{equation}
		On fait l'hypothèse que les sections planes restent planes ($\gamma_{xz}=\ 
		cste$) :
		\begin{equation}
		\gamma_{xz} = \beta_x+\dfrac{\partial w_0}{\partial x}
		\end{equation}
		
		\subsubsection{Contraintes}
		Toujours notre fameuse loi de Hooke, mais cette fois $\sigma_x$ a une 
		répartition constante \textit{et} une répartition linéaire en $z$ dans 
		la section transversale si $E$ est constant. Poisson encore et toujours 
		négligé. Par rapport à $\tau_{xz}$, cela dépend de si on tient compte 
		ou pas de l'hypothèse de Bernoulli.
		
	\subsection{Éléments de réduction : section constante}
	A l'aide de la loi de Hooke, nous avons
	\begin{equation}
	\begin{array}{ll}
	\sigma_x &=\displaystyle E\left(\dfrac{\partial u_0}{\partial x}+z\dfrac{
	\partial\beta_x}{\partial x}\right)\\
	&= \displaystyle \sigma_x^0 + Ez\dfrac{\partial\beta_x}{\partial x}
	\end{array}
	\end{equation}
	Calculons avant tout notre normale $N$
	\begin{equation}
	N = \int_A\sigma_x\ dA\qquad\ \Longrightarrow\qquad N = \sigma_x^0A
	\end{equation}
	En effet, $\sigma_x^0$ peut être sorti de l'intégrale, étant constant. 
	Comme on a fait dans les chapitres précédents, on suppose $\int_A z\ dA =0$ : 
	ceci est vrai si je passe par le centre géométrique de la figure. Le 
	raisonnement inverse est aussi acceptable : \textit{Cette condition doit 
	être vraie pour que mon effort normal ne dépende que de $\sigma$.}\\
	
	En faisant un raisonnement similaire pour $M_y$ :
	\begin{equation}
	M_y = \int_A \sigma_xz\ dA \qquad\Longrightarrow\qquad M_y = E\dfrac{
	\partial \beta_x}{\partial x}\int_A z^2\ dA
	\end{equation}
	Cette fois, le terme en $\sigma_c^0$ disparaît : car, encore, $\int_A z\ dA=0$.
	On a donc 
	\begin{equation}
	M_y = E\dfrac{\partial \beta_x}{\partial x}I_{zz}\qquad\text{ avec }\quad 
	I_{zz} = \int_A z^2\ dA
	\end{equation}		
	
		\subsubsection{En résumé}
		La poutre est soumise à un effort normal ($N$) \textit{et} à un moment 
		fléchissant ($M_y$). Comme nous avons :
		\begin{equation}
		\begin{array}{lll}
		\text{Pour la traction : } & \sigma_x^{(N)} &= \dfrac{N}{A}\\
		\text{Pour la flexion : } & \sigma_x^{(M)} &= \dfrac{M_y}{I_y}z		
		\end{array}
		\end{equation}
		On a donc
		\begin{equation}
		\sigma_x = \dfrac{N}{A}+\dfrac{M_y}{I_y}z
		\end{equation}
		
		\subsubsection{Synthèse pour une structure plane}
		Soit $xy$ le plan de la structure. Si un axe \textbf{principal d'inertie} 
		de la section transversale est perpendiculaire au plan d'une structure, 
		on a la superposition d'une traction simple $(N)$ et d'une flexion dans le 
		plan de la structure $(M)$.
		
		
\section{Répartition de la contrainte et noyau central}
	\subsection{Répartition de la contrainte $\sigma_x$}
	En toute généralité, une section n'est pas forcément rectangulaire : il faut 
	alors calculer la position du centre de figure et déterminer la mi-hauteur.\\
	En fonction de l'intensité de l'effort normal et fléchissant, trois situations 
	sont possibles\footnote{La somme donne toujours une droite, mais pas forcément 
	la même.}
	\begin{center}
	\includegraphics[scale=0.4]{ch7/image2.png}
	\end{center}
	\begin{enumerate}
	\item Ma droite passe par le zéro (à distance $y_0$) à l'extérieur de la 
	section (gauche). Dans ce cas, ma section sera uniquement en traction.
	\item Par chance, le zéro arrive juste au bord de la section (milieu). Le signe 
	est toujours le même, mais j'arrive à zéro au bord.
	\item Le zéro est avant le bord : j'ai de la traction \textbf{et} de la 
	compression (droite).
	\end{enumerate}
	Comment trouver alors la position de l'axe neutre (et donc notre fameux 
	point $y_0$, le point ou la contrainte axiale s'annule) ? Il suffit d'égaler 
	la contrainte axiale à zéro 
	\begin{equation}
	\sigma_x = 0\qquad\Longrightarrow\qquad y_0=-\dfrac{N}{M}\dfrac{I}{A}
	\end{equation}
	
	
	\subsection{Le noyau central}
	Considérons que l'axe $x$ soit au milieu de la section, section sur laquelle 
	j'applique un effort normal et un moment fléchissant. L'effet causé sera 
	\textbf{identique} à celui d'appliquer la même force normale, mais décallée 
	d'une certaine distance ($e$).
	\begin{center}
	\includegraphics[scale=0.4]{ch7/image3.png}
	\captionof{figure}{ }
	\end{center}
	La sollicitation "$M$ et $N$" est ainsi équivalente à un effort axial $N$ 
	(seul) excentré tel que 
	\begin{equation}
	M = Ne
	\end{equation}
	La position de l'axe neutre est alors 
	\begin{equation}
	y_0=-\dfrac{N}{M}\dfrac{I}{A}\qquad\Longrightarrow\qquad y_0=-\dfrac{I}{eA}
	\end{equation}
	Cherchons maintenant la zone ou, en appliquant cette translation de $e$, j'ai 
	toujours une normale de même signe (comme les deux premiers cas de la section 
	précédente). \\
	Je définis ainsi le \textbf{noyau central} comme étant le lieu des points 
	$P$ d’application de $N$ tels que $\sigma_x$ ne change pas de signe dans la 
	section.
	\begin{center}
	\includegraphics[scale=0.4]{ch7/image4.png}
	\captionof{figure}{ }
	\end{center}
	
	
	
	
	
	
	
	
	
	
	
	
	