
\chapter{Centrifugal pumps}
\section{Generalities}
\subsection{Description - Type of turbopump}
\minifig{ch2/1}{ch2/2}{0.3}{0.4}{0.2}{0.3}
The task of the turbopump is to transfer energy to a liquid. Above we can see a centrifugal and an axial turbopump. Between these two extremes, we can have a variety of of types depending on the requirements. Each turbopump is composed of one or several wheels that can be mounted in parallel (increase mass flow rate) or in series (higher energy transfer), see \autoref{ch1/2}. 

\subsection{Installation of a turbopump}
\minifig{ch2/3}{ch2/4}{0.3}{0.25}{0.2}{0.3}
The general scheme is shown here, observe that the flow enters at the middle in the rotating blade and is projected into the volute. This last has a growing section from the beginning to the end as the mass flow increases. The turbopump is commonly used to transfer liquid from a downstream reservoir to an upstream reservoir situated higher. We have to be careful to avoid cavitation (evaporation of the fluid due to too low pressures) and we also have a control valve at the suction section to always have a contact blade-fluid.

\subsection{Energy developed by the turbopump - flow rate}
\wrapfig{8}{l}{4.5}{0.2}{ch2/5} 
This is the type of curve we will have with the equations, where we see the characteristic curves of the pump and of the overall system (model of the resistance). These curves will be very similar to compressors. Depending on the rpm, we will consider different curves. The fundamental equations are simplified considering $\rho = cst \rightarrow \nu = cst$ for non compressible fluids: 

\begin{equation}
e = \frac{v_o^2 - v_i^2}{2} + \int _{p_i}^{p_o} \nu dp + g(h_o - h_i) = \frac{v_o^2 - v_i^2}{2} + \frac{p_o-p_i}{\rho} + g(h_o - h_i)
\end{equation}

The velocity is low in order to limit the head losses and thus the pressure term is the highest (height change in a compressor is low too). The energy delivered by the pump to the fluid can be rewritten in terms of the volumetric flow rate $Q$ [$m^3/s$]: 

\begin{equation}
e_p = \frac{p_a - p_i}{\rho} - \frac{p_a - p_o}{\rho} + \frac{A_i^2-A_o^2}{2A_iA_o}Q^2 + g(z_o-z_i)
\end{equation}

\subsection{Useful power or hydraulic power}
The power transfered from the input of the pump until the exit and the global efficiency of the pump are: 

\begin{equation}
P_h  = \dot{m}e = \rho Q e \quad [W] \qquad \eta = \frac{\rho Q e}{P_m}
\end{equation}

where $P_m$ is the mechanical power to drive the pump. 

\subsection{Working point of a turbopump}
Consider \autoref{ch2/4} and let's apply Bernouilli equation (kinetic energy equation) between $z'$ and $z_i$ then $z_o$ and $z^"$: 

\begin{equation}
\frac{v_i^2 - {v'}^2}{2} + g(z_i - z') = -\frac{p_i - p'}{\rho} -w'_{fa} \\ \frac{{v^"}^2 - {v_o}^2}{2} + g(z^" - z_o) = -\frac{p^" - p_o}{\rho} -w'_{fr}
\end{equation}

One can make the sum of the two expression and regroup the terms of the reservoirs in a new \textbf{energy requested by the circuit} $\bm{e_n}$. If we consider large reservoirs $v^" \approx v'$ and $p^"\approx p' \approx p_a$, we have: 

\begin{equation}
e_n = g(z^"-z') + \underbrace{w'_{fa} + w'_{fr}}_{w'_f} \qquad \Rightarrow e_p = e_n 
\end{equation}

This is always valid in \textbf{steady state}. 

\subsection{Characteristic of the hydraulic circuit}
The system curves on \autoref{ch2/5} plot $e_n$ which depends on the height difference and the mass flow rate (because $w'_f\propto v^2$ of the flow) and depends thus on the square of the volumetric mass flow rate. This is why we have a parabolic shape, the slope depends on the head loss coefficient $K$. If we have a valve, the closer the valve, the higher the slope. 

\subsection{Performance curve of a pump}
Similar curves can be established for the $e, Q$ relations at different rpm. With a control valve at the exit, and by fixing the rpm of the engine, we can find them and are plotted on \autoref{ch2/5}. 

\subsection{Working regimes}
Practical analysis shows that if two of the three parameters $e,Q,n$ are fixed, the working point too: $f(e,Q,n) = 0$

\subsection{Practical units}
\wrapfig{8}{l}{5}{0.45}{ch2/6}
Here we express the energy in $J/kg$ but we know that it is also $g\Delta z$ in [m]. Thus we will use instead of $e$, $H = e/g$ [m]. The energy transferred to the fluid is often called the \textbf{height} of the \textbf{head}. For example $H = 10$ m means that we transfer energy such that we increase $z$ of 10 m. As last remark, be aware that efficiency curves are provided by the manufacturer and the pump has to be chosen specifically to the circuit where it should operate to get the maximum efficiency. 

\section{The centrifugal pump}
\subsection{Organization of a centrifugal pump}
\wrapfig{8}{r}{4}{0.3}{ch2/7}
We have an inlet distributor D charged of guiding the fluid towards the entrance 1 of the rotor R or also called \textbf{impeller}. The rotor is made of one or two disks on which are mounted the blades beginning at a certain external radius $r_1$ and finishing at $r_2$. A fixed diffuser d composed of 2 parallel discs surrounding the rotor, connected with vanes surrounds the exit of the blades, sometimes it is not used. A \textbf{volute} or \textbf{collector} c with increasing volume directs the flow to the exit section of the machine. 

\subsection{The distributor}
If there is no vane in the distributor, the flow penetrates in the rotor axially since we assume no fluid particle to rotate before entering in the rotor, and becomes radial symmetrically at intrance 1. If there is vane, the direction of the flow is imposed by the vanes but we take the first case here. The equation of kinetic energy applied between i and 1 when neglecting the height difference is: 

\begin{equation}
\frac{v_i^2-v_1^2}{2} + \frac{p_i-p_1}{\rho} = w'_{fD}
\end{equation}

where $w'_{fD}$ represents the pressure losses in the distributor, proportional to the square of $Q$ and thus to $v_1^2$: $w'_{fD}= K_D \frac{v^2_1}{2}$ where $K_D \approx 5.10^{-3}$

\subsection{The rotor}
The impeller starts at $r_1$ and finish at $r_2$, the sections at these levels are: 

\begin{equation}
A_1 = 2\pi r_1b_1 e_1 \qquad A_1 = 2\pi r_2b_2 e_2
\end{equation}

where $e_1,e_2$ are blockage coefficients taking into account the  