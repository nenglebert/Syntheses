
\chapter{Dynamics}
\section{Basic equations}
	
	Until now we worked with static forces, we will see dynamic forces. Newton's second law for the applied forces states: 
	
	\begin{equation}
	b = m \frac{d ^2u}{dt^2} = m\ddot{u} \Rightarrow mb-m\ddot{u} = 0
	\end{equation}
	
	D'Alembert's principle of inertial forces consists in considering the term $-m\ddot{u}$ as $b^I$ (inertial forces). In addition we consider also the viscous damping forces $b^D = -\mu \dot{u}$ so that we get the equilibrium:
	
	\begin{center}
	\theor{
	\begin{equation}
	b + b^I + b^D = 0.
	\end{equation}
	}
	\end{center}
	
	When we have a deformable body, we use finite elements for the integration, since the derivation is temporal: 
	
	\begin{equation}
	u^h(x,t) = N(x) q(t) \qquad \dot{u}^h(x,t) = N(x) \dot{q}(t) \qquad \ddot{u}^h(x,t) = N(x) \ddot{q}(t)
	\end{equation}
	
	We can now use $f^V = \int _V N^T b\, dV$ for the computations: 
	
	\begin{equation}
	\begin{aligned}
	&f^I = \int _V N^T (-\rho \ddot{u}^h) dV = - \left(\int _V \rho N^T N\, dV \right) \ddot{q} = -M\ddot{q}\\
	&f^D = \int _V N^T (-\mu \dot{u}^h) dV = - \left(\int _V \mu N^T N\, dV \right) \dot{q} = -C\dot{q}
	\end{aligned}
	\end{equation}
	
	where $M$ is the mass matrix and $C$ the damping matrix. If we consider our $Kq = f$ with $f = f^{ext}+f^{I}+f^{D}$, we have: 
	
	\begin{center}
	\theor{
	\textbf{Dynamic equilibrium equation}
	\begin{equation}
	M\ddot{q} + C\dot{q} + Kq = f^{ext}
	\end{equation}}
	\end{center}
	
	This can be seen as the governing equations to a one DOF composed by a spring, a damping and a mass. As further development, the kinetic energy, the power of damping forces and the strain energy are given by: 
	
	\begin{equation}
	\left\{	
	\begin{aligned}
	&E^{h,kinetic} = \frac{1}{2}\int _V \rho \dot{u}^{hT}\dot{u}^h\, dV = \frac{1}{2} \dot{q} ^T \int _V \rho N^{T}N \, dV \, \dot{q} = \frac{1}{2} \dot{q}^T M \dot{q} > 0\\
	&P^{h,damping} = - \int _V \rho \dot{u}^{hT}\dot{u}^h\, dV = - \dot{q} ^T \int _V \mu N^{T}N \, dV \, \dot{q} = - \dot{q}^T C \dot{q} \leq 0\\
	&W^h = \frac{1}{2}\int _V \tau _{ij}^h \epsilon _{ij}^h\, dV = \frac{1}{2} q^T K q > 0
	\end{aligned}
	\right.
	\end{equation}
	
	If we multiply now our dynamic equation by $\dot{q}^T$ (left) we can find 
	
	\begin{center}
	\theor{
	\begin{equation}
	\dot{E}^{h,kinetic} - P^{h,damping} + \dot{W}^h = P^{h,ext} 
\end{equation}		
	}
	\end{center}
	
\section{Resolution}
	The general equation forms a system of Q second-order differential equations, where Q is the number of DOF. Initial boundary condition at $t = t_0$ is required. Three resolution approaches are available: 
	
	\begin{itemize}
	\item[•] \textbf{modal analysis:} eigenmodes and eigenvalues study;
	\item[•] \textbf{response in the modal basis:} projection onto the modal basis, getting Q decoupled equations;
	\item[•] \textbf{direct integration methods:} step by step incremental method. 
	\end{itemize}
	
\section{Eigenmodes and eigenvalues}
\subsection{Principles}
	The free vibrations of an undamped system will be studied: 
	
	\begin{equation}
	M\ddot{q} + K q = 0
	\end{equation}
	
	The solutions are harmonic $q(t) = \phi e^{i\omega t}$, by replacing: 
	
	\begin{equation}
	(K - \omega ^2 M) \phi = 0
	\end{equation}
	
	The ssystel has non-trivial solutions only if the determinant is null, leading to the eigenvalue problem 
	
	\begin{equation}
	K \phi _i = \lambda _i M\phi _i
	\label{eq:19.10}
	\end{equation}
	
	where $\lambda _i = \omega ^2$ are the eigenvalues and $\phi _i$ the corresponding eigenmodes. $\omega _i$ are the eigenfrequencies. Moreover, if $\phi _i$ are the modes, $K\phi _i$ too for any real value of K. Therefor the following normalization exist: \\
	
	\begin{itemize}
	\item[•] Euclidean normalization: $\phi _i | \phi _i^T \phi i = 1$
	\item[•] Mass matrix normalization: $\phi _i | \phi _i^T M\phi i = 1$\\
\end{itemize}	  

	The mass one is preferred because it takes into account the mass distribution. For vertical structures, the two first modes are generally bending modes and the third one a torsion mode, followed by more and more complex shapes. In addition, the modes can be asymmetric, we cannot use the symmetry for the computations. 
	
\subsection{Properties of the eigenmodes and eigenvalues}
\subsubsection{Rayleigh coefficient}
	If we multiply left by $\phi _i$ in \eqref{eq:19.10} and make the division (since the mass matrix is regular): 
	
	\begin{equation}
	\lambda _i = \frac{\phi _i^T K \phi _i}{\phi _i^T M \phi _i} \geq 0
	\end{equation}
	
\subsubsection{Case of distinct eigenvalues}
	Let us now consider $\lambda _i \neq \lambda _j$: 
	
	\begin{equation}
	K \phi _i = \lambda _i M\phi _i \qquad K \phi _j = \lambda _j M\phi _j
	\end{equation}
	
	By premultiplying the first by $\phi _j^T$ and the send by $\phi ^T_i$ and making (1)-(2):
	 
	 \begin{equation}
	 (\lambda _i - \lambda _j) \phi _j^T M \phi _i = 0 \qquad \Rightarrow \phi _j^T M \phi _i = 0
	 \end{equation}
	 
	 since the eigenvalues are different. The eigenmodes associated to distinct eigenvalues are mutually M-orthogonal. 
	 
\subsubsection{Case of multiple eigenvalues}
	If $\lambda _i = \lambda _j \equiv \lambda ^*$ with different eigenmodes, it is possible to get (same system as above but without premultiplying, but summing):
	
	\begin{equation}
	K(a\phi _i + b \phi _j) = \lambda ^* M(a\phi _i + b \phi _j)
	\end{equation}
	
	We see that a linear combination of the eigenvectors associated to the same eigenvalue are an eigenvector too (M-orthogonal).
	
\subsubsection{K-orthogonality}
	Consider the distinct eigenvalues case, the final result implies K-orthogonality: 
	
	\begin{equation}
	\phi _j ^T K \phi _i = 0
\end{equation}	 

\section{Modal basis}
	Let us consider the matrix $\Phi = [\phi _1 \ \phi _2 \ \dots \ \phi _Q]$.  containing the eigenmodes. Applying the orthogonality properties we get:
	
	\begin{equation}
	\Phi ^T M\Phi = I \qquad \Phi ^T K\Phi = \Lambda \qquad \Rightarrow \det (\Phi ^T) \det( M ) \det (\Phi ) = 1
	\end{equation}
	
	where $\Lambda$ is the matrix containing the eigenvalues on the diagonal. Since M is regular, $\det (\Phi) \neq 0$ meaning that the eigenmodes are linearly independent from each other and constitute the so-called \textbf{modal basis}. So any vector can be expressed as: 
	
	\begin{equation}
	q = a_1 \phi _1 + \dots + a_Q \phi _Q = \Phi a 
	\label{eq:19.17}
	\end{equation}
	
	and by premultiplying by $\Phi ^T M$ we get $\Phi ^T Mq = a$ and initial conditions can be expressed as: 
	
	\begin{equation}
	a_0 = \Phi ^T M q_0\qquad \dot{a}_0 = \Phi ^T M \dot{q}_0
	\end{equation}
	
	and the equations of motion becomes injecting \eqref{eq:19.17}:
	
	\begin{equation}
	M\Phi \ddot{a} + C\Phi \dot{a} + K\Phi a = f^{ext} \Leftrightarrow \ddot{a} + \Phi ^T C\Phi \dot{a} + \Lambda a = f^{ext} 
	\end{equation}
	
	for the last line we pre multiply by $\Phi ^T$. Assuming a proportional damping we can define damping ratios $\xi _i$ and define 
	
	\begin{equation}
	\Phi ^T C \Phi = \Gamma = \left[ 
	\begin{array}{cccc}	
	2\xi _1 \omega _1 & 0 & \dots & 0\\
	0 & 2\xi _2 \omega _1 & \dots & 0\\
	\vdots & \vdots & \vdots & \vdots\\
	0 & 0 & \dots & 2\xi _Q \omega _Q \\ 
	\end{array}	
	\right]
	\end{equation}
	
	Now the equation of motion is a system of decoupled scalar differential equations: 
	
	\begin{equation}
	\ddot{a} + 2\xi _i \omega _i \dot{a}_i + \lambda _i a_i = f_i^{ext} = \phi _i^T f^{ext}
	\end{equation}
	
\subsection{Numerical resolution of dynamic systems}
There is two way of solving, the first consist in a conversion in a first order differential equation system by posing $h = \dot{q}$, indeed we double the number of unknowns. The second way are the explicit and implicit methods, this is not seen in the lecture. 

\ \\


\begin{center}
\textbf{This is the end of a long way, I hope you'll get good marks.}
\end{center}