
\chapter{Emissions}
	When dealing with the emissions there is a strong trade-off between global (greenhouse gases, ...) and local (health, NOx, ...) effects.
	
\section{Emission formation}
	The ideal combustion case is: 
	
	\begin{equation}
	C_x H_yO_z + O_2 \rightarrow CO_2 + H_2O
	\end{equation}
	
	We can see that $CO_2$ is the product of the combustion at the same level of the $H_2O$ and should not be considered as pollutant. But in reality what happens is: 
	
	\begin{equation}
	C_x H_y O_z + O_2 + N_2 \rightarrow CO_2 + H_2O + N_2 + O_2 + CO + C_xH_yO_z + C_mH_nO_o + NO + H_2
	\end{equation}
	
	However, the $CO_2$ contributes to the greenhouse effect when coupled in the atmosphere and its minimization is also important from an efficiency viewpoint. It is not the only greenhouse effect gas, the other have more consequent effects. \\
	
	Pollutants have several negative effects, for example: 
	\begin{itemize}	
	\item[•] $CO$ is very toxic (asphyxiation).
	\item[•] $NOx$ which combined with $HC$ and sunlight produces \textbf{ozone}. Ozone can lead to respiratory or cardiac problems. 
	\item[•] $NO + O_3 \rightarrow NO_2$ formation occurs very quickly. It leads to lung malfunctioning, acid rains $HNO_3$ and formation of Nitro-Pah (carcinogenic).\\
	\end{itemize}
	
	In fact the smaller the particle the more treatment it requires because it penetrates easily into lungs. 
	
	\minifig{ch5/1}{ch5/2}{0.3}{0.3}{0.4}{0.3}
	
	Above we can see that SI and CI engines do not have the same emissions and we can also regroup all the parameters leading to emission in function of the equivalence ratio as shown on the plot. In SI engines, the important phenomenon leading to main emissions is the absorption of $HC$ by deposits and cylinder walls that free them at the emission point. 
	
\paragraph{CO}
	Carbon monoxide is mostly produced in SI engine when the fuel is rich ($\lambda > 1$) and is due to incomplete combustion. It is still present in lean fuels because of the dissociation reactions: 
	
	\begin{equation}
	CO + \frac{1}{2}O_2 \leftrightarrow CO_2 \qquad CO_2 + H_2 \leftrightarrow CO + H_2O
	\end{equation}
	
	that happen more easily in high temperature. It is thus possible to reduce it with combustion temperature decrease. 
	
\paragraph{NO} The $NO$ production mostly depends on the temperature, this is due to the kinematics of the reaction which shows a dramatic increase in production rate when above 1800 K. The equivalence ratio plays also a role since NO needs oxygen. The maximum is at $\lambda \approx 0.9$ since when higher, temperature increases but less oxygen and the contrary when $\lambda < 0.9$. It also needs times for production and thus is more present at low rpm. 

\paragraph{Unburned HC} The unburned hydrocarbons are mostly due to incomplete combustion. Other origins: "short circuit" between intake and exhaust valves, slow combustion or flame extinction, absorption desorption of fuel by oil, deposits or crevices. 

\paragraph{Soot} This emission is mostly due to not optimal combustion (ignition delay, speed) and is mostly observed in direct injection $\rightarrow$ CI, but also SI direct injection now. It increases dramatically in congested traffic and during idling. We remark also that SI engines produces PM. 

\section{Emission regulation}
	Emission legislation were implemented because of the concerns for the air quality and the impact on human health. From epidemiological studies, the upper limit before negative effects can be determined. Based on this limit and model on diffusion of pollutants, we can determine what is the limit on a global scale (country) and local scale. From these numbers and other assumptions, we can go back to the limit per source of pollutants, such as vehicles. \\
	
	Legislation differs from region to region, from the vehicle size, ... The difference between market is difficult with the manufacturer since he has to adapt the vehicle to each market, but they tend to harmonize. The legislation is based on $g/km$ emitted during a test cycle representative for real world driving (previously NEDC as project). 
	
\subsection{Chassis dyno cycle}
		\wrapfig{7}{l}{5}{0.25}{ch5/3}{ch5/3}
		The test are performed in a laboratory on a test bench. They are potentially non realistic but the legislators ask for durability of the results. But this is a way of harmonising the legislation as the tests are the same for everyone. To illustrate how dynamic a cycle is, we can plot for every second of the test what is the acceleration and speed of the vehicle. For NEDC, this is illustrated by small acceleration and a very simple pattern. For WLTC, the range of acceleration and speed is much bigger and variable. However, this is still a predictable cycle, which could be identified.
		
		\wrapfig{6}{l}{6}{0.25}{ch5/4}{ch5/4}
		The other problem of NEDC is that the entire operating map is not used. Here a comparison between Real Driving Emissions. It provides data with more insight and there is more random point on \autoref{ch5/3} (more difficult to cheat).
		
\subsection{Post-processing}
	As we have a lot of information we have to threat them. The first method is \textbf{EMROAD} (moving averageging window). The data is plotted on windows regarding their CO$_2$ accumulation, then we compute the average mg/km for all pollutants, then we find a value for the entire trip. \\
	
	The second method is \textbf{CLEAR} (Power Binning), the data is ranked in function of the instantaneous power bins. The post-processing is sensible to the windowing used. We use a setup to measure the dynamic profile of PM in GDI (gasoline direct injection), it relies on an impactor that can determine the number of particles of different sizes. 
	
\subsection{Cheating}
	\begin{itemize}
	\item[•] Thermal window defeat device: temperature below the must for test
	\item[•] Hot restart defeat device: if engine already hot
	\item[•] Cycle detection defeat device: if we already know the cycle. \\
	\end{itemize}
	
	For the VW scandal for example, it is easy to prove cheating by change of emission strategy (cycle known). We only have to remake the cycle but backwards, the results should be very similar with non-cheating car. In the VW case they were 6 times higher than the legislation limit. 
	