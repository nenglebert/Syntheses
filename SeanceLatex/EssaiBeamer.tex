\documentclass[11pt]{beamer}
\usetheme{Warsaw}
\usepackage[utf8]{inputenc}
\usepackage[french]{babel}
\usepackage[T1]{fontenc}
\usepackage{amsmath}
\usepackage{amsfonts}
\usepackage{amssymb}
\usepackage{physics}
\author{M@T}
\title{Mon beau slide}
\begin{document}

\begin{frame}
\titlepage
\end{frame}

\begin{frame}{Résultat final}
Nous avons alors
\begin{equation}
\bra{v}\left(\hat{A}\ket{u}\right) = \underbrace{(v_1^*\ v_2^*\ \dots\ v_n^*)\left(\begin{array}{ccc}
a_{11} & \dots & a_{1n}\\
\vdots &\ddots &\vdots\\
a_{n1} & \dots & a_{nn}
\end{array}\right)}_{(*)}\left(\begin{array}{c}
u_1\\
u_2\\
\vdots\\
u_n
\end{array}\right)
\label{eq:Joli}
\end{equation}
Le $(*)$ apparaissant dans \autoref{eq:Joli} est joli, n'est-ce-pas ? 
\end{frame}

\end{document}