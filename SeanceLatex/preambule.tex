%%%%%%%%%%%%%%%%
%%% Packages %%%
%%%%%%%%%%%%%%%%

%%% Général %%%
\usepackage[utf8]{inputenc}   
\usepackage[french,english]{babel}
\usepackage[T1]{fontenc}
\usepackage{mathpazo}
\usepackage[scaled=0.95]{helvet}
\usepackage{courier}
\usepackage{graphicx}

%%% Tableau %%%
\usepackage{tabularx} %Permet d'auto dimensionner les tableaux
%\usepackage{slashbox} %Slash dans les tableau


%%% Bibliographie %%%
\usepackage[style=alphabetic,backend=bibtex]{biblatex}
\usepackage[autostyle]{csquotes}
\DeclareNameAlias{sortname}{last-first}
\DeclareFieldFormat{url}{\space\url{#1}}
\DeclareNameAlias{labelname}{last-first}
\addbibresource{a.bib}



%%% Graphiques %%%
\usepackage{tikz}
\usepackage{pgfplots}
\usepackage{circuitikz}

%%% Mise en page %%%
\usepackage{amsmath}
\usepackage{amsfonts}
\usepackage{amssymb}
\usepackage{amsthm}
\usepackage[tt]{titlepic} % Centre le titre
\usepackage{fancyhdr} % Permet de modifier l'entête & footer
\usepackage{caption} % Permet d'ajouter des légendes en images sans les mettre en float + dans la marge
\usepackage{wrapfig}
\usepackage{fullpage}
\usepackage{multicol} % pour les liste sur plusieurs colonnes
\usepackage{subfigure} % alligne deux images cote a cote
\usepackage{float} %permet de mettre du texte entre les figures grace a [H]. Génial! 

%%% Math %%%
\usepackage{delarray} % Belles matrices
\usepackage{siunitx}
\sisetup{locale = FR,detect-all}
% Pour mettre siunitx en mode français (virgule plutôt que point etc.)




%%%%%%%%%%%%%%%%%
%%% Commandes %%%
%%%%%%%%%%%%%%%%%

%%% Physque %%%
\newcommand{\cst}{\text{cst}}
\newcommand{\E}{\vec E}
\newcommand{\B}{\vec B}
\newcommand{\F}{\vec F}
\newcommand{\module}[1]{||#1||}

%%% Math %%%
\newcommand{\oiint}{\int\!\!\!\!\!\!\! \:\!\subset\!\!\supset\!\!\!\!\!\!\!\int}
\newcommand{\rot}{\text{rot}\,}
\newcommand{\divv}{\text{div}\,}
\newcommand{\phas}[1]{\underline{#1}}
\newcommand{\RE}{\text{Re}}

%\pagestyle{headings} % Titre du ch et numéro page dans l'entete
\renewcommand{\proofname}{Démonstration}
\selectlanguage{english}

\addto\captionsfrench{\def\tablename{Tableau}}



%%%%%%%%%%%%%%%%%
%%% CHEAT OMG %%%
%%%%%%%%%%%%%%%%%
%Donne le style Book à un Article héhé
\makeatletter

\newcommand\frontmatter{%
    \cleardoublepage
  %\@mainmatterfalse
  \pagenumbering{roman}}

\newcommand\mainmatter{%
    \cleardoublepage
 % \@mainmattertrue
  \pagenumbering{arabic}}

\newcommand\backmatter{%
  \if@openright
    \cleardoublepage
  \else
    \clearpage
  \fi
 % \@mainmatterfalse
   }

\usepackage{eso-pic}

%%% Background %%%
\newcommand\BackgroundPic{%
\put(0,0){%
\parbox[b][\paperheight]{\paperwidth}{%
\vfill
\centering
\includegraphics[width=\paperwidth,height=\paperheight,%
keepaspectratio]{ulb.jpg}%
\vfill
}}}

%%%%%%%%%%%%%%%%%%%%%%%%%%%
%% QUEL SALAUD CE CEDRIC %%
%%%%%%%%%%%%%%%%%%%%%%%%%%%
\usepackage{abstract}
\AtBeginDocument{\renewcommand{\abstractname}{\Large Abstract}}