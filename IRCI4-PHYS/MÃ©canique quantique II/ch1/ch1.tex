\chapter{Opérateur densité}
\section{Introduction}
\subsection{Distinction entre superposition quantique et mélange statistique}
L'opérateur densité peut être vu comme une généralisation du vecteur d'état. 
En effet, pour décrire un système plus compliqué, on utilise la mécanique 
statistique : on considère un système d'axes $(x,p)$ que l'on peut subdiviser 
en plusieurs cases $dx\ dp$. On peut alors introduire $f(x,p)\ dx\ dp$ comme 
étant la probabilité de trouver la particules dans $dx\ dp$. Avant de nous 
intéresser à distribution $f(x,p)$, il est nécessaire d'introduire un nouveau 
formalisme.\\

Le problème est que nous ne savons pas où est une particule et on aimerait 
associer la notion de probabilité à un formalisme quantique
\begin{equation}
\ket\psi = \int dx\ \psi(\vec{r})\ket{\vec{r}}
\end{equation}
Nous ne savons pas exactement dans quel état nous nous sommes : les probabilités 
reflètent ainsi le manque connaissance de la position et l'impulsion.\\

Définissons la notion d'\textbf{état pure}. Tous les états considérés jusqu'ici 
étaient considérés comme tel. Il s'agit d'un état qui peut être décrit à partir 
des vecteurs propres d'un ECOC. Mais il se peut que l'on n'ai pas toujours la 
parfaite connaissance d'un observable de notre ECOC : nous avons ainsi 
un manque de connaissance qui se traduit par un \textbf{état mixte}. \\

La nécessite d'introduire la notion d'état mixte se justifie expérimentalement : 
l'expérience de Stern et Gerlach ne peut être comprise si cette notion n'est pas
introduite.

\subsubsection{Expérience de Stern et Gerlach}
Cette expérience consiste à envoyer un jet d'atome d'argent (spin $1/2$) et les 
envoyer à travers un champ magnétique. Si l'on procède ensuite à l'analyse de 
la projection du spin, par exemple le long de l'axe $z$, on peut observer deux 
taches à cause du spin $1/2$
\begin{equation}
+\frac{\hbar}{2}\ (50\%),\qquad\qquad\qquad -\frac{\hbar}{2}\ (50\%)
\end{equation}
Cette projection sur l'axe $z$ dénonce la quantification, ces deux spins sont 
équiprobables : nous avons une chance sur deux d'avoir l'une ou l'autre. Ceci 
peut être décrit par le vecteur d'état
\begin{equation}
\ket \psi = \frac{1}{\sqrt{2}}\ket{\frac{1}{2},+\frac{1}{2}}+\frac{1}{\sqrt{2}}\ket{\frac{1}{2},-\frac{1}{2}}
\end{equation}
où l'on utilise la notation efficace $\ket{s,m_s}$.  On retrouve bien cette 
équiprobabilité
\begin{equation}
S_z\ \left\{\begin{array}{lll}
\DS p\left(S_z = \frac{\hbar}{2}\right) &\DS= \left| \bra{\frac{1}{2},\frac{1}{2}}\ket{\psi}\right|^2&=\DS\frac{1}{2}\vspace{2mm}\\
\DS p\left(S_z = -\frac{\hbar}{2}\right) &\DS= \left| \bra{\frac{1}{2},-\frac{1}{2}}\ket{\psi}\right|^2&=\DS\frac{1}{2}
\end{array}\right.
\end{equation}
Nous avons ici travaillé avec des états purs. L'axe $z$ étant totalement arbitraire 
(on doit avoir le même résultat peu importe l'axe choisi), analysons la situation 
selon l'axe $x$
\begin{equation}
S_x = \frac{\hbar}{2}\left(\begin{array}{cc}
0&1\\
1&0
\end{array}\right)
\end{equation}
Nos deux états propres sont
\begin{equation}
\left\{\begin{array}{ll}
\ket{+} &\DS\equiv \frac{1}{\sqrt{2}}\ket{\frac{1}{2},+\frac{1}{2}}+\frac{1}{\sqrt{2}}\ket{\frac{1}{2},-\frac{1}{2}}\vspace{2mm}\\
\ket{-} &\DS\equiv \frac{1}{\sqrt{2}}\ket{\frac{1}{2},+\frac{1}{2}}-\frac{1}{\sqrt{2}}\ket{\frac{1}{2},-\frac{1}{2}}
\end{array}\right.
\end{equation}
Calculons maintenant les probabilités
\begin{equation}
\left\{\begin{array}{lll}
p\left( S_x = +\frac{\hbar}{2}\right) &= |\bra{+}\ket{\psi}|^2 &= 1\\
p\left( S_x = -\frac{\hbar}{2}\right) &= |\bra{-}\ket{\psi}|^2 &= 0
\end{array}\right.
\end{equation}
Or, ceci n'est pas ce qui est observé expérimentalement, cette modélisation de $\ket\psi$ n'est pas 
modélisable avec des états purs, d'où une première motivation à l'utilisation des états mixtes.\\

\subsubsection{Oscillateur harmonique (2D)}
Dans ce cas, la quantification de l'énergie est donnée par
\begin{equation}
E_{n_x, n_y} = (1+n_x+n_y)\hbar\omega
\end{equation}
Supposons que l'on ai deux états et que l'on souhaite le décrire. Trois candidats basés sur les états 
purs sont possibles
\begin{equation}
\ket{n_x=1,n_y=0},\qquad\ket{n_x=0,n_y=1},\qquad \alpha\ket\dots + \beta\ket\dots
\end{equation}
où $|\alpha|^2+|\beta|^2 = 1$. Nous ne savons pas dans lequel de ces trois états possibles nous nous 
trouvons, nous sommes bien dans la situation d'un manque de connaissance.


\section{Opérateur densité d'un état pur}
Montrons tout d'abord comment ce formalisme fonctionne lors de l'utilisation d'états purs. Ensuite, 
nous étendrons cette notions pour les états mixtes.

\subsection{Trace, propriété de $\rho$, Liouville quantique, mesure de l'état $\rho$}

\subsubsection{Trace d'un opérateur}
La trace d'un opérateur correspond à la trace de la matrice de cet opérateur. Soit $\hat{A}\in \mathcal{H}$, 
un opérateur hermitien et $\left\{\ket n\right\}$ une base orthonormée de $\mathcal{H}$. On définit alors la 
\textbf{trace de $\hat{A}$} 
\begin{equation}
\Tr(\hat{A}) \equiv \sum_n \bra{n}\hat{A}\ket{n}
\label{eq:DefTR}
\end{equation}
Le choix de la base étant arbitraire, nous devrions retrouver ce résultat peu importe la base choisie.  
Considérons une autre base orthonormée $\left\{\ket{\overline{n}}\right\}$ tel que nous pouvons écrire comme 
relation de fermeture
\begin{equation}
\sum_{\overline{n}}\ket{\overline{n}}\bra{\overline{n}} = \hat{\mathbb{1}}
\end{equation}
La définition \eqref{eq:DefTR} est équivalente à
\begin{equation}
\Tr(\hat{A}) = \sum_n\sum_{\overline{n}}\sum_{\overline{m}}\bra{n}\ket{\overline{n}} \bra{\overline{n}}\hat{A}\ket{\overline{m}}
\bra{\overline{m}}\ket{n}
\end{equation}
En réorganisant les termes
\begin{equation}
\Tr(\hat{A}) = \sum_{\overline{n}}\sum_{\overline{m}} \bra{\overline{n}}\hat{A}\ket{\overline{m}}\sum_n
\underbrace{\bra{\overline{m}}\overbrace{\ket{n}\bra{n}}^{\mathbb{1}}\ket{\overline{n}}}_{\delta_{\overline{m},
\overline{n}}}
\end{equation}
Par propriété de $\delta_{\overline{m},\overline{n}}$, on en tire que
\begin{equation}
\Tr(\hat{A}) = \sum_{\overline{n}} = \bra{\overline{n}}\hat{A}\ket{\overline{n}}
\end{equation}
ce qui montre que la définition de la trace est indépendante du choix de la base. Une autre propriété intéressante est que 
\begin{equation}
\Tr(\hat{A}\hat{B})  = \Tr(\hat{B}\hat{A})
\end{equation}
\begin{proof}\ \\
Insérons la relation de fermeture entre $\hat{A}$ et $\hat{B}$. Nous avons alors
\begin{equation}
\begin{array}{ll}
\Tr(\hat{A}\hat{B}) = \sum_n\bra{n}\hat{A}\hat{B}\ket{n} &=\DS \sum_n\sum_m \bra{n}\hat{A}\ket{m}\bra{m}\hat{B}\ket{n}\\
&=\DS \sum_n\sum_m \bra{m}\hat{B}\ket{n}\bra{n}\hat{A}\ket{m}\\
&=\DS \sum_m \bra{m}\hat{B}\hat{A}\ket{m} = \Tr(\hat{B}\hat{A})
\end{array}
\end{equation}
\end{proof}
Considérons le cas particulier ou l'on considère $\hat{A}=\ket{\psi}\bra{\varphi}$
\begin{equation}
\begin{array}{ll}
\Tr(\ket{\psi}\bra{\varphi}\hat{B}) &=\DS \sum_n \bra{n}\ket{\psi}\bra{\varphi}\hat{B}\ket{n}\\
&=\DS \sum_n \bra{\varphi}\hat{B}\ket{n}\bra{n}\ket{\psi} \\
&=\DS \bra{\varphi}\hat{B}\ket{\psi}
\end{array}
\end{equation}
Ce qui revient au calcul d'un élément de matrice.\\

Définissions maintenant l'\textbf{opérateur densité d'état}? Soit un système dans un état $\ket{psi}$ (normalisé). L'opérateur densité associé à cet état est 
\begin{equation}
\underline{\hat{\rho} = \ket{\psi}\bra{\psi}}
\end{equation}
\subsubsection{Propriétés de $\hat{\rho}$}
Cet opérateur a plusieurs propriétés indifférentes
\begin{enumerate}
\item[i] $\hat{\rho} = \hat{\rho}^\dagger$  (valeur propre réelle)
\item[ii] $\hat{\rho} \geq 0$ (valeur propre positive)
\begin{equation}
\begin{array}{lll}
\forall\ket{\varphi} :&\DS \bra{\varphi}\hat{\rho}\ket{\varphi} &\geq 0\\
&\DS \bra{\varphi}\ket{\psi}\bra{\psi}\ket{\varphi} &\DS\geq 0\quad \Leftrightarrow\quad |\bra{\psi}\ket{\varphi}|^2 \geq 0
\end{array}
\end{equation}
Il s'agit bien d'un opérateur positif
\item[iii] $\Tr(\hat{\rho}) = 1$ (somme des valeurs propres vaut 1)
\begin{equation}
\Tr(\hat{\rho})=Tr(\ket{\psi}\bra{\psi}) = \bra{\psi}\ket{\psi} = 1
\end{equation}
Ceci est une conséquence directe de la normalisation de $\ket{\psi}$.
\end{enumerate}
On remarque que ces trois propriétés correspondent à la définition d'une probabilité : les valeurs propres
de l'opérateur $\hat{\rho}$ vont jouer un rôle analogue aux distributions de probabilités. On peut dès lors 
en tirer trois autres propriétés :
\begin{enumerate}
\item Les éléments de matrices diagonales  de l'opérateur $\hat{\rho}$ correspondent aux probabilité de mesures du système\footnote{La matrice de l'opérateur densité $\hat{\rho}$ est la \textit{matrice densité}.}. La probabilité 
de mesurer la particule au point $\vec{r}$ est donnée par
\begin{equation}
\underline{p(\vec{r}) = \bra{\vec{r}}\hat{\rho}\ket{\vec{r}}}
\end{equation}
\begin{proof}
\begin{equation}
\begin{array}{ll}
p(\vec{r}) = |\psi(\vec{r})|^2 = |\bra{\vec{r}}\ket{\psi}|^2 &=\DS \bra{\vec{r}}\ket{\psi}^*\bra{\vec{r}}\ket{\psi}\\
&=\DS \bra{\vec{r}}\ket{\psi}\bra{\psi}\ket{\vec{r}}\\
&=\DS \bra{\vec{r}}\hat{\rho}\ket{\vec{r}}
\end{array}
\end{equation}
\end{proof}
Vérifions que ceci est consistant
\begin{equation}
\int p(\vec{r})\ d\vec{r} = \int \bra{\vec{r}}\hat{\rho}\ket{\vec{r}}\ d\vec{r} = \Tr\hat{\rho} = 1
\end{equation}
Ceci étant connecté à la notion de probabilité, la normalisation prend tout son sens. 
Ceci était pour l'observable position, voyons pour un autre observable. 
\item Soit $\hat{A}$, de valeurs propres $a_i$ et $\DS \hat{P_i} = \sum_{j=1}^{d_i} \ket{\varphi_{ij}}\bra{\varphi_{ij}}$ (le projecteur associé à la valeur propre $a_i$). La probabilité de mesurer $a_i$ peut s'écrire
\begin{equation}
\underline{p(a_i) = \Tr(\hat{\rho}\hat{P_i})}
\end{equation}
\begin{proof}
\begin{equation}
\begin{array}{ll}
p(a_i) = \|\hat{P_i}\ket{\psi}\|^2 &=\DS \bra{\psi}\hat{P_i}\ket{\psi}\\
&=\DS \sum_k \bra{\psi}\hat{P_i}\ket{u_k}\bra{u_k}\ket{\psi}\\
&=\DS \sum_k \bra{u_k}\underbrace{\ket{\psi}\bra{\psi}}_{\hat{\rho}}\hat{P_i}\ket{u_k} = \Tr(\hat{\rho}\hat{P_i})
\end{array}
\end{equation}
\end{proof}
Ceci est cohérent avec ce que nous avions obtenu avec l'observable $\vec{r}$
\begin{equation}
\hat{P_{\vec{r}}} = \ket{\vec{r}}\bra{\vec{r}}\quad\rightarrow\quad p(\vec{r}) = \Tr(\hat{\rho}\ket{\vec{r}}\bra{\vec{r}})
= \bra{\vec{r}}\hat{\rho}\ket{\vec{r}}
\end{equation}
\item Montrons que ceci est consistant avec la notion de valeur moyenne de l'opérateur $\hat{A}$. Cette dernière s'exprime
\begin{equation}
\underline{\langle\hat{A}\rangle = \Tr(\hat{\rho}\hat{A})}
\end{equation}
\begin{proof}
\begin{equation}
\begin{array}{ll}
\langle\hat{A}\rangle = \sum_i a_ip(a_i) &=\DS \sum_i a_i\Tr(\hat{\rho}\hat{P_i})\\
&=\DS \Tr(\hat{\rho}\underbrace{\left(\sum_i a_i\hat{P_i}\right)}_{\hat{A}})
\end{array}
\end{equation}
où nous avons  utiliser la décomposition spectrale de l'opérateur $\hat{A}$.
\end{proof}
\end{enumerate}
Ceci est un formalisme différent, mais équivalent à celui vu précédemment.

\subsubsection{Équation de Liouville quantique}
Reprenons l'équation de Schrödinger temporelle
\begin{equation}
i\hbar \frac{d}{dt}\ket{\psi(t)} = \hat{H}(t)\ket{\psi(t)}
\end{equation}
Intéressons nous à la dérivée temporelle de $\hat{\rho}$, multipliée par $i\hbar$
\begin{equation}
\begin{array}{ll}
i\hbar\dfrac{d\hat{\rho}}{dt} &=\DS i\hbar\left(\frac{d}{dt}\ket{\psi(t)}\right)\bra{\psi(t)} + i\hbar\ket{\psi(t)}\left(
\frac{d}{dt}\bra{\psi(t)}\right)\vspace{2mm}\\
&=\DS \hat{H}\underbrace{\ket{\psi}\bra{\psi}}_{\hat{\rho}}-\underbrace{\ket{\psi}\bra{\psi}}_{\hat{\rho}}\hat{A}\\
&=\DS [\hat{H},\hat{\rho}]
\end{array}
\end{equation}
où nous avons utilisé l'équation de Schrodinger dépendante du temps ainsi que son équation conjuguée pour 
exprimer les dérivées des vecteurs d'états. Il s'agit de l'\textbf{équation de Liouville quantique}. Ce 
formalisme est identique à celui de la mécanique quantique ; nous avons ici vu un "autre langage" d'écrire les 
états purs, que nous allons étendre aux états mixtes. \\

Notons qu'il est possible d'obtenir le théorème d'Ehrenfest à partir de la relation suivante (vu en séance d'exercices)
\begin{equation}
\langle\hat{A}\rangle = \Tr(\hat{\rho}\hat{A})\qquad\Leftrightarrow\qquad i\hbar\frac{d}{dt}\langle\hat{A}\rangle = 
i\hbar\frac{d}{dt}\Tr(\hat{\rho}\hat{A})
\end{equation}

%%%%%%%


\newpage
	
\section{Opérateur densité d'un mélange statistique}
Un état pure d'un système, par exemple $\ket{\psi_i}\bra{\psi_i}$ est caractérisé par le déterminisme de la 
mécanique classique dans l'évolution de l'état quantique\footnote{Sauf lorsqu'il y a une mesure}. Par contre, 
un état mixte est un mélange statistique d'états purs. On verra que l'on peut toujours écrire un état mixte 
\begin{equation}
\sum_i p_i\ket{\psi_i}\bra{\psi_i}
\end{equation}
où les $p_i$ sont les probabilités du mélange statistique, et les $\ket{\psi_i}\bra{\psi_i}$ les différents 
états purs du système\footnote{Source : \textsc{Wikipedia}}.

\subsection{Définition, propriétés de $\hat{\rho}$, interprétation (diagonalisation)}
\subsubsection{Définition}	
Soit $\{\ket{\psi_k},p_k\}$ un \textbf{ensemble statistique} où les $p_k$ représentent les probabilités $\left(p_k \geq 0, 
\sum_k p_k=1\right)$ et où les $\ket{\psi_k}$ ne sont pas forcément orthogonal (mais normalisés). On va définir, comme 
annoncé dans l'introduction, les \textbf{états mixtes} :
\begin{equation}
\hat{\rho} = \underline{\sum_k p_k\ket{\psi_k}\bra{\psi_k}}
\label{eq:DefRho2}
\end{equation}
où $\ket{\psi_k}\bra{\psi_k} = \hat{\rho_k}$ est un projecteur. Cette définition nous dit que l'opérateur densité 
est une combinaison complexe d'\textit{opérateur densité purs} $\hat{\rho_k}$, pondérée par une probabilité.

\subsubsection{Propriétés de $\hat{\rho}$}
\begin{itemize}
\item[$\bullet$] Mesure de $\hat{A}$. Rappelons-nous que
\begin{equation}
p(a_i|k) = \bra{\psi_k}\hat{P_i}\ket{\psi_k} = \Tr(\ket{\psi_k}\bra{\psi_k}\hat{P_i}) = \Tr(\hat{\rho_k}\hat{P_i})
\end{equation}
où $\hat{\rho_k}$ est l'opérateur densité correspondant à l'état pur $\ket{\psi_k}$. Nous voulons maintenant\footnote{Multiplication de deux probas ? Pas clair.}
\begin{equation}
p(a_i) = \sum_k p_k\ p(a_i|k) = \sum_k p_k \Tr(\hat{\rho_k}\hat{P_i}) = \Tr(\left(\sum_k p_k\hat{\rho_k}\right)\hat{P_i}) = 
\Tr(\hat{\rho}\hat{P_i})
\end{equation}
où nous avons utilisé \eqref{eq:DefRho2}. Ceci nous montre que le formalisme introduit ici est général, la forme 
obtenue étant la même que pour les états purs.\\

\item[$\bullet$] La valeur moyenne de l'opérateur $\hat{A}$ s'obtient
\begin{equation}
\langle\hat{A}\rangle = \Tr(\hat{\rho}\hat{A})
\end{equation}
\begin{proof}
\begin{equation}
\langle\hat{A}\rangle = \sum_k p_k \bra{\psi_k}\hat{A}\ket{\psi_k} = \sum_k p_k\Tr(\underbrace{\ket{\psi_k}\bra{\psi_k}}_{\hat{\rho_k}}\hat{A}) = \Tr(\left(\sum_k p_k\hat{\rho_k}\right)\hat{A})
\end{equation}
\end{proof}

\item[$\bullet$] L'équation de Liouville quantique a elle aussi une forme similaire à celle obtenue pour les états purs
\begin{equation}
\begin{array}{ll}
i\hbar\dfrac{d\hat{\rho}}{dt} &=\DS i\hbar\frac{d}{dt}\left(\sum_k p_k \hat{\rho_k}\right)\vspace{2mm}\\
&=\DS  \sum_k p_k \underbrace{i\hbar \frac{d\hat{\rho_k}}{dt}}_{[\hat{A},\hat{\rho_k}]}\vspace{2mm} \\
&\DS= \left[\hat H, \sum_k p_k\hat{\rho_k}\right] = [\hat{H},\hat{\rho}]
\end{array}
\end{equation}
\end{itemize}
Si nous calculons la trace de $\hat{\rho_k^2}$, nous obtenons
\begin{equation}
\Tr(\hat{\rho_k^2}) = \hat{\rho_k} = 1
\end{equation}
car $\hat{\rho_k}$ est idempotent. Ceci est vrai \textbf{pour un état pur} mais en général
\begin{equation}
\Tr(\hat{\rho^2}) \leq 1
\end{equation}
Il s'agit de la seule différence entre un état pur et un état mixte. Pour un état mixtes, nous avons donc $\hat{\rho}
=\hat{\rho}^\dagger, \hat{\rho}\geq 0, Tr(\hat{\rho})=1$ mais $\Tr(\hat{\rho^2})\leq 1$.


\subsubsection{Diagonalisation de $\hat\rho$ (interprétation)}
A l'aide du théorème de décomposition spectrale, nous pouvons écrire $\hat{\rho}$ à l'aide de ses valeurs 
et vecteurs propres
\begin{equation}
\hat{\rho} = \sum_i \Pi_i \ket{\chi_i}\bra{\chi_i}
\end{equation}
où les $\Pi_i \geq 0$ sont les valeurs propres de l'opérateur densité et les $\ket{\chi_i}$ sont les vecteurs propres 
(orthogonaux) de ce même opérateur. \\

Ceci peut s'interpréter comme un mélange des $\ket{\chi_i}$. Nous pouvons remarquer que toutes les expressions 
précédemment obtenues ne dépendaient toujours que de $\hat{\rho}$ et jamais des $\ket{\psi_k}$ : tout est 
caractérisé par $\hat{\rho}$. Ceci montre qu'il peut y avoir des cas où il n'est pas possible d'extraire l'état 
dans laquel le système a été préparé : on peut comprendre que l'expérience de S\&G  est une combinaison de spin 
\textit{up} et \textit{down} mais l'axe sur lequel on prend la mesure n'a pas d'importance car, peu importe 
celui-ci, la matrice densité contient toute l’information, les propriétés physiques ne dépendent de rien d'autre 
que de $\hat{\rho}$.\\


Un autre point à soulevé est que la notion de \textit{phase globale} disparaît totalement lors du traitement des 
états mixtes. Soit $\ket{\psi'} = e^{i\alpha}\ket{\psi}$
\begin{equation}
\hat{\rho}' = \ket{\psi'}\bra{\psi'} = e^{i\alpha}\ket{\psi}\bra{\psi}e^{-i\alpha} = \ket{\psi}\bra{\psi}= \hat{\rho}
\end{equation}
Il n'y a donc plus de notion de phase globale : ceci montre que $\hat{\rho}$ est l'unique représentation du système. 
L'opérateur densité $\hat{\rho}$ est \textbf{complet} et l'\textbf{unique} représentation d'un état du système\footnote{Unique 
car il n'y a pas de phase globale irrelevante.}.\\

Interprétons maintenant l'experience de Stern et Gerlach pour des atomes non polarisés. Empiriquement, la 
probabilité de mesurer un spin \textit{up} ou \textit{down} est identique et vaut $1/2$. Soit la projection 
du spin selon l'axe $\vec{u}$
\begin{equation}
\hat{S_{\vec{u}}} = \hat{\vec{S}}.\vec{u}
\end{equation}
La matrice densité décrivant l'état de notre système s'écrit
\begin{equation}
\hat{\rho} = \left(\begin{array}{cc}
\frac{1}{2}&0\\
0&\frac{1}{2}
\end{array}\right) = \frac{1}{2}\hat{\mathbb{1}}
\end{equation}
Une rotation de la matrice identité reste une matrice identité : peu importe l'axe de mesure, les probabilités de 
mesures seront les mêmes, nous avons correctement décrit notre système. \\

Si l'on s'intéresse à la valeur moyenne 
\begin{equation}
\begin{array}{ll}
\langle \hat{S_{\vec{u}}}\rangle =\langle \hat{\vec{S}}.\vec{u}\rangle &\DS= \langle S_x\rangle u_x+
\langle S_y\rangle u_y+\langle S_z\rangle u_z\vspace{2mm}\\
&=\DS \Tr(\hat{\rho}.S_x)u_x+\Tr(\hat{\rho}.S_y)u_y+\Tr(\hat{\rho}.S_z)u_z\vspace{2mm} \\
&=\DS \frac{u_x}{2}\Tr(S_x)+\frac{u_y}{2}\Tr(S_y)+\frac{u_z}{2}\Tr(S_z)\vspace{2mm}\\
&=0
\end{array}
\end{equation}
car les traces des matrices de Pauli sont nulles. On voit donc que peu importe la direction, on observe le même
caractère équiprobable. \\

Nous savons maintenant que la représentation suivante est fausse
\begin{equation}
\ket{\psi} \neq \dfrac{1}{\sqrt{2}}\ket{\frac{1}{2},\frac{1}{2}}\ \ +\ \ \dfrac{1}{\sqrt{2}}\ket{\frac{1}{2},-\frac{1}{2}}
\end{equation}
Pour le montrer, considérons la matrice densité associé à ce $\ket{\psi}$
\begin{equation}
\hat{\rho} = \ket{\psi}\bra{\psi} = \left(\begin{array}{cc}
\frac{1}{2} & \frac{1}{2}\vspace{1mm}\\
\frac{1}{2} & \frac{1}{2}
\end{array}\right)
\end{equation}
La diagonalisation de la matrice dans la base des états propres donne alors
\begin{equation}
1\ket{+}\bra{+} \ \ +\ \  0\ket{-}\bra{-}\qquad\rightarrow\qquad \left(\begin{array}{cc}
1&0\\
0&0
\end{array}\right)
\end{equation}
ce qui ne correspond clairement pas à la réalité physique observée : on se retrouve dans un autre état en diagonalisant 
dans la base des états propres.

\subsubsection{Exemples physiques}
Ces exemples montrent des situations classiques traduit dans le cadre de la mécanique quantique.\\

\textsc{A. Ensemble micro-canonique}\\
Sot un ensemble micro-canonique
\begin{equation}
\mathcal{F}\subset\mathcal{H}
\end{equation}
L'énergie (connue) correspond à un sous-espace de l'espace de Hilbert. Dans le cas d'un ensemble micro-canonique, 
l'opérateur densité s'écrit
\begin{equation}
\hat{\rho} = \frac{1}{d}\hat{\mathbb{P}_f} = \frac{1}{d}\sum_{i=1}^d \ket{e_i}\bra{e_i}
\end{equation}
Sa matrice n'est rien d'autre qu'une matrice diagonale composée uniquement de $1/d$ : tous les états sont bien 
équiprobables.\\

\textsc{B. Ensemble canonique}\\
Un tel ensemble décrit un système en équilibre thermodynamique avec le réservoir (régit par la distribution de 
Boltzmann). L'opérateur densité est donné par
\begin{equation}
\hat{\rho} = \dfrac{e^{-\beta\hat{H}}}{\Tr(e^{-\beta\hat{H}})}
\end{equation}
où $\hat{H}$ est l'Hamiltonien et $\beta = \dfrac{1}{k_BT}$. Si l'on cherche à diagonaliser :
\begin{equation}
\hat{H}\ket{\psi_n} = E_n\ket{\psi_n}\qquad \text{où }\ \hat{H} = \sum_n E_n\ket{\psi_n}\bra{\psi_n}
\end{equation}
A l'aide de la relation de fermeture, nous pouvons obtenir la relation suivante
\begin{equation}
e^{-\beta\hat{H}} = \sum_n e^{-\beta E_n}\ket{\psi_n}\bra{\psi_n}\qquad\Rightarrow\qquad \Tr(e^{-\beta \hat{H}}) = 
\sum_n e^{-\beta E_n}
\end{equation}
L'opérateur densité peut alors se réécrire
\begin{equation}
\hat{\rho} = \sum_n \dfrac{e^{-\beta E_n}}{\sum_m e^{-beta E_m}} \ket{\psi_n}\bra{\psi_n}
\end{equation}
où l'on reconnaît la fonction de partition de Boltzmann. Il s'agit d'un état mixte (chaque état est pondéré par
une probabilité). 

 

\section{Distribution de Wigner}
\subsection{Définition, propriétés de $W(x,p)$, négativité de $W(x,p)$}
\subsubsection{Définition}
Nous allons ici présenter une autre façon de représenter un système dans un état pure et mixte. Pour se faire, 
considérons une particule dans une espace $3D$ et choisissons une base - ici la base position -de sorte à exprimer 
son opérateur densité. En utilisant deux fois la relation de fermeture, nous pouvons écrire
\begin{equation}
\hat{\rho} = \iint d^3\vec{r} d^3\vec{r'}\ \ket{\vec{r}}\underbrace{\bra{\vec{r}}\hat{\rho}\ket{\vec{r'}}}_{\hat\rho(\vec{r},\vec{r})}\bra{\vec{r'}}
\end{equation}
où $\hat\rho(\vec{r},\vec{r})$ est une fonction complexe qui pondère chaque $\ket{\vec{r}}\bra{\vec{r}}$. Regardons l'élément 
diagonal de la matrice densité de cette fonction complexe
\begin{equation}
\hat{\rho} = \bra{\vec{r}}\hat{\rho}\ket{\vec{r}} = \mathbb{P}(\vec{r}) = \Tr(\hat{\rho}\ket{\vec{r}}\bra{\vec{r}})
\end{equation}
En intégrant sur  $\vec{r}$ et par linéarité de la trace
\begin{equation}
\int d\vec{r}\ \rho(\vec{r},\vec{r}) = \Tr(\hat{\rho}\hat{\mathbb{1}}) = 1
\end{equation}
Il devient possible interpréter $\hat \rho(\vec{r})$ comme la probabilité de se trouver à une certaine position. Il serait 
intéressant de pouvoir traiter l'impulsion de façon similaire à la position. Le problème est similaire à celui introduit 
en début de chapitre : la probabilité d'être dans un domaine infinitésimal $dx\ dp$ est donné par $f(x,y)\ dx\ dy$. 
Intéressons-nous à ce que pourrait être une fonction jouant le rôle joué par $f(x,y)$. Une telle fonction, une telle 
distribution, peut être la (quasi) distribution de Wigner.\\

Par \textbf{définition}
\begin{equation}
\underline{W(\vec{r},\vec{p}) = \dfrac{1}{(2\pi\hbar)^3}\int d^3\vec{a}\ \hat\rho\left(\vec{r}-\frac{\vec{a}}{2},\vec{r}
+\frac{\vec{a}}{2}\right)e^{\frac{i}{\hbar}\vec{a}.\vec{p}}}
\end{equation}
Cette fonction prend en compte deux positions choisies proche l'une de l'autre. Nous savons que la partie diagonale donne 
la probabilité de mesurer $\vec{r}$ et nous avons introduit le vecteur $\vec{a}$ qui est la différence entre les deux 
positions ; la transformée de Fourier est ensuite effectuée de sorte à avoir l'impulsion.
On a deux position, on les prends proche l'une de l'autre et on sait que la partie diagonale donne la probabilité de mesurer r et on a introduit le vecteur a qui est la différence entre les deux et on prend la transformée de Fourier pour avoir l'impulsion.

\subsubsection{Propriétés de $W(x,p)$}
La (quasi) distribution de Wigner possède les propriétés suivantes
\begin{itemize}
\item[$\bullet$] $W(\vec \hat r,  \vec \hat p)$ est réel :
\begin{equation}
\begin{array}{ll}
W^*(\vec{r},\vec{p}) &\DS= \frac{1}{(2\pi\hbar)^3}\int d^3\vec{a}\ \hat{\rho}^\dagger\left(\vec{r}-\frac{\vec{a}}{2},\vec{r}
+\frac{\vec{a}}{2}\right)e^{-\frac{i}{\hbar}\vec{a}.\vec{p}}\\
&\DS= \frac{1}{(2\pi\hbar)^3}\int d^3\vec{a}\ \hat{\rho}^\dagger\left(\vec{r}+\frac{\vec{a}}{2},\vec{r}
-\frac{\vec{a}}{2}\right)e^{-\frac{i}{\hbar}\vec{a}.\vec{p}}
\end{array}
\end{equation}
où nous avons utilisé le fait que $\hat{\rho}^\dagger(\vec{r},\vec{r'}) = \hat{\rho}(\vec{r'},\vec{r})$ :
\begin{equation}
\hat{\rho}^\dagger(\vec{r},\vec{r'}) = \bra{\vec{r}}\hat{\rho}\ket{\vec{r'}}^*= \bra{\vec{r'}}\hat{\rho}\ket{\vec{r}} 
= \hat{\rho}(\vec{r'},\vec{r})
\end{equation}
car $\hat{\rho}$ est hermitien. Nous pouvons conclure\footnote{Pas de signe négatif avec $d^3\vec{a}$, le jacobien 
étant pris en valeur absolue.} en effectuant le changement de variable $\vec{b}=-\vec{a}$
\begin{equation}
W^*(\vec{r},\vec{p}) = \dfrac{1}{(2\pi\hbar)^3}\int d^3\vec{b}\ \hat\rho\left(\vec{r}-\frac{\vec{b}}{2},\vec{r}
+\frac{\vec{b}}{2}\right)e^{\frac{i}{\hbar}\vec{b}.\vec{p}} = W(\vec{r},\vec{p})
\end{equation}
\item[$\bullet$] $W(\vec{r},\vec{p})$ étant une distribution, elle doit être normalisable : sa trace doit valoir 1.
\begin{equation}
\begin{array}{ll}
\DS\iint d^3\vec{r}d^3\vec{p}\ W(\vec{r},\vec{p}) &=\DS \int d^3\vec{r}\int d^3\vec{a}\ r\left(\vec{r}+\frac{\vec{a}}{2},\vec{r}
-\frac{\vec{a}}{2}\right) \overbrace{\frac{1}{(2\pi\hbar)^3}\int d^3\vec{p}\ e^{\frac{i}{\hbar}\vec{a}.\vec{p}}}^{\delta^{(3)}(\vec{a})}\vspace{2mm}\\
&=\DS \int d^3\vec{r}\ \hat{\rho}(\vec{r},\vec{r}) = \Tr(\hat{\rho}) = 1
\end{array}
\end{equation}
où nous avons utilisé $\int e^{iax}\ dx = 2\pi \delta(x)$.
\item[$\bullet$] La valeur moyenne de la position se calcule
\begin{equation}
\langle\vec{r}\rangle = \Tr(\hat{\rho}\hat{\vec{r}}) = \int d^3\vec{r}\ \bra{\vec{r}}\hat{\rho}\hat{\vec{r}}\ket{\vec{r}} = 
\int d\vec{r}\ \vec{r}\underbrace{\bra{\vec{r}}\hat{\rho}\ket{\vec{r}}}_{\mathbb{P(\vec{r})}}
\end{equation}
Ceci est équivalent à 
\begin{equation}
\iint d\vec{r}d\vec{p}\ \vec{r}\ W(\vec{r},\vec{p}) = \int d\vec{r}\ \vec{r}\ \hat{\rho}(\vec{r},\vec{r})
\end{equation}
où nous avons utilisé (montré lors de la propriété précédente)
\begin{equation}
\iint d^3\vec{r}d^3\vec{p}\ \ W(\vec{r},\vec{p}) = \int d^3\vec{r}\ \ \hat{\rho}(\vec{r},\vec{r})
\end{equation}

\item[$\bullet$] Nous pouvons obtenir une équation similaire pour calculer la valeur moyenne de l'impulsion. Nous avions obtenu
\begin{equation}
\langle\vec{p}\rangle = \Tr(\hat{p}\hat{\rho}) = \int d\vec{p}\ \bra{\vec{p}}\hat{p}\hat{\rho}\ket{\vec{p}} =
\int d\vec{p}\ \ \vec{p}\underbrace{\bra{\vec{p}}\hat{\rho}\ket{\vec{p}}}_{\mathbb{P}(\vec{p})}
\end{equation}
L'idée est de ré-obtenir une forme semblable en utilisant notre distribution de Wigner. Pour se faire, partons de la 
définition de la valeur moyenne de $\hat{\rho}$ et utilisons la relation de fermeture
\begin{equation}
\bra{\vec{p}}\hat{\mathbb{1}}\hat{\rho}\hat{\mathbb{1}}\ket{\vec{p}}= \int d\vec{r}d\vec{r}'\ \ 
\underbrace{\bra{\vec{p}}\ket{\vec{r}}}_{\frac{1}{(2\pi\hbar)^{3/2}}e^{-\frac{i}{\hbar}\vec{p}.\vec{r}}}\bra{\vec{r}}\hat{\rho}\ket{\vec{r}'}\underbrace{\bra{\vec{r}'}\ket{\vec{p}}}_{\frac{1}{(2\pi\hbar)^{3/2}}e^{
\frac{i}{\hbar}\vec{p}.\vec{r}}}
\label{eq:ReecritVMRho}
\end{equation}
Après ré-écriture, nous obtenons
\begin{equation}
\bra{\vec{p}}\hat{\rho}\ket{\vec{p}}=\frac{1}{(2\pi\hbar)^3}\iint d\vec{r}d\vec{r'}\ \
\bra{\vec{r}}\hat{\rho}\ket{\vec{r'}}e^{\frac{i}{\hbar}\vec{p}(\vec{r'}-\vec{r})}
\end{equation}
Effectuons le changement de variable suivant
\begin{equation}
\left\{\begin{array}{ll}
\vec{r} &= \vec{u}-\frac{\vec{v}}{2}\\
\vec{r'} &= \vec{u}+\frac{\vec{v}}{2}
\end{array}\right.\qquad\Rightarrow\qquad \mathcal{J} = \left(\begin{array}{cc}
1&1/2\\
1&-1/2
\end{array}\right)\quad \rightarrow\quad |\mathcal{J}| = 1
\end{equation}
Nous obtenons
\begin{equation}
\bra{\vec{p}}\hat{\rho}\ket{\vec{p}}= \iint d\vec{u}d\vec{v}\ \ \hat{\rho}\left(\vec{u}-\frac{\vec{v}}{2},
\vec{u}+\frac{\vec{v}}{2}\right)e^{\frac{i}{\hbar}\vec{p}.\vec{v}}
\end{equation}
En remplaçant $\vec{v}\to\vec{a}$, on retrouve $W(\vec{r},\vec{p})$
\begin{equation}
\bra{\vec{p}}\hat{\rho}\ket{\vec{p}}=\int d\vec{u}\ \ W(\vec{u},\vec{p}) = \int d\vec{r}\ \ W(\vec{r},\vec{p})
\end{equation}
Nous pouvons alors ré-écrire \eqref{eq:ReecritVMRho}
\begin{equation}
\langle p \rangle = \iint d\vec{r}d\vec{p}\ \ \vec{p}\ W(\vec{r},\vec{p})
\end{equation}
La valeur moyenne de $\hat{p}$ est donc obtenue par simple intégration de $\vec{p}\ W(\vec{r},\vec{p})$. Ceci 
peut se généraliser : si l'on souhaite la valeur moyenne de n'importe quelle fonction de la position où 
l'impulsion, on retrouve
\begin{equation}
\begin{array}{ll}
\langle f(\vec{r})\rangle &= \Tr(\hat{\rho}f(\vec{r})) = \iint d\vec{r}d\vec{p}\ \ f(\vec{r})\ W(\vec{r},\vec{p})\\
\langle g(\vec{p})\rangle &= \Tr(\hat{\rho}g(\vec{p})) = \iint d\vec{r}d\vec{p}\ \ g(\vec{p})\ W(\vec{r},\vec{p})
\end{array}
\end{equation}
\end{itemize}
En plus d'être une fonction réelle, on peut montrer que
\begin{equation}
\left\{\begin{array}{ll}
\DS\int d\vec{p}\ W(\vec{r},\vec{p}) &= \mathbb{P}(\vec{r})\\
\DS\int d\vec{r}\ W(\vec{r},\vec{p}) &= \mathbb{P}(\vec{p})\\
\DS\int d\vec{r}d\vec{p}\ W(\vec{r},\vec{p}) &= 1
\end{array}\right.
\end{equation}
Peut-on dès lors voir $W(\vec{r},\vec{p})$ comme la probabilité d'occuper une cellule $d\vec{r}d\vec{p}$ ? 
La réponse est \textbf{non} car $W(\vec{r},\vec{p})$ n'est pas toujours positive. C'est la raison pour 
laquelle on parle de \textit{quasi} distribution. Néanmoins, pour les états purs, on peut utiliser le 
théorème de Hudson-Piquet
\begin{equation}
W(\vec{r},\vec{p}) \geq 0\ \forall \vec{r},\vec{p}\qquad\Leftrightarrow\qquad \hat{\rho} \text{ est un 
état gaussien pur}
\end{equation}
Pour considérer un exemple pratique : l'état fondamental $\ket{\psi_0}$ de l'oscillateur harmonique est 
un état gaussien : il est possible d'écrire l'opérateur densité $\hat{\rho}$ correspondant pour ensuite 
écrire $W_n(\vec{x},\vec{p})$ et l'utiliser comme bon nous semble.
