\chapter{Méthodes de résolution approchée du problème à $N$ corps}
Bien souvent, il n'est pas possible de calculer de façon exacte les problèmes à $N$ corps ; il va 
falloir recourir à des méthodes d’approximations. L'une de ces méthodes est la "Density Fonctionnal 
Theory" (DFT). Avec celle-ci, une fonctionnelle $\rho_e(\vec{r})$ est créée : celle-ci sera vue comme 
une fonction de l'énergie qu'il faudra minimiser. Une autre approche, que nous allons maintenant 
découvrir, est celle du champ moyen.

\section{Modèle à champ moyen (self-consistant)}

L'idée est que l'on considère chaque électron dans un champ qui correspond à l'attraction des nucléons mais 
aussi la répulsion de tous les autres électrons. Ainsi, chaque électron est indépendant dans son "champ propre" 
qui prend compte les interactions des autres. Deux grandes méthodes utilisent ce modèle : la méthode d'Hartree (non utilisé aujourd'hui mais pédagogiquement intéressant pour introduire la suivante) et la méthode d'Hartree-Fock. 
Il s'agit de deux méthodes variationnelles pour laquelle nous allons maintenant faire un bref rappel.

\subsection{Rappels : problèmes à $N$ corps, méthode variationnelle}
\subsubsection{Problème à $N$ corps}
L'Hamiltonien que nous avons à traiter est
\begin{equation}
\hat{H} = \sum_{i=1}^Z \hat{H_i} + \sum_{i>j}^N V_{ij}\qquad\text{ où }\quad \left\{\begin{array}{ll}
\hat{H_i} &\DS= \frac{p_i^2}{2m}-\frac{Ze^2}{r_i}\\
\hat{V}_{ij} &=\DS \dfrac{e^2}{|\vec{r_i}-\vec{r_j}|}
\end{array}\right.
\end{equation}où $Z$ est le nombre atomique (soit le nombre d'électrons), $\hat{H_i}$ contient le terme de 
répulsion du noyau et $\hat{V_{ij}}$ contient la répulsion entre les électrons. Ceci donne lieu à une 
fonction d'onde $\psi(1,2,\dots Z)$ totalement antisymétrique. \\

Afin d'interpréter de champ moyen, nous pouvons ré-écrire l'Hamiltonien comme (ajout et suppression du 
même terme)
\begin{equation}
\hat{H} =\sum_i\left( \frac{p_i^2}{2m}-\dfrac{Ze^2}{r_i}+V(\vec{r_i}\right)+\underbrace{\sum_i\left(\sum_{j>i} 
\dfrac{e^2}{|\vec{r_i}-\vec{r_j}|}-V(\vec{r_i}\right)}_{\approx 0}
\end{equation}
Le but sera alors d'approcher $V(\vec{r_i})$.


\subsubsection{Méthode variationnelle}
Soit $W$, une fonctionnelle du \textit{ket} $\ket{\phi}$ 
\begin{equation}
W\left[\ket{\phi}\right] = \dfrac{\bra{\phi}\hat{H}\ket{\phi}}{\bra{\phi}\ket{\phi}},\qquad\qquad\Leftrightarrow
\qquad\qquad \min_{\ket{\phi}} W\left[\ket{\phi}\right] = E_0
\end{equation}
Le théorème de Ritz nous informe que cette fonctionnelle est stationnaire pour tous les états propres:
\begin{equation}
\delta W\left[\ket{\phi}\right] = 0\qquad\qquad\Leftrightarrow\qquad\qquad \hat{H}\ket{\phi}=E\ket{\phi}
\end{equation}
Nous allons extrémiser sous la contrainte de normalisation à l'aide des multiplicateurs de Lagrange
\begin{equation}
\begin{array}{lll}
&\DS\delta[\bra{\phi}\hat{H}\ket{\phi}-E\bra{\phi}\ket{\phi}] &=0\vspace{2mm}\\
\Leftrightarrow&\bra{\delta\phi}\hat{H}\ket{\phi} + \bra{\phi}\hat{H}\ket{\delta \phi} - E\bra{\delta\phi}\ket{\phi}-E
\bra{\phi}\ket{\delta\phi}&=0\vspace{2mm}\\
\Leftrightarrow&\DS \bra{\delta\phi}\hat{H}-E\ket{\phi}+\bra{\phi}\hat{H}-E\ket{\delta\phi}&=0
\end{array}
\end{equation}
où nous avons utilisé le fait que $\hat{E}$ et $E$ permutent. Considérons le conjugué de cette expression : celui-ci
doit être valable $\forall \ket{\delta\phi}$ :
\begin{equation}
\bra{\delta\phi}\hat{H}-E\ket{\phi}+\bra{\delta\phi}\hat{H}-E\ket{\phi}^*\qquad\forall\ket{\phi}
\label{eq:NConj}
\end{equation}
où $\hat{H}$ est hermitien. Ceci étant vrai $\forall\ket{\phi}$, procédons au changement de variable
$\ket{\delta\phi} = i\ket{\delta\phi}$
\begin{equation}
-i\bra{\delta\phi}\hat{H}-E\ket{\phi}+i\bra{\delta\phi}\hat{H}-E\ket{\phi}^*\qquad\forall\ket{\phi}
\end{equation}
Après division par $-i$
\begin{equation}
-\bra{\delta\phi}\hat{H}-E\ket{\phi}+\bra{\delta\phi}\hat{H}-E\ket{\phi}^*\qquad\forall\ket{\phi}
\label{eq:OConj}
\end{equation}
Comme \eqref{eq:NConj} et \eqref{eq:OConj} sont valable $\forall\ket{\phi}$, nous pouvons les sommer. 
On en tire que ($\forall\ket{\phi}$)
\begin{equation}
\bra{\delta\phi}\hat{H}-E\ket{\phi}=0\qquad\qquad\Rightarrow\qquad\qquad \hat{H}\ket{\phi}=E\ket{\phi}
\end{equation}
Ce résultat nous montre qu'il n'est pas nécessaire de tenir compte du complexe conjugué, celui-ci n'étant 
plus présent dans cette dernière expression.


\subsubsection{Hartree et Hartree-Fock}
La seule différence entre ces deux méthodes se situe dans la façon d'écrire. Pour Hartree la 
fonction d'essai est un produit d'état (non symétrique, car fermions)
\begin{equation}
\ket{\psi} = \ket\dots\ket\dots\dots \ket\dots
\end{equation}
Alors que pour Hartree-Fock $\ket\psi$ est écrit sous la forme d'un déterminant de Slater.






\subsection{Méthode de Hartree}
Pour cette méthode, on considère que
\begin{equation}
\psi(1,2,\dots)=\phi_1(1)\phi_2(2)\dots\phi_N(N)
\end{equation}
où
\begin{equation}
\phi_i(i) = \phi_i(\vec{r_i})\chi(n_i) \quad \dashrightarrow \quad\phi_i(\vec{r_i})
\end{equation}
Nous ne tenons pas compte du spin ici afin d'alléger le formalisme.\\

Comme discuté dans la précédente sous-section, nous allons maximiser sous contrainte de 
normalisation sauf que cette fois nous n'avons plus une, mais $N$ contraintes, chacune 
des fonction du produit devant être normalisés :  nous avons $N$ fonctions d'onde
\begin{equation}
14
\end{equation}
Revenons à notre fct de Lagrange. On s'en fou du terme de drote, c'est que le complexe conjugué (ou on 
aplique la delta sur le ket mais osef ici), on note juste +c.c.
\begin{equation}
15
\end{equation}
On obtient alors N équations de Schrod, une pour chaque delta phi i. Comme ceci doit etre vrai forall i. 
Si on extrait dans la derniere ligne le terme correspondant à delta phi i :
\begin{equation}
16
\end{equation}
On a un terme additionnel. Le systeme 'Hartree va etre un systeme de N équations.

HARTREE EQUATIONS
\begin{equation}
17
\end{equation}
On l'avait dit : tout se passe comme si les électrons ne sont pas couplé et chaque électron subit 
dans son hamp moyen la répulsion e tous les autres électrons. Tout le point est que V i (vec r) 
est auto consistant. Si on a pas phi j c'est mort pour V i donc on va procéder par ittération\\

On commence par un état initial, un pour chaque électron
\begin{equation}
18
\end{equation}
On doit ittérer. C'est self consistant. Ona besoin de résoudre schro mais pour chaque électron CHque 
electron est dans V i mais pour ça on a besoin de phi j qu'on a pas. On part 'un état initial arbitraire (enfin N, un pour chaque e-). On peut calculer les N potentiel, résoudre scrhod et avoir un phi i puis recommencer et ça va converger : auto consistant.


\subsection{Méthode de Hartree-Fock}
La fonction est ici completement antisymétrique
\begin{equation}
19
\end{equation}
On va aussi utiliser la méthode 'extrama de Lagrange. L'idée est la meme mais la valeur moyenne de H sera plus compliquée
\begin{equation}
20
\end{equation}
On va d'abord regarder le One body term number $i$. On peut voir qu'ona des permutations à gauche te à droite mais si on 
considère que si les permutations sont pas identiques ça fera 0. Les seules permutations qui vont avoir une contrib c'est
quand on aura la meme dans le bra et le ket
\begin{equation}
21
\end{equation}
Ceci peut etre compris comme une permutations de toutes.Pour toutes les permutation on regarde la valeu rmoyenne de l'énergie 
de la particule numéro $i$. \\

Ceci c'est our le  terme i, mais si on regarde le "full one body term", c'est juste le terme pour N terme de chaque particules. 
On aura quelque chose de simple à la fin
\begin{equation}
22
\end{equation}
La façon dont on traite l'antisymétrie ne change pas pour le one body terme (c'est le meme pour les deux et c'est pas influence
par la symétrie). Ce qui va changer c'est le two body term.\\

On considère un terme doné : hashtag(carre)$i$, carre$j$. Vijn'agit que sur l'électron i et j.  
\begin{equation}
23
\end{equation}
On fait exactement le même raisonnement Le - vient de l'antisymétrie. Ceci est pour i et j, mais on peut le remplacer ce 
n'est qu'un nom. 

Two body Term
\begin{equation}
24
\end{equation}


SI la fonction d'état est un slater et pas un simple product, ona un terme en plus. On a tjs le terme e coulomb mais on a 
un terme en plus qui peut pas etre une interaction car on a le meme H qu'avant.
\begin{equation}
25
\end{equation}
On a donc une équation par i
\begin{equation}
26
\end{equation}
On peut redre cette expression plus spécifique dans le cadre 'un atome
\begin{equation}
27
\end{equation}
Le terme en - c'est cet exchange.\\

On a donc N schrodinger équation (non interacting electrons).  On a pas besoin de les résoudre tous à la fois, on la 
résou pour chaque phi i mais cette équation est couplée à cause du terme d'interaction 'change : il faut la résoudre 
ittérativement exactement comme précéemment. C'est donc également une équation auto consistante. C'est une approx 
car on a ce champ d'interaction moyen.