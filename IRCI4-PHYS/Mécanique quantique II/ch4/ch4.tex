\chapter{Méthodes de résolution approchée du problème à $N$ corps}
Bien souvent, il n'est pas possible de calculer de façon exacte les problèmes à $N$ corps ; il va 
falloir recourir à des méthodes d’approximations. L'une de ces méthodes est la "Density Fonctionnal 
Theory" (DFT). Avec celle-ci, une fonctionnelle $\rho_e(\vec{r})$ est créée : celle-ci sera vue comme 
une fonction de l'énergie qu'il faudra minimiser. Une autre approche, que nous allons maintenant 
découvrir, est celle du champ moyen.

\section{Modèle à champ moyen (self-consistant)}

L'idée est que l'on considère chaque électron dans un champ qui correspond à l'attraction des nucléons mais 
aussi la répulsion de tous les autres électrons. Ainsi, chaque électron est indépendant dans son "champ propre" 
qui prend compte les interactions des autres. Deux grandes méthodes utilisent ce modèle : la méthode d'Hartree (non utilisé aujourd'hui mais pédagogiquement intéressant pour introduire la suivante) et la méthode d'Hartree-Fock. 
Il s'agit de deux méthodes variationnelles pour laquelle nous allons maintenant faire un bref rappel.

\subsection{Rappels : problèmes à $N$ corps, méthode variationnelle}
\subsubsection{Problème à $N$ corps}
L'Hamiltonien que nous avons à traiter est
\begin{equation}
\hat{H} = \sum_{i=1}^Z \hat{H_i} + \sum_{i>j}^N V_{ij}\qquad\text{ où }\quad \left\{\begin{array}{ll}
\hat{H_i} &\DS= \frac{p_i^2}{2m}-\frac{Ze^2}{r_i}\\
\hat{V}_{ij} &=\DS \dfrac{e^2}{|\vec{r_i}-\vec{r_j}|}
\end{array}\right.
\end{equation}où $Z$ est le nombre atomique (soit le nombre d'électrons), $\hat{H_i}$ contient le terme de 
répulsion du noyau et $\hat{V_{ij}}$ contient la répulsion entre les électrons. Ceci donne lieu à une 
fonction d'onde $\psi(1,2,\dots Z)$ totalement antisymétrique. \\

Afin d'interpréter de champ moyen, nous pouvons ré-écrire l'Hamiltonien comme (ajout et suppression du 
même terme)
\begin{equation}
\hat{H} =\sum_i\left( \frac{p_i^2}{2m}-\dfrac{Ze^2}{r_i}+V(\vec{r_i}\right)+\underbrace{\sum_i\left(\sum_{j>i} 
\dfrac{e^2}{|\vec{r_i}-\vec{r_j}|}-V(\vec{r_i}\right)}_{\approx 0}
\end{equation}
Le but sera alors d'approcher $V(\vec{r_i})$.


\subsubsection{Méthode variationnelle}
Soit $W$, une fonctionnelle du \textit{ket} $\ket{\phi}$ 
\begin{equation}
W\left[\ket{\phi}\right] = \dfrac{\bra{\phi}\hat{H}\ket{\phi}}{\bra{\phi}\ket{\phi}},\qquad\qquad\Leftrightarrow
\qquad\qquad \min_{\ket{\phi}} W\left[\ket{\phi}\right] = E_0
\end{equation}
Le théorème de Ritz nous informe que cette fonctionnelle est stationnaire pour tous les états propres:
\begin{equation}
\delta W\left[\ket{\phi}\right] = 0\qquad\qquad\Leftrightarrow\qquad\qquad \hat{H}\ket{\phi}=E\ket{\phi}
\end{equation}
Nous allons extrémiser sous la contrainte de normalisation à l'aide des multiplicateurs de Lagrange
\begin{equation}
\begin{array}{lll}
&\DS\delta[\bra{\phi}\hat{H}\ket{\phi}-E\bra{\phi}\ket{\phi}] &=0\vspace{2mm}\\
\Leftrightarrow&\bra{\delta\phi}\hat{H}\ket{\phi} + \bra{\phi}\hat{H}\ket{\delta \phi} - E\bra{\delta\phi}\ket{\phi}-E
\bra{\phi}\ket{\delta\phi}&=0\vspace{2mm}\\
\Leftrightarrow&\DS \bra{\delta\phi}\hat{H}-E\ket{\phi}+\bra{\phi}\hat{H}-E\ket{\delta\phi}&=0
\end{array}
\end{equation}
où nous avons utilisé le fait que $\hat{E}$ et $E$ permutent. Considérons le conjugué de cette expression : celui-ci
doit être valable $\forall \ket{\delta\phi}$ :
\begin{equation}
\bra{\delta\phi}\hat{H}-E\ket{\phi}+\bra{\delta\phi}\hat{H}-E\ket{\phi}^*\qquad\forall\ket{\phi}
\label{eq:NConj}
\end{equation}
où $\hat{H}$ est hermitien. Ceci étant vrai $\forall\ket{\phi}$, procédons au changement de variable
$\ket{\delta\phi} = i\ket{\delta\phi}$
\begin{equation}
-i\bra{\delta\phi}\hat{H}-E\ket{\phi}+i\bra{\delta\phi}\hat{H}-E\ket{\phi}^*\qquad\forall\ket{\phi}
\end{equation}
Après division par $-i$
\begin{equation}
-\bra{\delta\phi}\hat{H}-E\ket{\phi}+\bra{\delta\phi}\hat{H}-E\ket{\phi}^*\qquad\forall\ket{\phi}
\label{eq:OConj}
\end{equation}
Comme \eqref{eq:NConj} et \eqref{eq:OConj} sont valable $\forall\ket{\phi}$, nous pouvons les sommer. 
On en tire que ($\forall\ket{\phi}$)
\begin{equation}
\bra{\delta\phi}\hat{H}-E\ket{\phi}=0\qquad\qquad\Rightarrow\qquad\qquad \hat{H}\ket{\phi}=E\ket{\phi}
\end{equation}
Ce résultat nous montre qu'il n'est pas nécessaire de tenir compte du complexe conjugué, celui-ci n'étant 
plus présent dans cette dernière expression.


\subsubsection{Hartree et Hartree-Fock}
La seule différence entre ces deux méthodes se situe dans la façon d'écrire. Pour Hartree la 
fonction d'essai est un produit d'état (non symétrique, car fermions)
\begin{equation}
\ket{\psi} = \ket\dots\ket\dots\dots \ket\dots
\end{equation}
Alors que pour Hartree-Fock $\ket\psi$ est écrit sous la forme d'un déterminant de Slater.






\subsection{Méthode de Hartree}
Pour cette méthode, on considère que
\begin{equation}
\psi(1,2,\dots)=\phi_1(1)\phi_2(2)\dots\phi_N(N)
\end{equation}
où
\begin{equation}
\phi_i(i) = \phi_i(\vec{r_i})\chi(n_i) \quad \dashrightarrow \quad\phi_i(\vec{r_i})
\end{equation}
Nous ne tenons pas compte du spin ici afin d'alléger le formalisme.\\

Comme discuté dans la précédente sous-section, nous allons maximiser sous contrainte de 
normalisation sauf que cette fois nous n'avons plus une, mais $N$ contraintes, chacune 
des fonction du produit devant être normalisés :  nous avons $N$ fonctions d'onde
\begin{equation}
\langle\hat{H}\rangle = \bra{\phi_1\ \phi_2\ \dots\ \phi_N}\sum_i \hat{H_i}+\sum_{i>j}V_{ij}\ket{\phi_1\ \phi_2\ \dots\ \phi_N}
\end{equation}
La première sommation peut facilement se résoudre, ne portant que sur $i$\footnote{Pas d'indice $i$ sur 
$\hat{H}$ car celui-ci doit être symétrique et il est forcément le même pour toutes les fonctions propres.}
\begin{equation}
\langle\hat{H}\rangle = \sum_i\bra{\phi_i}\hat{H}\ket{\phi_i} + \sum_{i>j}\bra{\phi_i\phi_j}V_{ij}\ket{\phi_i\phi_j}
\end{equation}
Dans la base position, ceci est équivalent à
\begin{equation}
\langle\hat{H}\rangle = \sum_i \int d\vec{r}\ \phi_i^*(\vec{r})\left(-\frac{\hbar^2\nabla^2}{2m}-\frac{Ze^2}{r}\right)
\phi_i(\vec{r}) + \sum_{ij} \underline{\int d\vec{r}d\vec{r'}\ \phi_i^*(\vec{r})\phi_j^*(\vec{r'})\frac{e^2}{|r_i-r_j|}\phi_i(\vec{r})\phi_j(\vec{r'})}
\end{equation}
Le terme souligné porte le nom d'\textbf{intégrale de Coulomb} qui est parfois notée
\begin{equation}
\mathcal{J}_{ij} = \iint d\vec{r}d\vec{r'}\ |\phi_i(\vec{r})|^2|\phi_j(\vec{r'})|^2\frac{e^2}{|r_i-r_j|}
\end{equation}
Revenons à notre fonction de Lagrange
\begin{equation}
\delta\left[\langle\hat{H}\rangle - \sum_{i=1}^N \varepsilon_i\bra{\phi_i}\ket{\phi_i} \right] = 0
\end{equation}
 Nous avons montré précédemment que le terme du complexe conjugué se 
simplifiait : il  n'est nécessaire d'appliquer la différentiation uniquement sur le \textit{bra}
\begin{equation}
\sum_i\bra{\delta\phi_i}\hat{H}\ket{\phi_i} + \sum_{j<i} \bra{\delta\phi_i\phi_j}\hat{V}\ket{\phi_i\phi_j}
+ \sum_{j<i} \bra{\phi_i\delta\phi_j}\hat{V}\ket{\phi_i\phi_j} - \sum_{i=1}^N \varepsilon_i\bra{\phi_i}\ket{\phi_i} 
+ \underbrace{c.c.}_{\text{inutile}} = 0\qquad \forall \ket{\delta\phi_i}
\end{equation}
Le terme de potentiel peut être ré-écrit (en utilisant sa symétrie et la permutation d'indices)
\begin{equation}
\sum_{j<i} \bra{\phi_i\delta\phi_j}\hat{V}\ket{\phi_i\phi_j} = \sum_{j>i} \bra{\phi_j\delta\phi_i}\hat{V}\ket{\phi_j\phi_i} = \sum_{j>i} \bra{\delta \phi_i\phi_j}\hat{V}\ket{\phi_i\phi_j}
\end{equation}
On peut alors ré-écrire l'équation comme
\begin{equation}
\sum_i \bra{\delta \phi_i}\hat{J}\ket{\phi_i} + \sum_{j\neq i} \bra{\delta\phi_i\phi_j}V\ket{\phi_i\phi_j} - 
\sum_i\varepsilon_i\bra{\delta \phi_i}\ket{\phi_i} = 0
\end{equation}
Nous obtenons alors $N$ équations de Schrödinger, une pour chaque $\bra{\delta\phi_i}$. Comme ceci doit être vrai 
$\forall i$, nous pouvons dire que\footnote{On "se débarrasse" des $\bra{\delta\phi_i}$.}
\begin{equation}
\forall i :\qquad \hat{H}\ket{\phi_i} + \sum_{j\neq i} \bra{\bullet \phi_j}V\ket{\phi_i\phi_j} = \varepsilon_i\ket{\phi_i}=0
\end{equation}
où $\bullet$ est une notation efficace pour dire que l'on  n'a pas "fermé" la première particule (voir note en 
bas de page). On observe un terme additionnel (?). Le système de Hartree va ainsi être un système de $N$ équations
\begin{equation}
\left(-\frac{\hbar^2V^2}{2m}-\frac{Ze^2}{r}\right)\phi_i(\vec{r}) + \sum_{j\neq i} \int d\vec{r'}\ \phi_j^*(\vec{r'})\frac{e^2}{|\vec{r}-\vec{r'}|}\phi_i(\vec{r})\phi_j(\vec{r'})
\end{equation}
où $\phi_j^*(\vec{r'})$ correspond à la probabilité que l'électron $j$ soit en $\vec{r'}$. On peut encore écrire
\begin{equation}
\left[-\frac{\hbar^2\nabla^2}{2m}-\underbrace{\frac{Ze^2}{r}+\sum_{i\neq j}\int d\vec{r'}\frac{e^2}{|\vec{r}-
\vec{r'}|}|\phi_j(\vec{r'})|}_{\equiv V_i(\vec r)}\right]\phi_i(\vec{r})=\varepsilon_i\phi_i(\vec{r})
\end{equation}
où $\frac{Ze^2}{r}$ est le terme de répulsion du noyau et $\frac{e^2}{|\vec{r}-
\vec{r'}|}|\phi_j(\vec{r'})|$ le terme de répulsion entre électron. Comme nous l'avions annoncé, tout se 
déroule exactement comme si les électrons ne sont pas couplé et que chacun d'entre eux "existe seul dans son champ 
moyen" constitué de la répulsion du noyau et des autres électron\footnote{On parle parfois de "screening effect".}. Le point clé est que $V_i(\vec{r}$ est auto-constitant. En effet, nous avons besoin de résoudre Schrödinger pour chaque électron et pour se faire, il faut connaître $V_i$ et pour cela il est nécessaire de connaître $\phi_j$ qui est inconnu. On va alors procéder par itération en partant d'un état initial pour chaque électron
\begin{equation}
\ket{\phi_1^{\text{init}}}\ket{\phi_2^{\text{init}}}\dots\qquad\rightarrow\qquad V_1(\vec{r})V_2(\vec{r})\dots\qquad\rightarrow\qquad \ket{\phi_1}\ket{\phi_2}\dots
\end{equation}
Se faisant, on pourra calculer les $N$ potentiels, résoudre Schrödinger, trouver un $\phi_i$ et recommencer le processus jusqu'à observer une convergence.\\

Si l'on procède à une mesure expérimentale du spectre avec cette méthode, on se rend rapidement compte qu'il y a une légère divergence entre la théorie et la pratique et ceci parce qu'il "manque un terme" à cette relation. Comme nous l'avions annoncé, la méthode de Hartree n'est présentée ici que dans le but (pédagogique) d'introduire une méthode plus cohérente, celle d'Hartree-Fock.


\subsection{Méthode de Hartree-Fock}
La fonction est également totalement antisymétrique mais donnée cette fois par un déterminant de Slater 
\begin{equation}
\psi(1,2,\dots, N) = \frac{1}{N!}\left|\begin{array}{ccc}
\phi_1(1)&\dots&\phi_1(N)\\
\vdots&\ddots&\vdots\\
\phi_1(N)&\dots&\phi_N(N)
\end{array}\right|
\end{equation}
où le spin est également négligé. Comme pour la méthode d'Hartree, nous allons utiliser la méthode des extréma de 
Lagrange. L'idée est identique, l'expression de la valeur moyenne de $\hat{H}$ sera cependant plus compliquée.
On va aussi utiliser la méthode des extrama de Lagrange. L'idée est la même mais la valeur moyenne de $\hat H$ sera plus compliquée
\begin{equation}
\langle\hat{H}\rangle = \left\{\frac{1}{\sqrt{N!}}\sum_p(-1)^p \bra{p_{(1)},p_{(2)}\dots p_{(n)}}\right\}\left(
\sum_i^N \hat{H}_i+\sum_{i>j}^{\frac{N(N-1)}{2}} V_{ij}\right)\left\{\frac{1}{\sqrt{N!}}\sum_p^N (-1)^p
\ket{p_{(1)},p_{(2)}\dots p_{(n)}}\right\}
\end{equation}
où $\ket{p_{(1)},p_{(2)}\dots p_{(n)}}$ est donnée par le déterminant de Slater. \\

Nous allons regardé le terme "un corps" numéro $i$ en considérant pour l'instant seulement le $\hat{H}_i$. Il peut y 
avoir toute sorte de permutation dans le \textit{bra} et le \textit{ket} mais si celles-ci ne sont pas identiques, 
cela donnera zéro. Les seuls contributions viendront du cas ou le \textit{bra} est bien le vecteur dual du 
\textit{ket} associé
\begin{equation}
\frac{1}{N!}\sum_p \bra{p_{(1)},p_{(2)}\dots p_{(n)}}\hat{H}_i\ket{p_{(1)},p_{(2)}\dots p_{(n)}} = \frac{1}{N}
\sum_j\bra{\phi_j}\hat{H}\ket{\phi_j}
\end{equation}
Ceci peut également être interpréter comme le fait de regardes, pour toutes les permutations, la valeur moyenne 
de l'énergie de la particule numéro $i$. Ici nous avons regardé seulement pour la particule $i$ mais si l'on 
s'intéresse au "full one body term" nous aurons alors $N$ termes :
\begin{equation}
\sum_j\bra{\phi_j}\hat{H}\ket{\phi_j}
\end{equation}
La façon de traiter l'anti-symétrie ne diffère en rien pour le terme à un corps\footnote{Ceci est rassurant, 
celui-ci étant identique dans les deux méthodes}. Ce qui va changer, c'est le terme à deux corps.\\

Considérons un terme donné ($i,j$ fixé). $V_{ij}$ n'agit que sur l'électron $i$ et $j$
\begin{equation}
\frac{1}{N!}\sum_p \bra{p_{(1)},p_{(2)}\dots p_{(n)}} V_{ij} \left\{\ket{\text{same}} - \ket{\text{échange $i\leftrightarrow j$}}\right\}
\end{equation}
où $\ket{\text{same}}$ rappelle que le seul terme qui survit est celui dont les permutations seront identiques. Ceci 
est fait pour $i$ et $j$ mais ce n'est qu'une dénomination arbitraire qui peut bien sûr être échangée. LE signe 
négatif
On fait exactement le même raisonnement Le - vient de l'antisymétrie. Ceci est pour i et j, mais on peut le remplacer ce 
n'est qu'un nom. 

Two body Term
24=
\begin{equation}
\begin{split}
\frac{1}{2}\sum_{i\neq j} &\left(\bra{\phi_i\phi_j}V\ket{\phi_i\phi_j}-\bra{\phi_i\phi_j}V\ket{\phi_j\phi_i}\right)\\
\left\langle H\right\rangle &= \sum_i H_i + \frac{1}{2} \sum_{i\neq j} (J_{ij}-K_{ij})\\
H_i &\equiv \bra{\phi_i}H\ket{\phi_i}
\end{split}
\begin{cases}
J_{ij} \equiv \overset{\text{Coucomb integral}}{\bra{\phi_i\phi_j}V\ket{\phi_i\phi_j}}\\
K_{ij} = \underset{\text{Excahnge integral}}{\bra{\phi_i\phi_j}V\ket{\phi_j\phi_i}}
\end{cases}
\end{equation}

25=

\begin{equation}
\begin{split}
\delta \left[\left\langle H\right\rangle - \sum_{i=1}^{N} c_i \bra{\phi_?}\ket{\phi_?}\right] &= 0\\
\sum \bra{\delta\phi_i}H\ket{\phi_i} + \overset{\displaystyle\ 1}{\cancel{\frac{1}{2}}} \sum_{i\neq j} &\sum \bra{\delta\phi_i\phi_j}U\ket{\phi_i\phi_j} - \overset{\displaystyle\ 1}{\cancel{\frac{1}{2}}} \sum_{i\neq j}\sum \bra{\sqrt{\phi_i}\phi_j}V\ket{\phi_j\phi_i}\\
-\sum_i \varepsilon_i\bra{\delta \phi_i}V\ket{\phi_i} +\frac{1}{2}\sum_{i\neq j}\sum &\cancel{\bra{\phi_?\delta\phi_j}V\ket{\phi_?\phi_j}} - \frac{1}{2}\sum_{i \neq j} \sum \cancel{\bra{\phi_i\delta\phi_j}\ket{\phi_j\phi_i}},\quad \forall \ket{\delta\phi}\ |\delta\\
V \text{ syn : } \phi_j &\delta\phi_?\,V\,\phi_j\phi_i = \delta\phi_?\phi_j\,V\,\phi_?\phi_j
\end{split}
\end{equation}

26=
\begin{equation}
\forall i\ :\ H(\phi_i) + \sum_{j\neq i}\bra{\bullet\ \phi_j}V\ket{\phi_i\phi_j}- \sum_{j\neq i} \bra{\bullet\ \phi_j}V\ket{\phi_j\phi_i} = \varepsilon_i\ket{\phi_?}
\end{equation}

27=
\begin{equation}
\begin{split}
\forall i\ :\ \left(-\frac{\hbar^2\Delta^2}{2m}-\frac{Z e^2}{r}\right)\phi_i(\vec r) + \overbrace{\left[\sum_{j\neq i}\int d\vec r\,\phi_j(\vec r)^*\frac{e^2}{|\vec r - \vec{r'}|}\phi_?(r)\phi_j(\vec {r'}) |\phi_j(\vec r)|^2\right]}^{\text{il y a du texte}} \\- \begin{cases}
\sum_{?} \int d\vec r \phi_j^*(\vec r)\frac{e^2}{|\vec r - \vec{r'}|} \phi_j(\vec r)\phi_i(\vec{r'})\\
\text{verso}
\end{cases}
\end{split}
\end{equation}


\begin{equation}
24
\end{equation}


SI la fonction d'état est un slater et pas un simple product, ona un terme en plus. On a tjs le terme e coulomb mais on a 
un terme en plus qui peut pas etre une interaction car on a le meme H qu'avant.
\begin{equation}
25
\end{equation}
On a donc une équation par i
\begin{equation}
26
\end{equation}
On peut redre cette expression plus spécifique dans le cadre 'un atome
\begin{equation}
27
\end{equation}
Le terme en - c'est cet exchange.\\

On a donc N schrodinger équation (non interacting electrons).  On a pas besoin de les résoudre tous à la fois, on la 
résou pour chaque phi i mais cette équation est couplée à cause du terme d'interaction 'change : il faut la résoudre 
ittérativement exactement comme précéemment. C'est donc également une équation auto consistante. C'est une approx 
car on a ce champ d'interaction moyen.