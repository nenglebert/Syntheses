\documentclass	[11pt, a4paper, openany]{book}
\usepackage[utf8]{inputenc}
\usepackage[francais]{babel}
\usepackage[T1]{fontenc}
\usepackage{amsmath}
\usepackage{amsfonts}
\usepackage{amssymb}
\usepackage{lmodern}
\usepackage{amsthm}

\usepackage{graphicx} % ajout image
\usepackage[tt]{titlepic}%Centre le titre

\usepackage{shorttoc} %Permet d'avoir de petites tables des matières


\usepackage{shapepar} %texte en keur
\usepackage{siunitx} %S.I.

\usepackage{graphicx}
\usepackage{caption} %Permet d'ajouter des légendes en images sans les mettre en float + ds la marge
\usepackage{delarray} % Belles matrices

\usepackage{fancyhdr} %Permet de modifier l'entête & footer
\usepackage{bbding} % Note marge
\usepackage{todonotes}
\usepackage{wrapfig}

\pagestyle{headings} % Titre du ch et numéro page dans l'entete
\usepackage{fullpage} %Utilise toute la page




\newcommand{\cst}{\text{cst}}
\newcommand{\E}{\vec E}
\newcommand{\F}{\vec F}

\newcommand{\questpm}[3]{#1. \textbf{#3} (p.#2)}
\newcommand{\exerc}[2]{\textbf{\Large Exercice #1\normalsize \\#2}}
\newcommand{\dif}{\mathrm{d}}
\newcommand{\comment}[1]{}
\newcommand{\rot}{\text{rot}\,}
\newcommand{\divv}{\text{div}\,}
\newcommand{\phas}[1]{\underline{#1}}
\newcommand{\RE}{\text{Re}}
\DeclareMathOperator{\arccot}{arccot}
\newcommand{\com}[1]{}

%\newcommand{\oiint}{\int\!\!\!\!\!\:\!\!\!\;\!\!\subset\!\!\supset\!\!\!\:\!\!\!\!\!\int}
\newcommand{\oiint}{\int\!\!\!\!\!\!\! \:\!\subset\!\!\supset\!\!\!\!\!\!\!\int}



\begin{document}
\renewcommand{\proofname}{Démonstration}
\frontmatter
\begin{titlepage}
\begin{center}	
	
	\newcommand{\HRule}{\rule{\linewidth}{0.5mm}}   			%Titre en gros
	\includegraphics[scale=0.11]{logo2.png}~\\[1cm]				%Logo

	\textsc{\LARGE Université Libre de Bruxelles}\\[1.5cm]
	\textsc{\Large Synthèse}\\[0.5cm]

	\HRule \\[0.4cm]
	{ \huge \bfseries Physique générale \\[0.4cm] }


	\HRule \\[1.5cm]
		\begin{minipage}{0.4\textwidth}
		\begin{flushleft} \large
		
		\emph{Auteur:}\\
			Nicolas \textsc{Englebert}\\
			\end{flushleft}
			\end{minipage}
			\begin{minipage}{0.4\textwidth}
			\begin{flushright} \large
			\emph{} \\		
			\textsc{}
			\end{flushright}
		\end{minipage}

	\vfill

% Bottom of the page
{\large 2014 - 2015 \\ Document non-officiel}

\end{center}
\end{titlepage}





\part{Introduction}
\chapter{A propos du document}
J'ai écrit ce document durant l'année 2013/2014, pour ne pas devoir reparcourir à chaque fois l'intégralité des syllabus lors de l'étude.\\

Même si je le juge assez complet, il n'est pas totalement exhaustif. Certaines démonstrations n'y figurent pas, mais bien souvent c'est mentionné! \\
Même si ce document peut servir à certains comme référence supplémentaire, faire sa propre synthèse est essentiel pour la réussite de ce cours ! Cela permet de s'assurer que les principes sont assimilés et compris.\\

Ce document comprend un résumé de chaque syllabus du cours de \textit{Physique Générale}, un rapport de laboratoire et les réponses aux questions préparatoires (celles à préparer avant chaque TP)(J'insiste néanmoins qu'il faut s'entraîner à les faire seule et utiliser ce document comme "vérification", sans quoi ce ne sera pas très utile!)\\ 
Ce document s'organise de la sorte :
\begin{enumerate}
\item Première partie - A propos
\begin{itemize}
\item Vous êtes occupé de la lire !
\end{itemize}
\item Seconde partie - Résumé du cours
\begin{itemize}
\item Thermodynamique
\item Électrostatique
\item Magnétostatique
\item Electromagnétisme
\end{itemize}
\item Troisième partie - Rapport de Labo
\begin{itemize}
\item Voltmètre et Ampèremètre
\item Charge et décharge du condensateur
\item Lentilles minces
\end{itemize}
\item Quatrième partie - Questions préparatoires aux tests de TP
\end{enumerate}

Je voudrais également remercier Cédric \textsc{Hannotier} et Pierre \textsc{Baudoux} pour leur aide à la réalisation des questions préparatoires, ainsi que toute ma série 3C wOOp wOOp (Sacha \textsc{Némégaire}, Nicolas \textsc{Dardenne}, Maxime \textsc{Libert}, Enes \textsc{Ulusoy}, Yacine \textsc{Tchabi}, Kasidi et Ahmet \textsc{Amiro}, Hajar \textsc{Salmi}, Élisabeth \textsc{Gruwé}, Julie \textsc{Vijghen}, Wail \textsc{Benaissa}, Fanny \textsc{Donny}, Görkem \textsc{Özcam} et Mariame \textsc{Sacko}) ! \\

Il se peut que quelques erreurs se soient glissées dans ce document, si c'est le cas contactez-moi ! (Par mail ou \textit{Nico Englebert}, sur Facebook).\\

Bonne étude ! 
\begin{flushright}
Nicolas \textsc{Englebert}\\
\includegraphics[scale=0.8]{gmail.png}
\end{flushright}

\mainmatter
\part{Synthèse du cours}
\chapter{Thermodynamique}
\section{Introduction}
Une brève introduction aux degrés de libertés. Une lecture attentive suffira.

\section{La pression}

\begin{wrapfigure}[10]{l}{4cm}
\includegraphics[width=4cm]{th/image1.png}
\captionof{figure}{Pression hydrostatique}
\end{wrapfigure}

Dans notre cas, le poids est la force engendrée par la colonne d'eau. Comme celle-ci est liquide, sont poids vaut $\underbrace{\rho V}_m  g$.\\
Par définition, de la pression : $P \equiv \frac{Poids}{Surface}$ on peut trouver la pression hydrostatique \footnote{Que pour des liquides incompressibles} :
\begin{equation}
\fbox{$P_{hydrostatique} = \rho g h $}
\end{equation}
Les unités de la pression sont les Pascals $\left(Pa = \dfrac{N}{m^2}\right)$. Notons également que 1 bar = $10^5\ Pa$ et que $1\ atm = 1,013\ bar$.

\subsection{Le baromètre}
\begin{wrapfigure}[10]{r}{3cm}
\includegraphics[scale=0.3]{th/image2.png}
\captionof{figure}{Baromètre}
\end{wrapfigure}
Son but est de mesurer la pression atmosphérique. La colonne d'air au dessus de nous à une hauteur de 100km. Si l'air était du mercure, cette colonne ferait 76mm de hauteur. \\

Ici l'air n'est pas un liquide incompressible, il faut donc passer par le formalisme intégral : 
\begin{equation}
Poids\ =\ \rho_{air}.g.s.dh
\end{equation}
Connaissant la définition de la pression, on peut calculer celle-ci
\begin{equation}
\int_0^h \frac{\rho_{air}g.S.dh}{S} = \rho_air.g.\Delta h
\end{equation}

\section{La température}
\subsection{Dilatation thermique}
Utilisation de la dilatation dans les thermomètres (Même si aujourd'hui on utilise de l'alcool, initialement c'était avec un gaz).

\subsection{Température absolue}
A l'aide d'un thermomètre à gaz à volume constant, \textit{Amontons} a mis en évidence une relation linéaire entre la pression et la température.
\begin{center}
\includegraphics[scale=0.15]{th/image3.png}
\captionof{figure}{Thermomètre à gaz à volume constant}
\end{center}
\begin{equation}
P = \alpha T + P_0
\end{equation}
où $P_0$ est la température à $0 deg. C$.\\
En extrapolant les résultats par une série de mesure répétitive, \textit{Amontons} a pu mettre en évidence l'existence du zéro absolu (-273,15 deg. C)
\begin{center}
\includegraphics[scale=0.3]{th/image4.png}
\captionof{figure}{Extrapolation du zéro absolu}
\end{center}

\section{Comportement des gaz}
\subsection{Loi de Gay-Lussac (Amontons)}
Cette loi est la même que celle découverte par \textit{Amontons}. Elle dit que la pression est proportionnelle à la température à volume constant.
\begin{equation}
P \propto T|_{V = cste}
\end{equation}

\subsection{Loi de Charles}
\begin{wrapfigure}[7]{l}{3cm}
\includegraphics[scale=0.23]{th/image5.png}
\captionof{figure}{Loi de Charles}
\end{wrapfigure}
Par le placement d'un poids libre de se mouvoir, la pression est maintenue constante. Ceci permis l'étude de l'élévation du piston à pression constante et de dériver la Loi de Charles :
\begin{equation}
V \propto T|_{P = cste}
\end{equation}

\subsection{Loi de Boyle et Mariotte}
Ces deux scientifiques ont mis en évidence une relation entre le volume et la pression à température constante :
\begin{equation}
PV = cste|_{T = cste}
\end{equation}

\subsection{Loi des gaz parfaites}
En rassemblant les trois lois obtenues ci-dessus, on obtient : 
\begin{equation}
PV = \alpha T
\end{equation}
Si l'on devait doubler le nombre de particules de gaz, la constante $\alpha$ serait proportionnelle au nombre de particules multiplié par une constante ($\alpha \propto N.cste$).\\
Pour des raisons détaillée ci-dessous, $\alpha = N.k_B$ de la sorte que
\begin{equation}
\fbox{$ PV = N.k_BT$}
\end{equation}

\subsection{Loi d'Avogadro}
Le but est de déterminer la valeur de la \textit{constante de Boltzman}, $k_B$ grâce à la loi d'Avogadro : Si la pression et la température sont constantes, la volume $V$ occupé par $N$ particules est le même.\\ Ce n'est donc pas le nombre d'atomes qui détermine le volume. En isolant $V$ :
\begin{equation}
V = k_B . \frac{NT}{P}
\end{equation}
Or, $\frac{NT}{P}$ donnant une valeur unique pour un gaz, $k_B$ est une constante universelle. \\En posant $P=1atm, T = 237,15K, V = 22,4 L$ on trouve :
\begin{equation}
k_B = 1,38.10^{-23}\ \frac{Pa.m^3}{K}\ \ \  \left(= \frac{N.m^3}{m^2.K} = \frac{J}{K}\right)
\end{equation}

\subsection{Constante universelle des gaz parfaits}
Le nombre de moles étant donné par $n = N/N_A$, on déduit que $N = n.N_A$. En substituant dans la loi des gaz parfaits\footnote{Valabie si $P$ pas trop grande et $T$ pas trop petite} :
\begin{eqnarray}
PV &=& n.\underbrace{N_A.k_B}_R T\\
PV &=& nRT
\end{eqnarray}
où $R\ =\ 8,314\ J/K$.

\section{Théorie cinétique des gaz parfaits}
Un gaz est un ensemble de particules en mouvement incessant (dit parfait s'il $\nexists$ d'interactions entres particules). Chaque particule cause un impact générant une pression.
\begin{equation}
Force\ d'impact\ : f = m\frac{dv}{dt}
\end{equation}

\subsection{Incidence normale}
Lors d'un impact, la particule s'écrase et regagne une accélération.
\begin{center}
\includegraphics[scale=0.4]{th/image6.png}
\captionof{figure}{Incidence normale : Position - Vitesse - Accélération}
\end{center}
En prenant l'hypothèse que l'accélération est constante $\ddot{x} = a = cste$ on peut trouver la vitesse par intégration  : $\dot{x} = at + C$.\\
Grâce à $f = ma$, on peut calculer la force d'impact (on suppose $dv/dt = \partial v / \partial t$ (hypoth. de modél.comme $V = cste$)) sachant que la vitesse vaut = $v = -v_0 - v_0 = -2v_0$ (voir schéma ci-dessus)
\begin{equation}
f = m\ddot{x} = m\frac{dv}{dt} = m\frac{\partial v}{\partial t} = -m\frac{2v_0}{\partial t}
\end{equation}
Comme $\overline{f_i} = -f$, la force d'impact moyenne vaut :
\begin{equation}
\overline{f_i} = m \frac{2v_0}{\partial t}\vec{1_x}
\end{equation}
Le signe est positif car la force se dirige vers les $x > 0$  (Action/réaction)

\subsection{Incidence oblique}
Cela ne change rien, il suffit de considérer une vitesse vectorielle pour s'en rendre compte
\begin{equation}
\vec v = v_x \vec{1_x} + v_y \vec{1_y} + v_z \vec{1_z}
\end{equation}
Comme seule la composante en $x$ varie, nous retrouverons bien le résultat précédent 
\begin{equation}
f_i = m\frac{2v_x}{\partial t}
\end{equation}

\subsection{Force d'impact périodique}
\begin{wrapfigure}[7]{l}{3cm}
\includegraphics[scale=0.4]{th/image7.png}
\captionof{figure}{Aller/retour}
\end{wrapfigure}
On veut connaître le temps d'un aller-retour. Connaissant la formule du MRU $d = vt \Rightarrow t = \frac{d}{v}$.
$\\
\left\{\begin{array}{l}
d = L\\
v = v_x = v.\cos\theta
\end{array}\right.
$ \ \ \ $\Rightarrow \Delta t = \dfrac{L}{v.\cos\theta}$
Comme nous nous intéressons à un aller/retour, il convient de considérer une longueur double :
\begin{equation}
\Delta t = \frac{2L}{v_x}
\end{equation}
\subsection{Force continue}
Il y a beaucoup de particules : impossible de tous les traiter individuellement. On va alors passer au concept de la \textit{force continue}.

\subsection{Force d'impact moyenne}
Comment éliminer $\delta t$ (le temps du choc) ? On imagine qu'une particule tape périodiquement sur un piston.
\begin{itemize}
\item Le piston monte : $(f - mg)$
\item Le piston descend (entre impact) : $(-mg)$
\end{itemize}
Dans une position d'équilibre ou le poids sur le piston reste fixe, la gravité contre parfaitement la montée du piston 
\begin{equation}
g = f_i - mg
\end{equation}
On étudiant les extrêmes, on peut trouver la vitesse moyenne nulle, c'est à dire une position d'équilibre.\\
$\left\{\begin{array}{l}
Variation\ totale\ v = at  \Rightarrow \dfrac{1}{M}\left(f_i - Mg\right)\delta t\\
Variation\ pensenteur\ seule  \Rightarrow g(\Delta t -  \delta t)
\end{array}\right.$\\
En considérant la situation d'équilibre
\begin{eqnarray}
\frac{1}{M}(f_i - mg)\delta t &=& g(\Delta t - \delta t)\\
(f_i - mg)\delta t &=& mg(\Delta t - \delta t)\\
f_i\Delta t - mg\delta t &=& - mg\delta t + mg \Delta t\\
f_i \delta t &=& Mg\Delta t
\end{eqnarray}
Cette dernière équation traduit que l'action de la force d'impact sur un temps $\delta t$ équilibre la gravité sur le temps $\Delta t$.\\

Le piston étant à l'équilibre, on peut dire que sa force d'impact moyenne $\overline{f_i} = Mg$. On peut voir ceci comme la force que je dois appliquer pour lever le piston, la force moyenne à fournir.  On a donc (Sur base de la dernière équation)
\begin{equation}
\overline{f_i} = f_i \frac{\delta t}{\Delta t}
\end{equation}
où $f_i$ est la force intermittente.

\subsection{Force totale}
$\left\{\begin{array}{l}
f_i = m\dfrac{2v_x}{\delta t}\\
\overline{f_i} = fi\dfrac{\delta t}{\Delta t}
\end{array}\right.$\ \ \ $\Rightarrow \overline{f_i} = m\dfrac{2v_x}{\delta t}\frac{\delta t}{\Delta t}\ \ \ \Rightarrow \overline{f_i} = m\frac{2v_x}{\Delta t}$\\
Connaissant $\Delta t = 2L/v_x$, on trouve : 
\begin{equation}
\overline{f_i} = \frac{mv_x^2}{L}
\end{equation}
Pour $N$ particules (On multiplie et divise par $N$ pour faire apparaître la moyenne)
\begin{equation}
F_{tot} = \sum \frac{mv_x^2}{L} = \frac{Nm}{L}<v_x^2>
\end{equation}
$F_{tot}$ représente l'ensemble des force sur une période. On considérant les vitesses $v_x, v_y$ et $v_z$ isotropes :
\begin{equation}
<v^2> = <v_x^2> + <v_y^2> + <v_z^2>\ \ \ \ \ \ \ \ \Rightarrow \ \ \ <v_x^2> = \frac{1}{3}<v^2>
\end{equation}
Nous trouvons ainsi notre expression finale de la force totale 
\begin{equation}
f_{tot} = \frac{Nm}{3M}<v^2>
\end{equation}
\subsection{Calcul de la pression}
En se basant sur la définition de la pression
\begin{equation}
P = \frac{Nm}{3\underbrace{SL}_V}<v^2>\ \ \ \ \ \Rightarrow \ \ \ \ \ PV = \frac{Nm<v^2>}{3}
\end{equation}
\subsection{Théorie - expérience}
En égalant l'expression obtenue ci-dessus avec $PV = Nk_BT$, on peut retrouver l'expression de la température
\begin{equation}
T = \frac{m<v^2>}{3k_B}
\end{equation}
On retrouve implicitement l'expression de l'énergie cinétique, nous permettant de ré-écrire cette expression de la sorte
\begin{equation}
\fbox{$ T = \frac{2E_{cin}}{3k_B}$}
\end{equation}
Ceci démontre de façon formelle que la température est bien proportionnelle à l'énergie cinétique des particules.

\section{L'énergie thermique}
On va appeler l'énergie interne d'un gaz $U$ la somme de toutes les énergie cinétique d'un gaz. Sachant que $T = \dfrac{2E_{cin}}{3k_B}$ :
\begin{equation}
\fbox{$ U = N.E_{cin} = \frac{3}{2}Nk_BT$}
\end{equation}


\subsection{Expérience de Joule}
En faisant bouger de l'eau, on augmente son énergie cinétique causant une augmentation de la température $\Rightarrow$ augmentation de l'énergie interne, $U$.

\subsection{Échauffement et chaleur}
L'échauffement est un transfert \textit{collisionnel} d'énergie cinétique désordonnée. La chaleur \textbf{transmise} se note $Q\ [J]$.

\subsection{Équilibre thermique}
La propagation par transfert collisionnel homogène du aux mouvements erratiques se nomme l'\textit{équipartition}. Il conduit à une uniformisation de la température et défini le \textbf{Premier principe de la thermodynamique}
\begin{center}
\fbox{\textit{\textsc{Principe zéro :} équipartition de l'énergie thermique.}}
\end{center}

\subsection{Capacité calorifique}
Il s'agit de la quantité de chaleur apportée $Q$ sur la variation de température 
\begin{equation}
\Delta T : C_V = \dfrac{Q}{\Delta T}|_{v=cste} = \frac{3}{2}Nk_B
\end{equation}
Il faut $12.5J$ pour chauffer une mole de gaz \textbf{monoatomique} de $1K$.

\subsection{Gaz polyatomique}
Il y a plus de degré de liberté pour ce type de gaz. Il faudra dès lors plus de chaleur pour chauffer une mole de gaz car cette énergie ira à l'énergie de rotation, de vibration et de translation. En effet :
\begin{equation}
U \equiv N.E_{cin} + N.E_R + N.E_V
\end{equation}
Pour tenir compte de ce plus grand nombre de degré de liberté, il suffit d'utiliser cette expression :
\begin{equation}
U = \frac{n_d}{2}Nk_BT\ \ \ \ où \ \left\{\begin{array}{l}
n_d = 3 \ \ \ (gaz\ monoatomique)\\
n_d = 7 \ \ \ (gaz\ polyatomique)
\end{array}\right.
\end{equation}
Compte-tenu de ce résultat, la capacité \textit{molaire} à volume constant aura pour expression :
\begin{equation}
\fbox{$ C_V = \frac{n_d}{2}R$}
\end{equation}

\subsection{Capacité calorifique massique}
Par définition :
\begin{equation}
\fbox{$ Q \equiv c_v m \Delta T$}
\end{equation}
où $c_v$ est la quantité de chaleur nécessaire à l'élévation de $1K, 1kg$ de gaz à volume constant.
\begin{equation}
c_v = \frac{n_d}{2}\frac{R}{M_m}\ \ \ \left(\left[\frac{J}{kg.K}\right]\right)
\end{equation}

\subsection{Capacité calorifique de l'eau}
Il faut $4186\ J$ pour augmenter de $1\ K$ un kilogramme d'eau de 14.5 à 15.5 degré celsius.\\
On définit ainsi $1\ cal = 4.186J$ d'où $1\ kcal = 4186J$.
\begin{equation}
c_{H_2O} = 4186\ \frac{J}{kg.K}
\end{equation}
La grande capacité de l'eau est du à sa polarité : briser ses liaisons demande une énergie plus importante (Plus une capacité est grande, plus le $\Delta T$ sera en effet faible).

\subsection{Mécanisme de transfert de l'énergie thermique}
\subsubsection{La conduction}
\begin{wrapfigure}[11]{l}{3cm}
\includegraphics[scale=0.4]{th/image8.png}
\captionof{figure}{Conductivité thermique}
\end{wrapfigure}
\textit{Fourrier} a mis en évidence que le tranfert collisionnel est du à l'agitation thermique. Il a créé un dispositif permettant de connaître la capacité calorifique des matériaux grâce à la connaissance de celle de l'eau.\\

Pour se faire, il a fait un \textbf{bilan énergétique} : on regarde ce qui entre et ce qui sort sur un temps $\Delta t$. On notera $H$ la quantité de chaleur transférée par conduction par unité de temps. On remarque expérimentalement :\\
$\left\{\begin{array}{l}
H \propto \Delta T\\
H \propto S\\
H \propto \dfrac{1}{L}
\end{array}\right.$\\
On introduit dès lors une constante de proportionnalité, $k_T$ comme étant la \textit{conductivité thermique}.
\begin{equation}
\fbox{$ H = k_TS\frac{\Delta T}{L}$}
\end{equation}

\section{Transformations thermodynamiques}
\subsection{Système thermodynamique}
\subsubsection{Premier principe de la thermodynamique}
\begin{wrapfigure}[6]{r}{5cm}
\includegraphics[scale=0.4]{th/image9.png}
\captionof{figure}{Premier principe}
\end{wrapfigure}
Le premier principe de la thermodynamique se base sur la conservation de l'énergie, la quantité totale d'énergie n'est jamais modifiée.\\
\begin{center}
\fbox{\textsc{Premier principe :} $\Delta U = Q - W$}
\end{center}
La production d'un travail épuise l'énergie interne. On ne peut avoir un travail infini car la chaleur $Q$ ne l'est pas.\\
Tout système violant ce principe portera le doux nom de \textit{machine à mouvement perpétuel de première espèce.}

\subsection{État d'un système et transformation}
On utilise des variables d'état pour caractériser un gaz ($P, V, T, N$).

\subsection{La transformation isobare}
\begin{wrapfigure}[8]{l}{5cm}
\includegraphics[scale=0.34]{th/image10.png}
\captionof{figure}{Isobare}
\end{wrapfigure}
Il s'agit d'une transformation à pression constante. Le volume du gaz se verra modifié suite à une modification de la température. \\
Le calcul du travail est simple car la pression est constante 
\begin{equation}
W_{isobare} = F.\Delta L = P. \Delta V
\end{equation}
Quel est la relation entre $Delta T$ et $W$ ?\\
$\left\{\begin{array}{l}
T_i =  \frac{P}{Nk_B}V_i\\
T_f = \frac{P}{Nk_B}V_f\\
T_i - T_f = \Delta T
\end{array}\right.$\ \ \ \ \ $\Rightarrow\ \ \Delta T = \frac{P}{Nk_B}\Delta V = \frac{W}{Nk_B}$.

\subsubsection{Bilan détente isobare}
On sait : $\left\{\begin{array}{l}
W = P\Delta V\\
\Delta U = Q - W\\
Gaz\ parfait\ : \Delta U = \frac{n_d}{2}Nk_B\Delta T\\
\Delta T = \frac{W}{Nk_B}
\end{array}\right.$ d'où on tire : $Q = \frac{n_d}{2}W + W = W(\frac{n_d}{2}+1)$\\
Le rendement vaut donc 
\begin{equation}
r = \frac{W}{Q} = \frac{1}{1 + \frac{n_d}{2}}
\end{equation}

\subsubsection{Capacité calorifique molaire à pression constante}
Par définition 
\begin{equation}
Q \equiv C_p n \Delta T
\end{equation}
Que vaut $C_p$? Nous savons que $Q = \Delta U + P\Delta V \Rightarrow \frac{Q}{\Delta T} = \frac{\Delta U}{\Delta T} + \frac{P\Delta V}{\Delta T} \equiv C_p (1)$\\
Or $U = \frac{n_d}{2}RT \rightarrow \Delta U = \frac{n_d}{2}R\Delta T (2)$.\\
On réinjectant (2) dans (1), nous trouvons l'expression de la capacité molaire à pression constante.
\begin{equation}
C_p = \frac{n_d}{2}R + R = R(\frac{n_d}{2}+1)
\end{equation}
On remarque que, pour un gaz parfait, $C_p - C_v = R$. $C_p > C_v$ car il y a un travail mécanique à fournir en plus.


\subsection{La transformation isochore}
\begin{wrapfigure}[8]{l}{4cm}
\includegraphics[scale=0.24]{th/image11.png}
\captionof{figure}{Isochore}
\end{wrapfigure}
Caractérisée par un volume constant : $\Delta V = 0 \Rightarrow W = 0$.\\
Différence de température $T$ pour une chaleur $Q$ fournie ? Sachant que $Q = \Delta U + W$ où $W = 0$, on sait que $Q = \frac{n_d}{2}nR\Delta T$\footnote{On travaille avec des moles ($n$) ici !}.\\Nous avons donc :
\begin{equation}
\Delta T = \frac{2Q}{n_d Rn}
\end{equation}

\subsubsection{Travail d'une transformation quelconque}
\begin{wrapfigure}[8]{r}{3cm}
\includegraphics[scale=0.34]{th/image12.png}
\captionof{figure}{Transformation quelconque}
\end{wrapfigure}
L'expression du travail d'une transformation quelconque peut être vue approchée comme étant une suite d'isobare ($W = P\Delta V$) et d'isochore ($W = 0$) : $W = \sum P_n \Delta V_n$\\ En faisant tendre $\Delta V \rightarrow 0$ on retrouve la définition de l'intégrale (intégrale de circulation de la force transmise par le gaz)
\begin{equation}
W = \int_{V_i}^{V_f} P(V).dV
\end{equation}
\subsection{La transformation isotherme}
Il s'agit d'une transformation dite \textit{idéale}. Elle se base sur l'hypothèse de la présence d'une paroi mince et d'une transformation \textbf{lente} : le gaz à toujours la même température. 
\begin{center}
\includegraphics[scale=0.3]{th/image13.png}
\captionof{figure}{Exemple de transformation isotherme (gauche) et son diagramme PV (droite)}
\end{center}
Si l'on lève le piston de $\Delta l$ le volume augmente et comme $PV=cste$, la pression va diminuer.\\
Intuitivement, en levant le piston on \textit{amorti} les particules causant une diminution de la pression et de la température (comme l'$E_{cin}$ diminue).

\subsubsection{Bilan énergétique de la transformation isotherme}
Calculons tout d'abord le travail de cette transformation à l'aide de la relation obtenue ci-dessus ($T_r$ = température du réservoir).
\begin{equation}
W = \int_{V_i}^{V_f} P(V).dV\ \ où\ \ P = \frac{Nk_BT_r}{V}
\end{equation}
Après intégration, nous trouvons le travail d'une isotherme
\begin{equation}
W_{isoth} = Nk_BT_r \ln\left(\frac{V_f}{V_i}\right)
\end{equation}
Il n'y a pas de différence de température ($\Delta T = 0$) dans cette transformation $\Rightarrow \Delta U = \frac{n_d}{2}Nk_B\Delta T = 0$.\\
Dans une transformation isotherme $Q = W$ : le rendement est dès lors maximum.
\begin{equation}
r = \frac{W}{Q} = 1
\end{equation}

\subsection{La transformation adiabatique}
A l'inverse de la transformation isotherme, celle-ci se déroule \textbf{sans} échange de chaleur : $Q = 0$. C'est \textit{Pierre-Simon Laplace} qui a trouvé qu'il existe une relation particulière en l'absence d'échange de chaleur. Son génie est d'être passé par la décomposition infinitésimale.
\begin{eqnarray}
dU &=& dQ - dW\ \ \ où\ \ dQ = 0\\
  &=& - dW \ \ \ \ où\ \ dU = \frac{n_d}{2}Nk_B dt = - PdV
\end{eqnarray}
Sachant que (Loi des gaz parfaits) $P = \dfrac{Nk_BT}{V}$ on peut écrire 
\begin{eqnarray}
\frac{n_d}{2}Nk_Bdt = - \frac{Nk_BT}{V}dV\\
\frac{n_d}{2}\frac{dT}{T} = - \frac{dV}{V}
\end{eqnarray}
On intégrant de part et d'autres
\begin{eqnarray}
\int_{T_i}^{T} \frac{dT}{T} &=& -\frac{2}{n_d}\int_{V_i}^{V} \frac{dV}{V}\\
\ln\left(\frac{T}{T_i}\right) &=& -\frac{2}{n_d}\ln\left(\frac{V}{V_i}\right)\\
\ln\left(\frac{T}{T_i}\right) &=& \ln\left(\frac{V}{V_i}\right)^{-\frac{2}{n_d}}\\
\frac{T}{T_i} &=& \left(\frac{V_i}{V}\right)^{\frac{2}{n_d}}\\
TV^{\frac{2}{n_d}} &=& T_iV_i^{\frac{2}{n_d}}
\end{eqnarray}
On en tire ce que l'on appelle la \textit{Loi de Laplace}
\begin{equation}
TV^{\frac{2}{n_d}}\ =\ cste
\end{equation}

\subsubsection{Diagramme PV de la transfo adiabatique}
\begin{wrapfigure}[8]{r}{3cm}
\includegraphics[scale=0.44]{th/image14.png}
\captionof{figure}{Adiabatique}
\end{wrapfigure}
Pour pouvoir tracer le diagramme PV, il faut ré-exprimer la loi de Laplace en fonction de $P$ et $V$. Sachant que : \\
$\left\{\begin{array}{l}
TV^{\frac{2}{n_d}}\ =\ cste\\
T = \dfrac{PV}{Nk_B}
\end{array}\right.$\ \ \ \ $\Rightarrow \frac{PV}{Nk_B}V^{\frac{2}{n_d}} = cste$\\
Ce que l'on peut ré-écrire $PV^{1 + \frac{2}{n_d}} = cste$. En posant $(1 + \frac{2}{n_d}) = \gamma$ (coefficient adiabatique) :
\begin{equation}
\fbox{$PV^\gamma = cste$}
\end{equation}
\newpage
\section{Les cycles thermodynamiques}
\begin{wrapfigure}[9]{r}{3cm}
\includegraphics[scale=0.44]{th/image15.png}
\captionof{figure}{Cycle thermo}
\end{wrapfigure}
Un cycle thermodynamique est une suite de transformations formant un cycle dans l'espace des variables d'état. Le travail d'un cycle se défini de la sorte :
\begin{equation}
W_{cycle} = \oint P(V).dV
\end{equation}
Tout ce qui est gagné dans un cycle est perdu à un autre moment. Dès lors 
\begin{equation}
\Delta U_{cycle} = Q_{cycle} - W_{cycle} \equiv 0
\end{equation}
\subsection{Le cycle de Carnot}
\begin{wrapfigure}[6]{l}{5cm}
\includegraphics[scale=0.64]{th/image16.png}
\captionof{figure}{Machine de Carnot}
\end{wrapfigure}
Il s'agit d'une machine idéale divisée en 4 étapes :
\begin{enumerate}
\item Détente isotherme $T_H$
\item Détente adiabatique
\item Compression isotherme $T_B$
\item Compression adiabatique
\end{enumerate}
\ \\
\begin{wrapfigure}[1]{r}{5cm}
\includegraphics[scale=0.54]{th/image17.png}
\captionof{figure}{Cycle de Carnot}
\end{wrapfigure}
\subsubsection{Détente isotherme}
$\left\{\begin{array}{l}
W_{ab} = \int P(V)dV \rightarrow W_{ab} = Nk_BT_H\ln\left(\frac{V_b}{V_a}\right)\\
\Delta U = 0
\end{array}\right.$

\subsubsection{Détente adiabatique}
On déconnecte de $T_H$ mais ça monte encore un petit peu (comme par "inertie")\\
$\left\{\begin{array}{l}
\Delta U_{bc} = \frac{n_d}{2}Nk_B(T_H-T_B)\\
W_{bc} = \frac{Nk_B}{\gamma - 1}(T_H - T_B)
\end{array}\right.$

\subsubsection{Compression isotherme}
$\left\{\begin{array}{l}
W_{ab} = Nk_BT_B\ln\left(\frac{V_d}{V_c}\right)\\
\Delta U = 0
\end{array}\right.$

\subsubsection{Compression adiabatique}
$\left\{\begin{array}{l}
W_{bc} = \frac{Nk_B}{\gamma - 1}(T_B - T_H)\\
\Delta U = - W_{da}
\end{array}\right.$

\subsubsection{Bilan énergétique du moteur de Carnot}
Ceci est très algébrique et se trouve dans le syllabus. Seul le résultat final est exposé ici (la démo est à connaître pour l'examen!!) :
\begin{eqnarray}
W_{carnot} = Nk_B\ln\left(\frac{V_b}{V_a}\right).\left[T_H - T_B\right]\\
Q_B = -\frac{T_B}{T_H}Q_H
\end{eqnarray}


\subsubsection{Rendement du cycle de Carnot}
\begin{equation}
r_c = 1 - \frac{T_B}{T_H}
\end{equation}
Ceci montre que si la température du réservoir froid tend vers zéro kelvin, on tend vers un rendement maximal.


\subsection{Le moteur à combustion interne}
\begin{wrapfigure}[9]{r}{4cm}
\includegraphics[scale=0.54]{th/image18.png}
\captionof{figure}{Cycle d'Otto}
\end{wrapfigure}
Connu sous le nom du cycle d'\textit{Otto}, ce-dernier se compose de quatre étapes :
\begin{itemize}
\item[AB] Adiabatique (montée rapide du piston)
\item[BC] Isochore (Explosion)
\item[CD] Adiabatique (détente)
\item[DA] Isochore (refroidissement)
\end{itemize}
Encore une fois, seule les résultats finaux sont repris ici mais le développement est à connaître pour l'examen!!
\begin{eqnarray}
Q_B = -\left(\frac{V_b}{V_a}\right)^{\gamma -1}Q_H\\
r = \frac{W}{Q_H} = 1 - \frac{|Q_B|}{Q_H} = 1 - \left(\frac{V_b}{V_a}\right)^{\gamma -1}
\end{eqnarray}


\subsection{Le moteur à turbine à gaz (Cycle de Brayton)}
Le cycle se compose de quatre étapes (encore une fois je ne fais que citer, mais la flemme de recopier tous les dev' mathématiques!)
\begin{enumerate}
\item Gaz dans un cicuit fermé mis en circuit par un compresseur 
\item Gaz chauffé une fois compressé
\item Gaz détendu dans la turbine
\item Gaz refroidit
\end{enumerate}

\subsubsection{Le réfrigérateur (Cycle de Brayton inversé)}
Le principe est celui du moteur à turbine à gaz si ce n'est que c'est exactement l'inverse !
\begin{enumerate}
\item Gaz comprimé adiabatiquement
\item Gaz comprimé chaud refroidi dans un échangeur thermique
\item Gaz détendu de façon adiabatique
\item Le gaz refroidi se réchauffe en pompant la chaleur
\end{enumerate}

\newpage
\section{Entropie et second principe}
\subsection{Analyse de \textit{Clausius} avec le cycle de \textit{Carnot}}
\textit{Clausius} s'est intéressé à la relation particulière du cycle de Carnot.
\begin{equation}
r_c = 1 - \frac{|Q_B|}{Q_H} = 1 - \frac{T_B}{T_H}
\end{equation}
On peut en déduire la définition de l'entropie ; tout ce qui rentre, sort.
\begin{equation}
\frac{Q_B}{T_B} = -\frac{T_B}{T_H} \equiv Entropie
\end{equation}
On dira, dans ce cas, que l'entropie est conservée :
\begin{equation}
\fbox{$\frac{Q_B}{T_B} + \frac{Q_H}{T_H} = 0$}
\end{equation}

\subsection{Généralisation à tout cycle}
Il suffit de remplacer le cycle par une suite d'isotherme et d'adiabatique afin de retrouver $Q_B$ et $Q_H$.
\begin{equation}
\frac{\delta Q_H}{\delta T_H} + \frac{\delta Q_B}{\delta T_B} = 0
\end{equation}
En prenant la limite infinitésimale, on retrouve le résultat précédent sous sa forme intégrale
\begin{equation}
\fbox{$\Delta S =\oint \frac{dQ}{T} = 0$}
\end{equation}

\subsection{Définition de l'entropie}
\begin{wrapfigure}[7]{l}{4cm}
\includegraphics[scale=0.44]{th/image19.png}
\captionof{figure}{Cycle thermodynamique}
\end{wrapfigure}
Nous allons ici montrer que l'entropie est une \textbf{fonction d'état} : elle ne dépend que des états extrêmes.
\begin{eqnarray}
\oint \frac{dQ}{T} = 0 \Leftrightarrow \int_{0 \rightarrow A} \frac{dQ}{T} + \int_{A \rightarrow 0} \frac{dQ}{T} = 0\\
\int_{0 \rightarrow A} \frac{dQ}{T} = \int_{0 \rightarrow A} \frac{dQ}{T}
\end{eqnarray}
Ce qui est bien le résultat recherché : celui-ci est valable pour tout cycle.


\subsection{Moteurs et entropie}
L'entropie, $S(A)$ est nulle sur un cycle (tout comme $\Delta U$) :
\begin{equation}
W = \oint dW > 0\ \ \ \Rightarrow\ \ \ Q = \oint dQ > 0
\end{equation}
Il faut plus de chaleurs entrante que sortante sur le cycle d'un \textbf{moteur}. Ce n'est pas possible de ne pas rejeter $Q$ car sinon $\oint \frac{dQ}{T}$ ne serait pas nulle.\\
Si un moteur viole ceci, elle portera le nom de \textit{moteur à mouvement perpétuel de deuxième espèce.}

\subsection{Illustrations}
\subsubsection{Transformation isotherme}
\begin{equation}
\Delta Q = Nk_BT_r\ln\left(\frac{V_f}{V_i}\right)\ \ \ \ \ \ \ \Delta S_{isoth} = NK_b\ln\left(\frac{V_f}{V_i}\right)
\end{equation}

\subsubsection{Transformation adiabatique}
\begin{equation}
\Delta Q =0\ \ \ \ \ \ \ \Delta S_{adiab} = 0
\end{equation}

\subsubsection{Transformation isochore}
Attention, ici la température n'est pas constante ! Il faudra intégrer.
\begin{equation}
\Delta Q = \Delta U = \frac{n_d}{2}Nk_B\Delta T\ \ \ \ \ \ \ \Delta S_{isoc} = \Delta U = \frac{n_d}{2}Nk_B\int \frac{dT}{T}
\end{equation}
Ce qui nous donne
\begin{equation}
\Delta S_{isochore} = \frac{n_d}{2}Nk_B\ln\left(\frac{T_f}{T_i}\right)
\end{equation}


\subsubsection{Transformation isobare}
Le raisonnement est le même que pour l'isochore, mais on utilisera ici $C_p$ pour avoir le lien entre $Q$ et $T$ : $dQ = nC_p \Delta T$ où $C_p = (\frac{n_d}{2}+1)R$
\begin{equation}
\Delta S_{isobare} = \left(\frac{n_d}{2}+1\right)R\ln\left(\frac{T_f}{T_i}\right)
\end{equation}

\subsubsection{Cycle d'Otto}
Nous savons que l'entropie devra être nulle. Ici, il ne faut pas tenir compte des adiabatique mais juste des isochores (on les somme).
\begin{equation}
\Delta S_{Otto} = \frac{n_d}{2}Nk_B\left(\frac{T_cT_a}{T_bT_d}\right)
\end{equation}
En jouant avec la loi de Laplace des adiabatiques, on trouve que l'argument du $\ln$ vaut 1.
\begin{equation}
\Delta S_{Otto} = 0
\end{equation}


\subsection{Irréversibilité et entropie}
\subsubsection{La thermalisation}
Les énergies internes ($U$) vont s'additionner. On obsèrve une augmentation de l'entropie : \textit{la thermalisation produit $\Delta S$}.\\
Le système étant isolé, $dQ = 0$. Pourquoi alors l'entropie n'est-elle pas nulle ? Car il s'agit d'un \textbf{processus irréversible}.
\newpage
\subsubsection{Isotherme}
\begin{wrapfigure}[7]{l}{4cm}
\includegraphics[scale=0.44]{th/image20.png}
\captionof{figure}{Isotherme}
\end{wrapfigure}
Ici $\Delta S = 0$ car l'entropie perdue dans le réservoir vaut celle gagnée dans le gaz. Il s'agit d'une situation \textbf{réversible} (car absence de déséquilibre).

\subsubsection{La détente libre irréversible}
On prend une boîtes à deux compartiment, un vide et un rempli de gaz. On perce un trou entre les deux. L'énergie cinétique reste inchangée car $V' = 2V$ et $P' = P/2$. \\
Cette détente est irréversible, on imagine mal les particules de gaz repasser par le trou pour retrouver la situation initiale.\\

\begin{center}
\includegraphics[scale=0.7]{th/image21.png}
\captionof{figure}{Détente libre irréversible}
\end{center}
Cette détente est complexe, mais on ne va considérer que l'état initial et final de sorte à utiliser l'entropie d'une \textbf{isotherme équivalente}.
\begin{equation}
\Delta S = Nk_B\ln\left(\underbrace{\frac{V_f}{V_i}}_{2}\right) > 0
\end{equation}
\subsubsection{Conclusion}
On observera une production d'entropie $\Delta S$ si :
\begin{enumerate}
\item Déséquilibre
\item Irréversibilité
\item Augmentation du désordre
\end{enumerate}
\begin{center}
\fbox{\textit{\textsc{Deuxième principe :} L'entropie d'un système isolé ne peut que rester stable ou augmenter.}}
\end{center}
























\chapter{Électrostatique}
\section{Introduction}
Encore une fois, une bonne lecture attentive suffit !

\section{La force électrique}
\subsection{La force de Coulomb}
Celle-ci a été déterminée à l'aide d'une balance à torsion. Par prise de mesures successive de l'angle fait par le fil en fonction de la charge de la bille, on a pu déterminer les relations de proportionnalités suivantes :
\begin{wrapfigure}[7]{r}{4cm}
\includegraphics[scale=0.44]{es/image1.png}
\captionof{figure}{Loi de décroissance naturelle}
\end{wrapfigure}
\begin{eqnarray}
F \propto \frac{1}{r^2}\\
F \propto q
\end{eqnarray}
Ces forces ont toujours été obtenues en considérant deux charges. Dès lors :
\begin{equation}
F \propto \frac{q_1q_2}{r^2}
\end{equation}
Pour changer cette proportionnalité en égalité, on définit la constante de Coulomb, $k_0 = 8,987*10^{-9} Nm^2C^{-2}$.
\begin{equation}
\fbox{$ F = k_0 \dfrac{q_1q_2}{r^2}$}
\end{equation}

\subsection{Principe de superposition}
\begin{wrapfigure}[7]{l}{4cm}
\includegraphics[scale=0.44]{es/image2.png}
\captionof{figure}{Principe de superposition}
\end{wrapfigure}
La convention de signe associée à la force de Coulomb est : $\left\{\begin{array}{l}
+ ; Répulsion\\
- ; Attraction
\end{array}\right.$\\

Notons que la plus petite distance entre les deux charge vaut les deux rayons des charges, sinon l'on aurait une force infinie.\\
La présence d'autre charge n'influent pas sur le résultat :  superposition :
\begin{equation}
\vec{F_m} = k_0\sum_{n\neq m} \frac{q_nq_m}{r^2_{mn}}\vec{1_{r_{mn}}}
\end{equation}

\section{Le champ électrique}
Comme première approche, on peut faire une analogie entre la force électrique $F_e$ et la force gravitationnelle :
\begin{equation}
\vec{F_G} = -G \frac{mm_0}{r^2}\vec{1_r}
\end{equation}
Si la $F_G$ crée un champ gravitationnel, on peut concevoir qu'il en sera de même pour $F_e$. La différence majeure est que $F_e$ peut être positive ou négative.

\subsection{Champ de force}
Il s'agit de l'ensemble des charges que l'on pourrait mesurer dans l'environnement de cette charge sur une deuxième charge mobile (dite "charge d'essai").

\subsection{Champ électrique}
\begin{wrapfigure}[9]{l}{4cm}
\includegraphics[scale=0.34]{es/image3.png}
\captionof{figure}{Champ électrique}
\end{wrapfigure}
Toujours devoir utiliser deux charges (la charge $q_1$ et la charge d'essai, $q_2$) est très contraignant, c'est pourquoi on passe à la notion de champ qui va caractériser l'environnement indépendamment de la charge d'essai.
\begin{equation}
\fbox{$\vec{E} = \dfrac{\vec{F}}{q} = k_0\dfrac{q}{r^2}\vec{1_r}$}
\end{equation}
L'unité de $E$ sont des $NC^{-1}$. Notons que la force n'a pas toujours le même sens que le champ.


\subsection{Lignes de champ}
\begin{wrapfigure}[9]{r}{6cm}
\includegraphics[scale=0.34]{es/image4.png}
\captionof{figure}{Signe du champ}
\end{wrapfigure}
Il s'agit de courbes en tout point tangent au champ électrique. Elles ne donnent en rien la force d'interaction, elles représentent juste le champ (orientation).\\

Celles-ci ne se croisent jamais car un champ ne peut prendre plusieurs valeurs à la fois (car donné de façon univoque par la loi de Coulomb).

\subsection{Champ électrique dans les conducteurs}
\begin{wrapfigure}[7]{l}{5cm}
\includegraphics[scale=0.34]{es/image5.png}
\captionof{figure}{Conducteur}
\end{wrapfigure}
Un conducteur est une "substance" qui conduit l'électricité. Leurs électrons (ou ions) peuvent circuler librement.\\
Imaginons un conducteur sphérique. Sa surface peut être vue comme une enceinte de gaz d'électrons libres de se mouvoir. Ajouter des électrons va créer une répulsion, créant un excès de charge positive. Cet excès de charge positive va augmenter jusqu'à avoir compensé exactement la charge négative en excès.\\
Le champ va dès lors se mettre perpendiculairement à la surface à l'équilibre, de façon à ne plus bouger. Si ce n'était pas le cas, les charges se déplacerait et créerait de la chaleur par effet Joule : violerait la conservation d'énergie.\\
L'intérieur d'un conducteur est toujours neutre et le champ est toujours nul.

\subsection{Distribution de charge continue}
\subsubsection{Charges de volume}
Si l'espace est une répartition de charge ponctuelles, on va pouvoir appliquer le principe de superposition.
\begin{equation}
\vec E = k_0 \sum \frac{q_e}{r^2_n}\vec{1_{rn}}
\end{equation}
Il nous faut connaître $r$, la distance entre la charge et le lieu ou l'on calcule le champ. On peut la trouver grâce à la norme (application de Pythagore)\footnote{$(x,y,z)$ sont les coordonnées du point de calcul du champ et $(x_n, y_n, z_n)$ l'emplacement de la charge}
\begin{equation}
r = |\vec{x} - \vec{x_n}| = \sqrt{(x-x_n)^2 + (y-y_n)^2 + (z-z_n)^2}
\end{equation}
Étant l'espace, on va utiliser la densité de charge pour le caractériser : $\rho(\vec{x_n}) = \dfrac{\Delta q_m}{\Delta V_m}$. On peut en tirer $\Delta q = \rho(\vec{x_n)}).\Delta V_m$. En remplaçant dans l'expression du champ et en passant à la limite, on trouve l'expression recherchée.
\begin{equation}
\vec{E}(\vec x) = k_0 \int \frac{\rho(\vec x)}{||\vec{x} - \vec{x'}||^2}\vec{1}_{(x-x')}.dV
\end{equation}

Pour des charges de surface ($\sigma$) et linéique ($\lambda$) le principe est le même ! 


\section{Loi de Gauss}
\subsection{Analogie entre champ électrique et flux de particules}
Une charge ponctuelle émet un flux de photons virtuels (dit "particules", pour ne pas rentrer dans la théorie quantique de l'électromagnétisme).

\subsubsection{Densité de particules}
\begin{wrapfigure}[7]{rl}{5cm}
\includegraphics[scale=0.34]{es/image6.png}
\captionof{figure}{Densité de particules}
\end{wrapfigure}
Calculons le flux total à travers une sphère de rayon $r$.
\begin{equation}
\Delta V = \left(\underbrace{4\pi r^2}_{surface}\right) . \underbrace{v.\Delta t}_{épaisseur\ cercle}
\end{equation}
Le nombre de particules s'obtient de la sorte
\begin{equation}
\Delta N = \eta(r).\Delta V
\end{equation}
En remplaçant l'un dans l'autre
\begin{equation}
\Delta N = \eta (r) 4\pi r^2 v\Delta t
\end{equation}
Le flux est le nombre de particules par unité de temps 
\begin{equation}
\Phi = \frac{\Delta N}{\Delta T} = \eta (r) 4\pi r^2 v
\end{equation}
Comme la vitesse est constante, le flux est indépendant de $r$ : pas d'accumulation.\\

Introduisons la \textit{densité de flux de particules}, c'est à dire le nombre de particules traversant une surface par unité de temps : $\vec{F} = \eta (r)\vec{v}$. En isolant :
\begin{equation}
\vec{F} = \eta (r)\vec{v} = \frac{Q}{4\pi r^2}\vec{1_r}
\end{equation}

\subsubsection{Densité de flux}
Insistons sur le fait qu'il s'agit d'une notion locale ! Il faut donc intégrer si l'on veut pour toute la surface.
\begin{equation}
\vec{F} = \frac{Q}{S}\vec{1_r}
\end{equation}
Le flux est constant mais pas $\vec{F}$ qui diminue quand $r$ augmente. On peut voir une ressemblance entre la densité de flux et la charge de Coulomb ($1/r^2$):
$\left\{\begin{array}{l}
\vec{F} = \frac{Q}{4\pi r^2}\vec{1_r}\\
\vec{E} = k_0 \frac{q}{r^2}\vec{1_r}
\end{array}\right.$


\subsection{La permittivité}
Pour des raisons qui apparaîtrons plus tard, posons un changement de variable :
\begin{equation}
k_0 = \frac{1}{4\pi \epsilon_0}
\end{equation}
Recommençons la comparaison : $\left\{\begin{array}{l}
\vec{F} = \frac{q}{4\pi r^2}\vec{1_r}\\
\vec{E} = \frac{q}{4\pi \epsilon_0 r^2}\vec{1_r}
\end{array}\right.$\ \ \ $\Rightarrow\ \ \ \ \dfrac{q}{\epsilon_0} \leftrightarrow \Phi$.

\subsection{Intégrale de flux}
Plutôt que de considérer une sphère, on va s'intéresser à une surface quelconque.
\begin{center}
\includegraphics[scale=0.5]{es/image7.png}
\captionof{figure}{Flux à travers d'une surface}
\end{center}
Le nouveau volume parcouru peut être calculé simplement
\begin{equation}
\Delta V = hv\Delta t l \cos\theta = S v \Delta t \cos\theta
\end{equation}
Comme précédemment, $\Delta N = \eta . \Delta V = \eta S v \Delta t \cos\theta$. Nous avons donc comme flux
\begin{equation}
\Phi = \eta v \Delta S_m \cos\theta
\end{equation}
En remarquant l'expression de la densité de flux de particules et la présence d'un produit scalaire, on peut ré-écrire cette expression de façon plus élégante.
\begin{equation}
\Phi_{\Delta_{S_m}} = \vec{F}.\vec{\Delta S_m}
\end{equation}
En sommant toutes ces surfaces élémentaires, on retrouve l'expression du flux à travers toute la surface.
\begin{equation}
\Phi^S = \oint_S \vec{F}(\vec x).d\vec{S}
\end{equation}

\subsection{Flux du champ électrique}
Le résultat obtenu est généralisable pour tout champ vectoriel, y crompsi le champ électrique.
\begin{equation}
\Phi^S_E = \oint_S \vec{E}(\vec{x}).d\vec S
\end{equation}

\subsubsection{Flux au travers d'une surface fermée}
Voyons si ce résultat est cohérent avec l'expression de la densité de flux ($\vec F$)\footnote{Notons qu'ici d$\vec{S}$ étant radial, il vaut $dS.\vec{1_r}$}.\\
\begin{equation}
\Phi = \oint \frac{Q}{4\pi r^2} dS = \Phi\ \ \ \ \ \ \ \ \ \Rightarrow \ \ OK !
\end{equation}
Ce résultat peut être généralisé à toutes surface (si la vitesse est constante).\\
Voyons ce que donne le calcul de cette intégrale. 
\begin{equation}
\oint \vec{E}(\vec x).d\vec{S} = \frac{q}{4\pi r^2 \epsilon_0}\int dS = \frac{q}{\epsilon_0}\ \ \ \ \ \forall\ surfaces !
\end{equation}
Rappelons que cette analogie est possible grâce à la dépendance en $\frac{1}{r^2}$.

\subsubsection{Flux dû à plusieurs charges}
Simple application du principe de superposition
\begin{equation}
\Phi^S_E = \frac{1}{\epsilon_0}\sum_1^N q_n
\end{equation}

\subsubsection{Flux dû à des charges externes}
Tout ce qui va rentrer d'un côté va forcément ressortir de l'autre. Le flux en sera nul ($\Phi = 0$).


\subsubsection{Charges négatives}
Les charges vont "aller" dans l'autre sens, celui opposé à $d\vec S$ : le signe du flux sera négatif. Mais comme la charge sera aussi de signe négatif, les deux moins vont s'annuler et cela ne change au final rien du tout!

\subsubsection{Loi de Gauss}
Ces généralisations étant faites, on peut désormais dire que la Loi de Gauss est généralisée :
\begin{equation}
Loi\ de\ Gauss : \oint_S \vec{E}.d\vec{S} = \dfrac{1}{\epsilon_0}\sum_{m=1}^N q_m
\end{equation}

\subsection{Distributions de charge continue}
Comme vu à la section \textit{3.6}, on peut utiliser $\rho, \sigma\ et\ \lambda$.
\begin{equation}
\Phi^S_E = \oint_S \vec{E}.d\vec{S} = \frac{1}{\epsilon_0}\int_{V_S}\rho(\vec x).dV
\end{equation}

\subsection{Application de la loi de Gauss : calcul de champ}
Voir \textit{Annexe A}, tout y est !
\newpage
\subsection{Forme locale de la Loi de Gauss}
Il s'agit de la version différentielle de la Loi de Gauss : celle-ci appliquée en un point.
\subsubsection{Champ unidirectionnel}
\begin{wrapfigure}[7]{r}{3cm}
\includegraphics[scale=0.44]{es/image8.png}
\captionof{figure}{Loi de gauss locale}
\end{wrapfigure}
Supposons pour commencer que le champ ne dépend que de $x$ : $\vec{E}(\vec{x}) = \vec{E}(x,y,z) = E_x(x)\vec{1_x}$.\\
Le flux sera ainsi nul pour les autres surfaces. Selon le schéma ci-contre
\begin{equation}
\Phi^S_E = \oint_S \vec{E}.d\vec{S} = \int_{S_x} -E_x(x)ds +  \int_{S_x} E_x(x+\Delta x)dS
\end{equation}
Les surfaces étant infiniment petites, on peut considérer le champ comme constant.
\begin{equation}
-E_x(x)S_x + E_x(x+\Delta x)S_x = \left[E_x(x+\Delta x) - E_x(x)\right].S_x
\end{equation}
Utilisons l'artifice mathématique consistant à multiplier et diviser par $\Delta x$.
\begin{equation}
\Phi^S_E = \left[\frac{E_x(x+\Delta x) - E_x(x)}{\Delta x}\right]\underbrace{S_x\Delta X}_{\Delta V}
\end{equation}
On retrouve ainsi l'expression de la dérivée du champ par rapport à $x$ en faisant tendre $\Delta x \rightarrow 0, \Phi_E^S = \dfrac{dE_x}{dx}\Delta V$.\\
Appliquons la Loi de Gauss
\begin{eqnarray}
\dfrac{dE_x}{dx}\Delta V = \frac{1}{\epsilon_0}\int \rho(\vec x) dV\\
\dfrac{dE_x}{dx}\Delta V =\int \rho(\vec x) \Delta V =
\frac{dE_x}{dx} = \frac{1}{\epsilon_0}\rho(\vec{x})
\end{eqnarray}
Il s'agit la de la loi "locale" car le lien entre $E$ et la densité de charge $\rho$ est donné en un point de l'espace $\vec{x}$.

\subsubsection{Généralisation 3D}
On recommence le même raisonnement pour $x$ et $y$. En mettant tous les résultats ensemble, on obtient la loi de Gauss locale. 
\begin{equation}
\fbox{$\dfrac{\partial E_x(x,y,z)}{\partial x} + \dfrac{\partial E_y(x,y,z)}{\partial y} + \dfrac{\partial E_z(x,y,z)}{\partial z} = \frac{1}{\epsilon_0}\rho(x,y,z)$}
\end{equation}

\subsection{La divergence}
En définissant l'opérateur divergence, $\vec{\nabla}$ on peut noter la loi locale de Gauss sous sa forme compacte.
\begin{equation}
\vec{\nabla}\vec{E} = \frac{1}{\epsilon_0}\rho(\vec x)
\end{equation}
On appelle ceci \textit{divergence} car si le vecteur diverge des élément quittent la surface et le flux en sera positif.
\begin{equation}
\fbox{$div\ \vec{E} = \dfrac{1}{\epsilon_0}\rho(\vec{x})$}
\end{equation}
\textit{La quantité $div\ \vec{E}$ est le flux de particules du champ normalisée (car divisé par sa surface) par le volume défini par la surface fermée captant ce flux.}


\section{Le potentiel électrique}
\subsection{L'énergie potentielle gravitationnelle}
Par définition du travail : $W = F_g.\vec{1_z}\Delta h \rightarrow mg\Delta h$.\\
Par conservation de l'énergie, on peut retrouver la vitesse. En effet : $\frac{1}{2}mv^2 : mgh \Leftrightarrow v = \sqrt{2g\Delta h}$.

\subsection{L'énergie potentielle électrique}
Sa forme est grandement similaire : $W_p(z) = q||\vec{E}||.z (+cste)$. Notons que la force exercée sur la charge est l'opposée de la dérivée de l'énergie potentielle électrique.
\begin{equation}
F_e = - \frac{dW_p}{dz}\vec{1_z}
\end{equation}

\subsection{Potentiel électrique}
Par définition, le potentiel électrique est l'énergie potentielle électrique divisée par la charge.
\begin{equation}
\fbox{$V \equiv \dfrac{W_p}{q_0} = ||\vec E||.z (+cste)$}
\end{equation}
L'unité du potentiel est le Volt ($V$).

\subsection{Potentiels de champs non-uniformes}
On décompose la trajectoire en déplacement infinitésimaux. Le travail élémentaire correpondant au déplacement nous intéressant est donné par le produit scalaire de la force $\vec{F_a}$ (l'inverse de la force électrique que subit la charge, c'est à dire $-qE$) par le vecteur déplacement $\vec{dl}$.
\begin{center}
\includegraphics[scale=0.6]{es/image9.png}
\captionof{figure}{Décomposition infinitésimale de la trajectoire}
\end{center}
Comme $W = F.L$, en tenant compte de l'angle (et en passant à l'infinitésimal) : $dW = ||\vec{dl}||.||\vec{F}||.\cos\theta = \vec{F}.\vec{dl} = -q\vec{E}.\vec{dl}$.\\
Comme $V = \frac{W}{q}$ :
\begin{equation}
\Delta V = - \int_{i \rightarrow f} \vec{E}.\vec{dl}
\end{equation}

\subsection{Potentiel coulombien}
Nous savons que : $\left\{\begin{array}{l}
\vec{E} = \frac{q}{q\pi \epsilon_0 r^2}\vec{1_r}\\
\Delta V = - \int_{i \rightarrow f} \vec{E}.\vec{dl}
\end{array}\right.$. Injectons l'un dans l'autre :
\begin{equation}
\Delta V = -\frac{q}{4\pi \epsilon_0}\int_{r_i}^{r_f} \frac{dr}{r^2}
\end{equation}
L'idée du potentiel coulombien est de faire venir une charge de l'infini ($r_i = \infty$), c'est à dire quand on ne "ressent" plus les effets de la charge. Ce qui donne, après intégration\footnote{Souvent utilisé en pratique : $W = qV(r)$}
\begin{equation}
V(r) = \frac{q}{4\pi \epsilon_0 r}
\end{equation}
Il s'agit d'un champ scalaire (et non vectoriel comme $\vec{F}$ et $\vec E$).\\
Rappelons qu'on travail négatif signifie que la charge se fait attirer, c'est à dire qu'elle \textit{reçoit} de l'énergie.

\subsubsection{Chemin quelconque}
\begin{wrapfigure}[6]{l}{5.5cm}
\includegraphics[scale=0.44]{es/image10.png}
\captionof{figure}{Chemin quelconque}
\end{wrapfigure}
On remarque grâce au schéma ci-contre que $\vec{E}.\vec{dl} = ||\vec{E}||.||\vec{dl}||.\cos\theta = ||\vec{E}||.dr$.\\
La trajectoire est identique à celle radiale.\\

Le potentiel est donc indépendant du chemin suivi, ce qui implique que la force électrique est conservative.\\
\ \\

\subsection{Potentiel dû à plusieurs charge}
On applique encore et toujours le principe de superposition !
\begin{equation}
\Delta V = \sum_1^N \Delta V_n
\end{equation} 

\subsubsection{Potentiel du dipôle}
Les calculs se trouvent dans le syllabus, voici ici le résultat compte-tenu de l'approximation du premier-ordre.
\begin{equation}
V(x) \approx \frac{q}{4\pi \epsilon_0}\frac{xd}{r^3}
\end{equation}

\subsection{Potentiel d'une distribution de charges continue}
Même principe que précédemment. Par exemple, pour $\rho$ :
\begin{equation}
V(x) = \frac{1}{4\pi \epsilon_0}\int\frac{\rho (x')}{||x-x'||}dV'
\end{equation}

\subsection{Le moment dipolaire}
Par analogie au centre de masse et au moment (\textit{Cf. mécanique rationelle I}), le moment dipolaire est défini par
\begin{equation}
\vec{p} = \int \rho (\vec{x}).x'.dV
\end{equation}
Un autre résultat (souvent utilisé en pratique) est : 
\begin{equation}
\vec{p} = Q_+ <x_n^+> - Q_-<x_n^->
\end{equation}

\newpage
\subsection{Moment de force sur un dipôle}
Un couple de force s'appliquant sur une "barre" n'a qu'un seul effet possible : causer un déplacement en rotation (du au moment de force, \textit{Cf. ConFond, Meca I}).

\begin{center}
\includegraphics[scale=0.6]{es/image11.png}
\captionof{figure}{Couple de force et bras de levier}
\end{center}
Plus l'angle $\theta$ est petit, plus l'effet sera faible.  La composante de cette force en $x$ vaut $q.E\cos(\frac{\pi}{2} - \theta) = qE\sin\theta$.\\
Connaissant la définition du moment de force $\tau$ (\textit{bras de levier * force $\perp$, $\tau = d.F_\perp$}) nous avons : \\
\begin{equation}
\tau = d.q.E\sin\theta
\end{equation}
En travaillant avec des demi-bras de levier ($\frac{d}{2}$) et en les sommant, on trouve
\begin{equation}
\tau = qE\frac{d}{2}\sin\theta + qE\frac{d}{2}\sin\theta
\end{equation}
Ce que l'on peut ré-écrire (vectoriellement ici, par propriété du produit vectoriel)
\begin{equation}
\vec{\tau} = qd\vec{1_p}\times \vec{E}
\end{equation}
En posant $\vec p = qd\vec{1_p}$ on trouve l'expression générale du moment de force sur un dipôle.
\begin{equation}
\fbox{$\vec{\tau} = \vec{p} \times \vec{E}$}
\end{equation}

\subsubsection{Généralisation}
On retombe exactement sur le même résultat en passant par la décomposition infinitésimale ($\tau = \int d\tau$).\\
Ce sont les dipôles qui explique l'alignement des molécules d'eau dans le même sens.

\subsection{Détermination du champ électrique à partir du potentiel}
Partons de la relation bien connue donnant le potentiel en fonction du champ et dérivons la.
\begin{equation}
\Delta V = - \int_{i\rightarrow f} \vec{E}.\vec{dl}\ \ \ \ \Rightarrow\ \ \ \ dV = - \vec{E}.\vec{dl}
\end{equation}
Qu'est ce que $dV$ ? Il s'agit d'un déplacement infinitésimal 
\begin{equation}
dV = \frac{\partial V(x,y,z)}{\partial x}dx + \frac{\partial V(x,y,z)}{\partial y}dy + \frac{\partial V(x,y,z)}{\partial z}dz
\end{equation}
Connaissant le vecteur $\vec{dl} = dx\vec{1_x} + dy\vec{1_y}+dz\vec{1_z}$, on voit que l'on peut écrire $dV$ comme le produit scalaire suivant :
\begin{equation}
dV = \vec{\nabla}V.\vec{dl}
\end{equation}
Pour rappel : 
\begin{equation}
\vec{\nabla} \equiv \frac{\partial}{\partial x}\vec{1_x} + \frac{\partial}{\partial y}\vec{1_y} + \frac{\partial}{\partial z}\vec{1_z} \Rightarrow \vec{\nabla}V \equiv \frac{\partial V}{\partial x}\vec{1_x} + \frac{\partial V}{\partial y}\vec{1_y} + \frac{\partial V}{\partial z}\vec{1_z}
\end{equation}
On remarque que $\vec{\nabla}.\vec{dl}$ donne la même expression que $dV$
\begin{equation}
\vec{\nabla}.\vec{dl} = \frac{\partial V(x,y,z)}{\partial x}dx + \frac{\partial V(x,y,z)}{\partial y}dy + \frac{\partial V(x,y,z)}{\partial z}dz
\end{equation}
Dès lors, on peut dire que 
\begin{equation}
\vec{\nabla}V.\vec{dl} = -\vec{E}.\vec{dl}\ \ \ \ \Rightarrow\ \ \ \ \vec{E} = - \vec{\nabla}V
\end{equation}
L'opérateur $\vec{\nabla}$ est aussi appelé \textit{gradient}.
\begin{equation}
\fbox{$\vec{E} = -grad\ V$}
\end{equation}

\subsubsection{Interprétation du gradient}
Le gradient donne la direction de la plus grande pente positive.

\section{Capacité électrique}
\subsection{Générateur de Van de Graff}
\begin{wrapfigure}[10]{r}{3cm}
\includegraphics[scale=0.34]{es/image12.png}
\captionof{figure}{Générateur de Van de Graff}
\end{wrapfigure}
On remarque que plus on charge la sphère, plus le champ électrique augmente. Plus celui-ci est fort, plus la force électrique est importante. Forcément, le travail nécessaire pour ajouter une nouvelle charge est plus conséquent.\\
On remarque ainsi un lien de proportionnalité :
\begin{equation}
W \propto Q
\end{equation}
Rappelons que les charges se répartissent selon une densité surfacique de sorte que le champ soit chaque fois perpendiculaire à la surface.

\subsection{Capacité électrique}
Comme $W \propto Q$ et que $V = \frac{W}{q}$ on peut dire que $V \propto Q$. En introduisant un coefficient de proportionnalité $Q$, la capacité électrique, on définit l'égalité suivante :
\begin{equation}
\fbox{$Q = CV$}
\end{equation}

\subsection{Capacité de la sphère}
Grâce à la loi de Gauss, on peut trouver le champ d'une sphère 
\begin{equation}
\Phi = 4\pi r^2 ||\vec{E}|| = \frac{q}{\epsilon_0}\ \ \ \ \Rightarrow\ \ \ \ E = \frac{q}{4\pi \epsilon_0 r^2}
\end{equation}
Le potentiel associé à ce champ vaut 
\begin{equation}
V_{sphère} = \frac{q}{4\pi \epsilon_0 r}
\end{equation}
Or, $Q = CV\ \ \rightarrow\ \ V = \frac{Q}{C}$. Par comparaison, on trouve que la capacité de la sphère vaut :
\begin{equation}
C_{sphère} = 4\pi \epsilon_0 r
\end{equation}
Si la capacité $C$ augmente, le travail $W$ diminue et plus on pourra rajouter de nouvelles charges facilement. Ainsi, plus $C$ est grand, plus la différence de potentiel sera réduite.

\subsection{Énergie de la charge}
On apporte à une sphère des petites charges infinitésimales $dq$. Le travail nécessaire pour apporter ces charges vaut $V.q = V(q).dq$
\begin{equation}
W = \int_0^q V(q).dq = \int_0^q \frac{q}{C}dq = \frac{q^2}{2} 
\end{equation}
Comme $Q = CV$, on retrouve l'expression faisant l'objet de tous vos fantasmes :
\begin{equation}
\fbox{$W_e = \dfrac{CV^2}{2}$}
\end{equation}
Notons que l'unité de la capacité électrique est le $Farad, F$.


\subsection{Capacité électrique et influence électrostatique}
\begin{wrapfigure}[12]{l}{3.5cm}
\includegraphics[scale=0.44]{es/image13.png}
\captionof{figure}{Influence électrostatique}
\end{wrapfigure}
Plus on charge, par exemple, une sphère, plus le travail nécessaire augmente. Il est intéressant de pouvoir augmenter la capacité.\\
En approchant une deuxième sphère non chargée, celle-ci va subir la répulsion des charges négatives et l'attraction des charges positives. Il s'agit de l'\textbf{influence électrostatique}.\\
Les lignes de champ sont plus denses entre les deux sphères, le champ à donc obligatoirement diminué dans les zones plus éloignée de l'entre-deux sphère (résultat découlant de la Loi de Gauss).\\
Comme il existe des endroit ou apporter des charges est plus simple, on peut dire que le potentiel à diminué et donc, la charge étant constante ($C = Q/V$), la capacité électrique a augmenté.\\ Si l'on connecte la deuxième sphère au sol, elle gagne une capacité infinie et la capacité de notre première sphère se voit encore augmentée. Notons qu'à l'équilibre, les charges seront égales mais opposée : c'est le principe du \textbf{condensateur}.

\subsection{Le condensateur électrique plan}
\begin{wrapfigure}[12]{r}{3.5cm}
\includegraphics[scale=0.24]{es/image14.png}
\captionof{figure}{Condensateur plan}
\end{wrapfigure}
Le principe est le même, mais la géométrie est plus simple.  La justification du champ nul à l'équilibre s'obtient en prenant une surface de Gauss cylindrique contenant les deux plans de charges. Le flux est négatif mais par influence électrostatique il diminue en valeur absolue jusqu'à devenir nul.\\
A l'équilibre, le flux est nul et le champ extérieur l'est également : ils ne subissent plus de force de coulomb.\\
La condition d'équilibre est d'avoir une charge globale nulle.

\subsection{Calcul de la capacité du condensateur plan}
Nous avons connaissance du champ d'une plaque seule\footnote{Que si le plan est considéré comme infini et que l'on calcule le champ loin des bords}
\begin{equation}
\vec{E}_{plaque} = \frac{\sigma}{\epsilon_0}\vec{1_S}
\end{equation}
Pour utiliser cette relation, on va considérer le plan infini pour négliger les distorsions des lignes de champs aux extrémités. \textbf{Hypothèses : $h, l >> e$}.\footnote{$e$ est la distance entre les deux plaques}

\begin{wrapfigure}[19]{l}{6cm}
\includegraphics[scale=0.44]{es/image15.png}
\captionof{figure}{Condensateur plan}
\end{wrapfigure}
Calculons le champ grace au principe de superposition : on considère le champ généré par chaque plaque et l'on somme les résultats.\\
La plaque de gauche a une charge négative, son champ sera dès lors attractif. L'autre plaque sera donc chargé positivement et aura un champ répulsif.\\
On remarque qu'en dehors de l'entre-deux plaques, le champ est nul (les lignes de champ se compensent).
\begin{equation}
E_{cond_{ext}} = 0
\end{equation}

Par contre, entre les deux plaque, le champ est doublé et vaut
\begin{equation}
\fbox{$\vec{E}_{cond} = \dfrac{\sigma}{\epsilon_0}\vec{1_x}$}
\end{equation}
Notons que tant que les deux plaques ne sont pas à l'équilibres, des électrons sont poussés à l'extérieur.\\


Le champ du condensateur étant connu, calculons sa capacité électrique. Nous savous que $V = - \int \vec{E}.\vec{dl}$ où $E = \frac{\sigma}{\epsilon_0}$.
\begin{equation}
V = -\frac{\sigma}{\epsilon_0}\int_0^e dx = -\frac{\sigma e}{\epsilon_0} \overbrace{=}^{Q = -\sigma S} = \frac{Qe}{\epsilon_0 S}
\end{equation}
Sachant que $C = \frac{Q}{V}$ on trouve
\begin{equation}
\fbox{$C_{cond} = \dfrac{\epsilon_0 S}{e}$}
\end{equation}
C'est donc l'épaisseur $e$ qui détermine l'influence électrostatique. Plus elle est petite, plus l'influence sera grande et la capacité sera élevée.

\subsection{Milieux diélectriques et polarisation}
On peut encore augmenter la capacité électrique en incorporant un milieu non-conducteur, dit \textit{conducteur diélectrique}. \textbf{Hypothèse :} le milieu diélectrique est constitué d'une juxtaposition d'atomes identiques.
\begin{center}
\includegraphics[scale=0.48]{es/image16.png}
\captionof{figure}{Champ induit du au milieu diélectrique}
\end{center}
Le champ électrique du condensateur va causer une légère déformation de l'atome, qui va acquérir un moment dipolaire (\textsc{Figure 2.16.1})\\
On constate l'apparition d'un champ électrique induit qui s'oppose au champ extérieur (généré par les deux plaques) (\textsc{Figure 2.16.2}).\\

Les atomes du diélectrique ne peuvent se rapprocher à une distance inférieure à $d$ : formation d'une pellicule chargée d'épaisseur $d$ responsable d'une diminution du champ total (polarité inversée) (\textsc{Figure 2.16.3}).\\

Le champ total peut dès lors s'écrire (\textsc{Figure 2.16.1}) :
\begin{equation}
\vec{E} = \vec{E_0} + \vec{E_i}
\end{equation}
Comment calculer le champ induit $\vec{E_i}$ ? On sait que l'épaisseur $d$ sera proportionnelle au champ $E$. En introduisant la \textit{déformabilité $\alpha$}, on peut dire que $d = \alpha E$. Le moment dipolaire $\vec{p} = d.\vec{F}$ vaut donc :
\begin{equation}
\vec{p} =: \alpha q\vec{E}
\end{equation}
Connaissant $\left\{\begin{array}{l}
\eta = \frac{Q}{S}\\
Q = V.\rho = \eta_a q d s
\end{array}\right.$ on peut dire que $\sigma = \eta_a q d = \eta_a q \alpha E$ où $E = \frac{\sigma}{\epsilon_0}$.
Le champ induit vaut donc\footnote{Signe négatif car toujours opposé à $E_0$} :
\begin{equation}
E_i = -\frac{\eta_a q \alpha E}{\epsilon_0}
\end{equation}
En introduisant la susceptibilité $\chi\ \ \Rightarrow\ \ \vec{E_i} = -\chi \vec{E}$.\\
Mais que vaut $\vec{E}$? Sachant que $\vec{E} = \vec{E_0} + \vec{E_i} = \vec{E_0} - \chi \vec{E} \Leftrightarrow \vec{E}(1+\chi) = \vec{E_0}$\\
Le champ total $\vec{E}$ peut être calculé :
\begin{equation}
\fbox{$\vec{E} = \dfrac{\vec{E_0}}{(1+\chi)}$}
\end{equation}

Le champ rempli d'un diélectrique est donc plus faible d'un facteur $(1+\chi)$, où $E_0$ est le champ du condensateur vide (= sans diélectrique). D'autres éléments doivent être modifier pour tenir compte du di-électrique :
\begin{eqnarray}
V &=& \frac{V_0}{(1+\chi)}\\
C &=& (1+\chi)\frac{\epsilon_0 S}{e}\\
 &=& \epsilon \frac{S}{e}\ où\ \epsilon\ =\ permittivité : \epsilon_0(1+\chi)\\
\epsilon_r &=& \frac{\epsilon}{\epsilon_0} = 1+\chi
\end{eqnarray}

\subsection{Le condensateur en pratique}
\textit{Voir annexe A}

\subsection{Énergie stockée}
Même chose que précédemment (conservation d'énergie)
\begin{equation}
W_e = \frac{CV^2}{2}
\end{equation}
\subsubsection{Densité d'énergie électrique}
Le but est d'exprimer l'énergie stockée en fonction du champ sachant  que
\begin{equation}
\left\{\begin{array}{l}
W = \frac{CV^2}{2}\\
V = e\vec{E}\\
C = \frac{\epsilon_0 S}{e}
\end{array}\right.
\end{equation}
On peu en tirer que 
\begin{equation}
w = \frac{1}{2}\epsilon||E||^2
\end{equation}


\section{Résistance électrique}
\subsection{Définition du courant}
\begin{wrapfigure}[7]{l}{6cm}
\includegraphics[scale=0.44]{es/image17.png}
\captionof{figure}{Courant électrique}
\end{wrapfigure}
Le courant est un flux de charges électriques 
\begin{equation}
\fbox{$I \equiv \dfrac{\Delta q}{\Delta t}$}
\end{equation}
Le nombre de particules sur un temps $\Delta t$ vaut : $\Delta N =  \eta S v\Delta t$. On multiplie par la charge d'une particules pour avoir la charges : $\Delta q = \eta q_e S v\Delta t$\\
Connaissant la définition du courant $I$, on trouve ce-dernier.
\begin{equation}
I = \eta q_e vS
\end{equation}
L'unité du courant électrique est l'ampère ($A$).\\
\textbf{Attention !} Le sens du courant descend le potentiel (d'où $E = -grad\ V$) \textit{mais} les électrons remontent le potentiel (à cause de leur signe négatif...).

\subsection{Densité de courant}
On utilise souvent en pratique la notion de densité de courant. Il s'agit du courant divisé par la surface.
\begin{equation}
J = \frac{I}{S}= \eta e_e v
\end{equation}
On écrit souvent $\vec{J}$ sous sa force vectorielle. Son sens est toujours le même que celui du champ. \\
Une expression très utilisée en pratique est l'expression du courant sous sa forme intégrale que voici :
\begin{equation}
I = \int_S \vec{J}.d\vec{S}
\end{equation}

\subsection{Source de tension continue : la pile électrique}
Le champ va du $+ \rightarrow -$, mais les électrons font l'inverse.

\subsection{La conduction électrique}
L'accélération des électrons augmente l'énergie cinétique et donc la température. 
\begin{equation}
\vec{v_m} = -\mu \vec{E}
\end{equation}
où $\mu$ est la mobilité et $\vec{v_m}$ la vitesse moyenne des électrons.

\subsection{Forme locale de la loi d'Ohm}
On part de la densité de courant $\vec{J} = \eta q \vec{v}$ où $\vec{v} = -\mu \vec{E}$.\\
On peut ré-écrire $\vec{J} = \eta q \mu \vec{E}$. On pose $\sigma = \eta q \mu$ comme étant le conductivité et l'on tombe sur la loi d'Ohm locale.
\begin{equation}
\fbox{$\vec{J} = \sigma\vec{E}$}
\end{equation}

\subsection{Loi d'Ohm}
$I = ||\vec{J}||.S$ et $\left\{\begin{array}{l}
I = \sigma S ||\vec{E}||\\
||\vec{E}|| = \frac{V}{L}
\end{array}\right.$\ \ \ $\Rightarrow\ \ \ I = \frac{\sigma S V}{L}$ où $\frac{L}{\sigma S} \equiv R$, la résistance électrique.\\
On retrouve la loi d'Ohm, bien connue en secondaire.
\begin{equation}
\fbox{$V = RI$}
\end{equation}
L'unité de la résistance est l'Ohm, $\Omega$.

\subsection{Les résistances}
On définit la résistivité comme étant l'inverse de la conductivité : $\rho_e = \frac{1}{\sigma_e}$.
\begin{equation}
R = \frac{L}{\sigma_e S} = \frac{\rho_e L}{S}
\end{equation}

\subsubsection{Le potentiomètre}
Permet de diviser la tension.

\subsubsection{Association de résistance}
\textit{Cf. Annexe A}

\subsection{Puissance dissipée : effet Joule}
\begin{wrapfigure}[10]{l}{3cm}
\includegraphics[scale=0.44]{es/image18.png}
\captionof{figure}{Puissance dissipée}
\end{wrapfigure}
Toute la charge qui rentre vaut celle qui sort : il y a \textit{conservation de la charge}. La différence de potentiel d'un fil est nulle $\Delta V_{fil} = 0$ car la résistance du fil est nulle ($V = RI$).\\

Dans une résistance, la diminution de l'énergie potentielle ne se converti pas en énergie cinétique (car la vitesse est fixée, $\vec{v_m} = -\mu \vec{E}$) mais en chaleur : transfert collisionnel.\\
L'énergie potentielle (chute de potentiel) est donc convertie en chaleur ! Calculons le travail :
\begin{equation}
\Delta W = V.\Delta q = VI \Delta t
\end{equation}
La puissance étant $P = \frac{\Delta W}{\Delta t}$ :
\begin{equation}
\fbox{$P = IV$}
\end{equation}
Comme V = RI, plusieurs expression de la puissance dissipée peuvent être trouvées :
\begin{equation}
P = RI^2 = \frac{V^2}{R} = IV
\end{equation}
Une résistance à température constante (= à régime) dissipe un débit de chaleur $H = IV$.

\subsection{Circuit RC}
\subsubsection{Décharge du condensateur}
\begin{wrapfigure}[6]{l}{4.5cm}
\includegraphics[scale=0.44]{es/image19.png}
\captionof{figure}{Circuit RC}
\end{wrapfigure}
On a tout d'abord besoin d'une résistance (comme une turbine qui diminue l'$E_C$ pour fournir un $W$) dite "utile" $\rightarrow$ résistance de charge. Nous avons :\\
$\left\{\begin{array}{l}
V = \frac{Q}{C}\\
V = RI\\
I = -\frac{dQ}{dt}
\end{array}\right.$\ \ \ $\Rightarrow\ \ RI = \frac{Q}{C}$.\\
En passant à l'expression différentielle
\begin{wrapfigure}[6]{r}{3cm}
\includegraphics[scale=0.44]{es/image20.png}
\captionof{figure}{Décharge du RC}
\end{wrapfigure}
\begin{equation}
\frac{dQ}{dt} = - \frac{Q}{RC}
\end{equation}
Résolvons l'équation différentielle de premier ordre.
\begin{eqnarray}
\frac{dQ}{Q} &=& -\frac{dt}{RC}\\
\ln (Q) &=& \frac{t}{RC} + C\\
Q &=& Q(0).e^{-\frac{t}{RC}}
\end{eqnarray}

Cette équation différentielle exprime la vitesse ou la charge s'écoule, proportionnellement aux charges restantes dans le condensateur. On peut également trouver le potentiel associé :
\begin{equation}
V(t) = V(0).e^{-\frac{t}{RC}}
\end{equation}
\subsubsection{Charge du condensateur}
Le principe est le même, si ce n'est que l'équation différentielle n'est pas homogène.
\begin{equation}
\frac{dQ}{dt} = -\frac{Q}{RC} + \frac{V_0}{R}
\end{equation}
Le cours d'\textit{Analyse I} nous donne la S.G.E.H et la solution particulière. La Solution Générale de l'Equation Non-Homogène (S.G.E.N.H.) est la suivante (\textit{Cf. Syllabus + Analyse I})
\begin{equation}
V_C(t) = V_0\left(1 - e^{-\frac{t}{RC}} \right)
\end{equation}
\begin{center}
\includegraphics[scale=0.6]{es/image21.png}
\captionof{figure}{Charge du condensateur}
\end{center}














\chapter{Magnétostatique}
\section{Introduction}
Rien de bien important, une bonne lecture et ça ira!

\section{Courants électriques et champ magnétique}
Oersted a mis en avant que le courant électrique génère un champ magnétique perpendiculaire au sens du courant, grâce à l'expérience de la boussole.\\

\begin{wrapfigure}[7]{l}{4cm}
\includegraphics[width=4cm]{magneto/image0.png}
\captionof{figure}{Règle de la main droite}
\end{wrapfigure}

Le fait que la boussole tourne d'un sens et pas dans l'autre met en avant un \textit{problème de brisure de symétrie} qui à beaucoup préoccupé à l'époque. C'est en cherchant à résoudre ce problème qu'Ampère définissa la \textit{règle de la main droite} pour anticiper le sens du courant.

\subsection{Formule de Biot et Savart}
Ceux-ci découvrirent le lien de proportionnalité (expérimentalement, par la prise de mesures systématiques) entre le champ $B$ et $\frac{I}{r}$ grâce aux moments de forces.

\subsection{Interactions entre courants}
Si un courant provoque une force sur un aimant alors forcément l'aimant provoque une force sur le courant (conservation de l'énergie\footnote{Les courants inter-réagissent entre eux.}).\\\\
On peut en déduire que les courants interagissent entre eux et donc :
\begin{enumerate}
\item Si \textit{I} va dans le même sens, il y aura attraction entre ces fils
\item Si \textit{I} va dans le sens opposé, il y aura répulsion entre ces fils
\end{enumerate}
\ 


Pour en arriver à ces conclusions, Ampère a regarder la force magnétique d'un fil $B$ provoquée par un courant dans le fil $A$. (Notons que la force magnétique n'a pas la même direction que le champ qui la génère, elle lui est orthogonale)
\\
On a donc :
\begin{equation}
\Vert \vec{F_M} \Vert\ \propto\ l \frac{I_1 I_2}{d}
\end{equation}

\subsection{Courant dans un champ uniforme}
Considérons un fil passant entre l'entrefer d'un aimant afin de le placer dans un champ magnétique uniforme. En mesurant le module du champ, on remarque que $\vec{F_M}\ \perp\ \vec{1_x}\ \Rightarrow\ \vec{F_M}\ \propto\ I.l.B$.\\
De cette proportionnalité, par convention, on définit l'égalité suivante : $ \vec{F_M}\ \equiv Ilb$ où $[B] = \frac{N}{A.m} = T$\\

\textit{NB :} C'est cette convention qui fixe la définition du Tesla donnée ci-dessus.
\begin{center}
\includegraphics[scale=0.55]{magneto/image1.png}
\captionof{figure}{Fil électrique dans un champ magnétique}
\end{center}
Si le fil possède une inclinaison $\theta$ ($\neq \frac{\pi}{2}$), on remarque expérimentalement que la direction de la force magnétique reste \textbf{inchangée} mais que son module est proportionnel au sinus de l'angle : $\vec{F_M} = I.l.B.\sin(\theta)$.
\\

Introduisons un produit vectoriel pour rendre ceci compact. Sachant que $I = S.||\vec{J}||$ où $S$ est la section du fil (on fait apparaître un vecteur). \\
En introduisant le produit vectoriel (équivalent à la norme des deux vecteurs que multiplie le sinus de l'angle entre les deux), on peut écrire : 
\begin{equation}
\fbox{$\vec{F_M} = l.S.\vec{J} \times \vec{B} $}
\end{equation}

Ce qui respecte bien sur la règle de la main droite.

\subsection{Force de Lorentz}
Histoire que l'électrostatique nous serve à quelque chose, on l'introduit ici :
$\vec{F_M} =  l.S.\vec{J} \times \vec{B}$ et $\vec{J} =  \eta q \vec{v}$ (\textit{Rappel  : $I = \frac{\Delta q}{\Delta t}$ et $\delta q = \eta_c q_e v \delta t S$}.) 
\begin{center}
\includegraphics[scale=0.55]{magneto/image4.png}
\captionof{figure}{Électrostatique - Définition du courant}
\end{center}
En remplaçant l'un dans l'autre :
\begin{equation}
\vec{F_M} = \underbrace{lS\eta_c}_{N_e} q_e \vec{v} \times \vec{B} = N_eq \vec{v} \times \vec{B}
\end{equation}

Le $N_e$ (nombre de particules chargées) dans la dernière égalité nous montre que $F_M$ est la \textbf{somme} des forces magnétiques \textbf{individuelles} exercée sur chacune des particules ayant une vitesse $\vec{v}$. On appellera ainsi la \textit{force de Lorentz} la force magnétique \textbf{individuelle} :
\begin{equation}
\fbox{$f_M = q \vec{v} \times \vec{B} $}
\end{equation}

On peut conclure que le champ magnétique est responsable d'une force sur les charges en mouvement (si les charges n'étaient pas liées au fil, elles s'en iraient!).\\


\textit{NB : } Si un champ électrique $\vec{E}$ est présent en même temps, la force totale est simplement la somme de la force électrique et magnétique.

\textit{NB2 :} Cette force ne dépend pas du signe de la charge, le déplacements des électrons étant inverse au sens du courant. ($\vec{f_M} = (-q_e)(-\vec{v})\times \vec{B} = q_e \vec{v} \times \vec{B}$) \\

\subsection{Particule libre dans un champ uniforme}
Etudions le cas ou la particule n'est plus liée à un matériel conducteur (une charge dans un champ B par exemple).
\begin{center}
\includegraphics[scale=0.75]{magneto/image2.png}
\captionof{figure}{Rayon de courbure (1)}
\end{center}
\begin{wrapfigure}[15]{l}{4cm}
\includegraphics[width=4cm]{magneto/image3.png}
\captionof{figure}{Rayon de courbure (2)}
\end{wrapfigure}
L'égalité des deux expressions ci-dessus, multipliée de part et d'autre par $\vec{v}$ vaut zéro. Ceci est du que le produit vectoriel de $\vec{v}$ donne un vecteur perpendiculaire à lui-même (et donc le produit scalaire de ce nouveau vecteur sera nul). Ceci implique que \textit{la vitesse est constante}.

C'est du au fait que la force de Lorentz est à tout moment perpendiculaire à la vitesse de la particule, et donc à la trajectoire de celle-ci (le vecteur vitesse étant partout tangent à la trajectoire ($\vec{dl} = \vec{v}dl$) et donc cette force ne peut effectuer de travail (seulement possible si $\exists$ composante non nulle dans la direction du déplacement) $\Rightarrow$ pas de $W$, pas de $\Delta E_c$, pas de variation du module de la vitesse.\\


Comme la vitesse est constante, le déplacement décrit une trajectoire circulaire (si la vitesse ne varie pas, le module de $\vec{F}_m$ non plus $\Rightarrow$ force appliquée à la particule toujours perpendiculaire à la vitesse de celle-ci $\Rightarrow$ impose une courbure à la trajectoire). Comme la force centrifuge vaut $\frac{-mv^2}{R}\vec{1_\perp}$ et que nous savons que $\vec{f_m} + \vec{F_c} = 0$, on peut trouver le \textit{rayon de courbure :}
\begin{equation}
\fbox{$R = \frac{mV}{qB}$} 
\end{equation}


\subsection{Relation entre champ et courant : La perméabilité}
\textit{But :} Quantifier le champ magnétique par un courant donné.\\
Comme on sait que $\Vert \vec{B} \Vert \propto \frac{I}{r}$, on introduit un vecteur de proportionnalité $\frac{\mu_0}{2\pi}$ où $\mu_0$ est la \textit{perméabilité du vide}, une constante fondamentale de la nature reliant le courant électrique au champ magnétique qu'il produit.\\

Ampère, par mesures successives étudia la force subie par un fil $B$ du au champ magnétique généré par un fil $A$. Les deux fils étant parallèles, on peut utiliser la formule du champ uniforme : $||\vec{F_M}|| = I l ||\vec{B}||$. En remplaçant dans cette expression la \textit{formule} de Biot et Savart :
\begin{equation}
||\vec{F_M}|| = I_Bl\frac{\mu_0}{2\pi}\frac{I_A}{r}
\end{equation}




Par convention, la force entre les deux fils, avec $I_A = I_B = 1\ A, l = r = 1\ m$ vaut $10^{-7}\ N$. On dès lors trouve la constante recherchée :
\begin{equation}
[\mu_0] = \frac{T.m}{A} = 4\pi . 10^{-7}
\end{equation}

Le $4\pi$ est du au choix de l'unité de courant, l'ampère.\footnote{Le $C$ >> $T$, parce que les effet magnétiques sont très faibles par rapport aux effets électriques.}

\section{Loi d'Ampère}
Comme Gauss pour l'électrostatique, Ampère formula une loi équivalente pour la magnéto-statique.
\subsection{Circulation de champ magnétique}
Calculons l'intégrale de circulation du champ magnétique sur une trajectoire circulaire centrée sur le courant qui génère le champ. Le schéma ci-dessus montre  que le champ est orienté en $\vec{1_\theta}$ (vecteur unitaire polaire), nous permettant d'écrire : $\vec{B} = \frac{\mu_0 I}{2\pi r}\vec{1_\theta}$
\begin{center}
\includegraphics[scale=0.65]{magneto/image5.png}
\captionof{figure}{Circulation du champ magnétique}
\end{center}

C'est simple à calculer, la trajectoire étant une ligne de champ. \textit{Confond} nous apprend qu'il faut effectuer le produit $\vec{B}.\vec{dl}$. Cet accroissement $\vec{dl}$ vaut ici $rd\theta\vec{1_\theta}$. Le produit scalaire couplé à la formule de Biot et Savart nous donne :
\begin{equation}
\vec{B}.\vec{dl} = (B.\vec{1_\theta}).(rd\theta\vec{1_\theta}) = Brd\theta = \frac{\mu_0 I}{2\pi} d\theta
\end{equation}

En intégrant sur tout le contour, soit un angle de $2\pi$, il nous reste simplement $\mu_0 I$.
\begin{equation}
\oint \vec{B}.\vec{dl} = \int_0^{2\pi} \frac{\mu_0 I}{2\pi} d\theta = \mu_0 I
\end{equation}


\subsubsection*{Trajectoire fermée quelconque}
La trajectoire n'étant pas circulaire, il faut tenir compte de l'angle entre $\vec{B}$ et $\vec{dl}$. Nous sommes ici contraint à utiliser un vecteur déplacement généraliser : $\vec{dl} = dr \vec{1_r} + rd\theta \vec{1_\theta} + dz \vec{1_z}$. \\

Mais pas de panique, le champ étant orienté en $\vec{1_\theta}$ seule la composante polaire sera prise en compte ce qui nous ramène au cas étudié ci-dessus. 
La loi d'Ampère généralisée s'exprime ainsi :
\begin{equation}
\fbox{$\oint \vec{B}.\vec{dl} = \mu_0 I$}
\end{equation}


\subsection{Trajectoire ouverte}
Le raisonnement est identique à celui suivi précédemment si ce n'est que de passer de $0$ à $2\pi$ on intègrera de $\theta_i$ à $\theta_f$ :
\begin{equation}
\int_{i\rightarrow f} \vec{B}.\vec{dl} = \frac{\mu_0}{2\pi}I\Delta\theta
\end{equation}


\subsubsection{Courant extérieur}
On va dans un sens, puis dans l'autre ! Le résultat est logique :
\begin{equation}
\oint \vec{B}.\vec{dl} = 0
\end{equation}

Il faut donc que ce que l'on appellera plus tard le \textit{chemin ampérien} entoure la source de courant.

\subsubsection{Principe de superposition}
Également d'application en magnéto, il permet de généraliser la Loi d'Ampère à plusieurs courant.
\begin{equation}
\oint \vec{B}.\vec{dl} = \mu_0 \sum_{n=1}^N I_n
\end{equation}

Il faut tenir compte du sens du courant : si le sens de circulation correspond au sens opposé à celui de la main droite, alors il doit être considéré comme négatif.

\section{Loi de Biot \& Savart}
Ampère ne nous permet en réalité que de calculer l'expression du champ magnétique dans des configurations de courant particulières. La \textbf{Loi} deBiot \& Savart a été obtenue expérimentalement et permet de calculer le champ magnétique \textit{en un point} dans des configurations \textit{non symétrique}. Biot et Savart ont calculé le champ magnétique en une multitude de points sans trouver de relation. 

\begin{wrapfigure}[15]{r}{4cm}
\includegraphics[scale=0.55]{magneto/image6.png}
\captionof{figure}{Système de coordonnée sphérique}
\end{wrapfigure}

C'est Pierre-Simon Laplace qui a réussi à en trouver une en se basant sur un système de coordonnées sphériques. En gros, il a 'vu' cette  horreur : $ ||\vec{B}|| \propto Il sin\phi / r^2$. Il remarqua aussi la présence d'un coefficient de proportionnalité valant $\frac{\mu_0}{4\pi}$.
%\begin{center}
%\includegraphics[scale=0.55]{magneto/image6.png}
%\end{center}



Pour introduire une notation compacte, sachant que le champ est dirigé en $\vec{1_\theta}$ et vu la présence d'un sinus, on peut réécrire $sin\phi \vec{1_\theta}$ en $\vec{1_z} \times \vec{1_r}$. En introduisant le fait que le fil de courant est orienté en $\vec{1_z}$ on peut formuler la \textbf{Loi de Biot et Savart} sous sa forme intégrale :
\begin{equation}
d\vec{B} = \frac{\mu_0}{4\pi r^2}I\vec{dl} \times \vec{1_r}
\end{equation}

ou encore sous sa forme intégrale :
\begin{equation}
\fbox{$\vec{B} = \frac{\mu_0 I}{4\pi} \int \frac{1}{r^2}\vec{dl} \times \vec{1_r}$}
\end{equation}


\section{Calculs de champs magnétiques}
\subsection{Application de la loi d'Ampère}
\subsubsection{Le fil rectiligne}
Tentons de retrouver la formule de B\&S en partant de notre brave relation fraichement découverte.\\

Choisissons un contour Ampérien circulaire centré sur le fil. Ce dernier ayant une symétrie de révolution, il ne peut être \textbf{fonction} de la coordonnée polaire, ni même de la coordonnée en $z$, le fil étant supposé infini.


Il faut trouver maintenant la \textbf{direction} du champ (que nous avons justement oubliée). Si l'on procède à l'intégrale de circulation $\int_C \vec{B}.\vec{dl}$\footnote{$\vec{dl} = dr.\vec{1_r}$}, l'ouverture $\Delta\theta$ sera nul $\Rightarrow$ le champ sera nul $\Rightarrow$ $\vec{B}.\vec{1_r} 0 = \Rightarrow \vec{B} \perp \vec{1_r}$.\\


Un raisonnement similaire peut être tenu avec l'axe $z$ impliquant $\vec{B}.\vec{1_z} = 0$. Par élimination, le champ ne peut qu'être orienté selon le vecteur unitaire polaire : $\vec{B} = B(r).\vec{1_\theta}$. Comme $\vec{dl} = rd\theta\vec{1_\theta}$ le résultat est immédiat :
\begin{equation}
\oint_C Brd\theta = \mu_0 I \Leftrightarrow rB2\pi = \mu_0 I \Rightarrow B = \frac{\mu_0}{2\pi}\frac{I}{r}
\end{equation}



\subsubsection{La "paroi de courant"}
Une paroi de courant est une rangée plane de fils conducteurs transportant tous le même courant $I_0$. Ampère nous permet de calculer fil par fil pour ensuite utiliser le principe de superposition et calculer toute la plaque.

\begin{center}
\includegraphics[scale=0.55]{magneto/image7.png}
\captionof{figure}{Chemin ampérien}
\end{center}


Par symétrie, il apparaît évident que le champ est en tout point parallèle à la plaque. On choisira dès lors un contour ampérien qui est soit parallèle ($\vec{B}.\vec{dl} = Bdl$) au champ, soit perpendiculaire à celui-ci ($\vec{B}.\vec{dl} = 0$). \\
La longueur d'un côté parallèle au champ vaut L, le produit scalaire vaudra $BL$. Comme une plaque possède deux côtés (Sisi !), on a $2BL$ et ensuite, ce n'est plus que formalité ! 
\begin{equation}
\oint_C \vec{B}.\vec{dl} = \mu_0I \Leftrightarrow 2BL = \mu_0N I_0
\end{equation}

En faisant apparaître la notion de \textit{densité surfacique de courant}, c'est à dire le courant par unité de longueur transversant la paroi, soit $J_S = \frac{NI_0}{L}$ on trouve le module du champ généré par la paroi.
\begin{equation}
\fbox{$B = \frac{1}{2}\mu_0J_S$}
\end{equation}

\textit{NB :} Ce n'est le champ que d'un côté de la plaque !

\subsection{Application de la Loi de Biot et Savart}
Quand la configuration est trop complexe pour être traitée par la loi d'ampère, on utilisera B\&S.

\subsubsection{Champ au centre d'une spire}
(Notons avant tout que les calculs se font dans le cas ou $z$ est l'axe perpendiculaire au plan de la spire)
La difficulté de cette loi est de rendre son intégrale calculable. Pour le vecteur $\vec{dl}$, considérons la spire comme  un cercle. Ce vecteur vaudra ainsi $Rd\theta\vec{1_\theta}$. 

\begin{wrapfigure}[12]{r}{4cm}
\includegraphics[scale=0.55]{magneto/image8.png}
\captionof{figure}{Champ au centre d'une spire}
\end{wrapfigure}


Ces deux vecteurs étant perpendiculaire, leur produit vectoriel se calcule simplement : $\vec{dl} \times \vec{1_r} = Rd\theta\vec{1_z}$. Il reste donc à intégrer le résultat qui suit :
\begin{equation}
d\vec{B}(0) = \frac{\mu_0}{4\pi R^2}IRd\theta\vec{1_z}
\end{equation}

Comme tout est constant à l'exception de $d\theta$, tout peut sortir de l'intégrale et le résultat de l'intégrale vaut $2\pi$. Le champ au centre d'une spire peut ainsi être calculé.
\begin{equation}
\vec{B}(0) = \frac{\mu_0 I}{2R}\vec{1_z}
\end{equation}

\textit{NB :} On peut voir le vecteur $\vec{1_r}$ comme le vecteur \textit{pointant vers  notre œil}. En gros, il pointe vers le point ou on veut calculer le champ.

\subsubsection{Champ sur l'axe d'une spire}
Tout comme précédemment, $\vec{dl} \perp \vec{1_r}$, le produit vectoriel se calcul aisément ; $||d\vec{B}|| = \frac{\mu_0 I}{4\pi r^2}dl$.\\
La spire étant centrée, on peut toujours s'arranger pour obtenir un champ orienté le long de l'axe $z$ comme suggérer sur le schéma ci-dessous.

\begin{center}
\includegraphics[scale=0.70]{magneto/image9.png}
\captionof{figure}{Champ sur l'axe d'une spire}
\end{center}



Comme une paire d'éléments de longueur diamétralement opposée génère un champ de $2dB_z\vec{1_z}$, le champ magnétique total sera également en $\vec{1_z}$. Comme le montre le schéma, ce petit champ infinitésimal $d\vec{B}$ fait un angle de $\frac{\pi}{2} - \phi$ entre le vecteur unitaire radial $\vec{1_r}$ et l'axe $z$.\\

Cet angle représente une sorte d'\textit{ouverture angulaire} dont il faut tenir compte en multipliant le résultat obtenu ci-dessus par le cosinus de cet angle.
\begin{equation}
dB_z = \frac{\mu_0}{2\pi r^2}Idl\sin (\phi)
\end{equation}

Comme $dl = Rd\theta$ et que $\theta$ varie de $0 \rightarrow 2\pi$, on peut calculer le champ générer par une spire sur l'axe $z$.
\begin{equation}
B = \frac{\mu_0}{2 r^2} IR \sin (\phi)
\end{equation}

Il ne nous reste qu'à nous débarrasser de $r$ (à ne \textbf{pas} confondre avec R!) en nous servant de notre trigonométrie de base : $r.\sin (\phi) = R$.
\begin{equation}
\vec{B} = \frac{\mu_0}{2R}I \sin^3(\phi)\vec{1_z}
\end{equation}


Comme on pouvait s'en douter, le champ est maximal au centre de la spire, lorsque l'ouverture vaut 90 degrés. Quand on s'éloigne, le champ suit malheureusement pour lui une décroissance en $\sin^3\phi$.\\

\begin{wrapfigure}[9]{l}{4cm}
\includegraphics[scale=0.50]{magneto/image10.png}
\captionof{figure}{Représentation intuitive}
\end{wrapfigure}

Calculer le champ d'une spire en tout point de l'espace est trop complexe, on peut néanmoins se faire une idée intuitive en se basant sur le schéma ci-contre.


\subsubsection{Le solénoïde}
Il s'agit "simplement" d'un ensemble de spire parallèles parcourues par le même courant. Pour calculer le champ, utilisons le champ généré par une spire (trouvé un peu plus haut) et utiliser le principe de superposition.\\

Plaçons la spire en $z = z'$. Si le point de calcul est repérée par l'ordonnée $z$, on peut en déduire une relation remarquable liant $z'$ à $\phi$ : $R = (z - z')tan(\phi)$.

\begin{center}
\includegraphics[scale=0.70]{magneto/image11.png}
\captionof{figure}{Calcul du champ du solénoïde}
\end{center}

Pour calculer le champ, on va considérer que le solénoïde contient un grand nombre de spires très serrées caractérisée par une densité linéique de spires notée $e$. Ainsi, un élément de longueur $dz'$ contient $ndz'$ spires.\\

Le champ total est alors donné par la sommes sur $z'$ du début du solénoïde ($z' = z_{min}$) jusqu'à la fin ($z' = z_{max}$).
\begin{equation}
B(z) = \frac{\mu_0 nI}{2R}\int \sin^3\phi dz'
\end{equation}

Cette intégrale n'est pas calculable. Il faut procéder à un changement de variable pour ré-exprimer $dz'$ en fonction de $\phi$. Nous effectuons les opérations suivantes :
\begin{itemize}
\item $z - z' = \frac{R}{tg(\phi)} = Rcotg(\phi)$
\item $-dz' = R\frac{d}{d\phi}[cotg(\phi)]d\phi$
\item $dz' = \frac{R}{sin^2(\phi)}d\phi$
\end{itemize}
Ce qui nous donne :
\begin{equation}
B(z) = \frac{\mu_0 nI}{2}\int_{\phi_{min}}^{\phi^{max}} \sin\phi d\phi
\end{equation}

Pas besoin d'aide pour calculer l'intégrale, celle ci vaut (où $cos(\phi_{max}) = cos(\pi - \phi_{max})$) : 
\begin{equation}
B(z) = \frac{\mu_0 nI}{2}[cos(\phi_{min}) - cos(\pi - \phi_{max})]
\end{equation}

L'interprétation physique est immédiate  : quand on se situe à un point \textit{éloigné} $\rightarrow \phi_{max} = \pi$ et donc le champ est nul. Alors que si on se trouve au centre, $\rightarrow \phi_{min} = \frac{\pi}{2}$\\ \\
\emph{Milieu du solénoïde}\\\\
Le point centre du solénoïde se caractérise par l'égalité $\phi_{min} = \pi - \phi_{max}$ de sorte qu'on puisse exprimer le champ tel que : $B = \mu_0nIcos(\phi_{min})$\\

On remarque à partir de cette relation que plus $L >> R$, plus l'angle va tendre vers zéro, et donc, le champ tendra à devenir maximal. Lorsque la condition $L >> R$, on se trouve dans un \textit{solénoïde idéal} (c'est à dire arbitrairement long par rapport à son rayon de sorte que l'on puisse condidérer que $\phi_{min} = 0$ et $\pi - \phi_{max} = 0$ pour tous ses points intérieurs, sauf tout près de ses extrémités) dont le champ interne vaut  :
\begin{equation}
B_{int} = \mu_0nI\footnote{Le $n$ est bien la densité linéique de spire !}
\end{equation}

\emph{Extrémité du solénoïde}\\\\
A l'extrémité gauche, au niveau de la spire d'entrée du solénoïde idéal, $\phi_{min} = \frac{\pi}{2} \Rightarrow B = \frac{1}{2}\mu_0nIcos(\pi - \phi_{max})$. Le soso étant considérer idéal ; $\pi - \phi_{max} = 0$. On appellera $B_{ext}$ le champ aux extrémités pour rappeler ce fait.
\begin{equation}
\fbox{$B_{ext} = \frac{1}{2}\mu_0nI$}
\end{equation}


\begin{center}
\includegraphics[scale=0.60]{magneto/image12.png}
\captionof{figure}{Extrémité du solénoïde}
\end{center}

Introduisons la notion de \textit{densité de courant de surface}. Sachant que le nombre $N$ de fils est donné par $\frac{\Delta z}{d}$ où $d$ est le diamètre du fil. Le courant total correspond à cette expression multipliée par $I$. En divisant par la longueur qu'il occupe, soit $\Delta z$, on obtiens la densité de courant de surface du solénoïde : $J_S = \frac{NI}{\Delta z} = nI$.

\begin{equation}
\fbox{$B_{int} = \mu_0J_S$}
\end{equation}

\subsubsection{Interprétation des propriétés magnétiques de la matière}
Une lecture attentive des pages $53 - 56$ devrait suffire !

\section{Couple de force sur une spire de courant}
Un aimant plongé dans un champ subit un moment de force qui tend à l'aligner avec de champ. C'est le moment de le calculer ! \\
Le schéma ci-dessous représente une spire de courant plongée dans un champ magnétique uniforme avec un angle non nul et ses côtés verticaux perpendiculaire au champ.
\begin{center}
\includegraphics[scale=0.50]{magneto/image13.png}
\captionof{figure}{Couple de force sur une spire de courant}
\end{center}

Les forces magnétiques (en vert) des côtés horizontaux s'annulent, mais ce n'est pas le cas des deux autres (elles n'ont en effet pas le même support).\\

\begin{wrapfigure}[13]{l}{4cm}
\includegraphics[scale=0.50]{magneto/image14.png}
\captionof{figure}{Calcul du moment de force}
\end{wrapfigure}

\textit{Confond} nous apprend que le moment se calcule en multipliant la force par le bras de levier. Dans notre cas, il faut tenir compte de l'angle $\phi$ présent entre $\vec{B}$ et $\vec{1_S}$.\\



Le moment calculé vaut donc : $\tau = ||\vec{F_M}||l_h sin\phi$ ou en remplaçant la force magnétique par sa valeur on trouve : $\tau = I l_v B l_h sin\phi$.\\
En introduisant la surface de la spire $ S = l_v l_h$ on a :
\begin{equation}
\fbox{$\tau = ISBsin\phi$}
\end{equation}

Cette formule est valable pour une surface quelconque. Il suffit d'imaginer pleins de spires rectangulaires élémentaires. Les côtés tangents aux spires verront leurs courants devenir nul et ne restera que les courants de surfaces.
\subsection{Moment magnétique dipolaire}
Le moment est une grandeur vectorielle, généralisons l'expression en une expression vectorielle en faisant disparaître le sinus grâce à un... produit vectoriel !
\begin{equation}
\fbox{$\tau = IS\vec{1_S} \times \vec{B}$}
\end{equation}
On appellera le facteur $IS\vec{1_S}$ le moment magnétique noté $\vec{m_M}$ de la sorte que l'on puisse écrire $\tau = \vec{m_M} \times \vec{B}$.

\subsection{Champ magnétique dans la matière}
\subsubsection{Paramagnétisme}
Les électrons s'associent par paires opposées causant l'annulation du moment magnétique. Mais parfois, leur nombre est impair et il se crée alors un moment magnétique. Si l'on plonge un tel matériau dans un champ magnétiques, les atomes vont s'aligner au champ et subir une \textit{magnétisation}.\\
\\
En gros : \textit{"Les matériaux paramagnétiques sont les matériaux dans lesquels le phénomène d'alignement induit par un champ magnétique extérieur se produit en compétition avec l'agitation thermique".} En effet, l'agitation thermique peut empêcher l'alignement des atomes et donc leur magnétisation.\\

Le matériel aimanté peut être vu comme un aimant (un ensemble de spires générant des courants de surfaces) générant un \textit{champ magnétique induit} $\vec{B_i}$ qui va s'ajouter au champ initial $B_0$ pour former le champ total (on somme les deux). Pour exprimer ceci, on utilise généralement la perméabilité relative $\mu_r = \frac{\mu}{\mu_0}$. Logiquement, celle-ci est toujours supérieur à l'unité pour les matériaux paramagnétiques.\\

\textit{NB :} Ici, contrairement à l'électrostatique, les deux champs vont dans le même sens expliquant la présence de $\mu_0$ au numérateur.

\subsubsection{Ferromagnétisme}
Certains atomes ayant un nombre impair d'électrons ont des propriétés $\neq$ du aux interactions entre les moments magnétiques des atomes voisins.  Cela conduit à un alignement spontané des moments magnétiques ou certains "domaines" du matériau "reçoivent" une aimantation permanente.\\

\textit{Un matériau est ferromagnétique si certaines parties deviennent des aimants même après avoir retiré $\vec{B}$.} Ici, l'agitation thermique ne suffit plus pour désorienter les moments magnétiques.

\subsubsection{Le moteur à courant continu}
Encore une fois, une bonne lecture suffit ! (\textit{Cf. page 69 - 71})

\section{Forme locale de la loi d'Ampère}
La difficulté des lois d'Ampères et de B\&S vient du fait qu'elles sont non-locales : le champ est calculé à une certaine distance des courants. Existe-t-il une loi locale liant $\vec{B}$ et $I$ \textbf{au même} point ?

\subsection{Distribution de courant continue}
\begin{wrapfigure}[7]{l}{4cm}
\includegraphics[scale=0.50]{magneto/image15.png}
\captionof{figure}{Densité de courant}
\end{wrapfigure}

La loi d'Ampère développée quelques pages avant ne s'applique qu'aux courants dans les fils. Pour la généraliser, on va préférer la notation de densité de courant $\vec{J}$.\\





On définit une \textit{surface sous-tendue par le contour} C avec laquelle on va calculer le flux.
\begin{equation}
\oint \vec{B}.\vec{dl} = \mu_0 \int_{S_C} \vec{J}.\vec{dS}
\end{equation}


\subsection{Contour élémentaire}
Considérons une distribution de courant continue ou $\vec{J}(\vec{x})$ est la valeur de la distribution au point $\vec{x}$. On va appliquer Ampère à un contour élémentaire autour de $\vec{x}$. Prenons-en un dans le plan $y-z \rightarrow \perp \vec{1_x}$ (d'ou le contour $C_x$ (\textit{cf. ci-dessous}).
\begin{center}
\includegraphics[scale=0.60]{magneto/image16.png}
\captionof{figure}{Choix d'un contour élémentaire}
\end{center}
La loi d'Ampère pour ce contour vaut : 
\begin{equation}
\oint_{C_x} \vec{B}.\vec{dl} = \mu_0\int_{\Delta S} \vec{J}d\vec{S}
\end{equation}

Calculons le membre de droite. Comme $C_x \rightarrow 0 \Rightarrow \vec{J}(x) = cste$ ; on peut le sortir de l'intégrale. Cette dernière ne contient que $\vec{dS}$ ce qui nous donne $\Delta \vec{S} = \Delta y\Delta z \vec{1_x}$.
\begin{equation}
\oint_{C_x} \vec{B}.\vec{dl} = \mu_0J_x \Delta y\Delta z
\end{equation}

Calculons maintenant le champ grâce au contour ampérien :
\begin{equation}
\oint \vec{B}.\vec{dl} = \int \vec{B}.\vec{dl}_1 + \int \vec{B}.\vec{dl}_2 + \int \vec{B}.\vec{dl}_3 + \int \vec{B}.\vec{dl}_4 = \mu_0J_x \Delta y\Delta z
\end{equation}

Commençons par les côtés 1 et 3.\\
Comme $\vec{dl}_1 \parallel \vec{1_z}$, le produit scalaire $\vec{B}.\vec{dl}_1$ fait apparaître la composante en $z$ du champ magnétique :
\begin{equation}
\vec{B}.\vec{dl}_1 = B_z(x, y + \frac{\Delta y}{2}, z')dz'
\end{equation}

\begin{center}
\includegraphics[scale=0.60]{magneto/image17.png}\\
\captionof{figure}{Circulation le long d'un contour élémentaire}
\end{center}
En suivant le même raisonnement avec le côté 3
\begin{equation}
\vec{B}.\vec{dl}_3 = -B_z(x, y - \frac{\Delta y}{2}, z')dz'
\end{equation}

Plutôt que de garder z', considérons la largeur $\Delta z$ ; C'est identique comme nous somme dans le cas d'un accroissement infinitésimal. Le schéma suivant l'explicite bien :
\begin{center}
\includegraphics[scale=0.60]{magneto/image18.png}\\
\captionof{figure}{Approximation de la surface}
\end{center}

\textbf{Attention :} le passage de $z' \rightarrow z$ ne signifie pas que $z$ ne varie pas !\\

En sommant les deux expressions et en multipliant par $\frac{\Delta y}{\Delta y}$ :
\begin{equation}
\int \vec{B}.\vec{dl}_1 + \int \vec{B}.\vec{dl}_3 = [\frac{B_z(x, y + \frac{\Delta y}{2}, z)  - B_z(x, y - \frac{\Delta y}{2}, z)]}{\Delta y}\Delta z\Delta y
\end{equation}

Ce qui correspond à la dérivée partielle de $B_Z$ par rapport à $y$. Autrement dit :
\begin{equation}
\int \vec{B}.\vec{dl}_1 + \int \vec{B}.\vec{dl}_3 = \frac{\partial B_z}{\partial y}\Delta y\Delta z
\end{equation}


En suivant un raisonnement similaire pour le côté 2 et 4 :
\begin{equation}
\int \vec{B}.\vec{dl}_2 + \int \vec{B}.\vec{dl}_4 = -\frac{\partial B_y}{\partial z}\Delta y\Delta z
\end{equation}


En sommant le tout :
\begin{equation}
\oint_{C_x} \vec{B}.\vec{dl} = [\frac{\partial B_z}{\partial y} - \frac{\partial B_y}{\partial z}]\Delta y\Delta z = \mu_0J_x\Delta y\Delta z
\end{equation}


Ce qui donne après simplification :
\begin{equation}
\frac{\partial B_z}{\partial y} - \frac{\partial B_y}{\partial z}= \mu_0J_x
\end{equation}


\subsection{Loi d'Ampère locale}
On peut faire les mêmes raisonnements pour $C_y$ et $C_z$ :
\begin{center}
\includegraphics[scale=0.70]{magneto/image19.png}\\
\captionof{figure}{Loi d'Ampère locale}
\end{center}
En regardant la symétrie, on remarque que celle-ci est identique à celle du produit $\vec{A} \times \vec{B}$ où $\vec{A} = \vec{\nabla}$ (le fameux opérateur nabla)\\
\begin{equation}
\fbox{$Loi\ d'Ampère\ locale\ \ \ \ \ :\ \ \ \vec{\nabla} \times \vec{B} = \mu_0\vec{J}$}
\end{equation}


\subsection{Le rotationnel}
$\vec{\nabla} \times$ est un opérateur vectoriel nommé \textit{rotationnel} $\Rightarrow rot\ \vec{B} = \mu_0\vec{J}$.\\
Le terme \textit{rot} rappelle que les composantes du rotationnel d'un champ vectoriel sont obtenues par le calcul de la circulation de ce camp sur les contours fermés infinitésimaux.

\subsection{Interprétation du rotationnel}
En nous basant sur $C_x$ :
\begin{equation}
\oint_{C_x} \vec{B}.\vec{dl} = rot\ \vec{B}\vec{1_x}\Delta S_x \Rightarrow rot\ \vec{B}\vec{1_x} = \frac{\oint_{C_x} \vec{B}.\vec{dl}}{\Delta S_x}
\end{equation}

La composante en $x$ du rot est donc l'intégrale de circulation d'un contour élémentaire perpendiculaire à l'axe $x$ divise par la surface sous-tendue à ce contour.\\
\textit{Circulation normalisée par la surface sous-tendue le contour $C_x$}.

\subsection{Généralisation }
Facilement généralisable en changeant de repère et en repérant l'orientation de la surface élémentaire par le vecteur $\vec{1_{C_x}}$.\\

En faisant tendre $\Delta S_{C_x} \rightarrow 0$, cela revient à prendre la norme du rot multiplié par le cosinus de l'angle entre ce dernier et $\vec{1_{C_x}}$. (On voit que si l'angle est nul, on est au maximum. Tout le reste est toujours inférieur)\\

En conclusion, le rot en $\vec{x}$ est la circulation normalisée \textbf{la plus grande} que l'on puisse avoir au point $\vec{x}$ pour toute les orientations possibles. De plus, le rot pointe dans la direction de la circulation la plus grande. 

\section{Théorèmes d'analyse vectorielle}
\subsection{Théorème de Stokes}
Considérons la loi d'Ampères : $\oint_C \vec{B}.\vec{dl} = \mu_0 \int_{S_c} \vec{J}.\vec{dS}$ et l'expression du rotationnel : $rot\ \vec{B} = \mu_0\vec{J}$. En isolant $\vec{J}$ dans la seconde expression et en l'injectant dans la première :
\begin{equation}
\oint_C \vec{B}.\vec{dl} = \int_{S_C} rot\ \vec{B}.\vec{dS}
\end{equation}

\textit{Le flux du rot $\vec{B}$ au travers de la surface $S_C$ sous tendue par le contour $C$ est égale à l'intégrale de circulation de $\vec{B}$ sur ce contour.}\\
La généralisation (Pour une fonction vectorielle $\vec{F}$ est le \textbf{Théorème de Stokes :}
\begin{equation}
\oint_C \vec{F}.\vec{dl} = \int_{S_C} rot\ \vec{F}.\vec{dS}
\end{equation}


\subsection{Théorème d'Ostrogradski}
Théorème se rapportant à la divergence d'un champ vectoriel. Connaissant l'expression de la loi de Gauss : $\oint_S \vec{E}.d\vec{S} = \frac{1}{\epsilon_0} \int \rho(\vec{x})dV$ et son expression locale : $div.\vec{E} = \frac{1}{\epsilon_0}\rho(\vec{x})$. En remplaçant l'une dans l'autre : 
\begin{equation}
\oint_S \vec{E}.d\vec{S} = \int_{V_S} div.\vec{E}\ dV
\end{equation}

\textit{Le flux du champ électrique à travers d'une surface $S$ vaut l'intégrale de volume de sa divergence sur tout le volume $V_S$ enfermé par la surface $S$ est égale à l'intégrale de circulation de $\vec{B}$ sur ce contour.}\\

La généralisation (Pour une fonction vectorielle $\vec{F}$ est le \textbf{Théorème d'Ostrogradski :}
\begin{equation}
\oint_S \vec{F}.d\vec{S} = \int_{V_S} div.\vec{F}\ dV
\end{equation}

\subsection{Illustrations}
\subsubsection{Rotationnel du champ électrostatique}
Soit $\vec{E}$ et son potentiel $V \rightarrow \Delta V = - \int_i^f \vec{E}.\vec{dl}$. Or, sur un contour fermé, le potentiel est nul $\Rightarrow \oint \vec{E}.\vec{dl} \equiv 0$.\\

Or selon le Th. de Stokes : $\oint_C \vec{F}.\vec{dl} = \int_{S_C} rot\ \vec{F}.\vec{dS} \Rightarrow rot\ \vec{E} = \vec{0}$, c'est à dire que le champ est conservatif (\textit{Cf. méca}).

\subsubsection{Rotationnel d'un gradient}
Comme rot $\vec{E} = \vec{0}$ et $E = - grad\ V \Rightarrow rot\ \vec{E} = -rot[grad\ V] = \vec{0}$.\\
Ainsi, pour toute fonction vectorielle $\vec{F} : rot(grad(\vec{F})) = \vec{0}$

\subsubsection{Divergence du rotationnel}
Appliquons le Th. de Stokes à un contour ferme $C$ dont la longueur tend vers zéro.\\
$\lim\limits_{\substack{C \to 0}} \oint_C \vec{F}.\vec{dl} = \int_{S_C} rot\ \vec{F} d\vec{S} = 0 \Rightarrow$ Le Th. de Stokes nous indique que le flux du rot de $\vec{F}$ au travers de tout surface fermée $S$ est nul.\\
Appliquons le Th. d'Ostrogradski à $\vec{G}$ où $\vec{G} = rot\ \vec{F}$ :
\begin{equation}
\oint_S rot\ \vec{F}d\vec{S} = \int_{V_S} div(rot\ \vec{F})dV = 0\ \ \ (\forall\ \ V_S)\ \ \ \Rightarrow div[rot\ \vec{F}(\vec{x})] \equiv 0
\end{equation}


\section{Divergence du champ magnétique}
Considérons un élément de courant élémentaire véhiculant un courant $I$. La loi de Bios \& Savart nous informe que $\vec{B} = \frac{\mu_0}{4\pi r^2}I\Delta l\ \vec{1_l} \times \vec{1_r}$. Considérons une surface $S$ délimitant un \textit{tube de flux} représenté ci-dessus vu d'ensemble et vu du  haut.
\begin{center}
\includegraphics[scale=0.60]{magneto/image20.png}\\
\captionof{figure}{Divergence du champ magnétique - Courant extérieur}
\end{center}
Le flux d'entrée et de sortie sont égaux en valeur absolue. On peut exprimer l'équilibre de la façon suivante : $\vec{B}.\Delta\vec{S}_e + \vec{B}.\Delta\vec{S}_s = 0$.\\

Imaginons que l'espace soit découpé d'un ensemble de $N$ tubes: $\sum_{n=1}^N [\vec{B}.\Delta\vec{S}_e + \vec{B}.\Delta\vec{S}_s] = 0$. En passant à la limite infinitésimale, j'ai bien : $\oint_S \vec{B}.\vec{dS} = 0$. On en conclus qu'un \textbf{flux de champ magnétique sur une surface fermée extérieure est toujours nulle}.\\

Si on considère un élément de courant intérieur, certaines lignes de champ n'interceptent pas la surface et ne participent donc pas au flux. D'autres l'interceptent en sortant mais quand elles l'intercepte à nouveau pour rentrer, le flux s'annule.
\begin{center}
\includegraphics[scale=0.60]{magneto/image21.png}\\
\captionof{figure}{Divergence du champ magnétique - Courant intérieur}
\end{center}
$\Rightarrow \oint_C \vec{B}.\vec{dS} = 0$. Appliquons le Th. d'Ostrogradski :
\begin{equation}
\oint_S \vec{B}.d\vec{S} = \int_{V_S} div.\vec{B}\ dV = 0
\end{equation}

En faisant tendre $V_S \rightarrow 0$, la divergence peut être considérer comme constante et sortir de l'intégrale. Il ne reste plus que l'intégrale de $dV$ qui tend vers 0. (Considérons pour la ligne ci-dessus que $V_S \rightarrow 0$)
\begin{equation}
\int_{V_S} div\ \vec{B}dV = div\ \vec{B}\int_{V_S} dV = div\ \vec{B} V_S = 0
\end{equation}
\begin{equation}
\fbox{$div\ \vec{B} = 0 $}
\end{equation}
\\

Ce résultat est généralisable en sommant les $\vec{B}$ élémentaires.
















































\chapter{Électromagnétisme}
\section{Introduction}
Discipline scientifique décrivant les systèmes de charges électriques dans les situations les plus générales de charges et de courants variables.

\subsection{La dynamo}
Soit un courant en déplacement dans un champ magnétique uniforme $\vec{B}$ (par exemple, un fil effectuant une translation de vitesse $\vec{v}$. Les charges (considérons les positives) vont ainsi subir une force de Lorentz ($q\vec{v} \times \vec{B}$)orientée dans l'axe du fil causant la migration des charges libres transversalement.
\begin{center}
\includegraphics[scale=0.35]{em/image0.png}\\
\captionof{figure}{Fil plongé dans un champ magnétique uniforme}
\end{center}
La migrations des charges provoque une accumulation de charges positives dans le bas de la tiges mais aussi un défaut de charge positives sur la partie supérieure qui se charge négativement $\Rightarrow$ génération d'un champ électrostatique $\vec{E}_S$ dirigé vers le haut, s'opposant à la migration des charges (de par la force de Lorentz) créant un potentiel entre les extrémités. (\textbf{Attention !} Remarque \textit{page 3})
\begin{center}
\includegraphics[scale=0.55]{em/image1.png}\\
\captionof{figure}{Apparition de la force de Lorentz}
\end{center}
Ceci a été découvert expérimentalement par Faraday, qui inventa d'ailleurs la dynamo.\\

Le principe est de mettre un disque en rotation dans l'entrefer d'un aimant $\Rightarrow$ Forces de Lorentz $\Rightarrow$ Migration des charges $\Rightarrow$ Différence de potentiel établie $\Rightarrow$ Crée une source de tension.

\subsection{Force électromotrice}
Calculer le champ électrostatique $\vec{E}_S$ est simple : Sachant que $\vec{f_{E_S}} + \vec{F_M} = 0$ ($q\vec{E}_S + q\vec{v}\times \vec{B}$), on trouve $\vec{E}_S = -\vec{v} \times \vec{B}$.\\
Or, de par la géométrie du problème $\vec{v} \perp \vec{B} \Rightarrow ||\vec{v} \times \vec{B}|| = vB$. Si le fil est de longueur $l$, $V$ se trouve facilement.
\begin{equation}
\fbox{$Force\ électromotrice\ \equiv V = lvB $}
\end{equation}

\textit{NB :} Cette force n'est \textbf{pas} en Newton ! \textit{Electromotrice} traduit le fait que l'on est dans un mécanisme de séparation de charges.

\section{L'induction électromagnétique}
\subsection{Relativité galiléenne et force de Lorentz}
La relativité Galiléenne stipule que tout référentiel inertiel sont équivalents, ce qui n'est pas vérifié ici (particule au repos ou le champ magnétique se déplace à vitesse $-\vec{v}$)(En effet, il n'y a pas de raison de dire que c'est la tige qui bouge et non l'aimant.)\\
Dans une telle configuration, la vitesse de la particule est nulle, la force de Lorentz doit l'être aussi ($\vec{f}_m = q\vec{v}\times\vec{B}\ où\ \vec{v} = \vec{0}$) $\Rightarrow$ contradiction avec ce qui est dit ci-dessus.\\

Pour régler le problème, on introduit une nouvelle force, identique à celle de Lorentz mais non magnétique ($\vec{v} = \vec{0}$). $\vec{B}$ étant en mouvement, cette force serait du au mouvement, au "balayement" des lignes de champ "créant" cette nouvelle force. Faraday propose l'apparition d'un champ électrique induit.
\begin{center}
\includegraphics[scale=0.55]{em/image2.png}\\
\captionof{figure}{Calcul du champ induit}
\end{center}
Une charge électrique au repos dans une zone ou $\vec{B}$ est en mouvement suivra une force électrique remplaçant la force de Lorentz (son expression rappelle d'ailleurs cette dernière)
\begin{equation}
\fbox{$ Champ\ électrique\ induit\ \equiv \vec{E_i} = \vec{v} \times \vec{B}$}
\end{equation}

Le phénomène de balayage des lignes de champ est appelé \textit{induction électromagnétique}.\\
\textbf{Attention !} Le champ électrique induit n'est pas généré par des
charges électriques en tant que telles mais bien par un mouvement de champ
magnétique.

\subsection{Induction et force électromotrice}
Considérons une tige fixe véhiculant du courant et un aimant bougeant à la vitesse $^-\vec{v}$. Ci dessous, l'image de cette tige balayée par les lignes de champ magnétiques ; \textit{La ou je 'balaye', j'ai ce champ électrique induit qui apparait.}
\begin{center}
\includegraphics[scale=0.35]{em/image3.png}\\
\end{center}
La tige étant au repos, la vitesse des particules $\vec{v_c}$ l'est aussi ; la force de Lorentz également. \\

Pour résoudre ce 'souci Galiléen', Faraday proposa sous forme de conjecture qu'un balayement des lignes champ créait un champ induit provoquant une force électrique induite $\vec{f}_{E_i}$ remplaçant $\vec{f}_M$ (dirigée dans le sens de $\vec{E}_i$, ici vers le bas).
\begin{center}
\includegraphics[scale=0.45]{em/image4.png}\\
\captionof{figure}{Potentiel non-nul}
\end{center}

\subsection{Comparaison avec l'influence électrostatique}
Il est important de ne \textbf{pas} confondre champ induit et champ électrostatique.
\subsubsection*{Influence électrostatique}
Dans cette situation, $\vec{E_i}$ est du à la présence de charges de part et d'autre de la tige (séparation du au champ extérieur $\vec{E}_s$), provoquant une migration des charges libres jusqu'à ce que le champ électrique total soit nul (champ toujours dans un conducteur (apparition d'un champ opposé induit, ...).
\begin{center}
\includegraphics[scale=0.45]{em/image5.png}\\
\captionof{figure}{Influence électrostatique}
\end{center}
Comme on se trouve à l'équilibre $\Delta V = 0$. Rappelons aussi que $\vec{E} = \vec{E_i} + \vec{E_s} = \vec{0}$ (sinon on aurait violation de la conservation de l'énergie).\\
Ces deux champs étant tous deux \textit{électrique}, ça explique la d.d.p. nulle.

\subsubsection*{Induction électromagnétique}
Ici le champ électrostatique est en opposition au champ induit est responsable d'un potentiel valant $lvB$. 
\begin{center}
\includegraphics[scale=0.45]{em/image6.png}\\
\captionof{figure}{Induction électromagnétique}
\end{center}
Si ici $"\Delta V" \neq 0$ c'est parce qu'on ne prend en compte \textbf{que les champs électriques ce qui n'est pas le cas de $\vec{E}_i$}.\\
Quand on se trouve en situation de courant variable, la notion de potentiel électrique n'a qu'un sens très limité, juste à la composante électrostatique.\\
Pour re-donner un sens à tout cela, on va définir comme la \textbf{force électromotrice} le potentiel $V$ correspondant à $\vec{E_S}$ seul. Ce dernier valant $-\vec{E_i}$ dans un conducteur à l'équilibre nous donne l'expression de la force électromotrice :
\begin{equation}
\fbox{$ \xi \equiv \int_i^f \vec{E}_i.\vec{dl}$}
\end{equation}

Physiquement, cela illustre que c'est à cause de $\vec{E}_i$ que les charges se déplaceront.

\subsection{Variations temporelles du champ magnétique}
L'augmentation du courant rend le champ magnétique plus important : l'environnement devient plus dense en ligne de champ. On dira que ces "nouvelles" lignes de champ se déplace, comme ci-dessus, à vitesse $-\vec{v}$ provoquant un champ électrique induit $\vec{E}_i = \vec{v} \times \vec{B}$.
\begin{center}
\includegraphics[scale=0.45]{em/image7.png}\\
\captionof{figure}{Déplacement du champ magnétique $\vec B$}
\end{center}
Le schéma ci-dessus représente la situation vue localement. Une autre expression pour $\xi$ peut ainsi être :
\begin{equation}
\xi \equiv \int_i^f \vec{v} \times \vec{B}.\vec{dl}
\end{equation}


\textit{NB :} j'insiste encore une fois sur le fait que $\vec{E}_i = \vec{v} \times \vec{B}$ mais que le champ nous intéressant dans la force électromotrice, $\xi$ vaut $-\vec{v} \times \vec{B}$.


\section{Spires et induction}
Connaître la vitesse des lignes de champ n'est pas réalisable, il va falloir ruser.

\subsection{Champ homogène}
Imaginons une spire carrée se déplaçant dans un champ à la vitesse $\vec{v}$, causant un déplacement des lignes de champ en $-\vec{v}$. 
\begin{center}
\includegraphics[scale=0.45]{em/image8.png}\\
\captionof{figure}{Spire carrée se déplaçant dans un champ homogène}
\end{center}
Sur les côtés horizontaux, $\vec{f_m}$ est dirigée vers le bas mais les charges ne se déplacement pas le long du conducteur, donc pas de force électromotrice.\\

La force de Lorentz étant parallèle aux côtés verticaux, on aura une force électromotrice non-nulle mais celle-ci étant équivalente des deux côtés, $xi = 0$.
\begin{center}
\textit{Pas de force électromotrice accumulée sur un tour complet de la spire, quelque soit le point d'où on part pour faire ce tour.}
\end{center}
\begin{equation}
\xi \equiv \oint_C \vec{E}_i.\vec{dl} = \oint_C \vec{v} \times \vec{B}.\vec{dl} = 0
\end{equation}


\subsection{Champ inhomogène}
Même situation que la sous-section précédente sauf que le champ n'étant pas homogène, la densité de ligne de champ diffère d'un endroit à l'autre, causant une $\xi$ plus grande.
\begin{center}
\includegraphics[scale=0.45]{em/image9.png}\\
\captionof{figure}{Spire carrée se déplaçant dans un champ non-homogène}
\end{center}
La force électromotrice totale sera $\neq 0$ et se calcule simplement : $\xi = l(||\vec{E}_{i1}|| - ||\vec{E}_{i2}|| = lv(B_1 - B_2)$.\\

\subsection{Variations du flux magnétique : loi d'induction de Faraday}
On peut exprimer $\xi$ en fonction du flux de champ magnétique. Prenons encore une fois la situation précédente (vitesse spire $\vec{v}$, lignes de champ $-\vec{v}$, ...).\\
L'astuce de Faraday est d'exprimer sa loi en terme du \textit{flux de champ magnétique} sous-tendu à la surface $S_C$
\begin{equation}
\Phi_M = \int_{S_C} \vec{B}.d\vec{S}
\end{equation}

L'induction n'a lieu que si le champ varie : logique de considérer les variation du flux. Pour se faire, considérons un petit temps $dt$ durant laquelle la spire s'est déplacée de $vdt$.

\begin{center}
\includegraphics[scale=0.45]{em/image10.png}\\
\captionof{figure}{Évolution temporelle du flux}
\end{center}
Si \textbf{seul le côté gauche} était en mouvement (spire déformable), la surface diminuerait et forcément $\Phi_M$ aussi : diminution de $-lvdt$ sur le temps $dt$ ($l$ = largeur de la spire). Ensuite les lignes de champ "arrivent" aux deuxième côté qui lui engendrerait une augmentation (on imagine maintenant que seul ce côté bouge). \\
En sommant les deux variations : 
\begin{equation}
d\Phi_M = -lvdtB_1 + lvdtB_2
\end{equation}

En divisant les deux membres par $dt$, on obtient l'opposée de $\xi$ calculée plus haut (\textit{section 3.2})
\begin{equation}
\frac{d\Phi_M}{dt} = -lvB_1 + lvB_2 = - \xi
\end{equation}

Ce qui nous donne la loi de Faraday pour une spire carrée en  mouvement...
\begin{equation}
\xi = \oint_C \vec{E}_i.\vec{dl} = - \frac{d\Phi_M}{dt}
\end{equation}

\textit{NB :} Le mouvement de la tige vers la gauche crée une force magnétique vers le bas, poussant les charges + vers le bas ce qui signifie que le gradient du potentiel électrique induit est dirigé dans le sens opposé au sens de parcours de la spire. Le potentiel croît en descendant la tige mobile. Cette signification physique justifie le signe négatif. 
\begin{center}
\includegraphics[scale=0.45]{em/image11.png}\\
\captionof{figure}{Exemple concret}
\end{center}
\textit{Une variation positive de flux magnétique engendre un champ électrique induit dirigé dans le sens opposé au sens de parcours conventionnel (donné par la règle de la main avec le pouce dans le sens du champ magnétique).}

\subsection{Généralisation : contour quelconque}
Considérons une situation encore une fois similaire au précédente, mais cette fois-ci avec une spire de forme quelconque.
\begin{center}
\includegraphics[scale=0.45]{em/image12.png}\\
\captionof{figure}{Spire au contour quelconque}
\end{center}
Le flux au temps $t$ :
\begin{equation}
\Phi_M(t)= \int_{S_C(t)} \vec{B}.d\vec{S}
\end{equation}

\textit{Wiki : Il s'agit du flux de $\vec{B}$ à travers d'une surface orientée sous tendue à un contour $C$}.
Au temps $t+dt$ :
\begin{equation}
\Phi_M(t+dt)= \int_{S_C(t+dt)} \vec{B}.d\vec{S}
\end{equation}

On peut voit le flux au temps $t+dt$ comme le flux au temps $t$ additionné d'un petit flux infinitésimal : $\Phi_M(t+dt) : \Phi_M(t) + d\Phi_M(t)$.\\

A certains endroits, le balayage des lignes de champ provoque une augmentation du flux alors qu'à travers d'autre, c'est une diminution du flux. 
\begin{equation}
\Phi_M(t+dt) = \int_{S_C(t)} \vec{B}.d\vec{S} + \int_{dS_C} \vec{B}.d\vec{S}
\end{equation}

\begin{center}
\includegraphics[scale=0.45]{em/image29.png}\\
\captionof{figure}{Variation du flux de la spire}
\end{center}
Mais encore : 
\begin{equation}
\Phi_M(t+dt) = \Phi_M + \int_{dS_C} \vec{B}.d\vec{S}
\end{equation}


Ce qui nous permet de faire apparaître la \textit{surface différence} $dS_C$, c'est à dire la variation de surface entre $t$ et $t+dt$.

A la limite infinitésimale, une portion du contour va devenir un élément $\vec{dl}$. La somme de ceux-ci définissent le coutour $C$ et donc cette surface différence.

\begin{equation}
d\Phi_M = \Phi_M(t+dt) - \Phi_M(t) = \int_{dS_C} \vec{B}.d\vec{S}
\end{equation}

Il faut maintenant faire en sorte qu'un agrandissement de la surface engendre un flux > 0 en choisissant un vecteur surface adapté.\\

La surface à considérer est celle engendrée par le déplacement de la spire. La surface balayée par un élément $\vec{dl}$ est donnée par l'aire du parallélogramme construit sur les vecteurs $\vec{v}dt$ et $\vec{dl}$ ce qui peux s'exprimer comme $\vec{v}dt \times \vec{dl}$ soit $vdtdlsin\theta$.
\begin{center}
\includegraphics[scale=0.45]{em/image13.png}\\
\captionof{figure}{Vecteur surface}
\end{center}
On peut ré-exprimer en explicitant $d\vec{S}$ :
\begin{equation}
d\Phi_M = \int_{dS_C} \vec{B}.d\vec{S} = \oint_C \vec{B}.(\vec{v}dt \times \vec{dl})
\end{equation}

On obtient une $\oint$ comparable à une intégrale de circulation dans le sens ou la somme se fait sur les éléments de longueur $\vec{dl}$ sur toute la longueur du contour $C$ de la spire. \\

On va sommer les éléments $\vec{dl}$ sur le contour $C$ et notre expression du vecteur surface tiendra compte naturellement des flux entrants et sortants.\\
En divisant par $dt$ :
\begin{equation}
\frac{d\Phi_M}{dt} = \oint_C \vec{B}.(\vec{v} \times \vec{dl})
\end{equation}


Connaissant la règle du produit mixte ($\vec{A}.(\vec{B}\times\vec{C}) = -(\vec{B}\times\vec{A}).\vec{C}$), on peut ré-écrire cette expression et faire apparaître le champ induit :
\begin{equation}
\frac{d\Phi_M}{dt} = -\oint_C (\vec{v} \times \vec{B}).\vec{dl} = -\oint_C \vec{E}_i.\vec{dl}
\end{equation}

On en déduit la loi de Faraday appliquée au cas ou des spires sont en mouvement dans un champ magnétique stationnaire.
\begin{equation}
\frac{d\Phi_M}{dt} = -\oint_C \vec{E}_i.\vec{dl}
\end{equation}
Que l'on peut ré-écrire en introduisant la définition du flux magnétique et en tant compte que $S_C(t)$ est la seule chose variant dans le temps.
\begin{equation}
\fbox{$ oint_{C(t)} \vec{E}_i.\vec{dl} = -\frac{d}{dt}\int_{S_C(t)} \vec{B}.\vec{dS}$}\
\end{equation}

\subsection{Généralisation : contour déformable}
Facilement généralisable, il suffit de tenir en compte les variations spatiales et temporelles de la vitesse.

\subsection{Généralisation : champ variable}
Aussi d'application quand \textbf{seul} $\vec{B}$ varie. On considèrera simplement la vitesse $-\vec{v}$ du au lignes de champ, le résultat en découlera.\\
\textbf{Loi d'induction de Faraday}
\begin{equation}
\oint_{C(t)} \vec{E}_i(\vec{x}, t).\vec{dl} = -\frac{d}{dt}\int_{S_C(t)} \vec{B}(\vec{x}, t).d\vec{S}
\end{equation}
\textit{NB :} Les développements ci-dessus ont été réalisé en prenant $d\vec{S}$ dans le même sens que $\vec{B}$. Si ce n'est pas le cas, il faut se baser sur la règle de la main droite.

\subsubsection*{Illustrations}
Dans un solénoïde, quand le courant augmente, le champ magnétique augmente également $\Rightarrow$ augmentation de la densité de lignes de champ à vitesse $\vec{-v}$ (en bleu) provoquant un champ induit $\vec{E}_i$.\\
Si la spire est bien sur l'axe du solénoïde, une augmentation du courant dans ce dernier augmente le champ à l'intérieur. Les lignes "traversent" la spire et viennent se resserrer au milieu expliquant la vitesse $-\vec{v}$ (champ de plus en plus fort, de plus en plus de lignes 'entrent' et se resserrent).
\begin{center}
\includegraphics[scale=0.45]{em/image14.png}\\
\captionof{figure}{Flux au travers d'un solénoïde}
\end{center}
Le champ induit est dirigé dans le sens opposé de parcours (règle main droite) provoquant l'apparition d'un courant opposé. Il y a accumulation de charge des deux côtés de la borne à laquelle on connecte une résistance pour que le courant puisse passer.La spire fonctionne ainsi comme un générateur au même titre qu'une pile.

\begin{center}
\includegraphics[scale=0.45]{em/image15.png}\\
\captionof{figure}{Ressèrement des lignes de champ}
\end{center}

\section{Forme locale de la loi d'induction}
On peut localiser la loi grâce au Th. de Stokes (qui contient la def du rot (ORAL)). Il faut d'abord comprendre que le champ $\vec{E}_i$ qui apparaît dans la loi de Faraday ne nécessite pas la présence d'une spire pour exister mais a tout objet balayé par des lignes de champ. Par Stokes :
\begin{equation}
\oint_C \vec{E}_i.\vec{dl} = \int_{S_C} rot\ \vec{E}_i.d\vec{S} = -\frac{d}{dt}\int_{S_C} \vec{B}.d\vec{S}
\end{equation}
(Le - se justifie par dS dans le sens du pouce et dl dans le sens des doigts)
\\
Notons qu'ici $S_C$ désigne une surface ouverte, les bords constituant le contour $C$.
En considérant une surface $\Delta S$ arbitrairement petite : 
\begin{equation}
rot\ \vec{E}_i.\Delta\vec{S} = -\frac{\partial \vec{B}}{\partial t}.\Delta\vec{S}
\end{equation}

Ces deux surfaces étant identique, on obtient la loi locale de Faraday :
\begin{equation}
rot\ \vec{E}_i = -\frac{\partial \vec{B}}{\partial t}
\end{equation}

\subsection{Généralisation}
Cette expression ne se rapporte qu'au champ électrique induit $\vec{E}_i$, apparaissant lors de variations de champ magnétique. 
\begin{center}
\includegraphics[scale=0.45]{em/image16.png}\\
\captionof{figure}{Champ électrostatique et induit}
\end{center}
S'il y a présence de charges électrique au sein d'un système électromagnétique, il y aura un champ électrique $\vec{E}_S$ en plus du champ induit.\\
Sachant que $rot\ \underbrace{\vec{E}_S = 0}_{conservatif}$ et $rot\ \vec{E}_i = -\frac{\partial \vec{B}}{\partial t}$, on peut écrire :
\begin{equation}
rot\ \vec{E} = rot(\vec{E}_S + \vec{E}_i) = -\frac{\partial \vec{B}}{\partial t}
\end{equation}
On en conclus que la loi de Faraday locale est valable pour le champ électrique total.
\begin{equation}
\fbox{$rot\ \vec{E} = -\frac{\partial \vec{B}}{\partial t}$}
\end{equation}

\section{Loi de Lenz}
\subsection{Opposition du champ induit}
Comme la vitesse de balayement des ligne de champ vaut $-\vec{v}$ (cadre magnétostatique) on crée un champ induit $\vec{E}_i$ créant forcément un courant induit $I-i$ allant de $+ \rightarrow -$. Ce courant pousse les charge jusqu'à la résistance pour créer un petit dipôle et générer un champ $\vec{E}_S$ si le circuit est fermé.\\
Mais encore mieux : le courant induit $I_i$ crée un champ magnétique induit $\vec{B}_i$ \textbf{s'opposant à la variation de }$\vec{B}$.

\begin{center}
\includegraphics[scale=0.55]{em/image17.png}\\
\captionof{figure}{Flux au travers d'un solénoïde}
\end{center}
\textbf{Loi de Lenz :} le champ magnétique induit par la spire est opposé à la variation du champ extérieur $\vec{B}$.\\
Si le champ diminue, $I_i$ sera dans la même sens que $I \Rightarrow \vec{F}_M$ attractive pour "compenser" la diminution de $\vec{B}$.

\subsection{Conservation de l'énergie}
Quand un aimant est éloigné de la spire, il apparaît une force attractive si la spire est fermée sur un circuit. $\vec{B}_i$ va aller dans le même sens que $\vec{B}$ pour compenser sa diminution.
\begin{center}
\includegraphics[scale=0.45]{em/image18.png}\\
\captionof{figure}{Représentation de la conservation d'énergie}
\end{center}
Le fait que les deux aimants s'attirent va créer une $F_M$ : il faudra dès lors appliquer une puissance mécanique pour déplacer l'aimant qui peut facilement s'exprimer.
\begin{equation}
P = RI_i^2 = \frac{\xi^2}{R} = -\vec{F}_M.\vec{v}
\end{equation}
Ici la puissance sera dissipée à travers la résistance, le retrait de l'aimant dissipe la puissance dans la spire.

\section{Champ magnétique dans la matière}
\subsection{Diamagnétisme}
Observable seulement sur des matériaux ni ferromagnétiques, ni paramagnétiques $\Rightarrow$ il faut un nombre pair d'$e^-$ ($\vec{m}_M = 0$).\\
Considérons un atome avec un électron décrivant une boucle : spire microscopique. Si l'on rapproche un aimant, on va diminuer la vitesse orbitale de celui-ci. Comme le nombre d'$e^-$ est pair, il n'est jamais seul : Un autre $e^-$ va tourner dans l'autre sens mais ici la vitesse orbitale va augmenter $\Rightarrow$ induction responsable d'un déséquilibre de courant entre orbites. Cette deuxième orbite possédant un courant plus élevé fait que $\vec{B}_i$ sera opposé à $\vec{B}$.\\

Une fois que l'aimant aura fini de bouger, le phénomène d'induction s'arrête ($\vec{E}_i = \vec{0}$) mais $\vec{B}_i$ reste tant que le champ magnétique extérieur reste stationnaire.\\

\textit{NB :} Si la spire n'était pas macroscopique, il  n'y aurait pas de courant pour cause de résistivité du matériau de la spire freinant les $e^-$.\\

Dès lors, on peut dire $\vec{B} = \vec{B}_0 + \vec{B}_i$. Comme le champ induit s'oppose au champ extérieur, le champ total sera toujours inférieur au champ extérieur ce pourquoi on définit la perméabilité relative.
\begin{equation}
\vec{B} = \vec{B}_0 + \vec{B}_i = \frac{\mu}{\mu_0}\vec{B}_0 = \vec{\mu_r}\vec{B}_0
\end{equation}
Ceci explique notamment la diminution du champ magnétique dans l'eau, les courants orbitaux causant l'opposition à $\vec{B}_0$.

\section{Applications du phénomène d'induction}
Je ne parle ici que des deux cas les plus intéressants, les autres sont simple et vite lu ! (\textit{p. 46 - 47})
\subsection{Les courants de Foucault}
Considérons un disque ou seule une partie est soumise à un champ magnétique : $\xi$ n'est établi qu'aux limites de cette zone. Le champ induit déplaces les charges créant une différence de potentiel. Mais ici ces charges sont libres : elles suivent les lignes de champ électrostatique dipolaire créant des boucles de courants : \textit{courants de Foucault}. (\textit{cf. p48 pour autre explication})
\begin{center}
\includegraphics[scale=0.45]{em/image19.png}\\
\captionof{figure}{Représentation des courants de Foucault}
\end{center}
Ces courants affectent tous systèmes électromagnétiques et peuvent être très nuisible par dissipation de l'énergie par effet Joule. \\
\textit{NB :} $rot\ \vec{E}_i = -\partial\vec{B}/\partial t$. Si $\partial\vec{B}/\partial t$ est non nul, $\vec{E}_i$ l'est également $\Rightarrow \vec{J} = \sigma\vec{E}_i$ montre la présence de courants générés dans ce conducteur.

\subsection{Le générateur de tension alternative}
Basé sur la loi d'induction de Faraday appliquée au cas du contour $C(t)$ dans un champ stationnaire $\vec{B}(\vec{x})$.\\
Il s'agit d'une spire rotative en mouvement dans l'entrefer d'un aimant dont le mouvement provoque une force électromotrice.\\
La vitesse de rotation évolue avec le temps : $\theta (t) = \omega t$ où $\omega$ est la vitesse angulaire et $\omega T = 2\pi$ est la période de révolution. La période vaut donc : $T = \frac{2\pi}{\omega}$. En considérant $\vec{B}$ uniforme et en développant le produit scalaire : 
\begin{equation}
\int_{S_C(t)} \vec{B}(\vec{x}).d\vec{S} = Blhcos(\omega t)\ \ \ \ \Rightarrow \xi = Blh\omega sin(\omega t)
\end{equation}
Ce dispositif s'appelle un \textit{alternateur} car la tension est sans-cesse inversée.

\section{L'auto-induction}
\subsection{Introduction}
Si  une spire est alimentée par un courant variable, celle-ci subit la variation du flux magnétique qu'elle génère elle même par le passage de ce courant (Oui oui, c'est tordu) ; c'est l'auto-induction.\\

Soit une spire parcourue par un courant variable $I(t)$ (créant une d.d.p. (f.e.m.)) générant $B(t)$. D'après la loi de Faraday : 
\begin{equation}
\xi = \oint_C \vec{E}_i.\vec{dl} = -\frac{d\Phi_M}{dt}
\end{equation}
Pour se faire une idée, imaginons un contour intérieur à la spire extrêmement proche de la spire.

\begin{center}
\includegraphics[scale=0.55]{em/image20.png}\\
\captionof{figure}{Principe de continuité}
\end{center}

Ce contour imaginaire est balayé par les lignes de champ émanant de l'augmentation du champ (notée $-\vec{v}$ ci-dessus) correspondant à un ressèment des lignes de champ.\\
Le produit $\vec{v} \times \vec{B}$ indique que $\vec{E}_i$ est partout le même autant sur le contour qu'à l'extérieur : le champ à l'intérieur du conducteur est le même qu'à l'extérieur.\\
Prouvons le en appliquant ceci à un contour $C$ situé de part et d'autre de la surface du conducteur. Si $C$ à une largeur $\delta \rightarrow 0$ la circulation sera nulle. Or, selon le chemin ampérien, celle circulation vaut $E_il - E_I'l$

\begin{center}
\includegraphics[scale=0.47]{em/image21.png}\\
\captionof{figure}{Contour ampérien infiniment proche}
\end{center}
On peut en tirer que $E_i = E_i'$ :
\begin{equation}
\oint_C \vec{E}.\vec{dl} = -\int_{S_C}\frac{\partial\vec{B}}{\partial t}.d\vec{S} \Rightarrow E_il - E_i'l = \int_{S_C} \frac{\partial\vec{B}}{\partial t}.d\vec{S} \rightarrow 0 \Rrightarrow E_i = E_i'
\end{equation}
Comme le contour imaginaire est suffisamment proche, on pourra considérer qu'il fait partie de la spire et appliquer le résultat obtenu.

\subsection{Illustration}
Dans une spire en régime continu, le courant est donné par la loi d'Ohm $ V = RI$. Si $V$ diminue brusquement, le courant aussi : le champ induit va tendre à tirer les charges pour compenser sa diminution. Il faut prendre en compte cette force électromotrice (l'accumulation des charges tend à augmenter la d.d.p.).
\begin{equation}
V + \xi = RI
\end{equation}

\subsection{Inductance}
$\Phi_M \propto \vec{B} \propto I \Rightarrow \Phi_M \propto I$. On peut introduire un facteur de proportionnalité $L$, l'\textit{inductance}, tel que : $\Phi_M = LI$. (dépend essentiellement de la géométrie).
\begin{equation}
\Phi_M = LI\ \ \ \ \ [L] = \frac{Tm^2}{A} \equiv H\ (=\ Henry)
\end{equation}
La connaissance de $L$ (dur à déterminer) nous permet de connaître $\xi$ grâce à la relation infinitésimale :
\begin{equation}
\frac{d\Phi_m}{dt} = L \frac{dI}{dt} \Rightarrow \xi = -L\frac{dI}{dt}
\end{equation}
On voit que si $I$ augmente, la polarité de $\xi$ s'oppose à cela en tirant les charges positives dans le sens opposé au courant.
\begin{center}
\includegraphics[scale=0.50]{em/image22.png}\\
\captionof{figure}{Convention graphique}
\end{center}
Sur un circuit RL (résistance - inductance), $\xi$ se met dans le sens de $I$. La somme des tensions valant zéro : 
\begin{equation}
V + V_L + V_R = 0 \Rightarrow V + \xi - RI = 0 \Rightarrow V = RI + L\frac{dI}{dt}
\end{equation}

\subsection{Circuit RL}
Considérons un circuit RL au circuit initialement ouvert jusqu'au temps $t = 0$ ou le circuit est fermé et une tension $V$ est appliquée : $V = RI + L\frac{dI}{dt}$.\\
Ré-écrivons cette équation différentielle : $\frac{dI}{dt} = -\frac{R}{L}I + \frac{V}{M}$. Sa résolution est simple (\textit{Cf. Analyse I}) :
$I(t) = ae^{-\frac{R}{L}t} + \frac{V}{R}$
Comme nous savons que $I = 0$ à $t = 0 \Rightarrow a= - \frac{V}{R}$ :
\begin{equation}
I = \frac{V}{R}\left(1-e^{-\frac{R}{L}t}\right)
\end{equation}

\begin{center}
\includegraphics[scale=0.50]{em/image23.png}\\
\captionof{figure}{Charge du circuit RL}
\end{center}
Il faut attendre un certain temps pour que le courant soit "total". ($L/R$ est le temps caractéristique  définissant ce "temps").

\subsection{Décharge d'un inducteur}
Par raisonnement similaire, si ce n'est le coefficient $-R/L$ négatif : 
\begin{equation}
\fbox{$ I = \frac{V}{R}e^{-\frac{R}{L}t}$}
\end{equation}
Le courant n'est donc pas annulé instantanément, l'inducteur \textit{freine} cette diminution de $I$.

\subsection{Énergie stockée dans l'inducteur}
Un inducteur initialement chargé se déchargant va provoquer des pertes joules dans la résistances = $P_J = RI^2$, dépendantes du temps.
\begin{equation}
P_J(t) = RI(t)^2 \Rightarrow W = \int_0^\infty P_J(t) dt
\end{equation}
Sachant que $RI = -L\frac{dI}{dt}$, en multipliant tout par $I$ on obtient $RI^2 = -LI\frac{dI}{dt}$. En intégrant de $I(0)$ à $0$ (les $dt$ se simplifiant, laissant place à $dI$).
\begin{equation}
W = L\frac{I(0)^2}{2}
\end{equation}

Physiquement, on peut voir l'énergie stockée comme provenant de \textit{l'inertie} du courant du au phénomène d'auto-induction. Le champ induit réduit la diminution de courant comme la masse ralentissant la décélération d'un objet massique subissant une force opposée à sa vitesse.\\
L'énergie stockée est donc d'origine magnétique, notée $W_m$ :
\begin{equation}
\fbox{$W_m = \frac{1}{2}LI^2 $}
\end{equation}

\subsection{Énergie du champ magnétique}
Considérons un solénoïde idéal pour lequel $\vec{B}$ est connu : $\vec{B}(\vec{x}- = B\vec{1_Z}$ où $B = \mu_0nI$. En isolant $I$ et en remplaçant la densité de charge linéique $n$ par $N/l$ : $I = \frac{Bl}{\mu_0N}$.\\

Comme $W_M$ nécessite $L$, il faut le calculer. L'inductance étant le rapport entre le flux capté et le courant qui génère ce flux (flux total = $NBS$) et en remplaçant $B$ par sa valeur : 
\begin{equation}
\Phi_M = NBS = N\mu_0nIS \equiv LI \Rightarrow L = \mu_0\frac{N^2}{l}S
\end{equation}
Sachant que $I = Bl / (\mu_0N) et L = \mu_0\frac{N^2}{l}$, on peut exprimer $W_m$ en partant de $1/2LI^2$ :
\begin{equation}
\frac{1}{2}LI^2 = \frac{1}{2}\mu_0\frac{N^2}{l}S\frac{B^2l^2}{\mu_0^2N^2} = \frac{1}{2\mu_0}\underbrace{Sl}_{Volume}B^2
\end{equation}
En divisant par le volume (occupé par le champ), on obtient la densité d'énergie magnétique : 
\begin{equation}
w_m = \frac{1}{2}\frac{1}{\mu_0}||\vec{B}||^2
\end{equation}

\subsection{Loi courant-tension de l'inducteur}
La résistance sur les circuit RL représente la résistivité des fils. Appelons ainsi un \textit{inducteur idéal} un inducteur ne présentant pas de résistivité. On pourrait croire que le temps caractéristique tendrait vers l'infini mais il n'en est rien.\\
La source en tout temps est donnée par $V = -\xi$ ($V = RI + L dI/dt \ \ \ R \rightarrow 0$). Cette équation différentielle se résout facilement (Equadiff homogène de 1er ordre, on isole, on intègre et zou) : 
\begin{equation}
I = \frac{V}{L}t
\end{equation}
On a donc du \textit{recommencer} l'équadiff pour avoir une situation physiquement correct ou le temps caractéristique a disparu.\\
Comme rien ne retient le courant, $\xi$ compense exactement $V$ qui est donc constante.

\subsection{Généralisation}
La loi est généralisable au cas ou la tension est variable au cours du temps, il suffit de la considérer sous sa forme infinitésimale.
\begin{equation}
\frac{dI(t)}{dt} = \frac{1}{L}V(t)
\end{equation}

\subsubsection*{Illustration : tension sinusoïdale}
Imaginons une tension $V = V_m sin(\omega t)$. Il suffit d'appliquer ce que l'on vient de voir : 
\begin{equation}
\frac{dI}{dt} = \frac{1}{L}V_m sin(\omega t) \Rightarrow I(t) = -\frac{1}{\omega L}V_m cos(\omega t)
\end{equation}
\textbf{Attention :} Remarque page 70.
\subsection{Convention graphique}
Point
de vue graphique une tension négative signifie que la pointe de la flèche est tournée
vers le potentiel inférieur (le courant passe d’un potentiel inférieur vers un potentiel
supérieur).

\begin{center}
\includegraphics[scale=0.45]{em/image24.png}\\
\captionof{figure}{Convention graphique}
\end{center}

\section{Courants alternatifs}
\subsection{Introduction}
Le transport de l'électricité à haute tension est nécessaire pour éviter de trop grandes pertes joules.\\
Soit la résistances $R_t$ ($t$ pour transport) et $R_u$ l'énergie utile (consommation de l'énergie électrique). Sachant que $V = R_tI + V_u + R_tI$

\begin{center}
\includegraphics[scale=0.45]{em/image25.png}\\
\captionof{figure}{Transport de l'électricité}
\end{center}
On peut facilement voir que la tension utile s'exprime $V_u = V - 2 R_tI$ (Le facteur 2 vient de l'aller-retour). Comme $P = IV$, on peut calculer la puissance utile : $P_u = V_uI = VI - 2R_tI^2$.\\
Le rendement du transport de l'électricité $\eta_t$ peut ainsi se calculer aisément.
\begin{equation}
\eta_t = \frac{P_u}{P} = \frac{P - 2R_tI^2}{P} = 1 - \frac{R_tI^2}{P}
\end{equation}
Le rendement augmente quand $I$ diminue. Pour diminuer $I$ sans modifier $P_u$, il faut augmenter $V$ (comme $P = IV \Rightarrow I = P/V$-.

\subsection{Le transformateur}
On l'a inventer pour changer le courant transporté à haute tension en courant utilisable dans les foyers.

\subsubsection{Variation de flux de tension}
En générant un courant variable au cours d'un inducteur (idéal) connecté à un courant alternatif : elle génère un champ magnétique variable dont le flux au travers d'elle même vaut $LI$. ($\Phi_M(t) = \int_{S_C} \vec{B}(t).d\vec{S} \equiv LI(t)$)\\

La tension alternative valant $V(t) = V_m sin(\omega t)$ et grâce à la \textit{section 8.9} ($V = LdI / dt$), on peut calculer l'intensité : $I(t) = -\frac{1}{\omega L}V_m cos(\omega t)$. En substituant dans $\Phi_M = LI$ :
\begin{equation}
\Phi_M = -\frac{1}{\omega}V_m cos(\omega t)
\end{equation}
Dans un transformateur, on considère un circuit \textit{primaire} (l'inducteur) et un \textit{secondaire} ; ce dernier capte le flux magnétique générer. Ils sont suffisamment proche que pour capter le même $\Phi_M \rightarrow$ f.e.m. ($\xi_S$) opposé de la dérivée ce de flux magnétique. Comme nous savons que $d\Phi_M/dt$ vaut la tension $V$ appliquée au primaire : 
\begin{equation}
\xi_S = \oint_C \vec{E}_i.\vec{dl} = - \frac{d\Phi_M}{dt} = - V(t)
\end{equation}
En effectuant l'opposée de la dérivée de $\Phi_M$ trouvée ci-dessus, on trouve pour valeur de $\xi_S$ :
\begin{equation}
\xi_S = - V_m sin(\omega t)
\end{equation}

\subsubsection{Circuit secondaire à spires multiples}
Imaginons un circuit avec deux spires secondaires. L'intégrale de circulation devra être considérée sur un contour double et forcément $\xi_S$ en sera également doublée.
\begin{equation}
\xi_S = \oint_{2C} \vec{E}_I.\vec{dl} = -2\frac{d\Phi_M}{dt} \Rightarrow \xi_S = -2V_M sin(\omega t) \Rightarrow |\xi_S| = 2|V|
\end{equation}
En résumer, on peut augmenter la tension en plaçant plusieurs spires en série.

\subsubsection{Circuit primaire à spires multiples}
En pratique, on utilise des circuit primaire et secondaires avec un grand nombre de spires. Considérons une unique spire primaire traversée par une tension alternative $V_p$. Sachant que $(V_p + \xi_p = 0)$ et $\xi_p = -d\Phi_M/dt$, on peut écrire : 
\begin{equation}
\frac{d\Phi_M}{dt} = V_p
\end{equation}
Si l'on dispose maintenant de deux spires primaires, on aura $\xi_p = -2d\Phi_M/dt$ et donc : 
\begin{equation}
\frac{d\Phi_M}{dt} = \frac{1}{2}V_p
\end{equation}
En résumer, la variation de flux est deux fois plus faible car celle-ci est captée deux fois par le circuit primaire.\\
Si l'on a $N_p$ spires primaires, on peut écrire : $d\Phi_M/dt = V_p/N_p$.\\

En rajoutant $N_S$ spires secondaires, comme $V_S = - \xi_S$ on peut écrire :
\begin{equation}
V_S = \frac{N_S}{N_P}V_P
\end{equation}

\subsection{Résistances en courant alternatif}
Connaissant la loi d'Ohm ($V = RI$) et la tension alternative ($V(t) = V_M sin(\omega t)$), on peut facilement exprimer le courant passant à travers $R$.
\begin{equation}
I(t) = \frac{V_m}{R} sin(\omega t) = I_m sin(\omega t)
\end{equation}

$\omega$ est la \textit{pulsation} ou \textit{fréquence angulaire} déterminant la période valant $T : 2\pi /\omega$ d'où on tire la fréquence $f = 1/T = \omega / 2\pi$.

\subsubsection{Puissance dissipée et valeurs efficaces}
Sachant que $P = IV = RI^2$ :
\begin{equation}
P = I_mV_m sin^2(\omega t) = RI_m^2 sin^2(\omega t)
\end{equation}
On peut ré-écrire $sin^2(\omega t)$ en $[1-cos(2\omega t)]/2]$ montrant que la puissance dissipée évolue avec une période deux fois plus courte que la tension. En pratique, on considère $<P>$, les variations étant trop rapide.
\begin{center}
\includegraphics[scale=0.45]{em/image26.png}\\
\captionof{figure}{Puissance dissipée}
\end{center}
Le graphique ci-dessus montre que cette moyenne $<P>$ vaut $1/2$, le cosinus moyen étant nul.\\
En pratique, plutôt que d'utiliser $<P> = (1/2)RI^2_M$  on va introdire le \textit{courant efficace} $I_{eff} = I_M/\sqrt{2}$ pour écrire $<P> = RI_{eff}^2$.\\

\textit{NB :} Un ampère \textit{efficace} signifie que les amplitudes maximales de $I_m$ valent $\sqrt{2}$ ampères.\\
La tension efficace est définie de la même façon.
\begin{equation}
V_{eff} = \frac{V_m}{\sqrt{2}} \Rightarrow <P> = \frac{V2_{eff}}{R}
\end{equation}

\subsection{Inducteurs en courant alternatif}
Comme précédemment, on pour considérer le courant  $I(t)$ valant $I(t) = -\frac{1}{\omega L}V_m cos(\omega t)$. Si l'on définit l'\textbf{amplitude du courant} $I_m = -V_m/(\omega L)$ et en jouant avec la trigonométrie, on peut ré-écrire cette expression.
\begin{equation}
I(t) = I_m sin(\omega t - \frac{\pi}{2})
\end{equation}
Le courant est en retard d'un quart ($\frac{\pi}{2}$) de période  par rapport à la tension.
\subsubsection{Déphasage}
En mettant $\omega$ en évidence dans l'argument du sinus, on fait apparaître le retard temporel du courant par rapport à la tension.
\begin{equation}
I_m sin[\omega (t - \Delta t)]\ \ \ \ où\ \Delta t = \frac{\pi}{2 \omega}
\end{equation}
Le retard valant bien $T/4$ le courant est en \textit{quadrature de phase} avec la tension $\Rightarrow$ \textit{déphasage}, \textit{retard de phase} $\Delta\phi = - \frac{\pi}{2}$.

\subsubsection{Puissance dissipée}
Nous savons que $P = IV$. Or, $I = \Delta q / \Delta t \Rightarrow IV = V \Delta q / \Delta t $.
\begin{equation}
P(t) = IV = -I_mV_m \underbrace{cos(\omega t)sin(\omega t}_{sin(2\omega t)/2} \Rightarrow <P(t)> = 0
\end{equation}
La puissance étant sinusoïdale, il y a autant de valeur positive que négative : sa moyenne temporelle est forcément nulle.\\
Physiquement, en prenant la situation du courant négatif combiné à une tension positive signifie que les charges (+) remontent le potentiel et gagnent de l'énergie, impliquant une puissance dissipée négative.\\
Attention quand le courant est positif, l'énergie dissipée ne sera pas sous forme de chaleur mais va au champ magnétique de l'inducteur. En conclusion, un inducteur parfait ne consomme pas d'énergie en courant alternatif.

\subsubsection{Réactance}
Ce qui nous intéresse, ce sont les valeurs efficaces et comme le courant est maximal, il vaut la tension divisée par $\omega L$.
\begin{equation}
I_{eff} = \frac{V_{eff}}{\omega L}
\end{equation}
Le rapport courant-tension dépend de la pulsation : plus celle-ci est élevée, plus le courant est faible pour un $V$ donné.\\
Cette réaction nous rappelle la loi d'Ohm, $V = RI$, où $\omega L$ joue le rôle de $R$. Il s'agit d'une \textit{résistance variable} nommée \textit{réactance}($X_L$) pour les différencier.
\begin{equation}
X_L = \omega L\ \ ([X_L] = \Omega\ \ \ \ \Rightarrow\ \ \ \ \ V_{eff} = X_LI_{eff}
\end{equation}
Le grande différence est que la réactance dépend de la fréquence de la source de tension.

\subsubsection{Filtrage}
La réactance peut ainsi jouer le rôle de filtre. Imaginons une tension $V(t) = \sum_{}^{} V_m sin(\omega_m t)$ (= \textit{harmonique}) où $V_m$ sont les amplitudes de chaque composante fréquentielle $\omega_m$.\\

Chaque $V_m$ est responsable d'un courant $I_m = V_m / X_L$ : l'amplitude est d'autant plus faible que la fréquence est élevée : seule les basses fréquences passeront : inducteur = \textit{filtre passe-bas}.

\subsection{Condensateurs en courant alternatif}
Sachant que $Q = CV$ et que $I = dQ/dt$, on peut écrire $I = C \frac{dV}{dt}$. En explicitant la tension : 
\begin{equation}
V(t) = V_m sin(\omega t) \Rightarrow I(t) = C\omega V_m cos(\omega t)
\end{equation}
Et en introduisant \textbf{l'amplitude de courant} $I_m = C\omega V_m$, on peut écrire $I(t) = I_m sin(\omega t + \frac{\pi}{2}$.\\
En mettant $\omega$ en évidence, on remarque que cette fois le courant est \textit{en avance} sur la tension d'un quart de période
\begin{equation}
I(t) = I_m sin\left[\omega (t + \frac{\pi}{2\omega})\right]
\end{equation}
Cette fois-ci le déphasage $\phi$ vaudra $\frac{\pi}{2}$ tandis que la puissance dissipée sera toujours nulle.
\subsubsection*{Réactance}
De façon analogue à l'inducteur, on introduit la réactance du condensateur \begin{equation}
X_C = 1/(\omega C)
\end{equation}

Plus la fréquence est grande, plus la réactance est petite et plus le courant efficace est grand pour un $V$ donné $\Rightarrow$ le condensateur est un \textit{filtre passe-haut}.

\section{Conjecture de Maxwell : courant de déplacement}
\subsection{Problématique de la loi d'Ampère}
Selon Ampère, le courant est exprimé comme le flux de la densité de courant $\vec{J}$ au travers de la surface $S_C$ sous-tendue par le contour $C$ ($\oint_C \vec{B}.\vec{dl} = \mu_0 \int_{S_C} \vec{J}.d\vec{S}$).\\
Le souci était de savoir si, peut importe la surface, la loi était valable. Considérons deux surfaces $S_{C1}$ et $S_{C2}$.

\begin{center}
\includegraphics[scale=0.55]{em/image27.png}\\
\captionof{figure}{Choix du contour ampérien}
\end{center}
Pour respecter la convention de l'intégrale de circulation (comme on s'intéresse à $\vec{J}$ on préfère cette convention que celle des vecteurs dS sortants), on inverse le sens de $d\vec{S}_{C1}$ de sorte à ré-écrire l'égalité :
\begin{equation}
\oint_S \vec{J}.d\vec{S} = - \int_{S_{C1}} \vec{J}.d\vec{S}_{C1} + \int_{S_{C2}} \vec{J}.d\vec{S}_{C2} 
\end{equation}
Le souci, c'est que cette égalité peut se ré-exprimer  $\int_{S_{C1}} \vec{J}.d\vec{S}_{C1} = \int_{S_{C2}} \vec{J}.d\vec{S}_{C2}\ \ \ \Rightarrow\ \ \ \oint_S \vec{J}.d\vec{S} = 0$.\\

\textbf{La validité de la loi d'ampère impose que $\vec{J}$ soit nul pour toute surface fermée $S$}.\\
En appliquant le Th. d'Ostrogradski (flux d'un champ vectoriel sur une surface fermée $S$ est égal à l'intégrale de volume de la divergence de ce champ sur le volume $V_S$ enfermé par cette surface) :
\begin{equation}
\int_{V_S} div\ \vec{J} dV = 0\ \ \ \ \forall\ V_S
\end{equation}
Comme valide pour tout volume, avec un volume arbitrairement petit $\vec{J} = cste$. La loi d'Ampère sera d'application $\Leftrightarrow$
\begin{equation}
div\ \vec{J} = 0
\end{equation}
Ce qui est très restrictif : une accumulation de charge donnerait une divergence non nulle (négative).\\
Un bon exemple est l'espace entre les plaques d'un condensateur.

\subsection{Loi de conservation de la charge}
\subsubsection{Forme intégrale}
Pour rappel $\vec{J}$ est un vecteur représentant un flux de charge électrique. Si $Q$ est la charge totale enfermée par $S$, le flux $\vec{J}$ au travers de $S$ représente la variation temporelle de la quantité de charge $Q$ ; si le flux est > 0, cela correspond à une diminution de la charge Q.
\begin{equation}
\fbox{$ \oint_S \vec{J}.d\vec{S} = - \frac{dQ}{dt}$}
\end{equation}

\subsubsection{Forme locale}
Localement, on peut dire $Q = \int_{V_S} \rho dV$. En appliquant la formule ci-dessus :
\begin{equation}
\oint_S \vec{J}.d\vec{S} = - \int_{V_S} \frac{\partial\rho(\vec{x}, t)}{\partial t}dV
\end{equation}
En appliquant le Th. d'Ostrogradski : 
$\int_{V_S} div\ \vec{J}dV = -\int_{V_S} \frac{\partial\rho}{\partial t}dV$.\\
Comme d'habitude, en considérant un volume arbitrairement petit pour que $\vec{J}$ soit constant : $div\ \vec{J} = -\frac{\partial \rho}{\partial t}$.\\

On obtient une équation liant densité de charge et courant : l'\textit{équation de continuité}, c'est à dire la loi locale de conservation de la charge électrique.
\begin{equation}
\fbox{$div\ \vec{J} + \frac{\partial \rho}{\partial t} = 0 $}
\end{equation}

\subsection{Courant de déplacement}
On introduit les courants de déplacement pour régler le souci d'accumulation ($\partial\rho /  \partial t > 0$) (comme dans une antenne) ou la loi d'Ampère n'est pas d'application.

\subsubsection{Loi de Gauss}
Le génie d'ampère à été de combiner la loi de Gauss à celle de conservation de la charge : 
\begin{equation}
div\ \vec{E} = \frac{1}{\epsilon_0}\rho\ \ \ \ \ \ \ \ div\ \vec{J} + \frac{\partial \rho}{\partial t} = 0
\end{equation}
La première égalité peut être reformulée telle que $\rho = \epsilon_0div\ \vec{E}$ laissant place aux implications suivantes :
\begin{equation}
div\vec{J} + \epsilon_0 \frac{\partial}{\partial t}div\vec{E} = 0 \Rightarrow div\vec{J} + div\left(\epsilon_0\frac{\partial\vec{E}}{\partial t}\right) = 0 \Rightarrow div\left(\vec{J} + \epsilon_0\frac{\partial\vec{E}}{\partial t}\right) = 0
\end{equation}
\textbf{La densité de courant $\vec{J}$ complétée de la dérivée du champ électrique multiplié par $\epsilon_0$ constitue un champ vectoriel à divergence toujours nulle}.\\

Si le flux devait être annulé totalement (comme dans le cas de l'antenne), tout se passe comme si le flux de charges électriques était remplacé par le champ vectoriel $\epsilon_0 \partial\vec{E}/\partial t$.

\subsubsection{Loi d'Ampère-Maxwell}
On considère ce nouveau champ vectoriel dans la loi d'ampère.
\begin{equation}
\fbox{$ \oint_C \vec{B}.\vec{dl} = \mu_0   \int_{S_C}\left(\vec{J} + \epsilon_0 \frac{\partial \vec{E}}{\partial t}\right).d\vec{S}$}
\end{equation}

\subsubsection{Courant de déplacement}
Le terme correcteur est appelé \textit{densité de courant de déplacement} noté $\vec{J}_D$.
\begin{equation}
\vec{J}_D = \epsilon_0 \frac{\partial \vec{E}}{\partial t}
\end{equation}
Il s'agit d'une conjecture car cette loi a été basée sur une construction purement intellectuelle, vérifiée ensuite avec succès par l'expérience.\\

L'application du Th. de Stokes ($\oint_C \vec{B}.\vec{dl} = \mu_0 \int_{S_C}(\vec{J} + \vec{J}_D)$)  donne directement la forme locale de la loi d'Ampère-Maxwell.
\begin{equation}
rot\ \vec{B} = \mu_0 (\vec{J} + \vec{J}_D)
\end{equation}

\subsubsection{Illustration : charge du condensateur}
\textit{Cf. syllabus p 103 - 104}

\subsubsection*{Signification physique}
Dans les zones entourant le fil $\vec{B}$ est donné par la loi d'Ampère "normale" alors qu'entre les plaques on utilise uniquement le courant de déplacement.
\begin{center}
\includegraphics[scale=0.55]{em/image28.png}\\
\captionof{figure}{Courant de déplacement entre les deux plaques}
\end{center}
$\vec{B}$ ne souffre pas de discontinuité en passant du fil à l'entre-deux plaques.
\captionof{figure}{}
\newpage
\section{Équations de Maxwell}
Il s'agit de quatre lois fondamentales permettant de décrire tous les phénomènes électromagnétiques.

\subsection{Loi de Faraday}
Décrit le phénomène d'induction électromagnétique.
\begin{equation}
\oint_C \vec{E}.\vec{dl} = -\frac{\partial}{\partial t} \int_{S_C} \vec{B}.d\vec{S}
\end{equation}
Localement :
\begin{equation}
rot\ \vec{E} = -\frac{\partial\vec{B}}{\partial t}
\end{equation}

\subsection{Loi d'Ampère-Maxwell}
Montre que le courant de conduction (Ampère) et le champ électrique quand il varie (Maxwell) sont tous deux sources de champ magnétiques : en variant, ils constituent une source l'un pour l'autre.

\begin{equation}
\oint_C \vec{B}.\vec{dl} = \mu_0   \int_{S_C}\left(\vec{J} + \epsilon_0 \frac{\partial \vec{E}}{\partial t}\right).d\vec{S}
\end{equation}

\subsection{Loi de Gauss pour le champ électrique}
Équivalente à la loi de Coulomb, elle exprime le champ électrique.
\begin{equation}
\oint_S \vec{E}.d\vec{S} = \frac{1}{\epsilon_0}\int_{V_S} \rho(\vec{x})dV
\end{equation}
Localement :
\begin{equation}
div\ \vec{E} = \frac{1}{\epsilon_0}\rho(\vec{x})
\end{equation}

\subsection{Loi de Gauss pour le champ magnétique}
Exprime la non-existence de charge magnétique (lignes de champ forment toujours des boucles fermées).
\begin{equation}
\oint_S \vec{B}.d\vec{S} = 0
\end{equation}
Localement :
\begin{equation}
div\ \vec{B} = 0
\end{equation}




























































\part{Annexes}
\appendix
\chapter{Rapport de laboratoire}
\section{Voltmètre et ampèremètre}
\subsection{Rappel théorique}
\subsubsection{Notions de base}
Avant toute chose, définissions ce avec quoi nous allons travailler.
\begin{center}
\includegraphics[scale=0.5]{labo/image8.png}
\captionof{figure}{Circuit composé d'un voltmètre}
\end{center}
Sur ce magnifique schéma, nous avons un \textit{générateur de tension continue} ($V_G$), une \textit{résistance} ($R$), un \textit{ampèremètre} ($A$) en série et un \textit{voltmètre} ($V$) placé en parallèle.\\
Le sens du courant est conventionnel : $+ \rightarrow -$.

\subsubsection*{Loi d'Ohm}
Petit rappel de l'électrostatique :
\begin{equation}
V = RI
\end{equation}

\subsubsection*{Loi de Kirchoff}
Deux lois sont à retenir (et genre vraiment, pas pour rire) : 
\begin{description}
\item[Loi des nœuds] Un \textit{nœud} d'un circuit électrique est l'endroit où entrent un ou plusieurs fils électriques, et dont sortent un ou plusieurs fils électriques. La loi des nœuds stipule que la somme des intensités des courants électriques entrants est égale à la somme des intensités des courants électriques sortants.
\item[Loi des mailles] Une \textit{maille} d'un circuit électrique est constitué d'une succession de fils et éléments électriques formant un sous-circuit fermé. La loi des mailles stipule que la somme des d.d.p. (toutes prises dans le même sens) le long de la maille est nulle.
\end{description}


\subsubsection*{Association de résistances}
Les résistances en série obéissent à la relation :
\begin{equation}
R = R_1 + R_2
\end{equation}
alors que les résistance en parallèle suivent plutôt celle-ci :
\begin{equation}
\frac{1}{R} = \frac{1}{R_1} + \frac{1}{R_2}
\end{equation}

\subsubsection*{Diviseur de tension}
Quand deux résistances sont placées en séries, la tension fournie se sépare en deux.
\begin{center}
\includegraphics[scale=0.5]{labo/image9.png}
\captionof{figure}{Résistances en série}
\end{center}
\begin{enumerate}
\item Une fraction $V_1 = R_1I$
\item Une fraction $V_2 = R_2I$
\end{enumerate}
Comme on sait que la d.d.p. vaut la somme des d.d.p. on peut écrire : 
\begin{equation}
V_1 = \frac{R_1}{R_1 + R_2}V_G\ \ \ \ \ V_2 = V_G - V_1
\end{equation}

\subsubsection*{Diviseur de courant}
Le principe est identique : le courant sera divisé en deux.
\begin{center}
\includegraphics[scale=0.5]{labo/image10.png}
\captionof{figure}{Résistances en parallèle}
\end{center}
Comme le courant total vaut la somme des courants, on peut écrire : 
\begin{equation}
I_1 = \frac{R_2}{R_1 + R_2}I_G\ \ \ \ \ \ I_2 = I_G - I_1
\end{equation}

\subsubsection*{Voltmètre}
On peut diviser la tension à l'aide du principe vu ci-dessus. On mesurera ainsi une "portion" du courant qu'il faudra multiplier par un certain facteur pour obtenir la mesure réelle.
\begin{center}
\includegraphics[scale=0.5]{labo/image11.png}
\captionof{figure}{Diviseur de courant}
\end{center}
Le diviseur de tension, dans ce cas, donne :
\begin{equation}
V_0 = \frac{R_p}{R_p + R_s}V_N
\end{equation}
\subsubsection*{Ampèremètre}
Le principe est similaire à celui du voltmètre.
\begin{center}
\includegraphics[scale=0.5]{labo/image12.png}
\captionof{figure}{Ampèremètre}
\captionof{figure}{Voltmètre}
\end{center}
Le courant passant $I_N$ passe entièrement dans $R_P$.
\begin{equation}
V_0 = R_pI_N
\end{equation}

\subsection{Partie expérimentale}
\subsubsection*{Association de résistance}
\textbf{Mesure expérimentale 1}\\
Les mesures correspondent bien aux prédictions théoriques.\\

\textbf{Mesure expérimentale 2}\\
Diminuer le calibre augmente la précision des mesures. Quand on tente de mesurer $R_3$ on obtient 1. Ceci est dû au fait que la résistance que l'on essaye de quantifier est supérieure au calibre choisi.

\subsubsection*{Générateur de tension continue}
\textbf{Mesure expérimentale 3}\\
Peut importe l'endroit ou on mesure la tension, on obtient toujours $9,99 V$. C'est normal, car en utilisant un générateur de tension continue, la tension est la même partout.

\subsubsection*{Le diviseur de tension}
Dans un tel circuit, on peut calculer l'expression de la différence de potentiel aux bornes de la résistance de droite de la façon suivante = 
\begin{equation}
V = \frac{R_1}{R_1 + R_2}V_S
\end{equation}
Les tensions aux bornes de $R_I$ devrait valoir 8V, aux bornes de $R_3$ 2V et au générateur $V_G = 10V$ (formule ci-dessus).\\
Ces mesures sont bien vérifiées en pratique. La valeur du potentiel est plus faible après la première résistance, car celle-ci diminue déjà le potentiel.

\subsubsection*{Le diviseur de courant}
L'expression du courant $I$ dans la résistance de droite s'exprime : 
\begin{equation}
I = \frac{R_2}{R_1 + R_2}I_G
\end{equation}
Les prédictions théoriques se font grâce à $V = RI$. De cette façon, on obtient $5.10^{-3} A$ pour $R_1$ et $2.10^{-3} A$ pour $R_3$. Au bornes du générateur, $I_G = \frac{V_G}{R_{tot}}
$.
\begin{equation}
I_G = \frac{10}{\left(\frac{1}{20000} + \frac{1}{5000} \right)^{-1}} = 2,5.10^{-3} A
\end{equation}
Expérimentalement, ces mesures sont vérifiées. On voit que $I_G = I_1 + I_2$. 
On peut également voir que $I_1 < I_2$ car $R_1 > R_2$ (c'est proportionnel).

\subsubsection*{Réalisation d'un voltmètre de calibre donné}
Tout d'abord, voici l'expression de $R_S$ :
\begin{equation}
R_S = \frac{V_n R_P}{V_0} - R_P
\end{equation}
et l'expression de $R_N$ :
\begin{equation}
R_N = \frac{V_n R_P}{V_0} = R_p + R_S
\end{equation}
Avec les données fournies dans l'énoncé, l'expression de $R_S$ trouvée ci-dessus devient : 
\begin{equation}
R_S = 10\ R_P
\end{equation}

\subsubsection*{Utilisation du voltmètre réalisé}
\textbf{Partie Pratique 9}\\
Après réalisation de cette partie pratique, on remarque que l'intensité (A) augmente, car un introduit notre voltmètre qui fait diminuer la résistance totale. Il faut donc augmenter au maximum la résistance de notre voltmètre pour éviter qu'il n'influence le système. (Le voltage lui reste inchangé)\\

Avec cette forte résistance, peu de courant passerait par elle : le système ne serait pas perturbé et le peu de courant passant est suffisant pour déterminer la différence de potentiel. \\
Une résistance idéale serait donc infinie.

\subsubsection*{Réalisation d'un ampèremètre de calibre donné}
Comme dit dans le rappel théorique, $I_N$ passe entièrement dans $R_P$.
\begin{equation}
V_0 = R_P I_N
\end{equation}
L'expression analytique de $R_P$ vaut donc $\frac{V_0}{I_N}$.\\

Pour déterminer la résistance interne de notre ampèremètre on procède comme suit : 
\begin{equation}
\frac{1}{R_N} = \frac{1}{R_P} + \underbrace{\frac{1}{R_V}}_{= 0} = \frac{1}{R_P}\ \ \Rightarrow R_N = R_P
\end{equation}
Comme $R_V >> R_P$, on peut dire que la participation de $R_V$ dans la résistance totale est nulle.\\

On peut conclure que la résistance d'un ampèremètre idéal doit être la plus faible que possible, idéalement elle devrait valoir $0\ \Omega$. Ceci est "logique" dans le sens ou l'ampèremètre étant placé en série, il ne faudrait pas que sa résistance interne "fausse" les calculs.







\newpage
\section{Charge et décharge du condensateur}
\subsection{Rappel théorique}
\subsubsection*{Charge du condensateur}
Un condensateur sert à stocker des charges. Quand celui-ci acquiert une charge $+Q$ sur une plaque $-Q$ sur l'autre, une différence de potentiel s'établit : 
\begin{equation}
Q = CV_C
\end{equation}
\textbf{Attention !} Le condensateur attire les charges + sur une plaque et repousse ainsi les "+" de la charge opposée. Même si aucune charge ne traverse l'espace entre les plaques, un courant $I_{ch}$ s'installe dans le circuit RC.\\

Il va y avoir déplacement de charge jusqu'au moment ou la tension d'une plaque vaut celle de l'autre : $V_C = V_g = V_0$. \textbf{Intuitivement} on s'attend à une augmentation de la différence de potentiel au bornes de $C$ très rapide au début, puis de plus en plus lente.

\begin{center}
\includegraphics[scale=0.5]{labo/image13.png}
\captionof{figure}{Charge du condensateur}
\end{center}

Pour prouver ceci de façon plus formelle, on part du fait que $V_0 = V_R + V_C$. Connaissant la loi d'Ohm et la relation citée un peu plus haut, on peut écrire :
\begin{equation}
V_0 = RI + \frac{Q}{C}
\end{equation}
Comme $Q$ et $I$ sont fonction du temps, en dérivant de part et d'autre (sachant que $dQ/dt = I$): 
\begin{equation}
\underbrace{\frac{dV_0}{dt}}_{V_0 = cste} = 0 = R\frac{dI}{dt} + \frac{I}{C}
\end{equation}
D'où on tire : 
\begin{equation}
\frac{dI}{dt} = - \frac{1}{RC}I
\end{equation}
Ce qui donne après intégration : 
\begin{equation}
I(t) = I_0e^{\frac{-t}{RC}}
\end{equation}

où $I_0$ est le courant qui passe dans le circuit quand $t = 0$.

En réinjectant le tout dans l'équation du départ et en utilisant $I_0 = V_0/R$ :
\begin{equation}
V_C = V_0 - R\frac{V_0}{R}e^{-\frac{t}{RC}}
\end{equation}

On peut ainsi trouver les équations de la charge et celle du courant dans le circuit : 
\begin{equation}
V_C(t) = V_0\left(1-e^{-\frac{t}{RC}}\right)\ \ \ \ \ I(t) = I_0e^{-\frac{t}{RC}}
\end{equation}
où $\tau = RC$ qui est appelé \textit{constante de charge}.
\begin{center}
\includegraphics[scale=0.5]{labo/image14.png}
\captionof{figure}{Tension aux bornes du condensateur}
\end{center}

\subsubsection*{Décharge du condensateur}
Supposons qu'en $t = 0$ le condensateur soit chargé. Le générateur n'intervient plus dans l'équation, la branche du circuit sur laquelle il se trouve est ouverte.
\begin{center}
\includegraphics[scale=0.5]{labo/image15.png}
\captionof{figure}{Décharge du condensateur}
\end{center}
Lors de la décharge, une charge + de la plaque supérieure va aller traverser la résistance $R$ et aller neutraliser une charge - de l'autre plaque jusqu'à ce que le condensateur soit déchargé (plus de charge sur les plaques).\\
En faisant le même raisonnement que ci-dessus mais pour la décharge on obtient : 
\begin{equation}
V_C(t) = V_0e^{-\frac{t}{RC}}
\end{equation}
\begin{equation}
I(t) = I_0e^{-\frac{t}{RC}}
\end{equation}
\begin{center}
\includegraphics[scale=0.5]{labo/image16.png}
\captionof{figure}{Courant dans le condensateur}
\end{center}

\newpage
\subsection{Partie expérimentale}
\textbf{Avant les manipulations}\\
Le temps de demi-vie de la charge est le temps nécessaire pour que la tensions aux bornes du condensateur atteigne la moitié de sa valeur maximale, c'est-à-dire que $V = V_0/2$, $V_0$ étant l'amplitude maximale.
\begin{proof}
\begin{equation}
\frac{V_0}{2} = V_0 \left(1-e^{-\frac{t}{RC}}\right)
\end{equation}
\begin{equation}
1 - e^{-\frac{t}{RC}} = \frac{1}{2}
\end{equation}
\begin{equation}
e^{-\frac{t}{RC}} = \frac{1}{2}
\end{equation}
\begin{equation}
\tau_{1/2} = RC\ ln(2)
\end{equation}
\end{proof}

\textbf{Partie Pratique 1}\\
En réglant l'oscilloscope, on obtenait quelque chose dans ce genre la :
\begin{center}
\includegraphics[scale=0.5]{labo/image17.png}
\captionof{figure}{Signal en créneau}
\end{center}
L'utilité d'un signal en créneaux est de simuler la présence d'un interrupteur passant par intervalle de temps régulier sur \textit{ON}, puis \textit{OFF}, ... \\

\textbf{Partie Pratique 2}\\
Voici les résultats expérimentaux recueillis selon les trois méthodes : 
\begin{center}
 \begin{tabular}{|c|c|c|}
\hline 
\textbf{Méthode} & \textbf{Période} (ms) & \textbf{Amplitude} (V) \\ 
\hline 
\textit{Graduations} & 150 & 2,5 \\ 
\hline 
\textit{Curseurs} & 100 & 2,64 \\ 
\hline 
\textit{Programmes} & 100,5 & 2,8 \\ 
\hline 
\end{tabular}
\end{center}
La méthode semblant être la plus précise est la troisième, soit les mesures programmées au sein de l'oscilloscope.\\

\textbf{Partie Pratique 3}\\
Avant tout, définissions ce qu'est un câble coaxial. Un tel câble est constitué d'un conducteur intérieur cylindrique plan et d'un conducteur extérieur creux séparé par un matériau isolant. Le potentiel du conducteur extérieur est constant : c'est celui de référence nommé "masse". On le retrouve à la borne \textbf{noire} de l'extrémité du câble coaxial insérer dans le circuit.
\begin{center}
\includegraphics[scale=0.5]{labo/image18.png}
\captionof{figure}{Cable coaxial}
\end{center}
\textbf{Attention !} Un circuit compte deux de ces "câbles noirs". Ces deux-ci sont reliés l'un à l'autre, c'est-à-dire qu'entre eux deux ils créent une différence de potentiel nulle !\\


Revenons à nos moutons. Lorsque l'on connecte donc deux connecteurs coaxiaux, on doit s'arranger pour que leurs points de masses soient branchés au même endroit sinon on créerait une "zone de potentiel nul" ai sein du circuit.\\

\textit{Expliquer l'influence de la largeur du créneau (demi-période du signal carré). À quelles conditions va-t-on observer des signaux de charge et de décharge complets ?}\\
Après avoir monté un circuit RC et l'avoir alimenté, l'oscilloscope affiche le résultat suivant : 
\begin{center}
\includegraphics[scale=0.3]{labo/image19.png}
\captionof{figure}{Charge et décharge}
\end{center}
Ceci représente bien un signal de charge et de charge complet (même deux.\\
Si le créneau est grand, on a plus facile à observer les signaux de charge et de décharge. Attention néanmoins, si le créneau est trop large, on ne peut observer tout la charge/décharge.\\
Il faut également que le créneau soit assez large, sinon le condensateur commencera sa décharge alors qu'il n'était pas complètement chargé.\\

\textbf{Partie Pratique 4}
Les réglages effectués, voici une courbe de décharge : 
\begin{center}
\includegraphics[scale=0.5]{labo/image22.png}
\captionof{figure}{Courbe de décharge}
\end{center}
En prenant une série de points et en les plaçant sur papier semi-logarithmique ($VC/t$) on obtient une relation linéaire : le graphique vérifie donc la loi.\\
La constante de charge théorique $\tau{th} = R.L$ ce qui dans notre cas fait : $40.000 * 100 * 10^{-9} = 4*10^{-3}$.\\
L'expression à linéariser est la suivante : 
\begin{equation}
log\left(\frac{V_C}{V}\right) = \frac{-log()}{RC}t + log\left(\frac{V_0}{V}\right)
\end{equation}
Celle-ci se linéarise facilement en portant $log(Vc/V)$ on ordonnée et $t$ en abscisse.
\begin{center}
\includegraphics[scale=0.5]{labo/image23.png}
\captionof{figure}{Linéarisation de la courbe de décharge}
\end{center}
Graphiquement, ce temps de charge théorique correspond à la pente de notre droite.\\

\textbf{Partie Pratique 6}\\
En mesurant la tension aux bornes de la résistance, on obtient la courbe suivante pour la charge :
\begin{center}
\includegraphics[scale=0.5]{labo/image21.png}
\captionof{figure}{Tension aux bornes de la résistance}
\end{center}
Lors de la décharge, la tension au générateur est nulle : $V_G = 0$. La loi des mailles s'écrit alors $V_G = V_C + V_R = 0$, c'est-à-dire que $V_R = - V_C$ ce qui se vérifie bien sur le graphique ci-dessus.\\

\textbf{Partie Pratique 7}
Comme $ln(I)$ est porté en ordonnée, les points expérimentaux se placent selon une droite d'origine $ln(I_0)$ et de pente $-\frac{1}{RC}$.

De façon théorique, on peut calculer $I_{0, th}$ en partant de la loi d'Ohm $V/R = 38*10^{-4}\ A$.
\begin{center}
\includegraphics[scale=0.5]{labo/image24.png}
\captionof{figure}{Linéarisation}
\end{center}

 Graphiquement, sil s'agit de la pente de la droite.\\

\textbf{Partie Pratique 8}\\
\textbf{Si} on double la valeur de la capacité du condensateur, le temps de demi-vie augmente d'un facteur deux. \textbf{Si} on double la valeur de la résistance de décharge, le temps de demi-vie augmente d'un facteur deux. Si on double la valeur de la capacité du conducteur et que l'on divise par deux la valeur de la résistance le temps de demi-vie ne change pas de valeur.\\
\textit{Le temps de décharge est proportionnel à R et C}. C'est normal, la relation du temps de charge étant $\tau = RC$.

\textbf{Partie Pratique 9}\\
La loi d'association de condensateur en parallèle est : 
\begin{equation}
C = C_1 + C_2
\end{equation}
et en série : 
\begin{equation}
\frac{1}{C} = \frac{1}{C_1} + \frac{1}{C_2}
\end{equation}
La capacité du condensateur est plus faible en série qu'en parallèle $\Rightarrow \tau_{1/2}$ est plus petit en série qu'en parallèle.\\
En connaissant la résistance, il suffit de calculer le temps de demi-vie pour 







\newpage
\section{Lentilles minces}
\subsection{Rappel théorique}
\subsubsection*{Sources primaires et secondaires}
Il est important de bien s'accorder sur l'usage du mot "source" :
\begin{description}
\item[Source primaire/directe] Génère la lumière au départ d'une autre forme d'énergie (bougie par combustion, ampoule par électricité, ...)
\item[Source secondaire/indirecte] Ré-émet la lumière d'une autre source (directe ou indirecte) (lumière d'une lampe au plafond)
\end{description}

\subsubsection*{Sources ponctuelles et étendues}
Quand la taille d'une source peut être considérée comme petite par rapport à l’élément d'un montage (flamme d'une bougie dans une pièce, soleil depuis la Terre), on peut considérer la lumière émise d'un seul point : source ponctuelle.\\
A l'inverse, vu de "près" un écran d'ordinateur ne peut être considéré comme un source ponctuelle : ce sera une source étendue.

\subsubsection*{Rayons lumineux}
Quand les  hypothèses géométriques sont vérifiées, on peut considérer que l'énergie lumineuse se déplace en ligne droite. Pour une source ponctuelle, elle se déplace en même temps dans toutes les directions : il s'agit de \textit{rayons lumineux}.\\
Parfois on ne veut de la lumière que d'un côté, avec une lampe de poche par exemple : le système n'est plus isotrope.

\subsubsection*{Réflexion, réfraction, absorption, diffusion}
Quand un rayon passe d'un milieu à un autre, trois cas sont possibles : réflexion, transmission ou absorption. (En général, un phénomène domine les trois).\\
\begin{figure}[h]
\begin{center}
\includegraphics[scale=0.5]{labo/image26.png}
\end{center}
\caption{Réflexion, réfraction et absorption}
\label{rra}
\end{figure}\ \\
Il reste la \textit{diffusion}, à prendre en compote quand l'interface est très irrégulière : les rayons incidents donne lieu à une infinité de rayons lumineux dans toutes les directions.\\
\begin{figure}[h]
\begin{center}
\includegraphics[scale=0.5]{labo/image27.png}
\end{center}
\caption{Diffusion}
\label{diff}
\end{figure}\ \\
Comme toutes les directions sont possibles, la répartition n'est pas équilibrée et il n'est pas possible de déterminer une image

\subsubsection*{Lentille mince}
Certaines choses possèdent des propriétés remarquables : on les appelle "optiques" ou "dispositifs optiques" pour les intimes.\\

Une lentille est constitué d'un \textit{indice de réfraction} différent de l'air pour dévier les faisceaux lumineux grâce à la forme de celle-ci.\\

Son axe de symétrie de révolution est l'\textit{axe optique}. L'intersection de la lentille avec l'axe optique est le \textit{centre optique}, O. Une lentille mince est caractérisée par une \textit{distance focale} $f$.\\
Le \textit{foyer image/objet} sont les deux points situé à une distance $f$ du centre de l'axe optique et les plans focaux sont les plans orthogonaux à l'axe optique passant par les foyers.
\begin{center}
\includegraphics[scale=0.6]{labo/image28.png}
\captionof{figure}{Lentille convergente et divergente}
\end{center}
Deux règles peuvent servir à tracer tous les rayons qui traversent une lentille : 
\begin{enumerate}
\item Un rayon incident sur le centre optique de la lentille n'est pas dévié.
\item Deux rayons incidents parallèles sont déviés de telle manière qu'ils se croisent dans
le plan focal image
\end{enumerate}
\textit{Pour voir comment tracer les rayons, consulter le syllabus de laboratoire.}

\subsubsection*{Image optique}
On parle d'\textit{image optique} lorsqu'on peut établir une correspondance entre les points sources d'un plan objet et le croisement des rayons dans un autre plan appelé "plan image".\\
La \textit{loi de conjugaison}, reliant $d_i$ et $d_o$ à $f$ s'exprime : 
\begin{equation}
\frac{1}{f} = \frac{1}{d_i} - \frac{1}{d_o}
\end{equation}
\textit{NB :} la convention de signe pour $d_i$ et $d_o$ sont mesuré par rapport à la lentille, positivement dans le sens des rayons.\\

On peut également définir le facteur de grossissement $G$ :
\begin{equation}
G = \frac{h_i}{h_o} = \frac{d_i}{d_o}
\end{equation}
Si celui-ci est négatif, c'est que l'image est retournée. Lorsque $|G| > 1$, l'image apparaît plus grande que l'objet.

\subsubsection*{Retour inverse de la lumière}
Le principe du retour inverse de la lumière stipule que si l'on inverse le sens de propagation de la lumière, elle refera exactement le même trajet à travers les lentilles, mais en sens inverse.

\newpage
\subsection{Partie expérimentale}
\textbf{Partie Pratique 1}\\
En plaçant l'écran blanc assez loin de la source et en jouant avec la molette de la source, on voit apparaître le filament. Ceci est dû au fait que la source se compose d'une lentille interne convergente.\\

\textbf{Partie Pratique 2}\\
En posant l'objet, on ne peut pas avoir d'image, car la lumière passe par un verre qui modifie la trajectoire des rayons (changement de milieu).\\

\textbf{Partie Pratique 3}\\
On ne parvient qu'à obtenir une seule image, $d_o$ ne variant pas (pas d'inversion $d_i/d_o$.\\
Autrement dit, comme la distance focale $f$ et $d_o$ sont fixés, il ne peut y avoir qu'un seul $d_i$ et donc qu'un seul endroit ou l'image est nette.\\

\textbf{Partie Pratique 4}\\
En faisant glisser la lentille on pourra identifier deux images nettes. Ceci est dû au \textit{retour inverse de la lumière} : la distance image et objet s'inverse.\\
Plus mathématiquement, on sait que :
\begin{equation}
\frac{1}{f} = \frac{1}{d_i} - \frac{1}{d_o}\ \ \ \ et\ \ \ \ \ d_i + d_o = cste
\end{equation}
Comme $f$ est fixée, nous avons deux équations à deux inconnues $\rightarrow$ deux solutions sont possibles.\\

\textbf{Partie Pratique 5}\\
En plaçant la lentille de façon à faire apparaitre une image nette sur l'écran et en déplaçant la molette, rien ne change si ce n'est l'intensité lumineuse (trajectoire des rayons identiques).\\

\textbf{Partie Pratique 6}\\
\textit{Premier cas : crayon devant objet -} En faisant ceci, on obstrue une partie des rayons d'émaner de la source. L'image ne contient pas la "parties qui a été obstruée.\\

\textit{Deuxième cas : crayon devant lentille -} Ici, aucune modification n'est observée si ce n'est une légère baisse de l'intensité. Comme le crayon est placé à un $d_o$ plus petit que $f$, rien ne peut se produire sur l'écran. Physiquement, c'est dû au fait qu'obstruer une partie des rayons n'empêche pas tous les rayons de la parties obstruées de passer.\\

\textbf{Partie Pratique 7}\\
Voici un petit texte explicatif (Wiki.) pour bien comprendre comment déterminer la distance focale.\\

\textit{Un objet lointain est un objet dont la distance à la lentille est très supérieure à la distance focale de la lentille. Nous pouvons considérer que les rayons qui proviennent de cet objet sont parallèles.}

\textit{Sur un écran quelconque, regardons l'image d'un paysage lointain fournie par une lentille convergente. Nous observons une image retournée et nette du paysage quand l'écran se situe à la distance focale de la lentille. Une image nette n'est obtenue que pour une seule position de l'écran : l'image est localisée.}\\

\textit{Pour déterminer la distance focale d'une lentille convergente, il suffit donc de mesurer la distance entre la lentille et l'endroit où apparaît nette l'image d'un objet lointain.}\\

\textit{Remarque : avec une lentille divergente, on constaterait que l'on ne peut pas obtenir d'image, quelle que soit la distance entre la lentille et l'écran.}\\

Mathématiquement, ceci s'explique comme ceci :
\begin{equation}
\frac{1}{f} = \frac{1}{d_i} - \underbrace{\frac{1}{d_o}}_{\approx 0}\ \ \ \ \ \Rightarrow \ \ f = d_i
\end{equation}
En effectuant une telle opération ô combien risquée, nous obtenons une distance focale de $20\ cm$.\\

\textbf{Partie Pratique 8, 9}\\
En déplaçant la lentille, on peut modifier $d_{objet}$. En considérant la position ou l'image est nette, on peut déterminer $d_{image}$.\\
En reprenant différents points et en les plaçant en considérant $1/d_i$ en abscisse et $1/d_o$ en ordonnée on obtient bien une relation linéaire : la loi de conjugaison est bien vérifiée.\\
Encore mieux, l'ordonnée à l'origine étant $1/f$, on peut trouver la distance focale.\\

\textit{Ou se forme l'image dans le cas où l'objet est placé à une distance égale à la distance focale ? }\\
L'image se forme à une distance $d_i = + \infty$. En effet : 
\begin{equation}
\frac{1}{f} = \frac{1}{d_i} + \frac{1}{f} \Leftrightarrow \frac{1}{d_i} = 0\ \ \Rightarrow d_i = +\infty
\end{equation}

\textit{Ou se forme l'image si l'objet est placé à une distance inférieure à la distance focale ?}\\
Aucune image réelle ne sera formée. Par contre il y aura bien une image virtuelle, mais celle-ci n'est pas visible sur l'écran.\\
Pour bien comprendre comment une loupe "grossit", voici un schéma d'une qualité hors norme : 
\begin{center}
\includegraphics[scale=0.6]{labo/image32.png}
\end{center}\ \\

\textbf{Partie Pratique 10 - 11}\\
La position de l'écran ou s'affiche l'image nette vaut $x_{écran} = 6\ cm$. La taille de l'image mesurée vaut $1,8\ cm$. On remarque que l'orientation y est inversée comme pour une symétrie centrale.\\

Si on retire l'écran, en regardant dans la direction de la lentille on peut voir une image, en réalité virtuelle.\\

\textbf{Partie Pratique 12}\\
Cette image virtuelle peut être considérée comme un "objet" et être ainsi "utilisée" par une lentille. Pour le prouver, on mesure une distance objet $d_o = 34\ cm$ et une distance image $d_i = 38\ cm$.\\
En appliquant la loi de conjugaison, on retrouve une distance focale $f = 17,94\ cm$ qui, compte tenu des erreurs, vérifie bien que cette image est un objet.\\
\textit{La première image peut être vue comme un objet et peut être utilisée  par une lentille.}\\

\textbf{Partie Pratique 3.3.4}\\
Pour mesurer une distance focale négative, il faut utiliser une méthode plus complexe. Ceci est dû au fait que l'image se forme du même côté de l'objet (distance focale négative). On ne sait pas à proprement parler "la voir" car on couperait les rayons lumineux.\\
Pour réussir à mesurer une telle distance, il faudra utiliser deux lentilles. On crée un objet virtuel avec une première pour le "donner" à la lentille divergente.\\

Pour mesurer la distance focale $f$, on considère que $d_i$ est la distance entre la lentille divergente et l'écran et $d_o$ est la distance entre  l'image virtuelle est la lentille divergente.\\

En gros, il faut placer les éléments dans l'ordre : source - lentille convergente - lentille divergente - écran puis appliquer la loi de conjugaison (Source : \textit{Futura-sciences})
\begin{center}
\includegraphics[scale=0.5]{labo/image33.png}
\captionof{figure}{Mesure d'une distance focale négative}
\end{center}















































\chapter{Questions préparatoires}
\section{Thermodynamique}
\newcommand{\quest}[3]{#1. \textbf{#3} (p.#2)}
	\subsection{Théorie}
	\quest{1}{4}{Donnez l'expression de la pression hydrostatique correspondant à une hauteur d'eau $h$ et donnez la signification physique des grandeurs qui y apparaissent.}
	$$P(h) = \rho g h$$ où $P$ est la pression, $\rho$ est la masse volumique de l’eau, $g$ est l’accélération de la gravitation terrestre et $h$ la hauteur d’eau.
	
	\quest{2}{13}{Donnez, en degrés Celsius, la température la plus basse que l'on peut atteindre.}
	
	$$0\ \text K = -273,15\ \text C^o$$
	
	\quest{3}{17-23}{De quel facteur varie la température d'un gaz parfait dont la pression double à volume constant ?}
	
	Elle double : $$V = \cst : P_f = 2P_i \Rightarrow T_f = 2T_i $$
	
	\quest{4}{21-23}{De quel facteur varie le volume d'un gaz parfait dont la pression double à température constante ?}
	
	Il diminue de moitié : $$T = \cst : P_f = 2P_i \Rightarrow V_f = \frac{V_i}{2} $$
	
	\quest{5}{19-23}{De quel facteur varie le volume d'un gaz parfait dont la température double à pression constante ?}
	
	Il double : $$P = \cst : T_f = 2T_i \Rightarrow V_f = 2V_i $$
	
	\quest{6}{28}{Donnez la masse d'une mole de nucléons.}
	
	$$ 1 gt $$
	
	\quest{7}{28}{Donnez la masse d'une mole de carbone.}
	
	$$ 12 gr $$
	
	
	\quest{8}{31}{Établissez le lien entre qui existe entre la constante de Boltzmann et la constante universelle des gaz parfaits.}
	
	On sait que :
	\begin{eqnarray}
		PV = nRT\\ 
		\label{q8eq2}
		\text{et}\ PV = Nk_BT
	\end{eqnarray}
	or $ n = \frac{N}{N_A} \Rightarrow N = n N_A$.\\
	En utilisant ce résultat dans l'équation \ref{q8eq2}, on obtient : $ PV = n\underbrace{N_Ak_B}_
	{\mbox{R}}T$ et donc : \fbox{$R = N_Ak_B$}
	
	\quest{9}{51,56}{Donnez le lien existant entre la température et l'énergie cinétique moyenne des particules d'un gaz parfait.}
	
	$$ T = \frac{2}{3} \frac{E_C}{k_B}$$
	
	\quest{10}{67}{Donnez l'expression de l'énergie interne d'un gaz parfait monoatomique et donnez la signification physique des grandeurs qui y apparaissent.}
	
	$$ U \equiv NE_C = \frac{3}{2}Nk_BT$$ où $U$ est l’énergie interne du gaz, $N$ est le nombre de particules, $E_C$ est l’énergie cinétique moyenne des particules, $k_B$ est la constante de Boltzmann et $T$ est la température.
	
	\quest{11}{70}{Donnez l'expression de l'énergie interne d'un gaz parfait polyatomique et donnez la signification physique des grandeurs qui y apparaissent.}
	
	$$U \equiv NE_C+NE_R+NE_V=\frac{n_d}{2}Nk_BT$$ où $E_R$ est l’énergie de rotation moyenne des particules, $E_V$ est l’énergie de vibration moyenne des particules et $n_d$ le nombre de degrés de liberté du gaz. (+ notation q.10)
	
	\quest{12}{70}{Donnez l'expression de la capacité calorifique molaire à volume constantd'un gaz parfait et donnez la signification physique des grandeurs qui y apparaissent.}
	
	$$C_V=\frac{n_d}{2}R$$ où $C_V$ est la capacité calorifique molaire à volume constant et $R$ est la constante universelle des gaz parfaits. (+ notation q.11)
	
	\quest{13}{82}{Donnez la loi de la conduction thermique de Fourier et donnez la signification physique des grandeurs qui y apparaissent, en vous aidant d'un shéma.}
	
	$$H=k_TS\frac{\Delta T}{L}$$ où $H$ est le débit de chaleur, $k_T$ est la conductivité thermique du matériaux, $S$ est la surface de la paroi, $\Delta T = T_c-T_f$ est la différence de température et $L$ est l’épaisseur de la paroi.
	
	\quest{14}{91}{Donnez l'expression du premier principe de la thermodynamique pour un système pouvant recevoir de la chaleur et fournir un travail.}
	
	Toute variation de l’énergie interne $\Delta U$ est due à la chaleur entrante $Q$ et au travail sortant $W$ : $$\Delta U=Q-W$$
	
	\quest{15}{99}{Expliquez brièvement pourquoi la capacité calorifique molaire à pression constante d'un gaz parfait est plus grande que sa capacité calorifique molaire à volume constant.}
	
	À pression constante, une partie de l'énergie va être utilisée sous forme de travail d'expansion, ce qui n'est pas le cas à volume constant. Pour une même quantité de chaleur, le gaz s'échauffera donc moins à pression constante : \fbox{$C_p > C_v$}
	
	\quest{16}{103}{Donnez l'expression générale du travail fourni par un gaz parfait lors d'un détente quelconque.}
	
	De manière générale : $$ W = \int_{V_i}^{V_f}P(V) dV$$
	
	\quest{17}{95,104}{Donnez l'expression du travail fourni par la détente isobare d'un gaz parfait.}
	
	Pour une détente isobare : $$W=P\Delta V$$
	
	\quest{18}{114}{Donnez l'expression de la loi de Laplace de la transformation adiabatique en termes de pression et volume.}
	
	$$PV^\gamma = \cst$$ où $\gamma=1+\frac{2}{n_d}$ est le c\oe fficient adiabatique.
	
	\quest{19}{116}{Donnez l'expression du travail fourni par la détente adiabatique d'un gaz parfait en termes de températures initiale et finale de la transformation.}
	
	Pour une détente adiabatique : $$W=\frac{Nk_B}{\gamma-1}(T_i-T_f)$$
	
	\quest{20}{137}{Donnez la définition du rendement d'un moteur thermique et exprimez-le en fonction des chaleurs entrantes et sortantes.}
	
	Le rendement d’un moteur thermique est le rapport entre le travail mécanique fourni par le moteur $W_{cycle}$ (énergie utile) et la chaleur fournie par la source chaude $Q_H$ (énergie consommée). $$r \equiv \frac{W_{cycle}}{Q_H}=1-\frac{|Q_B|}{Q_H}$$
	
	\quest{21}{137}{Donnez l'expression du rendement du cycle de Carnot en fonction des températures haute et basse du cycle.}
	
	Pour le cycle de Carnot : $$r=1-\frac{T_B}{T_H}$$
	
	\quest{22}{160}{Donnez l'expression générale de la variation d'entropie sous forme intégrale.}
	
	Variation d'entropie : $$\Delta S \equiv \int \limits_{A\rightarrow B} \frac{dQ}{T}$$
	
	\quest{23}{163}{Donnez la définition de l'entropie sous forme différentielle.}
	
	Forme différentielle : $$dS = \frac{dQ}{T}$$
	
	\quest{24}{165}{Donnez l'expression de la production d'entropie provoquée par une compression adiabatique en fonction des pressions initiale et finale.}
	
	Pour une compression adiabatique : $$\Delta S = 0$$
	
	\quest{25}{166}{Donnez l'expression de la production d'entropie provoquée par l'échauffement isochore d'un gaz parfait en fonction des températures initiale et finale.}
	
	Pour un échauffement isochore : $$ \Delta S = \frac{n_d}{2}Nk_B\ln (\frac{T_f}{T_i})$$
	
	
	
	
	
	\subsection{Exercices préparatoires}

\exerc{A}{Calculez la variation de volume d’un cylindre de rayon $R$ qui passe d’une longueur $L$ à
une longueur $L' = L + \Delta x$.}

	$V = S \times L \Rightarrow \Delta V = S \times \Delta L$ avec $S = \pi R^2$ et $\Delta L = \Delta x$
	
	$\Rightarrow$ \fbox{$\Delta V = \pi R^2 \Delta x$}

\vspace{0,5cm}

\exerc{B}{\textit{a)} Calculez la pression exercée par un homme de masse $m = 80\ \text{kg}$ qui se situe sur sa
balance, considérant que la surface de ses pieds vaut $S = 400\ \text{cm}^2$.}
	
	\fbox{$P = \dfrac{mg}{S}$} $= \frac{80\ \text{kg}\ 9,81\ \text{m}\ \text{s}^{-2}}{0,04\ \text{m}^2} = 19\,620\ \text{Pa}$
	
	\textbf{b) Quelle est la pression totale sur la balance sachant que $P_{\text{atm}} = 1\ \text{atm}$ ?}
	
	$P_{\text{tot}} = P_{\text{atm}} + P = 101\,325\ \text{Pa} + 19\,620\ \text{Pa} = 120\,945\ \text{Pa}$

\vspace{0,5cm}

\exerc{C}{À partir du premier principe et de la loi des gaz parfaits, montrez que :\\ \textit{a)} lors d’une transformation isotherme, le travail $W = nRT \ln(\frac{V_f}{V_i})$.}
	
	$W = \int P(V)dV$ avec $P(V) = \frac{nRT}{V}$
	
	$W = \int \dfrac{nRT}{V} dV = nRT \int \frac{dV}{V} = nRT \ln(\frac{V_f}{V_i})$
	
	\textbf{\textit{b)} lors d’une transformation adiabatique, $pV^{\gamma} = \text{constante}$ (où $\gamma = 1 + \frac{2}{n_d}$).}
	
	$Q = 0 \Rightarrow \Delta U = -W \Rightarrow dU = -dW \Rightarrow \dfrac{n_d}{2}Nk_BdT = -P(V)dV $ avec $P(V) = \dfrac{Nk_BT}{V}$ :
	
	$$\dfrac{dT}{T} = -\dfrac{2}{n_d}\dfrac{dV}{V}$$
	
	Intégrons membre à membre :
	
	$$\int_{T_i}^{T} \frac{dT}{T} = -\frac{2}{n_d} \int_{V_i}^{V} \frac{dV}{V}$$
	
	$\Rightarrow \ln \left(\dfrac{T}{T_i}\right) = -\dfrac{2}{n_d} \ln \left(\dfrac{V}{V_i}\right) \Rightarrow \ln \left(\dfrac{T}{T_i}\right) =  \ln \left(\left(\dfrac{V}{V_i}\right)^{-\frac{2}{n_d}}\right) \Rightarrow \dfrac{T}{T_i} = \left(\dfrac{V_i}{V}\right)^{\frac{2}{n_d}} \Rightarrow TV^{\frac{2}{n_d}} = T_i V_i^{\frac{2}{n_d}}$
	
	Tout les états thermodynamiques par lesquels le gaz passe durant la transformation adiabatique sont donc caractérisés par la même valeur du produit $TV^{\frac{2}{n_d}}$.
	
	Autrement dit : \fbox{$TV^{\frac{2}{n_d}}=\cst$}.
	
	Sachant que $T = \dfrac{PV}{nR}$, on en déduit que $\dfrac{PV}{nR}V^{\frac{2}{n_d}} = \cst$. En tenant compte du fait que $nR$ est une constante et en posant $\gamma=1+\frac{2}{n_d}$ nous sommes conduits à la relation suivante :
	
	$$PV V^{\frac{2}{n_d}} = PV^{1+\frac{2}{n_d}}= PV^\gamma=\cst $$

\vspace{0,5cm}

\exerc{D}{À partir du premier principe et de la loi des gaz parfaits, retrouvez l’expression de la
capacité calorifique molaire à pression constante $C_p$}
	
	Définition de $C_p$ : $$Q \equiv C_p n \Delta T$$ La valeur de $C_p$ est la quantité de chaleur nécessaire pour chauffer une mole ($n=1$) de 1 kelvin ($\Delta T = 1\ \text{K}$). Le rapport $\dfrac{Q}{\Delta T}$ nous donneras l'expression de la capacité calorifique d'une mole.
	
	$$ \frac{Q}{\Delta T} = \frac{\Delta U}{\Delta T} + \frac{P\Delta V}{\Delta T}$$
	
	$U = \dfrac{n_d}{2}N_Ak_BT \Rightarrow U = \dfrac{n_d}{2}RT \Rightarrow \Delta U = \dfrac{n_d}{2}R\Delta T$
	
	$ PV = RT \Rightarrow P \Delta V = R \Delta T$
	
	On obtient donc : $$ C_p \equiv \frac{Q}{\Delta T} = \frac{n_d}{2}R + R = (\frac{n_d}{2}+1) R$$

\vspace{0,5cm}

\exerc{E}{À partir du premier principe et de la loi des gaz parfaits, retrouvez l’expression de la variation d'entropie $\Delta S$ lors d'une transformation : \\a) isotherme}
	
		$ U = \dfrac{n_d}{2}Nk_BT \Rightarrow \Delta U = 0$ (Pas de variation de température)
		
		Par le premier principe : $ Q = \Delta U + W = W = \int  \limits_{i \to f} P(V)\,dV$
		
		Or, $ P(V) = \dfrac{Nk_BT}{V} \Rightarrow Q = Nk_BT\ln \left(\dfrac{V_f}{V_i}\right)$
		
		$ \Delta S = \int \limits_{i \to f} \dfrac{dQ}{T} = \dfrac{1}{T} \int \limits_{i \to f} dQ = \dfrac{Q}{T} = Nk_B\ln \left(\dfrac{V_f}{V_i}\right)$
	
	\textbf{b) adiabatique}
		
		Ce cas est trivial car il y a absence de transfert de chaleur. Si $dQ$ est nul en tout point de la transformation, $\frac{dQ}{T}$ est à fortiori nul également et donc la variation d'entropie d'une transformation adiabatique est nulle. $$\Delta S = 0$$
		
	\textbf{c) isochore}
	
		$W = \int  \limits_{i \to f} P(V)\,dV = 0$ (Pas de variation de volume)
		
		Par le premier principe : $ Q = \Delta U + W = \Delta U = \dfrac{n_d}{2}Nk_B\Delta T \Rightarrow dQ = \dfrac{n_d}{2}Nk_BdT $
		
		$ dS = \dfrac{dQ}{T} = \dfrac{n_d}{2}Nk_B\dfrac{dT}{T} \Rightarrow \Delta S = \int \limits_{i \to f} \dfrac{dQ}{T} = \dfrac{n_d}{2}Nk_B \int \limits_{i \to f} \dfrac{dT}{T} = \dfrac{n_d}{2}Nk_B\ln \left(\dfrac{T_f}{T_i}\right)$
		
	\textbf{d) isobare}
		
		$ Q = nC_PT \Rightarrow dQ = nC_PdT$ avec $ C_P = (\frac{n_d}{2}+1)R$
		
		Sachant que $n = N/N_A$ et que $R = N_Ak_B$, on obtiens : $dQ = (\frac{n_d}{2}+1)Nk_BdT $
		
		$dS = \dfrac{dQ}{T} = (\frac{n_d}{2}+1)Nk_B\dfrac{dT}{T} \Rightarrow \Delta S = \int \limits_{i \to f} \dfrac{dQ}{T} = (\frac{n_d}{2}+1)Nk_B\ln \left(\dfrac{T_f}{T_i}\right)$

\vspace{0,5cm}

\exerc{F}{\textit{a)} Montrez que la variation d'énergie interne $\Delta U = \frac{n_d}{2}W$ lors d'une transformation isobare et déduisez-en son rendement.}
		
		$\Delta U = \dfrac{n_d}{2}Nk_B\Delta T $,
		
		On sait que $W = P\Delta V$ et que $ P\Delta V = Nk_B\Delta T$, donc :
		
		$ \Delta U = \dfrac{n_d}{2}W$
		
		$Q = \Delta U + W = (\frac{n_d}{2}+1)W \Rightarrow r \equiv \dfrac{W}{Q} = \dfrac{1}{\frac{n_d}{2}+1}$
		
	\textbf{\textit{b)} Montrez que la variation d'énergie interne $\Delta U = 0$ lors d'une transformation isotherme et déduisez-en son rendement.}
		
		L'énergie interne n'est constituée que de l'énergie thermique du gaz dont la température est constante. La variation d'énergie interne est donc, par définition, nulle dans la transformation isotherme : $\Delta U = 0$
		
		Premier principe : $Q = \Delta U + W = W \Rightarrow r \equiv \dfrac{W}{Q} = 1 = 100\% $
		

\newpage
\section{Électrostatique}
	\subsection{Théorie}
	\quest{1}{10}{Donnez l'expression générale, vectorielle, de la force électrique qui s'exerce entre deux charges électriques ponctuelles $q_1$ et $q_2$.}
	
	$$ \vec F_e = k_0 \dfrac{q_1 q_2}{r^2} \vec 1_r \left(= \dfrac{q_1 q_2}{4\pi \varepsilon_0 r^2} \vec 1_r \right) $$
	
	\quest{2}{10}{De quel facteur augmente la force électrique qui s'exerce entre deux charges ponctuelles lorsque la distance qui les sépare diminue de moitié ? Justifiez brièvement votre réponse.}
	
	D'un facteur 4 : $ ||\vec F_1|| = k_0 \dfrac{q_1 q_2}{r^2}\ \text{et}\ ||\vec F_2|| = k_0 \dfrac{q_1 q_2}{(\frac{r}{2})^2} = 4 k_0 \dfrac{q_1 q_2}{r^2}$
	
	$$\Rightarrow ||\vec F_2|| = 4\,||\vec F_1||$$
	
	\quest{3}{10}{Donnez les unités de la constante de force électrique $k_0$.}
	
	$$ k_0 = 8,987 \times 10^9\ \frac{\textbf{N} \textbf{m}^2}{\textbf{C}^2} $$
	
	\quest{4}{17}{Donnez l'expression générale du champ électrique généré par une charge ponctuelle.}
	
	$$ \vec E \equiv \dfrac{\vec F}{q_0} = k_0 \dfrac{q}{r^2} \vec 1_r $$
	
	\quest{5}{17}{Donnez les unités du champ électrique en termes des unités de base du SI (m, kg, s, A, K, mol), sachant que $ 1C = 1A \times 1s$.}
	
	$$ [\vec E ] = \dfrac{\text{N}}{\text{C}} = \dfrac{\text{Kg}\times \frac{\text{m}}{\text{s}^2}}{\text{A} \times \text{s}} = \dfrac{\text{Kg}\times \text{m}}{\text{A} \times \text{s}^3} $$
	
	\quest{6}{23-26}{Justifiez brièvement pourquoi le champ électrique est nul au sein d'un matériau conducteur.}
	
	\quest{7}{23-26}{Justifiez brièvement pourquoi le champ électrique est perpendiculaire à la surface d'un matériau conducteur.}
	
	\quest{8}{30}{Donnez les unités de la densité surfacique de charge électrique.}
	
	$$ \sigma (\vec x_m) \equiv \lim_{\Delta S_m \to 0} \dfrac{\Delta q_m}{\Delta S_m}\qquad[\textbf{C}/\textbf{m}^2]$$
	
	\quest{9}{33}{Donnez les unités de la densité linéique de charge électrique.}
	
	$$ \lambda (\vec x_m) \equiv \lim_{\Delta l_m \to 0} \dfrac{\Delta q_m}{\Delta l_m}\qquad[\textbf{C}/\textbf{m}]$$
	
	\quest{10}{37}{Donnez les unités d'un flux de particules.}
	
	$$[\Phi] = \text{s}^{-1} $$
	
	\quest{11}{38}{Donnez les unités d'une densité de flux de particules.}
	
	Attention à ne pas confondre $\vec F_e$ et $\vec F$ ! $$ [\vec F] = \text{m}^{-2}\,\text{s}^{-1}$$
	
	\quest{12}{40}{Donnez les unités de la permittivité du vide.}
	
	$$ \varepsilon_0 \equiv \dfrac{1}{4\pi k_0} = 8,854 \times 10^{-12}\ \dfrac{\textbf{C}^2}{\textbf{N}\textbf{m}^2}$$
	
	\quest{13}{42}{Soit un flux de particules dont la densité de flux $\F$ est uniforme. Donnez l'expression du flux de particules au 
	travers d'une surface plane $S$ dont la normale fait un angle $\theta$ avec la densité de flux $\F$.}
	
	$$ \Phi^{S} = \vec F \cdot \vec S = ||\vec F||\,S \cos (\theta)  $$
	
	\quest{14}{55}{Donnez l'expression mathématique générale (pour une distribution de charge continue quelconque) de la loi de Gauss de l'électrostatique.}
	
	$$ \Phi^S_E = \oint \limits_S \vec E \cdot d \vec S = \dfrac{1}{\varepsilon_0} \int \limits_{V_S} \rho (\vec x) \, dV$$
	
	\quest{15}{57}{Par application de la loi de Gauss, démontrez l'expression du champ électrique généré par un plan uniformément chargé d'extension infinie.}
	
	Prenons comme surface de Gauss un cylindre traversant perpendiculairement le plan.
	
	Le flux sur le coté du cylindre est nul car le champ y est perpendiculaire à la normale. Pour les bases, le flux est égal au produit de la surface et du champs car le flux et la normale ont la même direction ($\cos 0 = 1$). Mathématiquement: $$ \Phi = \oint \limits_S \vec E \cdot d \vec S = 2E\Delta S$$
	
	Nous savons aussi que : $$ \Phi =\dfrac{1}{\varepsilon_0} \int \limits_{V_S} \rho (\vec x) \, dV = \dfrac{\sigma \Delta S}{\varepsilon_0}$$
	
	NB : Pour obtenir ce résultat il faut se rappeler que l'intégrale n'est que l'expression formelle de la charge totale incluse dans la surface de Gauss. On sait que cette charge est donné par $\sigma \Delta S$.
	
	Nous obtenons donc : $$ 2E\Delta S = \dfrac{\sigma \Delta S}{\varepsilon_0} \Rightarrow E = \dfrac{1}{2\varepsilon_0}\sigma \Rightarrow \vec E = \dfrac{1}{2\varepsilon_0}\sigma \vec 1_S$$
	
	\quest{16}{58}{Par application de la loi de Gauss, démontrez l'expression du champ électrique généré par un fil uniformément chargé d'extension infinie.}
	
	Prenons comme surface de Gauss un cylindre qui "entoure" le fil.
	
	Le flux sur les bases est nul car le champ y est perpendiculaire à la normale. Pour le coté, le flux est égal au produit de la surface et du champs car le flux et la normale ont la même direction ($\cos 0 = 1$). Mathématiquement : $$ \Phi = \oint \limits_S \vec E \cdot d \vec S = E\times 2\pi RL $$
	
	Nous savons aussi que : $$ \Phi =\dfrac{1}{\varepsilon_0} \int \limits_{V_S} \rho (\vec x) \, dV = \dfrac{\lambda L}{\varepsilon_0}$$
	
	NB : Pour obtenir ce résultat il faut se rappeler que l'intégrale n'est que l'expression formelle de la charge totale incluse dans la surface de Gauss. On sait que cette charge est donné par $\lambda L$.
	
	Nous obtenons donc : $$ E\times 2\pi RL = \dfrac{\lambda L}{\varepsilon_0} \Rightarrow E = \dfrac{\lambda}{\varepsilon_0 2 \pi R} \Rightarrow \vec E = \dfrac{\lambda}{\varepsilon_0 2 \pi R} \vec 1_r $$
	
	\quest{17}{67}{Donnez la forme analytique de la divergence d'un champ électrique $\E$ décrit dans un repère cartésien orthonormé $(x,y,z)$.}
	
	$$ \text{div}\,\E = \vec \nabla \cdot \E = \dfrac{\partial}{\partial x} E_x + \dfrac{\partial}{\partial y} E_y + \dfrac{\partial}{\partial z} E_z$$
	
	\quest{18}{68}{Donnez la forme locale de la loi de Gauss de l'électrostatique.}
	
	$$ \text{div}\,\E = \vec \nabla \cdot \E = \dfrac{1}{\varepsilon_0}\rho (\vec x)$$
	
	\quest{19}{76}{Donnez sous forme intégrale le lien existant entre le champ électrique et le potentiel électrique.}
	
	$$ \Delta V = - \int \limits_{i \to f} \E \cdot d \vec l $$
	
	\quest{20}{78}{Donnez l'expression du potentiel électrique d'une charge ponctuelle (potentiel coulombien).}
	
	$$ V(r) = \dfrac{q}{4\pi\varepsilon_0}\dfrac{1}{r} $$
	
	\quest{21}{88}{Donnez l'approximation du premier ordre en $x$ de la fonction $(1+x)^n$, où $x$ prend une valeur beaucoup plus petite que l'unité.}
	
	$$ (1+x)^n \approx 1+nx $$
	
	\quest{22}{95}{Donnez la définition du moment dipolaire d'une distribution de charge continue quelconque.}
	
	$$ \vec p \equiv \int \rho (\vec x') \vec x' dV' $$
	
	\quest{23}{102}{Donnez l'expression du moment dipolaire électrique d'un système de deux charges ponctuelles opposées $q$ et $-q$ séparées de la distance $d$.}
	
	Plaçons un axe $x$ passant par les deux charges, ayant pour origine le point milieu au deux charges. Inutile de calculer l'intégrale du moment car nous savons que $\frac{\vec p}{q}$ est donné par la différence des vecteurs centres de charges positives et négatives : $ \frac{\vec p}{q} = \langle \vec x_n^+ \rangle - \langle \vec x_n^- \rangle $
	
	$$ \vec p = +q\dfrac{d}{2}\vec 1_x -q(-\dfrac{d}{2})\vec 1_x = qd\,\vec 1_x $$
	
	\quest{24}{105}{Donnez l'expression du moment de force exercé sur une distribution de charge de moment dipolaire $\vec p$ plongée dans un champ électrique uniforme $\E$.}
	
	$$ \vec \tau = \vec p \times \E $$
	
	\quest{25}{108}{Donnez sous forme locale le lien qui existe entre la distribution de potentiel électrique et le champ électrique généré par une distribution de charges quelconque.}
	
	$$ \E = -\text{grad}\,V = -\vec\nabla V = - \dfrac{\partial V}{\partial x} \vec 1_x - \dfrac{\partial V}{\partial y} \vec 1_y - \dfrac{\partial V}{\partial z} \vec 1_z $$
	
	\quest{26}{124}{Donnez l'expression de la capacité électrique d'une sphère conductrice de rayon $R$ isolée dans l'espace.}
	
	$$ C_{\text{sph\`ere}} = 4\pi\varepsilon_0 R$$
	
	\quest{27}{126}{Donnez les unités de la capacité électrique en termes des unités de base du SI (m, kg, s, A, K, mol), sachant que $ 1C = 1A \times 1s$.}
	
	$$ [C] = F \equiv \dfrac{C}{V} = \dfrac{C}{J\times C^{-1}} = \dfrac{C^2}{kg\times m^2\times s^{-2}} = \dfrac{A^2\times s^4}{kg\times m^2} $$
	
	\quest{28}{133}{Donnez l'expression de la capacité électrique d'un condensateur plan vide et donnez la signification physique des grandeurs qui y apparaissent.}
	
	$$ C_{\text{cond.\ plan}} = \dfrac{\varepsilon_0 S}{e} $$ où $S$ est la surface du condensateur, $\varepsilon_0$ la permittivité du vide et $e$ la distance séparant les deux plaques du condensateur.
	
	\quest{29}{145}{Donnez la loi d'association de deux condensateurs montés en parallèle.}
	
	$$ C_{\text{tot}} = C_1 + C_2 $$
	
	\quest{30}{147}{Donnez la loi d'association de deux condensateurs montés en séries.}
	
	$$ C_{\text{tot}}^{-1} = C_1^{-1} + C_2^{-1} $$
	

	
	\quest{31}{155}{Donnez le lien entre la densité de courant électrique et le courant électrique circulant dans un fil conducteur.}
	
	$$ I = \int \limits_{S} \vec J \cdot d\vec S $$
	
	\quest{32}{155}{Donnez les unités de la densité de courant électrique.}
	
	$$ [\vec J] = \dfrac{A}{m^2}$$
	
	\quest{33}{158}{Donnez la définition de la mobilité électronique.}
	
	\vspace{4cm}
	
	\quest{34}{158}{Donnez les unités de la mobilité électronique en termes des unités de bases du SI (m, kg, s, A, K, mol), sachant que $ 1C = 1A \times 1s$.}
	
	\vspace{3.5cm}
	
	\quest{35}{159}{Donnez la forme locale de la loi d'Ohm.}
	
	$$ \vec J = \sigma_e \E $$
	
	\quest{36}{160}{Donnez l'expression de la résistance $R$ d'un fil conducteur en fonction de sa longueur, de sa section et de sa conductivité.}
	
	$$ R \equiv \dfrac{L}{\sigma_e S} $$
	
	\quest{37}{167}{Donnez l'expression de la puissance dissipée dans un fil conducteur véhiculant un courant $I$ sous une différence de potentiel $V$}
	
	$$ P = R I^2 = \dfrac{V^2}{R}$$
	\newpage
	\subsection{Exercices préparatoires}
\exerc{A}{a) À faire sur papier}
	
	\textbf{b) Donnez l'expression analytique des forces $\vec F_{12}$ et $\vec F_{13}$.}
	
	$$\vec F_{12} = k_0 \dfrac{q_1q_2}{r_{12}^2} \vec 1_{r_{12}}\ \text{et}\ \vec F_{13} = k_0 \dfrac{q_1q_3}{r_{13}^2} \vec 1_{r_{13}}$$
	
	\textbf{c) Dans ces expressions, à quoi correspondent les vecteurs $\vec 1_{r_{12}}$ et $\vec 1_{r_{13}}$ étant donné le système d'axes imposé ?}
	
	$$\vec 1_{r_{12}} = - \vec 1_y\ \text{et}\ \vec 1_{r_{13}} = \vec 1_x$$
	
	\textbf{d) Déterminez l'expression analytique des composantes de la force résultante exercée sur $q_1$.}
	
	$$\vec F_1 = k_0 q_1 \left( \dfrac{q_3}{r_{13}^2} \vec 1_x - \dfrac{q_2}{r_{12}^2} \vec 1_y \right)$$
	
	\textbf{e) Déduisez-en la norme de cette résultante. Représentez-la sur le shéma.}
	
	$$||\vec F_1|| = \sqrt{F_x^2 + F_y^2} = k_0 q_1 \sqrt{ \left( \dfrac{q_3}{r_{13}^2} \right)^2 + \left(  \dfrac{q_2}{r_{12}^2} \right)^2 }$$
	
	\textbf{f) Application numérique : $q_1 = -1\ \text{C}$, $q_2 = -0,9\ \text{C}$ et $q_3 = 1,6\ \text{C}$. Les axes $x$ et $y$ sont gradués en cm.}

%%%%%%%%%%%%%%%%%%%%%%%%%%%%
\comment{\vspace{0,5cm}

\exerc{B}{a) À faire sur papier}
	
	\textbf{b) Donnez l'expression analytique des ces champs et exprimez ces vecteurs dans le système d'axe $Oxy$.}
	
	\textbf{c) Déterminez l'expression analytique des composantes du champ résultant $\E (\vec x_1)$ apparaissant à l'endroit où se trouve $q_1$.}
	
	\textbf{d) Déduisez-en la norme de ce champ.}
	
	\textbf{e) Déduisez de cette valeur la norme de la force résultante exercée sur $q_1$. Comparez avec l'exercice A.}
	
	\textbf{f) Application numérique : $q_1 = -1\ \text{C}$, $q_2 = -0,9\ \text{C}$ et $q_3 = 1,6\ \text{C}$. Les axes $x$ et $y$ sont gradués en cm.}}
%%%%%%%%%%%%%%%%%%%%%%%%%%%%

\vspace{0,5cm}

\exerc{C}{Par application de la loi de Gauss, démontrez l'expression du champ électrique généré par une charge ponctuelle $Q$ à une distance $r$ de celle-ci.}
	
	Nous utiliserons comme surface de Gauss une sphère centrée sur $Q$ et de rayon $r$.
	
	Le champ est partout parallèle au vecteur radial $\vec 1_r$. Nous avons donc : $$ \Phi = \oint  \limits_S \E \cdot d \vec S = E \times 4\pi r^2 $$
	
	Nous savons aussi que : $$ \Phi = \dfrac{1}{\varepsilon_0} \int \limits_{V_S} \rho (\vec x) \, dV = \dfrac{Q}{\varepsilon_0} $$
	
	Nous obtenons donc : $$ E = \dfrac{Q}{4\pi \varepsilon_0 r^2} \Rightarrow \vec E = \dfrac{Q}{4\pi \varepsilon_0 r^2} \vec 1_r $$

\vspace{0,5cm}

\exerc{D}{Quelle est la densité surfacique de charge $\sigma$ d'une sphère de rayon $R = 1\mathrm{cm}$ qui porte à sa surface une charge totale $Q = 10\mathrm{mC}$ ?}
	
	$$ \sigma = \dfrac{Q}{S} = \dfrac{Q}{4\pi R^2} = \dfrac{10\times 10^{-3}}{4\pi (10^{-2})^2} \dfrac{\mathrm{C}}{\mathrm{m}^2} = \dfrac{1}{4\pi}\ \dfrac{\mathrm{C}}{\mathrm{m}^2} \simeq 0,08\,\dfrac{\mathrm{C}}{\mathrm{m}^2}$$

\vspace{0,5cm}
		
		
\newpage
\renewcommand{\quest}[4]{#1. \textbf{#4} (p.#2, niv. #3)}
\section{Magnétostatique}

\quest{1}{20}{3}{Sur base du principe de conservation de l’énergie, démontrez qu’une particule libre chargée plongée dans un champ magnétique constant se déplace à vitesse constante (en module).}

$$\Delta E_C = W = \int \vec f_M \cdot {\rm d}\vec l$$

On sait que ${\rm d}\vec l = \vec v . dt$

Comme la force de Lorentz est à tout moment perpendiculaire à la vitesse de la particule on en déduit que $\vec f_M \cdot \vec v = 0$

On as donc $\Delta E_C = 0$, ce qui nous indique bien que $\vec v$ ne varie pas \emph{en module} car il ne peut y avoir variation du module de la vitesse sans variation d'énergie cinétique.

\quest{2}{21}{1}{Quelle est l’expression analytique du rayon de courbure $R$ de la trajectoire d’une particule de masse $m$ et de charge $q$ se déplaçant à la vitesse $v$ perpendiculairement à un champ magnétique uniforme $B$ ?}

$$ R = \dfrac{mv}{qB}$$

\quest{3}{21-23}{3}{Expliquez en quelques mots (2 lignes maximum) la fonction et le principe de fonctionnement du spectromètre de masse à l’aide d’un schéma, d’une formule et de quelques mots clés.}

Le spectromètre de masse est utilisé pour mesurer la masse d'une particule chargé. La particule passe dans un champ magnétique uniforme, subit la force de Lorentz et est déviée de sa trajectoire. La mesure du rayon de courbure permet alors de connaitre la masse de la particule.

\quest{4}{24}{1}{Donnez les unités de la perméabilité du vide dans le système d’unités international.}

$$ [\mu_0] = \dfrac{\text{kg}\times \text{m}}{\text{A}^2\times \text{s}^2}  $$

\quest{5}{24-25}{2}{Quel est le module $B$ du champ magnétique en un point situé à une distance d’un mètre d’un fil rectiligne qui véhicule un courant d’un ampère (précisez la formule de base utilisée) ?}

$$ B = \dfrac{\mu_0}{2\pi} \dfrac{I}{r} = \dfrac{\mu_0}{2\pi} = 2\times 10^{-7}\ \frac{\text{T}\times \text{m}}{\text{A}}$$

\quest{6}{25}{2}{Quelle est la force $F$ qui s’exerce entre deux fils rectilignes parallèles d’un mètre de long, séparés d’un mètre et véhiculant un courant d’un ampère (précisez la formule de base utilisée) ?}

$$ F = I_2 l \dfrac{\mu_0}{2\pi} \dfrac{I_1}{r} = 2\times 10^{-7}\ \text{N}$$

Formules de base : $||\vec F_M|| = IlB\sin \theta$ et $||\vec B|| = \dfrac{\mu_0}{2\pi} \dfrac{I}{r}$

\quest{7}{25}{2}{Donnez l’expression analytique du module $F$ de la force qui s’exerce entre deux fils rectilignes parallèles de longueur $L$ séparés de la distance $d$ et véhiculant des courants $I$ identiques (précisez la formule de base utilisée).}

$$ F = L \dfrac{\mu_0}{2\pi} \dfrac{I^2}{d}$$

Formules de base : voir Q6

\quest{8}{31}{1}{Donnez l’expression mathématique de la loi d’Ampère pour un circuit ampérien ouvert.}

$$ \int \limits_{i\rightarrow f} \vec B \cdot {\rm d}\vec l = \dfrac{\mu_0}{2\pi} I (\theta_f - \theta_i) = \dfrac{\mu_0}{2\pi} I \Delta \theta $$

\quest{9}{34}{3}{Donnez l’expression mathématique de la loi d’Ampère pour un circuit ampérien fermé et expliquez en quelques mots et/ou à l’aide d’un schéma la convention de signe de la loi d’Ampère.}

$$ \oint \vec B \cdot {\rm d}\vec l = \mu_0 I $$

Règle de la main droite : Quand le pouce est dans le sens de $I$, le sens de replis des doigts donne le sens \emph{positif} de parcours.

\quest{10}{36-37}{2}{Écrivez la loi de Biot et Savart appliquée à un élément infinitésimal de courant. Définissez toutes les grandeurs qui y apparaissent à l’aide d’un schéma explicatif.}

$$ {\rm d}\vec B = \dfrac{\mu_0}{4\pi r^2} I {\rm d}\vec l \times \vec 1_r $$

où ${\rm d}\vec B$ est l'élément de champ infinitésimal, $\mu_0$ la perméabilité du vide, $r$ la distance du point de calcul, $I$ l'intensité du courant, $\dif\vec l$ l'élément de longueur infinitésimal et $\vec 1_r$ le vecteur unitaire radial.

\

\quest{11}{42}{1}{Donnez l’expression du module de la densité de courant de surface d’une paroi de courant de largeur $L$ véhiculant un courant $I$.}

$$ J_S = \dfrac{N I}{L}$$

\quest{12}{42}{1}{Donnez les unités de la densité de courant de surface $J_S$.}

$$ [J_S] = \dfrac{\text{A}}{\text{m}} $$

\quest{13}{42}{1}{Donnez l’expression du module du champ magnétique émis par une paroi de courant d’extension infinie véhiculant une densité de courant de surface $J_S$.}

$$ ||\vec B|| = \dfrac{\mu_0}{2} J_S $$

\quest{14}{43}{3}{Donnez l’expression du module du champ magnétique au centre d’une spire circulaire de rayon $R$ véhiculant un courant $I$ (précisez la formule de base utilisée).}

Formule de base : Q10

Ici : ${\rm d}\vec l = R \dif\theta \vec 1_\theta$, donc : ${\rm d}\vec l\times \vec 1_r = R\dif\theta\vec 1_z$

D'où : ${\rm d}\vec B(0) = \dfrac{\mu_0}{4\pi R^2} I R \dif\theta \vec 1_z$. Il ne reste plus qu'à intégrer sur le tour complet pour obtenir : $$\vec B(0) = \dfrac{\mu_0}{2R} I \vec 1_z$$

\quest{15}{58-60}{1}{Donnez l’expression du module du moment de force qui s’exerce sur une spire plane de surface $S$ véhiculant un courant $I$ lorsque celle-ci est plongée dans un champ magnétique uniforme $B$ faisant un angle quelconque $\phi$ avec le vecteur normal au plan de la spire.}

$$ ||\vec \tau|| = ISB\sin\phi $$

\quest{16}{60}{1}{Donnez l’expression vectorielle du moment de force qui s’exerce sur une spire plane de surface $S$ véhiculant un courant $I$ lorsque celle-ci est plongée dans un champ magnétique uniforme $B$ faisant un angle quelconque avec le vecteur $\vec 1_S$ normal au plan de la spire.}

$$ \vec \tau = IS\vec 1_S\times\vec B = I\vec S\times\vec B$$

\quest{17}{60}{1}{Soit une spire plane de surface $S$ véhiculant un courant $I$. Donnez l’expression du module du moment de force agissant sur cette spire lorsqu’elle est plongée dans un champ magnétique uniforme parallèle au plan de la spire (précisez la formule de base utilisée).}

$$ ||\vec \tau|| = ISB\sin\frac{\pi}{2} = ISB $$

\

\quest{18}{60}{1}{Soit une spire plane de surface $S$ véhiculant un courant $I$. Donnez l’expression du module du moment de force agissant sur cette spire lorsqu’elle est plongée dans un champ magnétique uniforme $B$ perpendiculaire au plan de la spire (précisez la formule de base utilisée).}

$$ ||\vec \tau|| = ISB\sin0 = 0 $$

\quest{19}{63-65}{2}{Expliquez en quelques mots (style télégraphique, schéma) pourquoi les matériaux faits d’atomes possédant un nombre impair d’électrons sont paramagnétiques (matériaux attirés par les aimants).}

Électron non apparié $\rightarrow$ Moment magnétique de l'atome non nul $\rightarrow$ Moment magnétique total nul car tous orrientés différemment $\rightarrow$ Si champ alors moment de force $\rightarrow$ Les atomes "s'alignent" $\rightarrow$ Moment magnétique total non-nul $\rightarrow$ Champ induit

\quest{20}{63-65}{2}{Expliquez en quelques mots (style télégraphique, schéma) pourquoi les matériaux faits d’atomes possédant un nombre pair d’électrons ne sont pas paramagnétiques (matériaux pas attirés par les aimants).}

Aucun électron non-apparié $\rightarrow$ Moment magnétique de l'atome nul $\rightarrow$ Moment magnétique total toujours nul

\quest{21}{66}{2}{Expliquez en quelques mots (style télégraphique, schéma) la différence entre le paramagnétisme et le ferromagnétisme.}\\\\
\renewcommand{\arraystretch}{1.8}
\begin{tabular}{|p{7.55cm}|p{7.55cm}|}
\hline
 Paramagnétisme & Ferromagnétisme \\
 \hline
 Moments magnétiques (atomes) alignés  ssi le matériau est plongé dans un champ magnétique &  Moments magnétiques (atomes) alignés initialement sur des domaines du matériau\\
 \hline
 Disparition de l'effet d'aimantation (moments magnétiques plus alignés) après l'annulation du champ magnétique (dans lequel il a précédemment été  plongé) &  Conservation de l'effet d'aimantation (moments magnétiques aligné) après l'annulation du champ magnétique (dans lequel il a précédemment été  plongé) \\
 \hline
\end{tabular}\\

\quest{22}{81}{2}{Donnez l’expression des trois composantes du rotationnel d’une fonction vectorielle $\vec F$ décrite dans un repère cartésien orthonormé.}

$$ \nabla \times \vec F = \left(\frac{\partial F_z}{\partial y} - \frac{\partial F_y}{\partial z}\right)\vec 1_x + \left(\frac{\partial F_x}{\partial z} - \frac{\partial F_z}{\partial x}\right)\vec 1_y + \left(\frac{\partial F_y}{\partial x} - \frac{\partial F_x}{\partial y}\right)\vec 1_z $$

NB : $ \nabla \times \vec F = \begin{array}{|ccc|}
\vec 1_x & \vec 1_y & \vec  1_z \\ 
\partial_x & \partial_y & \partial_z \\ 
F_x & F_y & F_z
\end{array} $

\quest{23}{84}{2}{Définissez le rotationnel d’une fonction vectorielle à l’aide de son intégrale de circulation sur un circuit $C$ fermé élémentaire (de surface $\Delta S_C$ tendant vers zéro). Faites un schéma permettant de comprendre la signification des grandeurs qui y apparaissent.}

$$ \rot\ \vec B \cdot \vec 1_C \equiv \lim_{\Delta S_c. \rightarrow 0} \oint_{C} \dfrac{\vec B \cdot {\rm d}\vec l}{\Delta S_C} $$

\quest{24}{87-88}{2}{Écrivez l’expression mathématique du théorème de Stokes.}

$$ \oint_C \vec F \cdot {\rm d}\vec l = \iint_{S_C} \rot\ \vec F \cdot {\rm d} \vec S $$



\quest{25}{89}{2}{Définissez la divergence d’une fonction vectorielle $\vec F$ à l’aide de son intégrale de flux sur une surface fermée $S$ élémentaire (de volume intérieur $\Delta V_S$ tendant vers zéro). Faites un schéma permettant de comprendre la signification des grandeurs qui y apparaissent.}

$$ \divv\ \vec F \equiv \lim_{\Delta V_S \rightarrow 0} \int\!\!\!\!\!\!\!\subset\!\!\!\supset\!\!\!\!\!\!\!\int_{S} \dfrac{\vec F \cdot {\rm d} \vec S}{\Delta V_S}$$

\quest{26}{89-91}{2}{Écrivez l’expression mathématique du théorème d’Ostrogradski.}

$$\int\!\!\!\!\!\!\!\subset\!\!\!\supset\!\!\!\!\!\!\!\int_{S_V} \vec F \cdot {\rm d} \vec S = \iiint_{V} \divv \ \vec F \cdot {\rm d}V$$

\quest{27}{91-92}{3}{Démontrez que le rotationnel d’un champ électrostatique est nul en toute circonstance.}

$$ \oint_{C}\vec E\cdot\dif\vec l \equiv 0 \overset{\text{Stokes}}{\Longrightarrow}\iint_{S_C}\rot\,\vec E\cdot\dif\vec S = 0$$

On peut faire $C,S_C\rightarrow 0$ pour tout point $\vec x$ de l'espace, ce qui nous donne $\rot\,\vec E=0$.

\quest{28}{93}{3}{Démontrez que la divergence du rotationnel de toute fonction vectorielle $\vec F$ est nulle.}

$$\lim_{C\rightarrow0}\oint_C\vec F\cdot\dif\vec l\ \ \overset{\text{Stokes}}{=} \iint_{S_C}\rot\,\vec F\cdot\dif\vec S = 0 \Rightarrow \int\!\!\!\!\!\!\!\subset\!\!\!\supset\!\!\!\!\!\!\!\int_{S}\rot\,\vec F\cdot\dif\vec S = 0 $$

$$ \Rightarrow \int\!\!\!\!\!\!\!\subset\!\!\!\supset\!\!\!\!\!\!\!\int_{S}\rot\,\vec F\cdot\dif\vec S\ \ \overset{\text{Ostrogradski}}{=} \iiint_{V_S} \divv(\rot\,\vec F)\,\dif V = 0 $$

On peut faire $\Delta V_S \rightarrow 0$ pour tout point $\vec x$ de l'espace, ce qui nous donne $\divv(\rot\,\vec F)=0$.

\quest{29}{94-98}{2}{Donnez l’expression mathématique de la loi de Gauss du champ magnétique sous sa forme intégrale et locale.}

Forme intégrale : $$ \int\!\!\!\!\!\!\!\subset\!\!\!\supset\!\!\!\!\!\!\!\int_{S} \vec B \cdot {\rm d} \vec S = 0$$

Forme locale : $$ \divv\,\vec B = 0$$






	\subsection{Exercices préparatoires}
\begin{center}
\includegraphics[scale=0.85]{prepa/magneto1.png}\\
\end{center}
$$\vec{f}_M = q\vec{v} \times \vec{B}$$
\begin{center}
Il suffit de calculer le produit vectoriel.
\end{center}


\begin{center}
\includegraphics[scale=0.85]{prepa/magneto2.png}\\
\end{center}

$$\oint_C \vec{B}.\vec{dl} = \mu_0 \sum_i I_i$$
$$\vec{B} = \frac{\mu_0}{2\pi r}\left(I_V + I_N + I_B - I_R\right)\ \vec{1_\theta}$$




\newpage
\begin{center}
\includegraphics[scale=0.85]{prepa/magneto3.png}\\
\end{center}
\begin{center}
Les deux fils véhiculent un courant identique, mais opposé.
\end{center}
$$\vec{B} = \vec{0}$$

\begin{center}
\includegraphics[scale=0.85]{prepa/magneto4.png}\\
\end{center}
$$B=\frac{\mu_0}{4\pi}\int \frac{I\,\vec{dl}\times\vec{1_r}}{r^2}$$Ici $\vec{dl}=\vec{dz}$ et $\vec{dz}\times\vec{1_r}=\|\vec{dz}\| \|\vec{1_r}\|\sin(\varphi)=dz\sin(\varphi)$\\
Il faut trouver $dz$ en fonction de $\varphi$  $$\Rightarrow r=\frac{d}{\sin(\varphi)}\text{ et } \tan(\varphi)=\frac{d}{z^*-z}\Leftrightarrow dz=\frac{d}{\sin^2(\varphi)}$$
$$\Rightarrow B = \frac{\mu_0}{4\pi}\int_{\arccot\left(\frac{z^*+L/2}{d}\right)}^{\arccot\left(\frac{z^*-L/2}{d}\right)}\frac{I}{d}\sin^2(\varphi)$$
%CHELSEA FC !
		


\newpage
\section{Électromagnétisme}
	\subsection{Théorie}

\quest{1}{8}{1}{Définissez mathématiquement le champ électrique induit à l’aide d’un schéma permettant de comprendre la signification physique des grandeurs qui y apparaissent.}

$$ \vec E_i = \vec v\times \vec B$$

\quest{2}{14}{1}{Définissez mathématiquement la force électromotrice provoquée par le phénomène d’induction électromagnétique.}

$$ \mathcal{E} \equiv \int \limits_{i \to f} \vec E_i \cdot \dif\vec l $$

\quest{3}{29}{2}{Donnez l’expression mathématique générale de la loi de Faraday en termes des champs électrostatique et magnétique. Expliquez en quelques mots et à l’aide d’un schéma la convention de signe et de sens des vecteurs de la loi de Faraday.}

$$ \oint_{C(t)}\vec E_i(\vec x,t)\cdot\dif\vec l = -\dfrac{\dif}{\dif t}\iint_{S_C(t)}\vec B(\vec x,t)\cdot\dif\vec S $$

\quest{4}{36}{1}{Écrivez la forme locale de la loi de Faraday.}

$$ \rot\vec E = -\dfrac{\partial\vec B}{\partial t}$$
\
\quest{5}{63}{1}{Donnez l’expression mathématique de l’énergie électromagnétique stockée dans un inducteur.}

$$ W_m = \dfrac{LI^2}{2} $$

\quest{6}{68}{2}{Donnez l’expression analytique du courant $I(t)$ circulant dans un inducteur d’inductance $L$ soumis à une différence de potentiel constante $V$ (précisez la formule de base utilisée).}

$ V = L\frac{\dif I}{\dif t} $ est l'équation du circuit muni d'un inducteur et d'une source de tension contante. En résolvant cette équation différentielle on obtient : $$ I(t) = \dfrac{V}{L}t + I_0 $$



\quest{7}{69-71}{2}{Écrivez la loi courant-tension de l’inducteur et illustrez-la à l’aide d’un schéma permettant de comprendre la convention de signe associée au courant et à la tension.}

$$ V(t) = L\,\dfrac{\dif I(t)}{\dif t} $$


\quest{8}{70}{2}{Donnez l’expression analytique du courant harmonique $I(t)$ circulant dans un inducteur d’inductance $L$ soumis à une différence de potentiel harmonique\\$V(t) = V_m \sin (\omega t)$ (précisez la formule de base utilisée).}


$ \frac{\dif I}{\dif t} = \frac{V_m\sin (\omega t)}{L} $ (voir Q7) et donc : $$ I(t) = -\dfrac{1}{\omega L}\, V_m\cos (\omega t) $$


\quest{9}{82}{2}{Donnez, en termes de la résistance $R$ et du courant efficace $I_{\text{eff}}$, l’expression de la puissance dissipée dans une résistance $R$ véhiculant un courant harmonique $I(t) = I_m \sin (\omega t)$ (définissez au préalable le courant efficace).}

$ I_{\text{eff}}\equiv \dfrac{I_m}{\sqrt{2}} \left( = {\langle I^2 \rangle}^{1/2} \right) $ $$ \langle P \rangle = RI_{\text{eff}}^2 $$

\quest{10}{83}{2}{Donnez, en termes de la résistance $R$ et la tension efficace $V_{\text{eff}}$, l’expression de la puissance dissipée dans une résistance $R$ soumise à la tension harmonique $V(t) = V_m \sin (\omega t)$ (définissez au préalable la tension efficace).}

$ V_{\text{eff}} \equiv \dfrac{V_m}{\sqrt{2}} $ $$ \langle P \rangle = \dfrac{V_{\text{eff}}^2}{R} $$
\
\quest{11}{83}{2}{Donnez, en termes des courant et tension efficaces $I_{\text{eff}}$ et $V_{\text{eff}}$, l’expression de la puissance dissipée dans un circuit purement résistif alimenté en courant alternatif (définissez au préalable le courant et la tension efficaces).}

Voir Q9 \& Q10.
$$ \langle P \rangle = I_{\text{eff}} V_{\text{eff}} $$

\quest{12}{84-85}{3}{Démontrez que dans un inducteur soumis à la tension harmonique $V(t) = V_m \sin (\omega t)$ le courant est en retard d’un quart de période sur la tension.}

$ I(t) = -\frac{1}{\omega L}\, V_m\cos (\omega t)$ (voir Q8)

En définissant l'amplitude du courant $I_m = \frac{V_m}{\omega L}$ et en utilisant la relation trigonométrique de base $\cos (x) = -\sin (x-\frac{\pi}{2})$ on obtient : $$ I(t) =  I_m\sin \left(\omega [t-\dfrac{\pi}{2\omega}]\right) $$ Ce qui montre que la variation périodique de courant est en retard d'un quart de période sur la variation périodique de la tension (Rappel : $T=\frac{2\pi}{\omega}$).

\quest{13}{86}{3}{Soit un inducteur soumis à la tension harmonique $V(t) = V_m \sin (\omega t)$. Démontrez que la puissance moyenne délivrée à cet inducteur est toujours nulle.}

$ I(t) = -I_m\cos (\omega t)$ (voir Q8 \& Q12) et donc : $$ P(t) = I(t)V(t) = -I_m\cos (\omega t)V_m\sin (\omega t) = -I_mV_m\dfrac{\sin (2\omega t)}{2} \overset{*}{\Longrightarrow} \langle P(t)\rangle = 0 $$

* : La puissance prend autant de valeurs positives que de valeurs négatives. Sa moyenne temporelle est donc nulle.

\quest{14}{88}{2}{Donnez l’expression de la réactance d’un inducteur d’inductance $L$ alimenté en tension alternative de pulsation $\omega$ (précisez la formule de base utilisée).}

$I_{\text{eff}} = \frac{I_m}{\sqrt{2}}$, $V_{\text{eff}} = \frac{V_m}{\sqrt{2}}$ et $I_m = \frac{V_m}{\omega L} $ donc : $I_{\text{eff}} = \dfrac{V_{\text{eff}}}{\omega L}$.

On voit que le produit $\omega L$ joue le rôle d'une "résistance" ($I=V/R$). On l'appelle \emph{réactance}, noté : $$ X_L = \omega L $$

\
\quest{15}{90}{3}{Démontrez que dans un condensateur soumis à la tension harmonique $V(t) = V_m \sin (\omega t)$ le courant est en avance d’un quart de période sur la tension.}

$Q(t) = CV(t)$ et $I(t)=\frac{\dif Q}{\dif t} \Rightarrow I(t)=C\dfrac{\dif V}{\dif t} = CV_m\omega\cos (\omega t)$

En introduisant $I_m = C\omega V_m$ on obtient : $I(t)=I_m\cos (\omega t) = I_m\sin (\omega t+\frac{\pi}{2}) $ $$I(t)=I_m\sin \left( \omega [t+\frac{\pi}{2\omega}]\right)$$

On voit bien que le courant est cette fois en avance sur la tension d'un quart de période.

\quest{16}{91}{3}{Soit un condensateur soumis à la tension harmonique $V(t) = V_m \sin (\omega t)$. Démontrez que la puissance moyenne délivrée à ce condensateur est toujours nulle.}

$I(t)=I_m\cos (\omega t)$ (voir Q15) et donc : $$ P(t) = I(t)V(t) = I_m\cos (\omega t)V_m\sin (\omega t) = I_mV_m\dfrac{\sin (2\omega t)}{2} \overset{*}{\Longrightarrow} \langle P(t)\rangle = 0 $$

* : Voir Q13

\quest{17}{92}{2}{Donnez l’expression de la réactance d’un condensateur de capacité $C$ alimenté en tension alternative de pulsation $\omega$ (précisez la formule de base utilisée).}

$I_m = \omega CV_m \Rightarrow I_{\text{eff}} = \omega C V_{\text{eff}}$. On voit que $1/\omega C$ joue le rôle d'une "résistance" ($I=V/R$). On l'appelle \emph{réactance}, noté : $$ X_C = \dfrac{1}{\omega C} $$

\quest{18}{97}{2}{Écrivez la loi de conservation de la charge sous forme intégrale (illustrer-la à l’aide d’un schéma permettant d’identifier clairement les grandeurs physiques qui y apparaissent).}

$$ \oiint_S\vec J\cdot\dif\vec S = -\dfrac{\dif Q}{\dif t} $$

\quest{19}{99}{1}{Écrivez la loi de conservation de la charge sous forme locale (ou équation de continuité) (définissez les grandeurs physiques qui y apparaissent).}

$$ \divv\vec J + \dfrac{\partial\rho}{\partial t} = 0 $$

où $\vec J$ est la densité de courant et $\partial\rho / \partial t$ est la variation temporelle de la densité de charge.

\quest{20}{102}{1}{Donnez l’expression de la densité de courant de déplacement $\vec J_D$.}

$$ \vec J_D = \varepsilon_0\dfrac{\partial\vec E}{\partial t} $$

\quest{21}{102}{1}{Donnez l’expression de la loi d’Ampère-Maxwell sous forme intégrale.}

$$ \oint_C\vec B\cdot\dif\vec l = \mu_0\iint_{S_C}\left(\vec J+\varepsilon_0\dfrac{\partial\vec E}{\partial t}\right)\cdot \dif\vec S $$

\quest{22}{102}{1}{Donnez l’expression de la loi d’Ampère-Maxwell sous forme locale.}

$$ \rot\vec B = \mu_0\vec J+\mu_0\varepsilon_0\dfrac{\partial\vec E}{\partial t} $$

Et en introduisant le courant de déplacement : $$ \rot\vec B = \mu_0\left(\vec J+\vec J_D\right) $$

\quest{23}{104}{2}{Démontrez l’expression de la densité de courant de déplacement $J_D$ au sein d’un condensateur
vide d’épaisseur $e$ et de surface $S$ parcouru par un courant $I$.}

$\vec E(t)=\frac{\sigma (t)}{\varepsilon_0}\vec 1_S$ et $\sigma = \frac{Q(t)}{S}$ d'où $\dfrac{\dif\vec E}{\dif t} = \dfrac{1}{S\varepsilon_0}\dfrac{\dif Q(t)}{\dif t}\vec 1_S = \dfrac{I}{S\varepsilon_0}\vec 1_S$ et donc : $$ \vec J_D = \varepsilon_0\dfrac{\dif\vec E}{\dif t} = \dfrac{I}{S}\vec 1_S$$




	\subsection{Exercices préparatoires}
\begin{center}
\includegraphics[scale=1]{prepa/exoA.png}\\
\end{center}
$$\varepsilon=\oint_{i\rightarrow f}\vec{E_i}\,\vec{dl}\ \text{  et  }\  \vec{E_i}=\vec v\times\vec B$$
$$\vec v=R\omega\Rightarrow \vec{E_i}=\|\vec v\|\,\|\vec B\|\  (\vec{B} \perp \vec{v})=R\omega B$$
$$\varepsilon=\oint_{i\rightarrow f}\vec{E_i}\,\vec{dl}=\oint_0^RR\,\omega B=\frac{R^2\omega B}{2}$$
\begin{center}
\includegraphics[scale=1]{prepa/exoB.png}\\
\end{center}
\begin{align*}\oint_{i\rightarrow f}\vec{E}\,\vec{dl} & =-\frac{d}{dt}\int_{Sc}\vec B\,\vec{dS}\ \text{ où }\ \left\{\begin{array}{l}
\vec B=\lambda t\,\vec{1_z}\\
\vec{dS}=dS\,\vec{1_z}=dx\,dy\,\vec{1_z}\end{array}\right.\\
 & =-\frac{d}{dt}\int_0^l\int_0^L\lambda t\,dx\,dy=-\lambda L l\end{align*}

 \newpage
\begin{center}
\includegraphics[scale=1]{prepa/exoC.png}\\
\end{center}
\begin{align*}\oint_{i\rightarrow f}\vec{E}\,\vec{dl} & =-\frac{d}{dt}\int_{Sc}\vec B\,\vec{dS}\ \text{ où }\ \left\{\begin{array}{l}
\vec B=\lambda t\frac{x}{y}\,\vec{1_z}\\
\vec{dS}=dS\,\vec{1_z}=dx\,dy\,\vec{1_z}\end{array}\right.\\
 & =-\frac{d}{dt}\int_d^{d+l}\int_0^L\lambda t\frac{x}{y}\,dx\,dy=-\lambda\frac{L^2}{2}\ln\left(1+\frac{l}{d}\right)\end{align*}


\begin{center}
\includegraphics[scale=1]{prepa/exoE.png}\\
\end{center}
$$\vec{J}_D = \epsilon_0 \frac{d\vec{E}}{dt}$$
$$V(t) = \int \vec{E}.\vec{dl} = E.L \Rightarrow E(t) = \frac{V(t)}{e}\vec{1_x}$$
$$\vec{J}_D = 4000\pi \frac{4 \epsilon_0}{e} \cos\left(4000\pi t + \frac{\pi}{4}\right)$$

\newpage
\begin{center}
\includegraphics[scale=1]{prepa/exoD.png}\\
\end{center}
\begin{itemize}
\item[a) ] 
$$\int \vec{B}.\vec{dl} = \mu_0 N_1I$$
$$B(r) = \frac{\mu_0 N_1I}{2\pi r}
\Rightarrow \left\{\begin{array}{l}
r \in [a,b] \ \ \  B = \frac{\mu_0 N_1I}{2\pi r}\\
r < a\ \ \ \ \ \ \ B = 0\\
r > b\ \ \ \ \ \ \ B = 0\end{array}\right.\\$$


\item[b) ]
$$N_1. \phi_M(t) = \int \vec{B}\vec{dS}$$
$$\phi_M = N_2 \int B.dS\ \ \ (\vec{B}\ //\ \vec{dS})$$
$$\phi_M = \int_b B.dS = N_2 \int_b \frac{\mu_0 N_1 I}{2\pi x} dxdy$$
$$\phi = N_1 \frac{4\pi 10^{-7}}{2\pi}N_1 I h \int \frac{dx}{x}$$
$$\phi = N_1 N_2 I h\ 2.10^{-7}\ln\left(\frac{b}{a}\right)$$
\end{itemize}



\tableofcontents

\end{document}

