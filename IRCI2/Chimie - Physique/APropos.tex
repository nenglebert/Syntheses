\chapter*{Appel à contribution}
\begin{wrapfigure}[5]{l}{4.5cm}
\includegraphics[scale=0.5]{git.png}
\end{wrapfigure}
Ce document est grandement inspiré de l’excellent cours donné 
par Benoît Haut et Pierre Colinet à l’EPB (École Polytechnique de Bruxelles), faculté de l’ULB (Université 
Libre de Bruxelles). Il est écrit par les auteurs susnommés avec l’aide de tous les autres étudiants 
et votre aide est la bienvenue ! En effet, il y a toujours moyen de l’améliorer surtout que si le 
cours change, la synthèse doit être changée en conséquence. On peut retrouver le code source à l’adresse 
suivante
\begin{center}
\url{https://github.com/nenglebert/Syntheses}
\end{center}\ \\
Pour contribuer à cette synthèse, il vous suffira de créer un compte sur \textit{Github.com}. De
légères modifications (petites coquilles, orthographe, ...) peuvent directement être faites sur le
site. Pour de plus longues modifications, il est intéressant de disposer des fichiers : il vous 
faudra pour cela installer \LaTeX, mais aussi \textit{git}. Si cela pose problème, nous sommes 
évidemment ouverts à des contributeurs envoyant leur changement par mail ou n’importe quel autre 
moyen.\\
\\
Le lien donné ci-dessus contient aussi le \texttt{README} contient de plus amples informations, 
vous êtes invités à le lire si vous voulez faire avancer ce projet ! \\
\\
\\
\begin{flushright}
Merci ! 
\end{flushright}