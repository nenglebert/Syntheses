\section*{TP 3 : Maser à amoniac}
\begin{itemize}
	\item Equation de Schrödinger stationnaire à une dimension 
	      \begin{equation}
	      	H\psi = E\psi \quad \Leftrightarrow \quad \left[\frac{-\hbar ^2}{2m}\frac{d^2}{dx^2} + V(x)\right]\psi = E \psi
	      \end{equation}
	\item Pour un potentiel plus complexe de la forme 
	      \begin{equation}
	      	\left\{ 
	      	\begin{aligned}
	      		  & 0 \qquad a<   & |x| <b \\
	      		  & V_0 \qquad    & |x|<a  \\
	      		  & \infty \qquad & |x|>b  
	      	\end{aligned}
	      	\right.
	      \end{equation}
	      La résolution de l'équation de Schrödinger devra être suivie de l'application des \textbf{conditions de continuité} et des \textbf{conditions aux limites} qui, dans ce cas, serait du type
	      \begin{equation}
	      	\left\{ 
	      	\begin{aligned}
	      		\psi (x_0^+)  & = \psi (x_0^-)  \\
	      		\psi '(x_0^+) & = \psi '(x_0^-) 
	      	\end{aligned}
	      	\right.
	      	\qquad
	      	\mbox{et}
	      	\qquad 
	      	\left\{ 
	      	\begin{aligned}
	      		\psi (b)  & = 0 \\
	      		\psi (-b) & = 0 
	      	\end{aligned}
	      	\right.
	      \end{equation}
	      Ils sont ainsi les \textbf{conditions de quantification de l'énergie}.
	      		
	\item Pour obtenir l'équation de Schrödinger non-stationnaire, il suffit de multiplier par l'exponentielle
	      \begin{equation}
	      	-i\hbar \frac{d}{dt}\psi = H \psi \quad \Rightarrow \quad \phi _i(x,t) = \psi _i (x) \exp \left( \frac{-iE_i t}{\hbar}\right) 
	      \end{equation}
\end{itemize}

