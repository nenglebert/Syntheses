\section*{TP 4 : Marche de potentiel descendante et cyclotron}
Le rappel théorique est le même que celui du précédent. On doit seulement rappeler que 
\begin{itemize}
	\item Impulsion et relation de De Broglie
	      \begin{equation}
	      	p = \sqrt{2m(E-V_0)} \qquad  \vec{p} = \hbar \vec{k}
	      \end{equation}
	      		
	\item Coefficients de \textbf{réflexion} et de \textbf{transmission}\\
	      Pour une fonction de la forme
	      \begin{equation}
	      	\psi(x) = 
	      	\left\{
	      	\begin{aligned}
	      		Ae^{ikx}+Be^{-ikx} \\
	      		Ce^{ikx}+De^{-ikx} 
	      	\end{aligned}			
	      	\right.
	      	\quad			
	      	\Leftrightarrow 
	      	\quad
	      	\psi(x) = 
	      	\left\{
	      	\begin{aligned}
	      		  & e^{ikx}+Re^{-ikx} \\
	      		  & Te^{ikx}          
	      	\end{aligned}			
	      	\right.
	      \end{equation}
	      où $R$ est le coefficient de \textbf{réflexion} et $T$ le coefficient de \textbf{transmission}. Le coefficient $D$ est nul puisqu'on considère qu'il n'y a pas d'onde venant de la droite lors d'un \textbf{effet tunnel}.
	      		
	\item Le Volt 
	      \begin{equation}
	      	V = \frac{eV}{C}
	      \end{equation}
	      Il est toujours utile de savoir ça, surtout dans ce tp où on donne un voltage pour l'ion $H_2^+$. On doit donc multiplier par la charge de l'ion pour obtenir le potentiel en electron-volt. Dans ce tp, puisqu'on a un seul électron en trop, $V = eV$.
\end{itemize}