
\section*{TP 9 : Rotation et vibration des molécules diatomiques}
\begin{itemize}
		
	\item Vibration de molécules diatomiques
	      \begin{itemize}
	      				
	      	\item Approximation parabolique 
	      	      \begin{equation}
	      	      	V = \frac{1}{2}\mu \omega ^2 (R-R_0)^2
	      	      \end{equation}
	      	      					
	      	\item Energie de vibration valable à courtes distances 
	      	      \begin{equation}
	      	      	E_{n,vib} = (n+\frac{1}{2})\hbar \omega
	      	      \end{equation}
	      \end{itemize}
	      		
	\item Rotation de molécules diatomiques
	      			
	      \begin{itemize}
	      	\item Hamiltonien de rotation
	      	      \begin{equation}
	      	      	H_{rot} = \frac{L^2}{2I} \qquad \Rightarrow \qquad E_{l,rot} = \frac{\hbar ^2 l(l+1)}{2I} \qquad avec \ I = \mu r_0 ^2
	      	      \end{equation}
	      	      				
	      	\item Classification des différentes énergies
	      	      \begin{equation}
	      	      	E_{rot} < E_{vib} < Ryd
	      	      \end{equation}
	      \end{itemize}
	      			
	\item Transition dipolaire électrique $\Delta l = 1$
	      
	      \begin{itemize}
	      	\item Absorption : $l_f = l_i + 1$
	      	\item Emission : $l_f = l_i - 1$
	      	\item Les raies d'absorption et d'émission correspondent aux énergies de transition admises
	      \end{itemize}
\end{itemize}

\begin{figure}[h]
	\begin{center}
		\includegraphics[scale=0.5]{img/1}
		\caption{Spectre de vibration de la molécule $H_2$.}
	\end{center}
\end{figure}