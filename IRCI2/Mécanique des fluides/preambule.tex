%%%%%%%%%%%%%%%%
%%% Packages %%%
%%%%%%%%%%%%%%%%

%%% Général %%%
\usepackage[utf8]{inputenc}   
\usepackage[french]{babel}
\usepackage[T1]{fontenc}
\usepackage{mathpazo}
\usepackage{lmodern}
\usepackage{courier}
\usepackage{graphicx}

%%% Tableau %%%
\usepackage{tabularx} %Permet d'auto dimensionner les tableaux



%%% Bibliographie %%%
\usepackage[style=alphabetic,backend=bibtex]{biblatex}
\usepackage[autostyle]{csquotes}
\DeclareNameAlias{sortname}{last-first}
\DeclareFieldFormat{url}{\space\url{#1}}
\DeclareNameAlias{labelname}{last-first}
\addbibresource{sample.bib}


%%% Graphiques %%%
\usepackage{tikz}
\usepackage{pgfplots}
\usepackage{circuitikz}

%%% Mise en page %%%
\usepackage{amsmath}
\usepackage{amsfonts}
\usepackage{amssymb}
\usepackage{amsthm}
\usepackage[tt]{titlepic}% Centre le titre
\usepackage{fancyhdr}   % Permet de modifier l'entête & footer
\usepackage{caption}     % Permet d'ajouter des légendes en images sans les mettre en float + dans la marge
\usepackage{wrapfig}
\usepackage{fullpage}
\usepackage{multicol}   % pour les liste sur plusieurs colonnes
\usepackage{subfigure}  % alligne deux images cote a cote
\usepackage{float}      %permet de mettre du texte entre les figures grace a [H]. Génial! 
\usepackage{eso-pic}    % Fond d'écran page de garde
\usepackage{adjustbox}  % Empêche les box de sortir de la page


%%% Math %%%
\usepackage{delarray} % Belles matrices
\usepackage{siunitx}
\sisetup{locale = FR,detect-all}
% Pour mettre siunitx en mode français (virgule plutôt que point etc.)

%%% Codes %%%
\usepackage{listings}

%% Reference
\usepackage{hyperref}
\renewcommand*{\figureautorefname}{fig.}
\def\appendixautorefname{annexe}
\def\tableautorefname{tab.}
\renewcommand*{\chapterautorefname}{ch.}



%%%%%%%%%%%%%%%%%
%%% Commandes %%%
%%%%%%%%%%%%%%%%%

%%% Physque %%%
\newcommand{\cst}{\text{cst}}
\newcommand{\E}{\vec E}
\newcommand{\B}{\vec B}
\newcommand{\F}{\vec F}
\newcommand{\module}[1]{||#1||}

%%% Math %%%
\newcommand{\oiint}{\int\!\!\!\!\!\!\! \:\!\subset\!\!\supset\!\!\!\!\!\!\!\int}
\newcommand{\rot}{\overline{\text{rot}}\ }
\newcommand{\grad}{\overline{\text{grad}}\ }
\newcommand{\divv}{\text{div}\ }
\newcommand{\phas}[1]{\underline{#1}}
\newcommand{\RE}{\text{Re}}

%% Box
\newcommand{\theor}[1]{\adjustbox{minipage=\linewidth-2\fboxsep-2\fboxrule,fbox}{\textsc{Théorème : }#1}}
\newcommand{\retenir}[1]{\adjustbox{minipage=\linewidth-2\fboxsep-2\fboxrule,fbox}{\textbf{\textit{\textsc{A retenir} : }}#1}}
\newcommand{\corollaire}[1]{\ \\\begin{tabular}{|c}
    \begin{minipage}{\textwidth}
      \textsc{Corollaire : } \textit{#1}
    \end{minipage}
    \end{tabular}}
\newcommand{\prop}[1]{\adjustbox{minipage=\linewidth-2\fboxsep-2\fboxrule,fbox}{#1}}    
    

%\pagestyle{headings} % Titre du ch et numéro page dans l'entete
\renewcommand{\proofname}{Démonstration}
\selectlanguage{french}

\addto\captionsfrench{\def\tablename{Tableau}}


%%% Background %%%
\newcommand\BackgroundPic{%
\put(0,0){%
\parbox[b][\paperheight]{\paperwidth}{%
\vfill
\centering
\includegraphics[width=\paperwidth,height=\paperheight,%
keepaspectratio]{ulb.jpg}%
\vfill
}}}