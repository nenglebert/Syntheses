\chapter{Dynamique des fluides visqueux}
\section{Introduction}
Le \autoref{ch:ch5} nous avait montré que le fluide visqueux newtonien était défini par \autoref{eq:ComportementVisqNew} ou \autoref{eq:EqStokes}. Ces deux équations mettent 
en évidence les deux coefficients de viscosité $\lambda$ et $\mu$. Lorsque le fluide 
était incompressible, il y avait une relation entres ces deux coefficients de nos deux
équations : si on ajoute l'hypothèse de Stokes, ces relations sont vérifiées même si le
flide n'est pas incompressible.


	\subsection{Interprétation physique}
	Considérons $\vec{V}$ la vitesse de la plaque et $\vec{v}$ la vitesse du fluide. Le
	long de la plaque mobile $\vec{v}=\vec{V}$ : le fluide visqueux colle et est entrainé
	par la paroi, s'il était parfait ça ne serait pas le cas. Le log de la plaque fixe,
	$\vec{v}=\vec{0}$.\\
	
	On verra (spoil) que sous certaines conditions que 
	\begin{equation}
	\vec{V} = V\frac{z}{L}\vec{1_x}
	\label{eq:VitCh10}
	\end{equation}
	Reprenons la loi de comportement \autoref{eq:EqStokes} :
	\begin{equation}
	\tau_{ij} = -p\delta_{ij} + 2\mu\left(V_{ij} - \frac{1}{3}\delta_{ij}V_{kk}\right)
	\end{equation}
	où la vitesse est donnée par \autoref{eq:VitCh10}. Nous avons donc (où $w$, la vitesse
	selon $z$ est nulle (vitesse est toujours horizontale)) :
	\begin{equation}
	\begin{array}{ccccc}
	\tau_{xz} = 2\mu V_{xz} &\Leftrightarrow & \tau_{xz} = 2\mu\dfrac{1}{2}\left(\dfrac{
	\partial u}{\partial z}+\underbrace{\dfrac{\partial w}{\partial x}}_{=0}\right) &
	\Leftrightarrow & \tau_{xz}	= \mu \dfrac{\partial u}{\partial z}
	\end{array}
	\end{equation}
	où $\tau_{xz}$ est la contrainte tangentielle de frottement entre les filets fluides
	voisins, due à la différence de vitesse de ces filets fluides.
	
	\subsection{Unités}
	\textbf{Inclure tableau slide 7}
	
\section{Equation de mouvement}
	\subsection{Equation de Navier-Stokes}
	La démarche est identique que celle suivie pour établir les équations d'Euler pour un
	fluide parfait. Partons des équations générales du mouvement
	\begin{equation}
	\rho v_i^\bullet = f_i + \partial_j \tau_{ij}
	\end{equation}		
	Notre loi de comportement est toujours donnée par l'équation de Stokes 
	\autoref{eq:EqStokes}. La vitesse de déformation est donnée par 
	\begin{equation}
	V_{ij} = \frac{1}{2}(v_{i,j} + v_{j,i})
	\end{equation}
	Le fluide étant incompressible, $\rho^\bullet = 0$:
	\begin{equation}
	\rho^\bullet + \rho \partial_iv_i = 0 \rightarrow V_{ii} = 0 \rightarrow V_{kk}=0
	\end{equation}
	On peut alors se débarasser de $V_{kk}$ dans \autoref{eq:EqStokes}.\\
	Introduisons la définition de $\tau_{ij}$ dans les équations du mouvement : 
	\begin{equation}
	\rho v_i^\bullet = f_i + \partial_j[-p\delta_{ij} + 2\mu V_{ij}]
	\end{equation}
	En tenant compte de la définition des vitesses de déformations :
	\begin{equation}
	\rho v_i^\bullet = f_i-\partial_i p + \mu\partial_j(\partial_iv_j+\partial_jv_i)
	\end{equation}
	En effectuant la distribution, compte-tenu du résultat que nous venons d'obtenir 
	($\rho\partial_iv_i=0$), le terme $\mu\partial_i\partial_j v_j$ s'annule pour donner
	:\\
	
	\prop{\textsc{Equations de Navier-Stokes}
	\begin{equation}
	\rho v_i^\bullet = f_i - \partial_ip + \mu\partial_j\partial_jv_i
	\end{equation}}
	
		\subsubsection{Exemple : écoulement entre deux plaques parallèles}
		Commençons par énoncer ce que nosu avons : 
		\begin{enumerate}
		\item Ecoulement permanent $\rightarrow \frac{\partial}{\partial t}=0$
		\item Fluide incompressible :$\rho^\bullet=0 \rightarrow \rho^\bullet + \rho
		\divv\vec{v}=0 \rightarrow \divv \vec{v}=0$
		\item Fluide visqueux $\rightarrow$ équation de Navier-Stokes
		\item Ecoulement unidimensionnel $\rightarrow \vec{v} = (u,0,0)$
		\end{enumerate}				
		Comme nous avons $\divv \vec v=0$ où $\vec{v}=(u,0,0)$, cela implique :
		\begin{equation}
		\frac{\partial y}{\partial x} = 0 \Rightarrow u = u(z)
		\end{equation}
		Comme nous sommes en stationnaire et en unidimensionnel, les équations 
		de Stokes développées se simplifient fortement :
		\begin{equation}
		\left\{\begin{array}{ll}
		\mu \frac{\partial^2u}{\partial z^2} &= \frac{\partial \hat{p}}{\partial x}\\
		0 &= \frac{\partial \hat{p}}{\partial y}\\
		0 &= \frac{\partial \hat{p}}{\partial z}
		\end{array}\right.
		\end{equation}
		Imposons les conditions aux limites $u(0)=0, u(L)=V$ et considérons la première 
		équation de notre système : le premier membre est fonction de $z$ et le second de
		$x$ impliquant que les deux fonctions doivent être constantes.
		\begin{equation}
		\frac{d\hat{p}}{dx}= 0 \Rightarrow \hat{p}=ax+b
		\end{equation}
		Deux cas s'offrent à nous : 
		\begin{enumerate}
		\item $a=0$\\
		\begin{equation}
		u = V\frac{z}{L}\ \ \ ;\ \ \ v=0;\ \ \ w=0
		\end{equation}
		\item $a\neq0$\\
		\begin{equation}
		u = \frac{az}{2\mu}(z-L) + V\frac{z}{L}\ \ \ ;\ \ \ v=0;\ \ \ w=0
		\end{equation}
		\end{enumerate}
		\textbf{Inclure schéma slide 25} et interprétation.
		
		
	\subsection{Pression motrice}
	Fréquemment, la force de volume est la pensenteur. En définitssant $z$ comme l'axe
	vertical positif vers le haut :
	\begin{equation}
	\begin{array}{ccc}
	\rho\vec{F}=-\rho g\vec{1_z}, & \vec{F}=-\grad U, & U=gz
	\end{array}
	\end{equation}
	L'équation du mouvement s'écrit 
	\begin{equation}
	\rho v_i^\bullet = \rho F_i - \partial_i p +\mu\partial_j\partial_jv_i = -\grad\hat{p}
	+\mu\Delta\ \vec v
	\end{equation}
	où l'on utilise la pression motrice $\hat{p}=p+\rho gz$.\\
	Un bel exemple sur l'\textit{écoulement unidimensionnel} est donné slide 27-30. Même 
	si je ne le reprends pas ici, notons :
	\begin{itemize}
	\item Attention, ne pas oublier que le $z$ caché dans $\hat{p}$ est dirigé vers le 
	haut.
	\item $\hat{p}$ est constant le long de la perpendiculaire à l'écoulement ; quand les 
	lignes de courants sont parallèles dans un écoulement, $\hat{p}$ est perpendiculaire 
	à la direction de l'écoulement.
	\end{itemize}
	
	
	\subsection{Conditions aux limites}
	Intéressons nous aux ordres des équations d'un fluide parfait (Euler) et d'un fluide 
	visqueux newtonien incompressible :
	\begin{equation}
	\begin{array}{ll}
	\rho(\partial_0v_i + v_k\partial_kv_i) &= f_i-\partial_i p\\
	\rho(\partial_0v_i + v_k\partial_kv_i) &= f_i-\partial_i p + \mu\Delta\ v_i
	\end{array}
	\end{equation}
	Le seul terme différent est du à la viscosité qui, pas de bol, est le terme d'ordre 
	le plus élevé : si l'on rajoute de la viscosité, on change les équations \textbf{et 
	aussi} (ne pas oublier!) les conditions aux limites!
	
\section{Mouvement turbulent}
	\subsection{Nombre de Reynolds}
	Faisons un expérience de la pensée ! \textit{Considérons un tuyau d'arrosage cylindrique,
	transparent et en plastique. On s'arrange pour que l'eau entre d'un coté et sorte de l'autre.
	Injectons un liquide coloré (d'où l'importance du plastique, sinon trop dur (TWSS)). Si la
	vitesse d'écoulement est faible, il va colorer la ligne de courant. C'est l'écoulement 
	\textbf{laminaire} ; chaque ligne de courant est indépendante d'lune de l'autre.}\\
	
	\textit{Assez joué, augmentons la vitesse du fluide ! L aligne de courant va se mettre à onduler à
	cause de la viscosité des coefficients de frottements, mais cela reste laminaire. Si on augmente
	encore le débit l'eau claire va devenir totalement mélangée avec notre liquide coloré ; c'est 
	l'écoulement \textbf{turbulent}.}\ \\
	
	Le passage de l'état laminaire à l'état turbulent peut être prédit par la valeur d'un
	nombre sans dimensions : le nombre de Reynolds\footnote{Remarquons que le symbôle des
	nombres sans dimensions ont 'toujours' deux lettres! Comme Mach, $Ma$ notamment.}\\
	Pour un tube circulaire:
	\begin{equation}
	\text{Re} = \frac{UD}{\nu}
	\end{equation}
	où $U$ est la vitesse débitante (débit/section), $D$ le diamètre du tube et $\nu = \frac{
	\mu}{\rho}$ la 	viscosité cinématique.

	
	
	
	
	
	
	
	
	
	
	
	
	
	
	
	
	
	
	
	
	
	
	
	
	
	
	
	
	\subsection{Contrainte de turbulence}
	\subsection{Equations de Navier-Stokes moyennées}
	
\section{Les théorèmes de Bernouilli}