\chapter{Dynamique des fluides visqueux}
\section{Introduction}
Le \autoref{ch:ch5} nous avait montré que le fluide visqueux newtonien était défini par \autoref{eq:ComportementVisqNew} ou \autoref{eq:EqStokes}. Ces deux équations mettent 
en évidence les deux coefficients de viscosité $\lambda$ et $\mu$. Lorsque le fluide 
était incompressible, il y avait une relation entres ces deux coefficients de nos deux
équations : si on ajoute l'hypothèse de Stokes, ces relations sont vérifiées même si le
flide n'est pas incompressible.


\subsection{Interprétation physique}
Considérons $\vec{V}$ la vitesse de la plaque et $\vec{v}$ la vitesse du fluide. Le
long de la plaque mobile $\vec{v}=\vec{V}$ : le fluide visqueux colle et est entrainé
par la paroi, s'il était parfait ça ne serait pas le cas. Le log de la plaque fixe,
$\vec{v}=\vec{0}$.\\
	
On verra (spoil) que sous certaines conditions on a: 
\begin{equation}
	\vec{v} = V\frac{z}{L}\vec{1_x}
	\label{eq:VitCh10}
\end{equation}
Reprenons la loi de comportement \autoref{eq:EqStokes} :
\begin{equation}
	\tau_{ij} = -p\delta_{ij} + 2\mu\left(V_{ij} - \frac{1}{3}\delta_{ij}V_{kk}\right)
\end{equation}
où la vitesse est donnée par \autoref{eq:VitCh10}. Nous avons donc (où $w$, la vitesse
selon $z$ est nulle (vitesse est toujours horizontale)) :
\begin{equation}
	\begin{array}{ccccc}
		\tau_{xz} = 2\mu V_{xz} & \Leftrightarrow & \tau_{xz} = 2\mu\dfrac{1}{2}\left(\dfrac{ 
		\partial u}{\partial z}+\underbrace{\dfrac{\partial w}{\partial x}}_{=0}\right) &
		\Leftrightarrow & \tau_{xz}	= \mu \dfrac{\partial u}{\partial z}
	\end{array}
\end{equation}
où $\tau_{xz}$ est la contrainte tangentielle de frottement entre les filets fluides
voisins, due à la différence de vitesse de ces filets fluides.
	
\subsection{Unités}
\textbf{Inclure tableau slide 7}
	
\section{Equation de mouvement}
\subsection{Equation de Navier-Stokes}
La démarche est identique que celle suivie pour établir les équations d'Euler pour un
fluide parfait. Partons des équations générales du mouvement
\begin{equation}
	\rho v_i^\bullet = f_i + \partial_j \tau_{ij}
\end{equation}		
Notre loi de comportement est toujours donnée par l'équation de Stokes 
\autoref{eq:EqStokes}. La vitesse de déformation est donnée par 
\begin{equation}
	V_{ij} = \frac{1}{2}(v_{i,j} + v_{j,i})
\end{equation}
Le fluide étant incompressible, $\rho^\bullet = 0$:
\begin{equation}
	\rho^\bullet + \rho \partial_iv_i = 0 \rightarrow V_{ii} = 0 \rightarrow V_{kk}=0
\end{equation}
On peut alors se débarasser de $V_{kk}$ dans \autoref{eq:EqStokes}.\\
Introduisons la définition de $\tau_{ij}$ dans les équations du mouvement : 
\begin{equation}
	\rho v_i^\bullet = f_i + \partial_j[-p\delta_{ij} + 2\mu V_{ij}]
\end{equation}
En tenant compte de la définition des vitesses de déformations :
\begin{equation}
	\rho v_i^\bullet = f_i-\partial_i p + \mu\partial_j(\partial_iv_j+\partial_jv_i)
\end{equation}
En effectuant la distribution, compte-tenu du résultat que nous venons d'obtenir 
($\rho\partial_iv_i=0$), le terme $\mu\partial_i\partial_j v_j$ s'annule pour donner
:\\
	
\prop{\textsc{Equations de Navier-Stokes}
	\begin{equation}
		\rho v_i^\bullet = f_i - \partial_ip + \mu\partial_j\partial_jv_i
	\end{equation}}
	
\subsubsection{Exemple : écoulement entre deux plaques parallèles}
Commençons par énoncer ce que nosu avons : 
\begin{enumerate}
	\item Ecoulement permanent $\rightarrow \frac{\partial}{\partial t}=0$
	\item Fluide incompressible :$\rho^\bullet=0 \rightarrow \rho^\bullet + \rho
	      \divv\vec{v}=0 \rightarrow \divv \vec{v}=0$
	\item Fluide visqueux $\rightarrow$ équation de Navier-Stokes
	\item Ecoulement unidimensionnel $\rightarrow \vec{v} = (u,0,0)$
\end{enumerate}				
Comme nous avons $\divv \vec v=0$ où $\vec{v}=(u,0,0)$, cela implique :
\begin{equation}
	\frac{\partial u}{\partial x} = 0 \Rightarrow u = u(z)
\end{equation}
Comme nous sommes en stationnaire et en unidimensionnel, les équations 
de Stokes développées se simplifient fortement :
\begin{equation}
	\left\{\begin{array}{ll}
	\mu \frac{\partial^2u}{\partial z^2} &= \frac{\partial \hat{p}}{\partial x}\\
	0 &= \frac{\partial \hat{p}}{\partial y}\\
	0 &= \frac{\partial \hat{p}}{\partial z}
	\end{array}\right.
\end{equation}
Imposons les conditions aux limites $u(0)=0, u(L)=V$ et considérons la première 
équation de notre système : le premier membre est fonction de $z$ et le second de
$x$ impliquant que les deux fonctions doivent être constantes.
\begin{equation}
	\frac{d\hat{p}}{dx}= 0 \Rightarrow \hat{p}=ax+b
\end{equation}
Deux cas s'offrent à nous : 
\begin{enumerate}
	\item $a=0$\\
	      \begin{equation}
	      	u = V\frac{z}{L}\ \ \ ;\ \ \ v=0;\ \ \ w=0
	      \end{equation}
	\item $a\neq0$\\
	      \begin{equation}
	      	u = \frac{az}{2\mu}(z-L) + V\frac{z}{L}\ \ \ ;\ \ \ v=0;\ \ \ w=0
	      \end{equation}
\end{enumerate}
\textbf{Inclure schéma slide 25} et interprétation.
		
		
\subsection{Pression motrice}
Fréquemment, la force de volume est la pensenteur. En définissant $z$ comme l'axe
vertical positif vers le haut :
\begin{equation}
	\begin{array}{ccc}
		\rho\vec{F}=-\rho g\vec{1_z}, & \vec{F}=-\grad U, & U=gz 
	\end{array}
\end{equation}
L'équation du mouvement s'écrit 
\begin{equation}
	\rho v_i^\bullet = \rho F_i - \partial_i p +\mu\partial_j\partial_jv_i = -\grad\hat{p}
	+\mu\Delta\ \vec v
\end{equation}
où l'on utilise la pression motrice $\hat{p}=p+\rho gz$.\\
Un bel exemple sur l'\textit{écoulement unidimensionnel} est donné slide 27-30. Même 
si je ne le reprends pas ici, notons :
\begin{itemize}
	\item Attention, ne pas oublier que le $z$ caché dans $\hat{p}$ est dirigé vers le 
	      haut.
	\item $\hat{p}$ est constant le long de la perpendiculaire à l'écoulement ; quand les 
	      lignes de courants sont parallèles dans un écoulement, $\hat{p}$ est perpendiculaire 
	      à la direction de l'écoulement.
\end{itemize}
	
	
\subsection{Conditions aux limites}
Intéressons nous aux ordres des équations d'un fluide parfait (Euler) et d'un fluide 
visqueux newtonien incompressible :
\begin{equation}
	\begin{array}{ll}
		\rho(\partial_0v_i + v_k\partial_kv_i) & = f_i-\partial_i p                  \\
		\rho(\partial_0v_i + v_k\partial_kv_i) & = f_i-\partial_i p + \mu\Delta\ v_i 
	\end{array}
\end{equation}
Le seul terme différent est du à la viscosité qui, pas de bol, est le terme d'ordre 
le plus élevé : si l'on rajoute de la viscosité, on change les équations \textbf{et 
aussi} (ne pas oublier!) les conditions aux limites!
	
\section{Mouvement turbulent}
\subsection{Nombre de Reynolds}
Faisons une expérience de la pensée ! \textit{Considérons un tuyau d'arrosage cylindrique,
	transparent et en plastique. On s'arrange pour que l'eau entre d'un coté et sorte de l'autre.
	Injectons un liquide coloré (d'où l'importance du plastique, sinon trop dur (TWSS)). Si la
	vitesse d'écoulement est faible, il va colorer la ligne de courant. C'est l'écoulement 
	\textbf{laminaire} ; chaque ligne de courant est indépendante l'une de l'autre.}\\
	
\textit{Assez joué, augmentons la vitesse du fluide ! La ligne de courant va se mettre à onduler à
	cause de la viscosité des coefficients de frottements, mais cela reste laminaire. Si on augmente
	encore le débit l'eau claire va devenir totalement mélangée avec notre liquide coloré ; c'est 
	l'écoulement \textbf{turbulent}.}\ \\
	
Le passage de l'état laminaire à l'état turbulent peut être prédit par la valeur d'un
nombre sans dimensions : le nombre de Reynolds\footnote{Remarquons que le symbôle des
	nombres sans dimensions ont 'toujours' deux lettres! Comme Mach, $Ma$ notamment.}\\
Pour un tube circulaire:
\begin{equation}
	\text{Re} = \frac{UD}{\nu}
\end{equation}
où $U$ est la vitesse débitante (débit/section), $D$ le diamètre du tube et $\nu = \frac{
	\mu}{\rho}$ la 	viscosité cinématique.\\
		
	
Le nombre de Reynolds est tel que pour chaque cas particulier, il existe une valeur limite pour 
laquelle, si on est au dessus ou en dessous on est soit en laminaire soit en turbulent. On le 
définit à l'aide d'une vitesse caractéristique de l’écoulement (vitesse débitante) et de la longueur
caractéristique de l’écoulement (pas la longueur du tuyau mais le diamètre !)	:
\begin{equation}
	Re = \frac{\text{vitesse caractéristique\ \ \ \ \ longueur caractéristique}}{\text{viscosité 
	cinématique}}
\end{equation}
Ainsi, pour un tube cylindriques : $U$ est la vitesse moyenne et $D$ le diamètre du tube. Pour
une aile d'avion, $U$ est la vitesse de l'avion et $D$ la corde de l'aile.\\
Les vitesses, qu'on soit en laminaire ou turbulent, sont solutions des équations de Navier-
Stokes ; comme elles ne sont pas linéaires, on numérise!
	
\subsection{Contrainte de turbulence}
Considérons une grandeur $A(t)$ quelconque liée à l'écoulement, représenté par la courbe verte. 
La courbe courbe rouge est la valeur moyenne de $A(t)$ définie par\footnote{Si $T$ est grand, 
cette courbe sera lisse.} : 
est donné par 
\begin{equation}
	\overline{A}(t) = \frac{1}{T}\int_{-T/2}^{T/2} A(t+\tau)d\tau
\end{equation}
\textbf{INCLURE IMAGE SLIDE 37}
On peut voir la courbe verte comme étant la courbe rouge avec des fluctuations. On va choisir 
un intervalle $T$ sur lequel on va calculer la moyenne, cet intervalle $T$ doit être :
\begin{itemize}
	\item Assez grand pour que $\overline{A}(t)$ ne dépende pas de $T$.
	\item Petit par rapport aux constantes de temps de l'écoulement moyen.
\end{itemize}
La moyenne des fluctuations doit alors être égale à zéro. En décomposant $A(t)$ :
\begin{equation}
	A(t) = \overline{A}(t) + \underbrace{A'(t)}_{0}
\end{equation}
	
	
	
\subsection{Equations de Navier-Stokes moyennées}
Avant tout, passons en revue quelques propriétés mathématiques. Dans le cas d'un produit de deux 
grandeurs $A$ et $B$\footnote{"La moyenne d'un produit c'est ça. Je suis un peu rapide mais 
acceptez-le, vous les verrez en MA1".} :
\begin{equation}
	\overline{AB} = \overline{(\overline{A}+A')(\overline{B}+B')} = \overline{A}\overline{B} + 
	\overline{A'B'}
\end{equation}
Dans le cas des dérivées :
\begin{equation}
	\frac{\overline{\partial A}}{\partial t} = \frac{\partial}{\partial t}\overline{A}\ \ \ \ \ \ 
	\ \ \ \ \ \ \ \ \ \ \ \ \ \ \ \frac{\overline{\partial A}}{\partial x_i} = \frac{\partial}{
		\partial x_i}\overline{A}
\end{equation}
	
Reprenons : partons des équations de Navier-Stokes :
\begin{equation}
	\rho \frac{\partial v_i}{\partial t} + \rho v_k\partial_kv_i = f_i - \partial_ip + \mu \Delta
	v_i
\end{equation}
Comme nous sommes dans le cas incompressible, $\rho$ est constant : on peut le sortir lorsqu'on 
moyenne :
\begin{equation}
	\rho \frac{\overline{\partial v_i}}{\partial t} + \rho \overline{v_k\partial_kv_i} = f_i - 
	\overline{\partial_ip} + \mu \overline{\Delta v_i}
\end{equation}
Comme $\overline{AB} = \overline{A}\overline{B} + 	\overline{A'B'}$ (et la propriété pour la 
dérivée) on trouve :
\begin{equation}
	\rho \frac{\partial \overline{v_i}}{\partial t} + \rho \overline{v_k}\partial_k\overline{v_i} +
	\rho\overline{v_k'\partial_k v_i'} = f_i - \partial_i\overline{p} + \mu \Delta\overline{v_i}
\end{equation}
Un troisième terme s'est rajouté dans le membre de gauche : $\rho\overline{v_k'\partial_k v_i'}$.
Pour le calculer, repartons de l'équation de continuité moyennée :
\begin{equation}
	\overline{\partial_iv_i} = 0 \Leftrightarrow \partial_i\overline{v_i} = 0 \Rightarrow \partial_i
	v_i'=0
\end{equation}
Après quelques calculs assez fastidieux, on trouve :\\
	
\prop{\textsc{Équations de Navier-Stokes moyennées}
	\begin{equation}
		\rho \frac{\partial \overline{v_i}}{\partial t} + \rho \overline{v_k}\partial_k\overline{v_i} = 
		f_i - \partial_i\overline{p} + \partial_j\tau_{ij}' +\mu \Delta\overline{v_i}
	\end{equation}}\ \\
L'apparition des contraintes de turbulence signifie que les équations ont été moyennées\footnote{
Notons qu'il n'y Il n'y a pas de viscosité pour un fluide parfait.} ; on peut supprimer les barres 
des moyennes :
\begin{equation}
	\rho \frac{\partial v_i}{\partial t} + \rho v_k\partial_k v_i = f_i - \partial_i p + \partial_j
	\tau_{ij}' +\mu \Delta v_i
\end{equation}
La viscosité se traduit par le terme "laplacien" et par les contraintes de turbulence. Tout se
retrouve en effet dans ce terme de viscosité :
\begin{itemize}
	\item Les contraintes de turbulence résultent de l'échange de quantité de mouvement entre des 
	      filets fluides voisins, lorsque les particules passent par turbulence d'un filet de courant au 
	      filet voisin.
	\item Un fluide parfait est donc toujours en écoulement laminaire puisque ces échanges sont dus 
	      à la viscosité.
\end{itemize}
	
	
	
\section{Les théorèmes de Bernoulli}
Les slides commencent par un rappel des équations de Lamb ainsi que des effets sur la viscosité.
Venons-en directement aux faits :
	
\subsection{Rappels des deux premiers théorèmes}
Pour un fluide parfait, homogène, incompressible et dans la pesanteur.
	
\subsubsection{Théorème de Bernoulli 1}
Dans un écoulement permanent d'un fluide parfait homogène dans le champ de la pesanteur, 
l'énergie spécifique totale est constante le long d'une ligne de courant.
\begin{equation}
	\varepsilon = p + \rho_0gz + \rho_0\frac{v^2}{2} = C^{ste}
\end{equation}
		
\subsubsection{Théorème de Bernoulli 2}
Dans un écoulement permanent irrotationnel d'un fluide parfait homogène dans le champ de 
la pesanteur, l'énergie spécifique totale est constante dans tout l'écoulement.
\begin{equation}
	\varepsilon = p + \rho_0gz + \rho_0\frac{v^2}{2} = C^{ste}
\end{equation}
		
\subsection{Le théorème de Bernoulli 3}
Identifions une ligne de courant d'un écoulement permanent d'un fluide et calculons la charge en
un point puis après déplacement en un autre point, ... Si le fluide est parfait, la charge sera
constante. Si le fluide est visqueux, la charge en un point "plus haut" est plus élevée qu'en 
un point "plus bas" : il y a \textit{perte de charge}.
\begin{equation}
	\rho_0\dfrac{\partial \vec{v}}{\partial t} + 2\rho_0(\vec{\omega}\times\vec{v}) = -\grad \rho_0g
	\left[\underbrace{\dfrac{p}{\rho_0g}+z+\dfrac{v^2}{2g}}_{H}\right] + \vec{f'}
\end{equation}
où $H$ est la charge. Cette démonstration n'est pas à connaître "tellement elle est évidente avec 
pleins de \textit{voir MA1}".\\
		
\textbf{AJOUTER IMAGE SLIDE 51}
	
\subsection{Le théorème de Bernoulli 4}
Le théorème 3 est à utiliser dans un écoulement dans l'air, dans l'eau, ... mais \textbf{pas} dans
un tuyau ! En effet :
\begin{enumerate}
	\item La répartition des vitesses n'est pas uniforme dans une section transversale (la vitesse 
	      est nulle le long des parois).
	\item La seule grandeur utile est le débit, et pas les vitesses locales.
\end{enumerate}
On va donc cette fois s'intéresser non pas à une ligne de courant, mais à un \textit{tube de 
	courant}. La démonstration se trouve dans les slides 53 à 59. On conclut :\\
Le théorème de Bernoulli 4 s'énonce comme le 3, à condition :
\begin{itemize}
	\item De remplacer la vitesse $v$ par la vitesse débitante $U$.
	\item D'ajouter le coefficient $\alpha$ qui tient compte de la répartition non-uniforme des
	      vitesses dans chaque section\footnote{Dès que l'on voit un $\alpha$, on est d'office dans le 
	      cas d'un tube de courant et on a une vitesse moyenne.}.
\end{itemize}\ 
	
\prop{\textsc{Théorème de Bernoulli 4}
	\begin{equation}
		\left[\alpha_1\frac{U_1^2}{2g}+\frac{p_1}{\rho g} + z_1\right] = \left[\alpha_2\frac{U_2^2}{2g}+
		\frac{p_2}{\rho g} + z_2\right] + \Delta H_{1-2}
	\end{equation}}
	
	
\subsection{Les pertes de charge}
En pratique, il existe deux types de pertes de charge :

\subsubsection{Perte de charge répartie}
Il s'agit de la perte de charge entre deux sections, proportionnelle à la distance entre ces
deux sections :
\begin{equation}
	\Delta H = \frac{\lambda}{D}L\frac{U^2}{2g}
\end{equation}

\subsubsection{Perte de charge concentrée}
On trouve également des pertes de charge locales en raison des perturbations locales (tuyaux a 
forte courbe en un endroit, ...) :
\begin{equation}
	\Delta H = K\frac{U_{ref}^2}{2g}
\end{equation}
où $K$ est le coefficient dépendant du type de perturbation et $U_{ref}$ la vitesse caractéristique
de chaque type de configuration.
